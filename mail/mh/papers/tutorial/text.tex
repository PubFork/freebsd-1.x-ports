% begin text

\banner

\section{Acknowledgements}
The \MH/ system described herein is based on the original Rand \MH/ system.
It has been extensively developed (perhaps too much so) by Marshall Rose and
John Romine at the University of California, Irvine.
Einar Stefferud, Jerry Sweet, and Terry Domae provided numerous suggestions
to improve the UCI version of \MH/.

Parts of this document are taken from a Rand tutorial \cite{SPayn85} by
Sue Payne.

\section{Disclaimer}
The Regents of the University of California issue the following
disclaimer concerning the UCI version of MH:
\bigquote
Although each program has been tested by its contributor,
no warranty, express or implied,
is made by the contributor or the University of California,
as to the accuracy and functioning of the program
and related program material,
nor shall the fact of distribution constitute any such warranty,
and no responsibility is assumed by the contributor
or the University of California in connection herewith.
\endbigquote

\section{Scope}
This document assumes that you have no knowledge of \MH/.
However, to use \MH/ you should have some familiarity with the \unix/ 
operating system,
particularly with the way commands are given,
how files are named,
the jargon (e.g. {\it shell}, {\it argument}, {\it home directory},
{\it pathname\/}),
and how to use a text editor (such as \pgm{ex}, \pgm{vi}, or \pgm{emacs\/}).

This tutorial covers only basic material.
For additional information about \MH/,
consult the {\it User's Manual} \cite{MRose85a}.
Other documents of possible interest to you include
{\it The UCI BBoards Facility} \cite{MRose84}
and
the {\it MH Administrator's Guide} \cite{MRose85b}.

\section{How To Use This Tutorial}
Different typefaces and symbols are used in this document to denote the
kinds of things you (the user) must type on your keyboard.
\smallskip
{\advance\leftskip by\parindent
\item{1.} The names of programs are given in {\it text italics}:
\smallskip\hskip 1in \pgm{comp}\smallskip
\item{2.} Arguments to programs are given in {\tt typewriter style},
delimited by single-quotes:
\smallskip\hskip 1in \arg{msgs}\smallskip
\item{3.} \unix/ pathnames are given in {\sl slanted roman}:
\smallskip\hskip 1in \file{/usr/uci/}\smallskip
\item{4.} Text giving a full example is presented in {\tt typewriter style}:
\example comp\ -editor\ vi\endexample
The ``\hbox{\tt\char`\ }'' glyph % (visible space glyph)
is used to indicate an explicit space (the kind you make with the
space bar on your keyboard).
\smallskip}

\section{Introduction}
With \MH/ you can send messages to other people on your system
and read messages that other people send to you.
Depending on how things have been set up on your system,
it may be possible for you to send messages to people on remote systems.
You can also reply to messages that you have received,
review them,
organize them in {\it folders},
and delete them.

\MH/ differs from other mail programs in that it is composed of many
small programs instead of just one very large program.
Among new users this sometimes causes some confusion
along the lines of ``what program do I run?''
With \MH/, you use the shell to invoke one program at a time.
This means that when you handle mail,
the entire power of the shell is at your disposal
in addition to the facilities that \MH/ provides.
In the beginning, this may not make much sense or may not seem important.
However, we have found that as new users of \MH/ gain experience, 
they find this style of interface to be very useful.

\section{Summary}
The most minimal list of \MH/ commands that you can get by with is:
\smallskip
{\advance\leftskip by\parindent
\item{\pgm{inc}}	- incorporate mail (get new mail)
\item{\pgm{show}}	- show the first message
\item{\pgm{next}}	- show the next message
\item{\pgm{prev}}	- show the previous message
\item{\pgm{comp}}	- compose a new message to send
\item{\pgm{repl}}	- reply to a received message
\smallskip}
\pgm{Comp} and \pgm{repl} give enough prompting possibly to get you along.
However, it is suggested that you take the time to peruse this
tutorial before leaping into things.

\section{Messages and Folders}
A message takes the form of a memorandum,
and is composed of two major parts:
a {\it header},
which contains such information as
\eg{To} and \eg{From} addresses, \eg{Subject}, \eg{Date}, etc.;
and the {\it body},
which is the actual text of the message.
Each {\it component} in the header starts with a keyword followed by
a colon and additional information.
For example, in the message:
\example
    Date: 10 Oct 84 17:41:14 EDT (Wed)\\
    To: News@udel-dewey\\
    Subject: UCI Software Talk\\
    From: UCI Portal (agent: Marshall Rose) <uci@udel-dewey>\\\\
    This is the text.
\endexample
there are four header items, and one line of text in the body.
Note that a blank line separates the body from the headers.

\MH/ stores a message as an ordinary file in a \unix/ directory.
This directory is called a {\it folder}.
If you choose to keep and organize your messages,
you may create as many folders as you wish.
There is no limit as to the number of messages in a folder.
Typically messages are numbered from~1 up.
All of your personal folders,
along with some other information that \MH/ needs to know,
are kept in a special directory called \file{Mail} under your home directory.
Normally, \MH/ manages these files and directories automatically,
so you needn't muck around with them directly unless you really want to.

You won't have any folders until somebody sends mail to you, as a rule.
If you are anxious to try out \MH/, but no one has sent you mail yet,
try sending mail to yourself to start out with.

\section{Reading New Mail}
When you are notified that you have mail (usually when you log in),
perhaps with the message
\example You have mail.\endexample
then you know that messages are waiting in your {\it maildrop}.
To read these messages, you first have to {\it incorporate} the mail
into your ``in-box'' by typing the command:
\example inc\endexample
This incorporates the new mail from your mail drop to your in-box,
which is a folder named (naturally enough) \arg{+inbox}.
As \pgm{inc} incorporates your new mail,
it generates a {\it scan listing} of the mail:
$$\vbox{\tenpoint\tx\halign{\hfil#&#\hfil&&	\quad#\hfil\cr
\noalign{\noindent Incorporating new mail into inbox...\medskip}
2&+&	10/10&	WESTINE\%USC-ISIF&	RFC 916 Now Available&
					<<A new Request for Co\cr
3&&	10/10&	G B Reilly&		Gosling EMACS manual&
					<<Marshall, I am lookin\cr
4&&	10/11&	WESTINE\%USC-ISIF&	Internet Monthly Report&\cr
}}$$
Each time \pgm{inc} is invoked,
any new messages are added to the end of your \eg{+inbox} folder.

To read the first message,
use the \pgm{show} command:
\example show\endexample
This displays the current message.
To read each subsequent message,
use the \pgm{next} command:
\example next\endexample
If you want to back up,
the command \pgm{prev} shows the previous message.
Another way to read your messages is to name them all at once:
\example show\ all\endexample
This command displays them all, one after the other.
The \arg{all} argument to \pgm{show} above might also be replaced
with \arg{next} or \arg{prev}, as in
\example
    show\ next\\
    show\ prev
\endexample
which are respectively equivalent to the \pgm{next} and \pgm{prev}
commands.

If you have had occasion to type \pgm{inc} more than once, then
you will find that \eg{show\ all} is showing not only the new messages,
but also the old messages that you've already seen.
Therefore, you might find it better to use
\example show\ cur-last\endexample
instead.
This command displays messages from the current message (\arg{cur})
to the last message (\arg{last}).
Each time \pgm{inc} is invoked, it makes the first new message
the current message.
It should be noted here that the name \arg{all} given in a previous
example is equivalent to the {\it message range} \arg{first-last},
where \arg{first} is the name of the first message in \arg{+inbox}.
Also, \eg{show} by itself is equivalent to
\example show\ cur\endexample

As mentioned earlier,
with the \unix/ shell as your interface to \MH/,
it becomes easy to list a message on a line printer or to another file.
For example,
\example show\ all\ |\ lpr\endexample
lists all the messages in the current folder to the line printer.

To summarize, the preceding has introduced these important concepts:
{\it folders} (in particular, the \arg{+inbox} folder),
{\it messages},
{\it message names} (e.g. \arg{prev}, \arg{next}, \arg{cur}, \arg{last}),
and {\it message ranges} (e.g. \arg{cur-last}, \arg{all}).
More will be said about folders and messages in succeeding sections.

\section{Sending Messages}
To send a message, you compose a message {\it draft},
either by replying to a message that someone sent to you,
or by creating a draft from scratch.
The \pgm{send} command is used {\bf after} completing the final draft
of a message, 
in the same way that you mail a paper letter only after you are finished
writing it.
This is a common source of confusion among new \MH/ users who
may have had experience with other mail systems.

This section discusses how to originate messages
and how to reply to messages that were previously received,
along with a word or two about addresses.

\subsection{Originating Messages}
To create a message draft from scratch,
use the \pgm{comp} program.
You will be prompted for the header components
and then the body of the message.
If you make a mistake, you may correct it later with a text editor.
The draft will be sent only if you give an explicit \pgm{send} command,
so you do not have to worry about the draft getting away from you
prematurely.

To start, you simply type:
\example comp\endexample

{\bf To:}
First, the prompt \arg{To:} appears.
Here you type the address of the person to whom you wish the message sent.
If this person is on the same computer system as you,
then that person's login ID should serve as the address
(e.g. \arg{mrose} or \arg{jsweet}).

Here we digress briefly to discuss addresses.
A full discussion of addresses is beyond the scope of this
tutorial, but it should be mentioned that there are other
kinds of addresses besides login IDs.
To send messages to people on remote systems, 
the usual way is to type \arg{login-id@host} in the \arg{To:} component,
as in \arg{MRose@UCI-ICSA}.
Examples of \arg{host} names at UCI include
\arg{uci-icsa},
\arg{uci-icse},
and \arg{uci-cip1}.
Upper and lower case letters may be used interchangeably.
Sometimes a person's last name (e.g. \arg{Rose}, \arg{Sweet}) can be used
instead of a login ID,
but this cannot be relied upon in a world without unique surnames.

{\bf cc:}
After you have given an address to the \arg{To:} prompt, 
you are prompted for the \arg{cc:}
(``carbon copy''--an archaism)
address.
It is customary, but not required, to put your own address
here so that you get a copy of the message when it is sent.

To put more than one address in the \arg{To:} and
\arg{cc:} components,
just use a comma (``,'') between each address on a line.

{\bf Subject:}
The third prompt is for the \arg{Subject:} component.
Here a line of any descriptive text will do.
Once you have typed a line of text, a dashed line is printed,
and you are then expected to type the body of the message.
End the body with EOT (usually CTRL-D).

An example of a complete message draft, as it appears on your screen,
might be:
\example
    To: News\\
    cc: farber, mrose\\
    Subject: UCI Software Talk\\
    --------\\
    A presentation on the UCI software suite, including\\
    the Rand/UCI Mail Handling System (MH), will be given\\
    in CS220 on October 31st at 2:30 PM.  Refreshments\\
    will be served afterward.\\\\
    /mtr\\
    \^{}D
\endexample
(The ``\^{}D'' does not appear in the draft.)

At this point, you are asked
\example What\ now?\endexample
This is known as being at \whatnow/ level.
For now, there are probably only four options that will interest you:
\smallskip
{\advance\leftskip by\parindent
\item{\pgm{edit}} - edit the draft
\item{\pgm{list}} - list the draft on your screen
\item{\pgm{quit}} - quit, without sending the draft
\item{\pgm{send}} - send the draft, then quit
\medskip}
\noindent
All of these options take various arguments,
but only \pgm{edit} really needs an argument.

{\bf Edit:}
The \pgm{edit} option will let you edit the draft before sending it.
If your favorite text editor is \pgm{vi},
then you would use the \pgm{edit} option as:
\example edit\ vi\endexample
Just specifying \pgm{edit} with no argument
will only let you append text to the body of the
message draft.
Another editor (e.g. \pgm{vi}, \pgm{ex}, \pgm{emacs\/})
should really be run to finish the draft up.
When you leave the editor, you will come back to the \whatnow/ level,
where you can re-edit the draft, send it, list it, or simply quit
without sending the draft at all.

Caution: while in the editor,
you should not delete colons in the headers
or change the spelling of \arg{To:}, \arg{cc:}, or \arg{Subject:};
and do not leave blank lines between these lines.
Feel free to change the addresses that you typed previously,
or to add these lines if they are missing.
Do not delete the dashes that separate the header lines from
the text of the message.
You should not add additional header lines unless you understand
precisely what you are doing.  
This means particularly that you should not type or fill in a \arg{From:}
line.
When the message is sent, the system automatically adds this line.
Also, you should not type a \arg{Date:} line in the header.
When the message is sent, the system automatically adds the current
date and time.

{\bf Quit:}
If you \pgm{quit} without sending the draft,
the draft is saved in a file called \file{Mail/draft} under your
home directory.
This file can be recalled later using the \arg{-use} argument
to \pgm{comp}:
\example comp\ -use\endexample
The \whatnow/ level will permit you to do further editing
and to send the final draft when you are ready.

{\bf Send:}
When it is time to send the draft on its way,
use the \pgm{send} option by itself.
If there are any problems with the draft 
(for example,
if one or more of the people whom you specified in the \arg{To:} and \arg{cc:}
components do not exist)
then you will be notified at this time.

\subsection{Replying to Messages}
To reply to a message,
use the \pgm{repl} command.
For example,
\example repl\endexample
creates a reply to the current message.
You may also reply to a specific message (other than the current one)
by giving a {\it message number} (e.g. \arg{1}, \arg{4}, etc.)
or a {\it message name} (e.g. \arg{first}, \arg{last}, \arg{prev}):
\example repl\ prev\endexample
We haven't really introduced message numbers yet.
They will be discussed in the next section.

The process of replying to a message is very similar to composing
a message from scratch (see the previous section),
but \pgm{repl} conveniently constructs and displays the header
of the reply draft for you.
You need only type in the text of the reply.
An EOT (usually CTRL-D) indicates that you are done typing.
If you make a mistake, you may correct it later with a text editor.
The draft will be sent only if you give an explicit \pgm{send} command,
so you do not have to worry about the draft getting away from you
prematurely.

An example of a complete reply draft, as it appears on your screen might be:
\example
    To: MRose\\
    cc: JSweet\\
    Subject: Re: UCI Software Talk\\
    In-reply-to: Your message of 10 Oct 84 18:15:08 PDT (Wed).\\
    --------\\
    I'll be there.\\
    -jns\\
    \^{}D	
\endexample
(The ``\^{}D'' does not appear in the draft.)

At this point, you are asked
\example What\ now?\endexample
This is known as being at \whatnow/ level.
Refer to the previous section regarding how to edit,
display, or send the draft at this point.

As with \pgm{comp}, 
if you \pgm{quit} without sending the reply draft,
the draft is saved in a file called \file{Mail/draft} under your
home directory.
This file can be recalled later using the \arg{-use} argument
to \pgm{comp}:
\example comp\ -use\endexample
The \whatnow/ level will permit you to do further editing
and to send the final draft when you are ready.

\section{Scanning Messages}
The scan listing created by \pgm{inc} shows the {\it message number},
the date on which the message was sent,
the sender,
and the subject of the message.
If there is sufficient space remaining on the line,
the beginning of the text of the message is displayed as well,
preceded by two left angle brackets (``{\tenpoint\tx$<<$\/}'').
An example of a scan listing is:
$$\vbox{\tenpoint\tx\halign{\hfil#&#\hfil&&	\quad#\hfil\cr
1&+&	10/10&	WESTINE\%USC-ISIF&	RFC 916 Now Available&
					<<A new Request for Co\cr
2&&	10/10&	G B Reilly&		Gosling EMACS manual&
					<<Marshall, I am lookin\cr
3&&	10/11&	WESTINE\%USC-ISIF&	Internet Monthly Report&\cr
}}$$
Note that all messages have message numbers.

To generate your own scan listing, use the \pgm{scan} program.
Typing simply
\example scan\endexample
will list all the messages in the current folder.
To scan a subset of these messages,
you can specify the numbers of the messages that you consider interesting,
e.g.,
\example scan\ 2\ 3\endexample
Message names may be specified in addition to discrete message numbers.
The built-in message names recognized by \MH/ are:
\smallskip
{\advance\leftskip by\parindent
\item{\underbar{all}:}	all messages in the folder (\arg{first-last})
\item{\underbar{first}:}	the first message in the folder
\item{\underbar{last}:}	the last message in the folder
\item{\underbar{prev}:}	the message immediately before the current message
\item{\underbar{cur}:}	the current message
\item{\underbar{next}:}	the message immediately after the current message
\medskip}
\noindent

Message ranges may be specified in addition to discrete message numbers
or names by separating the beginning
and final message numbers with a dash (``-'').
For example,
\example scan\ 5-10\endexample
scans messages~5 through~10 inclusive.
A range of messages may also be specified by separating a beginning
message number and a relative number of messages with
a colon (``:'').
For example,
\example scan\ last:3\endexample
scans the last three messages in the folder.
Similarly,
\example scan\ first:3\endexample
scans the first three messages in the folder;
\example scan\ next:3\endexample
scans the next three messages;
\example scan\ cur:3\endexample
scans the three messages beginning from the current message;
\example scan\ 100:4\endexample
scans four messages beginning from message number 100.

To summarize, the important concepts that have been discussed
in the section are:
{\it message ranges},
{\it message numbers},
and {\it message names}.
When an \MH/ command is described as taking a \arg{msg} argument,
it accepts either a message name or a message number.
Most \MH/ commands are described as taking \arg{msgs} arguments,
meaning that more than one message or message range is accepted.

\section{Deleting Messages}
To delete a message, use the \pgm{rmm} program.
By default, \pgm{rmm} deletes the current message,
but you can give \pgm{rmm} a list of messages to be removed as well.
There is no corresponding ``\pgm{unrmm}'' program,
but clever users with a need will find out how to change the way \pgm{rmm}
works so that it simply moves messages to another folder
(say, \arg{+wastebasket}).

\section{Filing Messages}
The possibility of having folders other than \eg{+inbox} has been mentioned
previously.
The methods for moving messages between folders and manipulating folders
are discussed here.

The \pgm{refile} command moves messages from a {\it source folder} to one or
more {\it destination folders}.
By default, the current message is moved from the {\it current folder} 
(typically \arg{+inbox}) to another folder specified as an
argument to \pgm{refile}.
For example,
\example refile\ +todo\endexample
moves the current message from the current folder to the folder \eg{+todo}.
To move messages from a folder other than the current folder,
use the \switch{src +folder} switch, as in
\example refile\ -src\ +todo\ last\ +save\ +notes\endexample
which moves the last message in the \eg{+todo} folder to the folders
\eg{+save} and \eg{+notes}.
Note that this operation is a {\it move}, not a {\it copy};
it removes the message from the source folder.
To keep a copy in the source folder as well, use the \switch{link} switch
\example refile\ -link\ -src\ +todo\ last\ +save\ +notes\endexample

Whenever a folder argument is given to an \MH/ command,
that folder becomes the {\it current folder}.
To find out which folder is current, use the command
\example folder\endexample
The \pgm{inc} command sets the current folder back to \arg{+inbox}
by default.
To find out about all of a user's folders, use the command
\example folders\endexample
Since folders can contain other folders,
the command
\example folders\ -recurse\endexample
will recursively examine each folder for you.

To set the current folder, without doing anything else,
use the \pgm{folder} program with a folder argument.
Hence,
\example folder\ +inbox\endexample
makes \eg{+inbox} the current folder.

After a using \pgm{rmm} and \pgm{refile} on a folder a number of times,
there tend to be gaps in the numbering sequence.
To compress the numbers for the all messages in a folder,
use
\example folder\ -pack\endexample

\section{The Profile}
You can customize the \MH/ environment by editing your \profile/ file.
Although there are lots of options,
here are the most useful:
\smallskip
{\advance\leftskip by\parindent
\item{\underbar{Editor}:} lists the default editor that \pgm{comp} and
\pgm{repl} should use.
The default is
\example editor:\ prompter\endexample
but another editor might be preferred.

\item{\underbar{{\it editor}-next}:} lists the editor that should be used
after the last edit with {\it editor}.
Hence, if you have a profile entry
\example prompter-next:\ vi\endexample
after editing a draft with \pgm{prompter},
and being at \whatnow/ level,
you could say \eg{edit} (instead of \eg{edit vi})
to continue to edit the draft with \pgm{vi}.

\item{\underbar{Msg-Protect}:}
Whenever \MH/ creates a message (for example, with \pgm{inc\/}),
this is the octal protection mode that the message is created with.
The default is
\example Msg-Protect:\ 644\endexample
This protection mode permits all other users on the system to read
your messages.
To maintain privacy, the mode 600 should be used.
Note that changing the mode in the profile does not change the modes
of messages that have been created already.
Use the \unix/ command \pgm{chmod} to change the modes of your
existing messages.

\item{\underbar{Folder-Protect}:}
Whenever \MH/ creates a folder (for example, with \pgm{refile\/}),
this is the octal mode that the folder is created with.
The default is
\example Folder-Protect:\ 711\endexample
This mode permits other users on the system to make access to
specific messages in your folders.
To maintain stricter privacy, the mode 700 should be used.

\item{\underbar{{\it program\/}}:}
Each \MH/ program that reads user's \profile/ file
looks for an entry beginning with its own
name to determine its initial defaults.
For example,
if you want the default editor for \pgm{repl} to be \pgm{emacs},
the line
\example repl:\ -editor\ emacs\endexample
is sufficient.
Command line arguments tend to override profile settings.
Given the profile setting for \pgm{repl} above,
if you invoked \pgm{repl} with
\example repl\ -editor\ vi\endexample
\pgm{repl} would use the \pgm{vi} editor instead
of \pgm{emacs}.

\item{\underbar{signature}:}
When \MH/ posts mail for you,
it looks for this profile entry for your ``real world'' name.
For example,
\example signature:\ Marshall\ Rose\endexample
The contents of the \eg{signature:} entry in the profile should be a simple
phrase, with no embedded periods (e.g. ``Marshall T.~Rose'').
\medskip}
\noindent
Note that your profile resembles the header portion of a message.
Be sure that it is properly formatted by placing a colon after each entry
name,
and keep each entry on a single line.

\section{Conventions}
Now let's summarize the conventions that \MH/ programs use:
\smallskip
{\advance\leftskip by\parindent
\item{1.} Any \MH/ command that deals with messages can be given a
\arg{+folder} argument to say which folder to use.
However, only one \arg{+folder} argument may be given per command
in most cases.

\item{2.} If an \MH/ command accepts a \arg{msgs} argument,
then any number of messages can be given to the command.
The \MH/ command will expand all the ranges and process each message,
starting with the lowest numbered one and working its way to the message with
the highest number.

\item{3.} If an \MH/ command accepts a \arg{msg} argument,
then at most one message can be given.

\item{4.} Switches (options) to \MH/ commands start with a dash.
Unlike the standard \unix/ convention,
each switch consists of more than one character,
for example \switch{header}.
To minimize typing,
only a unique abbreviation of the switch need be typed;
thus for \switch{header}, \switch{hea} is probably sufficient,
depending on the other switches accepted by the command.

\item{5.} All \MH/ commands have a \switch{help} switch,
which {\it must} be spelled out fully.
When an \MH/ command encounters the \switch{help} switch,
it prints out the syntax of the command,
the switches that it accepts,
and version information.
In the list of switches,
parentheses indicate required characters.
For example,
all \switch{help} switches will appear as \switch{(help)},
indicating that no abbreviation is accepted.

\item{6.} Many \MH/ switches have both on and off forms,
such as \switch{format} and \switch{noformat}.
In these cases,
the last occurrence of the switch on the command line determines the setting
of the option.

\item{7.} All \MH/ commands that read your \MH/ profile operate the
same way:
\underbar{first},
the profile is consulted for an entry matching the name with which
the command was invoked;
\underbar{second},
if such an entry was found,
then the command immediately uses the arguments listed;
\underbar{third},
any arguments on the command line are then interpreted.
Since most switches have both on and off forms,
it's easy to customize the default options for each \MH/ command in the
\profile/,
and to override those defaults on the command line.
\smallskip}

\section{Online Documentation}
Each \MH/ program has its own \unix/ manual entry.
For example, to get information about \pgm{comp},
type
\example man\ comp\endexample
The manual entry for \man mh(1) lists all \MH/ commands,
while the manual entry for \man mh-chart(1) lists the syntax and switches for
all \MH/ commands.

In addition,
here are a few other manual entries might be found useful:
\smallskip
{\advance\leftskip by\parindent
\item{\man mh-alias(5)} to find out how aliases in \MH/ work;
\item{\man mh-mail(5)} to find out how \MH/ stores and interprets messages
(this manual entry explains all of the standard header components);
\item{\man mh-profile(5)} to find out about the \MH/ user-environment.
\smallskip}

The manual pages for \MH/ are in the standard \unix/ format,
but contain additional sections unique to \MH/.
Here's a summary of the sections one might find in an \MH/ manual entry:
\smallskip
{\advance\leftskip by\parindent
\item{\sc Name} command name and one-line description.

\item{\sc Synopsis} syntax of the command.\hbreak
All commands accept a \switch{help} switch.

\item{\sc Description} semantics of the command.

\item{\sc Files} files used by the command\hbreak
Almost always this includes \file{.mh\_profile}.

\item{\sc Profile} entries in the \profile/ used by the command;
\vskip -\parskip
\item{\sc Components} these do not include the profile entry for the
command itself.

\item{\sc See Also} other \unix/ manual entries (usually \MH/ programs) that
are related to this command.

\item{\sc Defaults} default arguments for the command\hbreak
If the command takes a \arg{+folder} argument,
this defaults to the current folder.
If the command takes a \arg{msg} argument,
this defaults to the current message.
If the command takes a \arg{msgs} argument,
this defaults to the current message or all messages,
depending on which one makes more sense.

\item{\sc Context} changes to your \MH/ context made by the command.

\item{\sc Hints} Helpful hints discussing the easy way to do things.

\item{\sc History} A historical perspective on why \MH/ works the way it does.

\item{\sc Bugs} Too embarrassing to mention.\hbreak
Just kidding.
\medskip}
\noindent
Obviously, not all \MH/ manual entries may have all of these sections.

\section{Reporting Problems}
If problems are encountered with an \MH/ program,
the problems should be reported to the local maintainers of \MH/.
When doing this,
the name of the program should be reported,
along with the version information for the program.
To find out what version of an \MH/ program is being run,
invoke the program with the \switch{help} switch.
In addition to listing the syntax of the command,
the program will list information pertaining to its version.
This information includes the version of \MH/,
the host it was generated on,
the date the program was loaded,
and the configuration options in effect when \MH/ was generated.
For example,
\example
    version: MH 6.1 \#1[UCI] (gremlin) of Wed Nov  6 01:13:53 PST 1985\\
    options: [BSD42] [MHE] [NETWORK] [SENDMTS] [MMDFII] [SMTP] [POP]\endexample
The \eg{6.1~\#1[UCI]} indicates that the program is from the UCI \mh6
version of \MH/.
The program was generated on the host \eg{gremlin} on
\eg{Wed Nov  6 01:13:53 PST 1985}.
It's usually a good idea to send the output of the \switch{help} switch along
with your report.

If there is no local \MH/ maintainer,
try the address {\tx Bug-MH}.
If that fails, use the Internet mailbox {\tx Bug-MH@UCI.ARPA}.

\section{More on MH}
There are myriad aspects of \MH/ that this tutorial hasn't touched upon.
Here are a few to whet your appetite:
\smallskip
{\advance\leftskip by\parindent
\item{1.} user-defined sequences\hbreak
Define {\it meaningful} message names and shorten type-in considerably
(see \man pick(1) for details).

\item{2.} draft folders\hbreak
Maintain a folder of drafts so that more than one draft can be edited at a
time,
and allow a draft to be edited over several \unix/ sessions independently of
other drafts
(see the {\bf Advanced Features} section of the \MH/ user's manual for
details).

\item{3.} draft pushing\hbreak
Post a draft in the background
and immediately free your terminal for other activities
(see the {\bf Advanced Features} section of the \MH/ user's manual for
details).

\item{4.} aliases\hbreak
Maintain one or more alias files containing the addresses of the people
frequently (or infrequently) sent to.
This lets you shorten type-in of addressees
and saves you from looking up
their addresses all the time.
(see \man mh-alias(5) for details).
\smallskip}
