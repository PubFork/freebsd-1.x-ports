% RCS LOG
%
% $Log: news.tex,v $
% Revision 1.1  1994/02/08 00:23:12  jkh
% Initial revision
%
% Revision 1.2  1991/10/29  19:27:27  db
% Added documentation of the -poly_make option to INSTALL.
%
% Revision 1.1  1991/10/22  20:32:00  db
% Initial revision
%

\documentstyle[a4,12pt]{article}

\title{Changes to the Edinburgh SML Library}
\author{Dave Berry}
\date{October 22, 1991}

\begin{document}

\maketitle

\begin{abstract}
This document presents chnages and additions to the specification
of the Edinburgh Standard ML Library since the original documentation
was published as a technical report.  It uses the same format as
the original report.
\end{abstract}

\tableofcontents

\newpage
\section{Installation}

The {\tt INSTALL} program has a new option, {\tt -poly\_make}.
This creates a sub-directory called {\tt poly\_make}, which
contains links to all the files in the library.  Where the
library contains versions specific to Poly/ML, it uses these.
If users create a symbolic link to this directory, they can use
the Poly/ML make facility ({\tt PolyML.make}) to load library
entries, if they first load the {\tt poly\_make.load} file.


\section{Top Level}
\begin{verbatim}
val libraryVersion: string
 (* libraryVersion; a string giving the current version number of
    the library, and the date and time that it was released. *)
\end{verbatim}

\section{AsciiOrdString}
Case sensitive comparison operations on strings. See FULL\_ORD.

\section{Boolvector}
See MonoVector.

\section{Bytevector}
See MonoVector.

\newpage
\section{CORE\_ARRAY}
\begin{verbatim}
signature CORE_ARRAY =

(* CORE ARRAY FUNCTIONS

Created by:     Dave Berry, LFCS, University of Edinburgh
                db@lfcs.ed.ac.uk
Date:           24 Jan 1991

Maintenance:    Author


DESCRIPTION

   This is the implementation of arrays agreed between the implementors
   of SML/NJ, Poly/ML and Poplog ML in Autumn 1990.  The main library
   adds more functionality.

*)

sig

  eqtype 'a array

  exception Size

  exception Subscript

  val array: int * '_a -> '_a array

  val arrayoflist: '_a list -> '_a array

  val tabulate: int * (int -> '_a) -> '_a array

  val sub: 'a array * int -> 'a

  val update: 'a array * int * 'a -> unit

  val length: 'a array -> int
end
\end{verbatim}

\newpage
\section{CORE\_VECTOR}
\begin{verbatim}
signature CORE_VECTOR =

(* CORE VECTOR FUNCTIONS

Created by:     Dave Berry, LFCS, University of Edinburgh
                db@lfcs.ed.ac.uk
Date:           24 Jan 1991

Maintenance:    Author


DESCRIPTION

   This is the implementation of vectors agreed between the implementors
   of SML/NJ, Poly/ML and Poplog ML in Autumn 1990.  The main library
   adds more functionality.


*)

sig

  eqtype 'a vector

  exception Size

  exception Subscript

  val vector: 'a list -> 'a vector

  val tabulate: int * (int -> 'a) -> 'a vector

  val sub: 'a vector * int -> 'a

  val length: 'a vector -> int
end
\end{verbatim}

\newpage
\section{EQ\_SET}
The following functions have been added:
\begin{verbatim}
(* ITERATORS *)

  val map: ('a -> 'b) -> 'a Set -> 'b Set
   (* map f s; builds a set by applying f to each element of s. *)

  val apply: ('a -> unit) -> 'a Set -> unit
   (* apply f s; applies f to each element of s. *)


(* REDUCERS *)

  val fold: ('a -> 'b -> 'b) -> 'b -> 'a Set -> 'b
   (* fold f s base; folds using f over the base element. *)

  val fold': ('a -> 'a -> 'a) -> 'a Set -> 'a
   (* fold' f s; folds using f over an arbitrary element of s.
      Raises (Empty "fold'") if s is empty. *)
\end{verbatim}


\newpage
\section{FULL\_ORD}
\begin{verbatim}
(*$FULL_ORD *)

signature FULL_ORD =
sig

(* A TYPE WITH ORDERING FUNCTIONS

Created by:     Dave Berry, LFCS, University of Edinburgh
                db@lfcs.ed.ac.uk
Date:           17 Sep 1991

Maintenance:    Author


DESCRIPTION

   This signature defines a type T and a full set of ordering functions.


SEE ALSO

   ORDERING, FULL_SEQ_ORD, SEQ_ORD, EQ_ORD, EQTYPE_ORD.


*)


(* TYPES *)

  type T


(* OBSERVERS *)

  val lt: T -> T -> bool
   (* lt x y; returns true if x is less than y; returns false otherwise. *)

  val le: T -> T -> bool
   (* le x y; returns true if x is less than or equal to y; returns
      false otherwise. *)

  val gt: T -> T -> bool
   (* gt x y; returns true if x is less than y; returns false otherwise. *)

  val ge: T -> T -> bool
   (* ge x y; returns true if x is less than or equal to y; returns
      false otherwise. *)

end;
\end{verbatim}


\newpage
\section{FULL\_SEQ\_ORD}
\begin{verbatim}
(*$FULL_SEQ_ORD *)

signature FULL_SEQ_ORD =
sig

(* A PARAMETERISED TYPE WITH ORDERING FUNCTIONS

Created by:     Dave Berry, LFCS, University of Edinburgh
                db@lfcs.ed.ac.uk
Date:           17 Sep 1991

Maintenance:    Author


DESCRIPTION

   This signature defines a type 'a T and a full set of ordering functions.


SEE ALSO

   SEQ_ORD, FULL_ORD, ORDERING

*)


(* TYPES *)

  type 'a T


(* OBSERVERS *)

  val lt: ('a -> 'a -> bool) -> 'a T -> 'a T -> bool
   (* lt p x y; returns true if x is less than y, using p to compare elements
      when necessary; returns false otherwise. *)

  val le: ('a -> 'a -> bool) -> 'a T -> 'a T -> bool
   (* le p x y; returns true if x is less than or equal to y, using p to
      compare elements when necessary; returns false otherwise. *)

  val gt: ('a -> 'a -> bool) -> 'a T -> 'a T -> bool
   (* gt p x y; returns true if x is less than y, using p to compare elements
      when necessary; returns false otherwise. *)

  val ge: ('a -> 'a -> bool) -> 'a T -> 'a T -> bool
   (* ge p x y; returns true if x is less than or equal to y, using p to
      compare elements when necessary; returns false otherwise. *)

end;
\end{verbatim}

\newpage
\section{LexOrdList}
Comparison operations on lists.  See FULL\_SEQ\_ORD.

\section{LexOrdString}
Case insensitive comparison operations on strings.  See FULL\_ORD.

\newpage
\section{MONO\_SET}
The following functions have been added:
\begin{verbatim}
(* ITERATORS *)

  val map: (Element -> Element) -> Set -> Set
   (* map f s; builds a new monoset by applying f to each element of s. *)

  val apply: (Element -> unit) -> Set -> unit
   (* apply f s; applies f to each element of s. *)


(* REDUCERS *)

  val fold: (Element -> 'b -> 'b) -> 'b -> Set -> 'b
   (* fold f s base; folds using f over the base element. *)

  val fold': (Element -> Element -> Element) -> Set -> Element
   (* fold' f s; folds using f over an arbitrary element of s.
      Raises (Empty "fold'") if s is empty. *)
\end{verbatim}

\newpage
\section{SET}
The following functions have been added:
\begin{verbatim}
(* ITERATORS *)

  val map: ('a -> 'b) -> 'a Set -> 'b Set
   (* map f s; builds a set by applying f to each element of s. *)

  val apply: ('a -> unit) -> 'a Set -> unit
   (* apply f s; applies f to each element of s. *)


(* REDUCERS *)

  val fold: ('a -> 'b -> 'b) -> 'b -> 'a Set -> 'b
   (* fold f s base; folds using f over the base element. *)

  val fold': ('a -> 'a -> 'a) -> 'a Set -> 'a
   (* fold' f s; folds using f over an arbitrary element of s.
      Raises (Empty "fold'") if s is empty. *)
\end{verbatim}

\newpage
\section{STRING\_LIST\_OPS}
The following exception was omitted, and has been added:
\begin{verbatim}
  exception Subscript of string * int
  (* Subscript (fn, n); raised if the function named fn is called with
     an out of range argument n. *)
\end{verbatim}

\end{document}

