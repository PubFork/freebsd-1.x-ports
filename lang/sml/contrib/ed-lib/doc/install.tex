% RCS LOG
%
% $Log: install.tex,v $
% Revision 1.1  1994/02/08 00:23:12  jkh
% Initial revision
%
%Revision 1.4  1991/10/29  19:46:27  db
%Added paragraph on how to *use* the poly_make version of the library.
%
%Revision 1.3  1991/10/29  19:27:27  db
%Added documentation of the -poly_make option to INSTALL.
%
%Revision 1.2  1991/10/22  20:01:04  db
%Some minor formatting changes.
%
%Revision 1.1  91/03/27  16:41:11  16:41:11  db (Dave Berry)
%Initial revision
%

\chapter{Installing the Library.}	\label{install}

\section{Distribution.}

The library is distributed with the following sub-directories:

\begin{description}
  \item[\tt signatures.] The signatures that the user can see.
  \item[\tt portable.] The portable implementations of the functors
	and structures that make up the library.
  \item[\tt doc.] Copies of this documentation, in LaTeX format.
  \item[\tt nj-sml.dist.] System-specific implementations of some entries
	for Standard ML of New Jersey.
  \item[\tt poly.dist.] System-specific implementations of some entries
	for Poly/ML.
  \item[\tt poplog.dist.] System-specific implementations of some entries
	for Poplog ML.
\end{description} 

\section{Installation.}

\newcommand{\install}{{\small\tt INSTALL}}

The \install\ program does most of the work of installation.
To use it, change directory to the directory containing the library,
and call \install\ with a command line like the following:
\begin{verbatim}
INSTALL -poly -nj-sml -poplog -poly_make target
\end{verbatim}
All the arguments to \install\ are optional.

\install\ copies the library to the {\tt target} directory.
If no target directory is given,
\install\ treats the current directory as the target.

If the {\tt -nj-sml} flag is given, \install\ creates a new
sub-directory called {\tt nj-sml}, and copies the compiler-specific
implementations to that directory.  Then it adds links to the
portable implementations of the remaining library entries, so
that the new sub-directory contains a complete version of the
library.  The {\tt -poly} and {\tt -poplog} flags have the same
effect for the other compilers.

If the {\tt -poly\_make} option is given, \install\ creates a
new sub-directory called {\tt poly\_make}, and
links all the files in the library to that directory.  Where the
library contains versions specific to Poly/ML, it uses these.
If users create a symbolic link to this directory, they can use
the Poly/ML make facility ({\tt PolyML.make}) to load library
entries.

Finally, \install\ creates the SML build files.  You can then load
the library into an SML session.  There are three build files that
you can choose from; normally you will want to use {\tt build\_make.sml}
or {\tt build\_all.sml}.  In addition, it's usually a good idea to
save a core image or a persistent database containing the loaded
library.  How to do this depends on which compiler(s) you are using,
and is described below.


\section{Building The Library.}

The basic method of building the library is simple, and varies only
slightly with the choice of compiler.  First, change directory to the
subdirectory for the version that you want to build -- either
the portable version or the version specific to the compiler
that you're using.  Then run SML and {\tt use} the file {\tt poly.load},
{\tt nj-sml.load} or {\tt poplog.load}, as appropriate.  This sets
up a consistent set of load functions that is used by the rest of
the library.  Then {\tt use} the build file that you want.

In New Jersey ML, the {\tt use} function is available at top-level.
In Poplog ML, it is in the {\tt Compile} structure. In Poly/ML, it is
in the {\tt PolyML} structure.

If you are using Poly/ML and you want to store the compiled library
in a child database, you have to create the database in one Poly/ML
session, (using the function {\tt PolyML.make\_database}), and then run
Poly/ML again with this database to actually build the library.  When you
leave Poly/ML, the database will contain everything that you evaluate
in this session. It's usually a good idea to make the database read-only
once you've built the library.

If you are using Standard ML of New Jersey, you can save the library
in a core image using the command {\tt exportML}.  The saved image can
be executed directly.

If you are using Poplog ML, you can save the library in a core image
using either {\tt PML.System.make} or {\tt PML.System.save}.  The core image
can be run by giving the file name as an argument to the {\tt pml}
{\small UNIX} command, preceded by a \verb-+- character.  For example,
on our systems, the file {\tt /usr/local/bin/pml-library} contains
the line:

{\tt pml +/usr/local/lib/poplog/library.psv}.

Once you have created a saved-state that contains the library, you
should add a user command to your system to start the version of SML
that contains the library.  You may wish to make this version the
default.

If you are using the {\tt -poly\_make} option to \install, then users
need to make a symbolic link to the {\tt poly\_make} directory, and
load the {\tt poly\_make.load} file.  They can then use {\tt PolyML.make}
to load the desired library entries.


\section{Choice of Build File.}

There are three ways that you can load the
library, depending on which build file you use.  You can load all
the files at once with the file {\tt build\_all.sml}; load just the
make system with the file {\tt build\_make.sml}, and let {\tt Make} do
the rest of the work when a new entry is needed; or use the file
{\tt build\_core.sml}, which just loads a
few functions that can find the library entries in the installation
directory.

If you load all the entries, then users will be able to use them
immediately.  Also, you can move the source to another directory,
since the system will never have to load it directly.  However,
the saved state will be fairly large (the library is around 4
megabytes).

If you load just the make system, then the saved state will be
quite small.  Users will load just the entries that they need.
The make system will load all entries needed by those
that the user wants.  However, the user will have to wait for
a short period while the entries are compiled, and the source
must remain in the installation directory.

The third method uses virtually no space in the saved state.
However, in addition to the disadvantages of the second option,
it also requires the user to ensure that all entries are loaded
in the correct order.  This is not a good choice for general use.
I would only recommend it for the final build of a stand-alone
package, in which case the user should access the library
directly instead of using a more generally useful version.

 
