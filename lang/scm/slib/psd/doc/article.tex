%
% $Id: article.tex,v 1.1 1994/02/16 08:44:36 hsu Exp $
%
% $Log: article.tex,v $
% Revision 1.1  1994/02/16 08:44:36  hsu
% Initial revision
%
% Revision 1.2  1993/10/08  11:12:09  pk
% Added To appear in Lisp Pointers.
%
% Revision 1.1  1993/10/08  11:07:25  pk
% Initial revision
%
%

\documentstyle[epsf,12pt]{article}
\makeatletter
\long\def\@makefntext#1{\parindent 1em\noindent#1}
\makeatother
\begin{document}
\epsfverbosetrue
\bibliographystyle{plain}
\newcommand{\scheme}[1]{{\tt #1}}
\newcommand{\foot}[2]{#1\footnotetext{\hbox to 1em{#1\hss}#2}}
\author{Pertti Kellom\"aki, {\tt pk@cs.tut.fi}\\ 
Tampere University of Technology \\
Software Systems Lab \\
Finland}
\title{Psd -- a Portable Scheme Debugger\foot{*}{To appear in Lisp Pointers}}
\maketitle
\pagestyle{empty}
\thispagestyle{empty}

\begin{abstract}
Psd is a portable debugger for the Scheme language. Debugging with Psd
is accomplished by instrumenting the original source program. When the
instrumented program is run, it presents the user with an interactive
interface that lets him or her examine and change values of variables,
set breakpoints, and single step evaluation. Psd is designed to be run
within GNU Emacs, which is used for displaying the current source code
position.
\end{abstract}

\section{Introduction}

There are numerous implementations of the Scheme language available.
While some of them have extensive debugging capabilities, many small
implementations have only limited support for it. Psd provides source
level debugging as an ``add on'', relying only on features described
in the Revised${}^4$ Report on the Algorithmic Language
Scheme~\cite{r4rs}. Psd does not use the macro proposal of the report,
so it should work with Revised${}^3$ Scheme implementations, also.

Psd works by transforming the original program into an operationally
equivalent program (modulo the debugging capabilities), that allows the
user to examine and change variables, set breakpoints, and single step
the evaluation process. To a calling procedure, a debugged procedure
behaves exactly like the original. This allows mixing debugged and
non-debugged code.

When a program is debugged with Psd, its source code is first given to
the instrumenting part of Psd. The instrumentation part writes an
instrumented version of the program to a file, which is then loaded
into the Scheme environment. When a procedure in the program is
invoked, it behaves as if it was executed under a conventional
debugger.

\section{Related Work}

There are a few other debugger that are implemented similarily. The
edebug package for GNU Emacs Lisp~\cite{edebug}, written by Daniel
LaLiberte, uses the the same ideas, but is much more tightly integrated
with GNU Emacs. Jurgen Heymann has implemented an instrumenting
debugger for the Simscript II.5 simulation
language~\cite{heymann:simscript-debugger}.

\section{The Emacs Interface}

Psd uses GNU Emacs as its user interface. The primary use for Emacs is
to provide source code debugging. A typical Psd session is shown in
figure~\ref{fig:session}.
\begin{figure}
\begin{center}
\leavevmode
\epsfbox{psd.dump.ps}
\end{center}
\caption{A Psd Session}
\label{fig:session}
\end{figure}

Psd uses the same interface to Emacs as the GNU project debugger
Gdb, and the Psd interface was actually modified from the
existing Gdb interface. When an instrumented program is run, it emits
specially formatted lines containing the source file and line number
of the current source line. Emacs interprets these lines by showing
the appropriate file in an editing window with an arrow indicating the
current line.

There is a small amount of Emacs Lisp code that interacts with the
Scheme environment. Emacs generates temporary file names and issues
instrumenting and loading commands. The instrumenting code is file
oriented, but with the Emacs interface it is possible to pick one
procedure from a source file to be debugged. The Emacs interface is
also used for setting breakpoints, with the Emacs Lisp code taking care
of the low level details like file names and line numbers.

\section{Accessing Variables by Name}

One of the main uses of a debugger is examining the values of
variables. In some Lisp environments it is easy to provide access to
variables by starting a new read-eval-print loop. The Scheme report
does not include \scheme{eval}, however, so a different strategy must
be used.  In Psd this problem is solved by inserting an access procedure
each time new variable bindings are made. This procedure performs the
mapping between symbols and actual program variables.
Figure~\ref{fig:access-vars} shows a \scheme{let} form and the code
that Psd generates for it.
\begin{figure}
\begin{verbatim}
(let ((x 1))
  (+ x 1))

(let ((x 1))
  (let ((psd-val 
         (lambda (sym)
           (case sym
             ((x) x)
             (else (psd-val sym))))))
  (psd-debug psd-val (lambda () (+ x 1)))))
\end{verbatim}
  
  \caption{Accessing variables by name}
  \label{fig:access-vars}
\end{figure}

\begin{sloppypar}
The procedure \scheme{psd-val} is passed to the debugger command loop
\scheme{psd-debug}. Using it the command loop gets access to variables
in the current lexical environment of the debugged program. The scope
rules come ``for free'', because the name
\scheme{psd-val} in the body of \scheme{psd-val} refers to the
lexical environment surrounding the \scheme{let} form.
\end{sloppypar}

Assignments to local variables are made using the same mechanism. A
setter procedure \scheme{psd-set!} is inserted each time local
variables are defined, and it is also passed to \scheme{psd-debug}.

Access to global variables is provided using the same idea. Every
instrumented Scheme file includes definitions for procedures similar
to \scheme{psd-val} and \scheme{psd-set!}. When the file is loaded,
the procedures are added to a global access procedure list. The global
definitions of \scheme{psd-val} and \scheme{psd-set!} call the access
procedures one by one until either the access succeeds or there are no
more access procedures.

The Psd runtime support includes access to all the essential
procedures\foot{$^*$}{The report distinguishes
between essential procedures that a conforming implementation must
provide, and non-essential procedures that are not required.}
  described in the Revised${}^4$ Report. The debugger command
loop includes a simple evaluator that can evaluate calls of the
procedures that are visible to it. The lack of a \scheme{bound?}
predicate or some other portable way of finding out whether an
identifier is bound prevents access to the non-essential names in the
report.

\section{Breakpoints and Single Stepping}
\label{sec:breakpoints}

In order to be able to single step the evaluation process, the
debugger must be able to gain control both before and after each
expression is evaluated. In Psd this is accomplished by packaging each
expression inside a procedure. This procedure is then passed to the
debugger command loop. When the user wants to continue, the command
loop simply calls the procedure that was passed to it. The command
loop then gains control again, and finally returns the value that the
procedure returned. For example, the expression
\scheme{(+ x 1)} in figure~\ref{fig:access-vars} is transformed to
\begin{verbatim}
  (psd-debug (lambda () (+ x 1)))
\end{verbatim}

The transformation is done recursively, so the expression
is really transformed into
\begin{verbatim}
  (psd-debug (lambda () ((psd-debug (lambda () +)
                         (psd-debug (lambda () x)
                         (psd-debug (lambda () 1)))))))
\end{verbatim}
In reality the debugger gets some more information (the current
location in source code etc.), but the basic idea is the same.

It may seem that there is no point in instrumenting expressions like
\scheme{+} and \scheme{1}, but the instrumentation is needed for
supporting breakpoints. Breakpoints are implemented by maintaining a
list of source code locations of breakpoints. Each time
\scheme{psd-debug} is called, it checks if there is a breakpoint for
the current source line, and starts a command loop if needed. If
primitive expressions like \scheme{x} would not be instrumented, there
would be source lines for which breakpoints could not be set, for
example in
\begin{verbatim}
  (foo bar
       baz
       zap)
\end{verbatim}

Single stepping is implemented similarily. Stepping by line is
implemented by keeping track of the line number corresponding to the
previous call to \scheme{psd-debug}.


\section{Runtime Support}

Psd needs some runtime support in the Scheme environment. The command
loop \scheme{psd-debug} is a closure containing state variables for
the debugger. Procedure application needs the procedure
\scheme{psd-apply}, and breakpoint support needs a global variable for
storing the breakpoint locations. 

The instrumentation code resides in the same Scheme environment as the
debugged program. This is not strictly necessary, but it has proved to
be a convenient way of working. For example, it is easy to cut down the size of
the instrumented files by assigning a
unique integer for each source file name and using it instead of the
full path name. 

\section{Catching Runtime Errors}

A typical use of a debugger is to let the program run until a runtime
error occurs and examine the program state to find out what went
wrong. With Psd, real runtime errors can not be allowed to happen,
since it relies on correct execution of the instrumented code.
Instead, if an expression would cause a runtime error to occur, the
command loop is called and an error message is issued.

Aside from syntactically incorrect expressions and causes outside the
scope of the language (exhaustion of memory, receiving a signal etc.),
the only place where a runtime error can occur is the procedure call.
When calling a user defined procedure, the only possible error is that
a wrong number of arguments is supplied. Although it would be possible
to detect at least some of these errors, Psd does not currently check
the number of arguments to a user procedure.

The number of primitive procedures is fixed, so they are easier to
handle. Psd transforms each procedure call \scheme{(proc args)} into
\scheme{(psd-apply proc args)}. Before \scheme{psd-apply} applies the
procedure to its arguments, it checks if the procedure is a primitive
procedure. If it is, \scheme{psd-apply} checks that the number of
arguments is correct and that the arguments are of correct type. If a
runtime error would occur, \scheme{psd-apply} calls the debugger
command loop. Runtime errors that occur in non-debugged code can not
be caught this way.

There are still some cases in the current implementation where a
runtime error can occur. For example, for the \scheme{assoc}
procedure, the second argument should be a list of lists. Currently,
it is only checked that it is a list.


\section{Tail Recursion and Continuations}

In Scheme, iteration is expressed as tail recursion. It is important
that the debugger maintains this property whenever possible, because
otherwise a debugged program might easily run out of memory.  During
single stepping Psd does not preserve tail recursiveness (because of
the way single stepping is implemented), but in other situations it is
preserved.

Tail recursiveness could be fully preserved by using breakpoints to
implement single stepping. This would add some complexity, though, and
since it would take quite a time to run out of memory by single
stepping a program by hand, it has not been judged worth the effort.

First class continuations are not a problem, since they are handled by
the underlying Scheme environment.

\section{Caveats and Limitations}

Psd shares the problem common to all debuggers: running the debugged
program is not exactly the same as running the same program without
the debugger. Psd tries to be true to the underlying environment, but
there is at least one aspect that would require access to the
underlying implementation: evaluation order.

In order to catch runtime errors, Psd transforms each procedure call
into a call to the procedure \scheme{psd-apply}, with the
subexpressions of the original combination as arguments. Because of
the way the Scheme language is defined, it cannot be guaranteed that the
evaluation order of the subexpressions is the same in the debugged
program as in the original program. There is not much that can be
guaranteed about the order of evaluation anyway, which makes this a
non-issue for well written programs. In practice, however, debuggers
are used for finding bugs in ill behaved programs, so it should be
addressed somehow. If an implementation always uses a left to right or
right to left evaluation, Psd preserves the evaluation order.

Another limitation is caused by the lack of a standard method for
accessing top level variables. Psd provides access to all variables
defined in the files that are being debugged, but all other top level
variables are inaccessible. A partial solution would be to detect all
nonlocal variables that are referenced in the debugged expressions.
This may be implemented in future versions of Psd.

Conventional debuggers can provide some help even when an error
happens in a part of a program that has not been compiled for debugging.
With Psd this is not possible, because debugging with Psd relies on
the correct execution of programs.

A yet unsolved problem is providing the user with backtrace
information. Access to local variables in the current lexical context
is easy to provide using closures, but access to nonlocal variales at
the calling procedure is more difficult. It would be possible to
collect backtrace by passing the backtrace as an extra parameter with
every procedure call. This is not a very good solution, because it
would not allow mixing debugged and undebugged code. Another solution
would be to collect the same backtrace by inserting assignments to a
global variable at each procedure entry and exit. This approach breaks
when \scheme{call-with-current-continuation} is used, because a
procedure invocation can be exited an arbitrary number of times.

An inherent problem with instrumenting the original source code is
that the resulting instrumented files are quite large. The extreme
case is the one line procedure
\begin{verbatim}
  (define (foo x) (+ x 1))
\end{verbatim}
that is expanded from 25 bytes to 963 bytes, giving an expansion
factor of 39. A more typical case is the instrumentation code of Psd
that was expanded from 18058 bytes to 259309 bytes, giving a factor of
14. The time taken to instrument the instrumentation code was little
over a minute on a Sun Sparcstation SLC using Aubrey Jaffer's scm
interpreter.

The instrumented code is so much slower than the original that it is
by no means practical to instrument all the procedures of a large
application. A binary tree implementation was instrumented, and the
slowdown caused by instrumentation was in the range of 170--290.
Usually the problem can be pinpointed to a few procedures with a fair
accuracy without a debugger, though, and the debugger can be applied
only to them. In practice the slowness has not been a serious problem.

\section{Availability of Psd}

Psd is available from the author using email, or from {\tt cs.tut.fi}
as the file {\tt /pub/src/languages/schemes/psd-1.1.tar.Z} using
anonymous ftp. Psd is placed under the GNU General Public
License,\foot{$^*$}{The General Public License is included in the Psd
distribution, or it can be obtained from the Free Software Foundation,
675 Mass Ave, Cambridge, MA 02139, USA} so it can be freely used and
distributed.

\bibliography{references}

\end{document}

