% doc.tex

\documentstyle[]{article}

\def\tty#1{{\tt #1}}
\def\bold#1{{\bf #1}}

\parindent 0in
\parskip 0.10in 
\textwidth 6.5in
\oddsidemargin 0in
\evensidemargin 0in
\textheight 8.175in
\topmargin -0.25in

\begin{document}

\begin{titlepage}
\begin{center}
\vspace*{2in}
\bf
\Huge
Xtrek Version 6.1

\vspace{2in}

\LARGE
Mike Bolotski $<$mikeb@salmon.ee.ubc.ca$>$ \\
Jon Bennett $<$jcrb@cs.cmu.edu$>$ \\
David Gagne $<$daveg@salmon.ee.ubc.ca$>$ \\
Daniel Lovinger $<$dl2n+@andrew.cmu.edu$>$ \\
Rob Ryan $<$rr2b+@andrew.cmu.edu$>$
\end{center}
\end{titlepage}



\tableofcontents \newpage

\section{Introduction}

Xtrek is one of the all time classic games for X Windows. This release
of the game at version 6.0 represents many  months of
work by various people at Carnegie Mellon University and the University
of British Columbia, and in our not so humble opinion constitutes a major
improvement over the previous versions. For those familiar with
versions 4.0 and 5.4, much of the game remains on the surface the same : you
still fly a ship 'round the galaxy blowing away your friends and
conquering planets. However, where the old xtrek was cast in stone (or
executable), this version is configurable in almost every respect from
the power of your torpedoes to the number of planets in the galaxy. A
whole new universe is only a few hours of configuration file hacking
away...

NOTE: At release 6.1, our involvement with the code ends. If you are interested
in becoming the next "keeper of the flame", please contact us so we can arrange
a formal handoff of the code.

The basic object of the game is to conquer sixteen (by default - as
with almost everything, this is configurable) planets for your empire.
Of course, everyone else is trying to achieve the same objective, and
that is where all of the fun comes in.

\section{Getting Started}

Xtrek is split up into two separate programs, xtrek and xtrekd. Xtrekd
is the main server process for the game and accepts as a command line
argument the name of a configuration file for the game. For instance

\tty{xtrekd cmu.config}

will start a game with the contents of the cmu.config file. The daemon
first tries to open the config file as an absolute pathname, and then
attempts to load the config file from the xtrek library directory.  If
no config file is specified, a default is read from the library. 

\subsection{Entering the Game}

Once xtrekd is running, players can connect to it by running xtrek with
the machine name the daemon is running on as the argument on the
command line. For instance, assume that a player has a running xtrekd
on ligonier.andrew.cmu.edu.  Then

\tty{xtrek ligonier.andrew.cmu.edu}

would send a request to the daemon on ligonier to open an xtrek window on the local
machine. Various errors can occur at this point, most of which have to
do with access protections on your X server - you must first xhost to
the machine running the daemon.

\subsection{Screen Layout}

When you start the game, you are given a window with five subwindows.  The big
left window is the local view from your ship updated in real time, and 
your ship will always appear the center of it.
Other ships will be displayed with their set ship bitmap, and will additionally
,on color workstations, also have a
different color.  The small hex (0-f) character to the right of the ship
represents the player number.  Round objects with names (which can be turned off) 
on them are the planets, and they also are the color of the owning team.  Small dots on
the screen are torpedoes or mines (torpedoes are in straight line motion).  Enemy torpedoes 
and mines are a small cross while friendly torpedoes and mines are a small dot.  
Exploding devices are represented by increasingly large explosion bitmaps proportional to
the kick behind it.  Phaser shots appear as lines emminating from the firing ship.

The large window on the right is a map of the whole galaxy.
Only ships and planets appear on it, and it is updated in real time.
Ships are just a player number (in hex) and team letter; however, cloaked
ships will not appear.

Directly below the galaxy window is a small status message window which will
occasionally contain warning messages like "Not enough fuel to fire
torp." 

Directly Below the warning window is the message window.  This window
will be described later.

The small window on the left side (bottom) is your status window.  
The status area contains your vessel's 'vital signs.'
If you are locked onto another ship, in addition to your own you
will get the status of the other vessel on the second line, minus
the \bold{Flags} field. There are eleven separate sections to the
full status line :

\begin{list}{}{
	\renewcommand{\makelabel}[1]{{\tt #1 \hfill}}
         \setlength{\leftmargin}{2.75cm}
         \setlength{\labelwidth}{\leftmargin}
         \setlength{\labelsep}{0in}
}
\item[Flags]
	This field contains information about ship activities.  There
	are twelve status characters that can be activated in this
	section (in order, left to right):

	\begin{tabbing}
        GYR \= \kill
	 \tty{S}\> ship has shields up \\
	 \tty{GYR }\> green, yellow, or red alert \\
	 \tty{L}\> ship has sensor or planet lock activated \\
	 \tty{R}\> ship is under repair \\
	 \tty{B}\> ship is bombing a planet \\
	 \tty{O}\> ship is orbiting a planet \\
	 \tty{C}\> ship is cloaked \\
	 \tty{W}\> ship weapons are over operating temperature \\
	 \tty{E}\> ship engines are over operating temperature \\
	 \tty{u}\> ship is beaming armies up from planet surface \\
	 \tty{d}\> ship is beaming armies down to planet surface \\
	 \tty{P}\> ship is allowing co-pilots \\
	\end{tabbing}

\item[Warp] Shows the ship's current warp speed in x.x format
             Ships move between warps 0 and 9.9.

\item[Damage] Shows the ship's current hull damage.
	Damage repairs slowly if shields
	are down, faster if ship is in repair mode.

\item[Shield] Shows the number of damage points your shield
	generators can absorb.They repair faster than hull damage.

\item[Torpedoes] Shows the number of torpedoes the ship has flying

\item[Kills] Shows the number of kills the ship has prosecuted in x.xx format

\item[Armies] Shows the number of armies currently onboard
Your ship can carry two times your current number of kills.
The maximum number of armies it can carry is ten.

\item[Fuel] Shows the current amount of fuel onboard

\item[Weapon Temp] Weapon temperature as a percentage of maximum

\item[Engine Temp] Engine temperature as a percentage of maximum

\item[Failed Systems]
	This field contains info about which of the ship's systems are currently
	non-operational due to excessive damage.  If a letter is shown, 
        the system is down.

	\begin{tabbing}
        GYR \= \kill
	 \tty{C}\> cloaking device \\
	 \tty{L}\> long range sensors \\
	 \tty{P}\> phaser banks \\
	 \tty{S}\> short range sensors  \\
	 \tty{T}\> torpedoes \\
	 \tty{c}\> cooling system \\
	 \tty{l}\> sensors (locking ability) \\
	 \tty{s}\> shields \\
	 \tty{t}\> transporters \\
	\end{tabbing}
\end{list}


\subsection{Help Window}

Typing 'h' will pop up a help window below the normal window.  The help
window lists commands you can type.  Its useful for small problems
like forgetting how to put your shields up, and for authors who forget
what command does what.  Typing 'h' again will get rid of it.

\section{Ship Details}

\subsection{Weapons}

Torpedoes move at a configurable warp (which should be higher than the max ship speed)
 and have a random life of three to five
seconds.  Torpedoes will detonate and do full damage if
they get close to a ship, and do less damage to objects at a
longer distances.  Torpedoes will not be detonated by a non-hostile
player, but they will do damage if something causes them to explode.

Mines are simply dropped by a player, and remain active until a hostile ship comes
within range (about one ship length).  They will then detonate and inflict damage
as per torpedoes to any ship in the local area.  Since their onboard power is finite,
they will detonate automatically after about 7-10 minutes of life, again doing
damage to any ship in the local area.

When exploding, torpedoes can damage {\em everyone} who is near them.
Thus it is possible to kill your own teammates and people you are at
peace with.

Phasers must be within about 10 degrees of their target to hit (this 'range' is doubled
if you have a sensor lock on the ship you are attacking).
Phasers that hit will maintain a line between the two ships.  The damage
inflicted is again inversely proportional to the distance between the
ships (linear for the sake of making the things work with reasonable
damage values).
It takes one second to recharge your phasers for
another shot.  As with torpedoes, Your phasers will affect teams 
you are at war with, or those who are at war with you.

Ships that explode will inflict damage as per a torpedo, albeit more powerful.  
Don't get too near one.

\subsection{Cooling}

Firing weapons and running your engines generates heat which must be
dissipated through your ship's interlinked cooling system. As your
systems heat up toward their maximum of 100% loaded, they try to dump
their excess heat off into the opposite cooling system. This can be
extremely useful when you are desperately trying to escape an enemy
trap, and need just that extra boost to get you into home space. Each
weapon shot 'costs' given amount of heat which is based on the power of
the weapon. Warp drive on the other hand costs an amount that increases
as you increase your warp factor. With all of the above interactions
taken into account, it is usually possible for a normal ship to pull
about warp four and fire weapons before starting to accumulate heat.

In addition to these standard sources of heating, if you are playing a
configuration with the teleport option, teleporting costs a large
amount.

The cooling systems can only operate at peak efficiency with ship's
shields down since the entire purpose of the shield system is to repel
energy. With shields up the cooling system will still operate, but at a
lower dissipation rate. Similarly, when the ship is cloaked the cooling
systems operate at a scaled back rate.

If the ships cooling system exceed 100\% capacity, there is a growing
probability during each time slice that the system being cooled will
shut off to protect itself from a meltdown. The capacity of the
cooling system must approach 0\% for the system to be reactivated. When
a system overheats, a flag is triggered on the status line, W for weapons,
E for engines.

\subsection{Planets and Armies}

Planets are conquered by first bombing them to reduce the number of
armies to a manageable level and then beaming down your own armies.  A
ship must be in orbit to bomb a planet.  A side effect of this
requirement is that shields are dropped and planetary defensive fire
damages the hull directly.  Planetary fire causes damage proportional
to the number of defending armies.  The actual formula is
$(\mbox{armies}/ 10) + 2$ damage points twice per second.

Planets stats are updated every minute.  There is a random chance that the
number of armies will increase.  There is also a random chance, that
there will be a major die-off of armies (more likely on planets with many armies).
If you have less than three armies
on the planet, they will not grow as fast.  Planets with no armies are
owned by the independent team and will have no armies on them until
someone beams one down.

Enemy planets do damage based on the number of armies they have.
This means planets with no armies do no damage.  

The teams' planets can provide
fuel and repair services, which roughly double each operation's speed.  
Obviously, this makes these planets
particularly valuable (and hence a lot of battles are fought around them).
Fuel replenishes at an additional 2x rate when orbiting a planet which is 
marked as {\em fuel} sources. The same applies for planets marked as {\em repair}
stations.

You can get information about any planet your team has orbited.
However, if the planet is taken by any other team, you will
lose information about the planet until it is reorbited.


\section{Xdefaults and options}

You can put the following options into your .Xdefaults file:

\begin{verbatim}
     xtrek.boldfont:         6x10b
     xtrek.font:             6x10
     xtrek.name:             DragonSlayer
     xtrek.reverseVideo:     off
     xtrek.showShields:      on
     xtrek.showStats:        on
     xtrek.stats.geometry:   +0+655
     xtrek.GAlertPattern:    0xf
     xtrek.YAlertPattern:    0xa 0x5
     xtrek.RAlertPattern:    0x0f 0x0f 0x0f 0x0f 0xf0 0xf0 0xf0 0xf0
     xtrek.ralert:           red
     xtrek.yalert:           yellow
     xtrek.galert:           green
     xtrek.border:           blue
     xtrek.background:       black
     xtrek.text:             white
     xtrek.warning:          red
     xtrek.unknown:          light grey
     xtrek.me:               white
\end{verbatim}

\begin{list}{}{
	\renewcommand{\makelabel}[1]{{\tt #1 \hfill}}
         \setlength{\leftmargin}{2.5cm}
         \setlength{\labelwidth}{\leftmargin}
         \setlength{\labelsep}{0in}
}

\item[Fonts] These options allow you to change fonts used in the game.
              Our advice: "Don't."

\item[name] This is your playername.

\item[reversevideo] On black and white monitors this can ease eye-strain.

\item[showShields] This causes the shields of you and other players to appear
                     as a circle around your ship when they are up.

\item[showStats] This option will put up a visual status window above 
          your display.  Some players find this easier to read than numbers.
          The stats.geometry uses standard X syntax for a screen position.

\item[Alert Patterns] These patterns are useful for showing your current 
          alert status based
on the border pattern.  It is used for black and white monitors only.
The alert colors are used for color monitors.  I think changing them
would be silly.

\item[Colors] These options allow you to set colors on various objects
so you can get much more information from your display.
Obviously, these only matter for color monitors.
\end{list}


\section{The Commands}
These are the various key and button commands in the game.
They are not currently remappable other than through the XKeyMap
function.

\begin{list}{}{
	\renewcommand{\makelabel}[1]{{\tt #1 \hfill}}
         \setlength{\leftmargin}{.5in}
         \setlength{\labelwidth}{\leftmargin}
         \setlength{\labelsep}{0in}
}
\item[0-9]{\em  Set ship's speed}.
The numbers zero through nine set your ship's speed.  It takes time for
your ship to accelerate or decelerate to the desired speed.  Damaged
ships can't use higher speeds.  When engines freeze, you can't set
speed.  Setting speed breaks you out of a planet's orbit.

\item[.<digit>]{\em Set fractional warp speed}.
This will set a fractional warp speed goal for your ship. For instance, if
you are traveling warp 7.5, hitting .1 would deaccelerate you to warp 7.1.

\item[k]{\em Set course}.
The letter $k$ or the right mouse button will set your course towards
the current mouse position.  Turning towards the desired course can
take time, depending on your current speed and damage.

\item[j]{\em Teleport}.
This command will randomly teleport you to a position within about a 
range specified in the configuration file from your current position. 
This should eat heat and fuel like candy (set
by config file). This option is only enabled if xtrekd has been compiled
with \verb|-DTELEPORT_OPTION|.

\item[T]{\em Turbo mode}
This will set your speed to a turbo speed for a given number of game ticks 
as defined by your configuration file. Heat and fuel gain/loss rates are
based on this new speed. This option is only enabled if xtrekd has been 
compiled with \verb|-DTURBO_OPTION|.

\item[p]{\em Fire phaser}.
The letter $p$ or the middle mouse button will fire your phasers
towards the current mouse position if you have enough fuel.  You may
only fire phasers once per second.

\item[t]{\em Launch torpedo}.
The letter $t$ or the left mouse button will launch a torpedo towards
the current mouse position if you have enough fuel and less than thirty
torpedoes currently out. This max usually isn't significant since there
is also a reload time built into the firing operation.

\item[m]{\em Drop Mine}.
This command will cause your ship to lay a mine at the current location.
This mine will have a lifetime for between 8 and 10 minutes, and will only
detonate when an enemy ship approaches, doing damage to any ship in range.
If time runs out, the mine will simply explode.

\item[d]{\em Detonate other torpedoes}.
This command will detonate other enemy torpedoes near you, doing less
than maximum damage to you. This costs fuel based on the power of
the enemy torpedoes.

\item[e]{\em Detonate your own torpedoes}.
This will turn your own torpedoes off so that you can fire new ones.

\item[D]{\em Detonate your mines}.
This will explode all of your own mines; however, this capability can be 
turned off in the configuration file.

\item[R]{\em Repair damage}.
Hitting this key will decelerate your ship to a stop and will begin
repairs when the ship is stopped.  Your shields are down when you are
repairing and you cannot fire weapons.  In a nutshell, you are a
sitting duck.  Putting your shields up or moving is the best way to get
out of repair mode.  If your ship is orbiting a friendly repair planet,
damage will repair at a faster rate.

\item[c]{\em Cloak your ship}.
This command toggles your cloaking device, and allows you to remain 
invisible on the windows of every player.  This would be nice except that you cannot
fire weapons while cloaked, or have shields up while the device is in operation.
To enforce this, shields will automatically be dropped on entering cloak mode, and
(since most times you uncloak you will be in battle) raised on exiting.
In addition, there is a small chance (which is based on range), that an enemy ship
may see you flicker as their sensors penetrate your device's field.

\item[u]{\em Toggle shields}.
This key will toggle your shields up and down.

\item[+/-]{\em Put shields up/down}.
This key will put your shields up or down.

\item[l]{\em Lock onto an object}.
This will lock your ship onto the object nearest the mouse cursor. If the object
is a planet, your course will change and you will be placed into orbit around
the planet when you arrive. If the object is a ship, you will start receiving
the ships 'vital statistics' as detailed in the {\bold Screen Layout} section.
This lock will also roughly double you phase's hit angle, and again roughly
double your chance of getting a sensor penetration of their cloaking device
if they activate it while you have the lock.

To break the lock, just set course normally.  You cannot lock
onto cloaked ships.

\item[f]{\em Follow locked ship}.
If you are currently locked onto a ship via $l$, this command will set your
ship's autopilot to follow that ship. Your course will automatically be updated.
The follow may be broken at any time by simply specifying a new course.

\item[b]{\em Bomb a planet}.
If an enemy planet has more than ten armies, you can bomb them.  With
four or less armies, you must beam down armies to defeat them.  Your
shields will go down to bomb, and you will take damage from the
planet.  You get kills for the armies destroyed.

\item[z]{\em Beam up armies}.
If the planet you are orbiting is owned by your team and has more than
four armies, you can beam them to your ship to carry to other planets.
You must be orbiting to beam up armies.  The number of armies you can
carry is equal to your kills * 2.  You can never carry more than ten
armies.

\item[x]{\em Beam down armies}.
This command will beam the armies that are on your ship down to a
planet you are orbiting.  If it is an hostile planet, you will kill his
armies.  If all enemy armies are killed and you land an army, you take
the planet for your team.  If it is a planet you own, you will simply
add to the armies already there.  If it is a friendly planet, owned by
other teams, you cannot beam armies down.

\item[C]{\em Throw a coup}.
Sometimes players will discover that all their planets have been
taken, or have been left with no armies.
In order to allow the team some form of return you can hold a coup
on your home planet assuming that the following criterion are met:
\begin{itemize}
\item Your team must have armies on no planets.
\item You must have more than one kill.
\item There must be less than five enemy armies (or zero friendly, should
you still own it) on your home planet.
\item You must be orbiting your home planet.
\end{itemize}
After the coup, the planet will have four of your armies on it.
As they come back you can take over more of your occupied worlds.
When your last planet is taken, you will have to wait from thirty
minutes to an hour before you can have a coup.

\item[i]{\em Info on object}.
This will pop up a window near the mouse cursor which contains
information about the nearest object (planet or ship).
To remove this window, just type 'i' again.

\item[o]{\em Orbit a planet}.
If you are near a planet and going warp two or less this will put in
orbit around it.  You must be in orbit around a planet to bomb it, beam
armies up or down to it, repair at it, or get fuel from it.  Enemy
planets will damage you if you get near them.  Setting a speed will
break out of orbit.

\item[Q]{\em Quit xtrek}.
This command will start a self-destruct sequence, and return you to the
entering screen, at which time you may click on the quit box to actually
terminate the game for your display. The time it takes to self destruct may
depend on the config file you are playing under (see \bold{Scenario Customization}).

\item[?]{\em repeat all of the previous messages}.
Message sending is detailed below.  This command allows you to
review the current set of them.

\item[@]{\em Allow/disallow copilots}.
This toggles a feature allows more than one player to fly a ship 
(on separate displays).

\item[L]{\em Get player list}.
This gives you a quick list of other players in a subwindow on
the short range screen.
To remove this window, just type 'L' again.

\item[P]{\em Get planet list}.
This gives you a list of planets in a subwindow on the short range
screen.
You get information only on planets which your team
owns or has previously orbited.  To remove this window,
just type 'P' again.

\item[S]{\em List score information}.
This gives you a list of interesting statistics on each player and
team currently in the game in a subwindow on the galaxy display window. 
It may be removed by hitting 'S' again.

\item[s]{\em Toggle visual status window}.
This will turn a visually oriented status window on and off.

\item[U]{\em Toggle showing of shields}.
This will toggle whether or not shields are shown around ships in the
short range view. By default, this is on.

\item[M]{\em Toggle galaxy map updating}.
This toggles whether or not xtrek is updating the galaxy map each tick.

\item[h]{\em Toggle the help window}.
Typing $h$ will pop up a help window below the normal window.  The help
window lists commands you can type.  Its useful for small problems
like forgetting how to put your shields up, and for authors who forget
what command does what.  Typing 'h' again will get rid of it.

\item[w]{\em Set war status}.
This command will give you a window that you can use to declare
war and peace with other teams.  It will be fully detailed below.
Hitting $w$ again will remove it.

\item[n]{\em Toggle bell}.
This toggles whether or not xtrek will try annoy you with beeps when
interesting things happen to your ship.

\item[N]{\em Toggle naming of planets}.
This toggles whether or not xtrek will put planet names under the circle
on the short range view. Depending on whether or not you are running
under X11R4, X11R3 with or without \verb|-DX11_R3_ENV|, this could speed
up updates considerably.

\item[G]{\em Turn on hostile mode}.
This will make you hostile to all ships in the game, and is separate from
the war mode. It is primarily useful in target practice with robots.

\item[g]{\em Turn off hostile mode}.
This undoes the effects of $G$.

\item [H]{\em Hose mode}.
This will start a robot for each empire defined in the config file, and insure
that at least one robot is always on the game for each. The robots will start
up in patrol mode, going from planet to planet in their empire.
\end{list}

\section{Nitty Gritty Details}

This section contains the details on certain features.

\subsection{Messages}

Messages are sent both from the daemon and from other players.
They appear in the message window at the bottom, right side of the display.
To send messages, just put the mouse into the message window.
You must type a character in that represents the recipient of your
message.  This can be 'A' for everyone,  the first letter of a team name
 to send to all members
of a given team, or [0-9abcdef] will send a message to an individual
player.  When you've typed this in, it will map to a proper address
and wait for you to type in some text.  When you type in a return,
your message is sent.
Typing an escape will abort the message in progress.

This is obviously a non-optimal addressing mode (for instance, a team 'Acturians'
screws things up), but it is the best we can think of, and also the original.

Now obviously, the next question is, "What if someone starts shooting
at me while I'm typing in my message?  Am I hosed?"
Simply move the mouse out of the window to defend yourself.  Messages
are only dealt with if the mouse is in the window.  You can stop
in the middle of the input.


\subsection{War and Peace}

You can declare war and peace with other teams.
The greatest use of this is that you can use planets belonging
to teams you are at peace with for fuel and repair.
It also allows two teams to gang up on another without risking
killing each other.

There are three states a player can be in versus any other team:
Peace, Hostile, and War.  Being at War is irrevocable.  You cannot
change to any other state until you reenter the game.  You get to
war when you are hostile to a team and you either bomb one of their
planets, damage one of their players with weapons, or beam armies
down to one of their planets.

If you are hostile towards a team, your weapons will hurt all members of
that team (whether or not they are at peace with you).  As soon as
you hurt them, you will be at war.

If you are peaceful towards a team, your weapons will only hurt members
of that team who are not peaceful towards you.  In other words, two
players who are at peace towards each other cannot fight each other.

Players default to being at peace with their own team and hostile
towards all the others.  Obviously, they can declare war on their own
team.

To change your settings, type 'w' and a war window will pop up.
Click the mouse in the boxes of the teams you want to change.
Click in "re-program" to save the results.  Teams will be notified
of your changes.  Finally, there is a ten second delay to declare
hostility towards another team so don't get too close while you make
the changes.

The easiest way to know if someone near you is hostile is to use
the alert status described below.  Info on a player will tell you
their status towards you.

\section{Scenario Customization}

All parameters are integers unless specified otherwise, and all 
heat costs and cooling rates are in hundredths of a degree. It should
be noted that it is far easier to modify an existing config than to
rewrite one from scratch.

\subsection{Ship Parameters}
\begin{list}{}{
	\renewcommand{\makelabel}[1]{{\tt #1 \hfill}}
         \setlength{\leftmargin}{2.5cm}
         \setlength{\labelwidth}{\leftmargin}
         \setlength{\labelsep}{0in}
         \setlength{\itemsep}{.2em}
}
\item[turns]   Ship maneuverability (turn radius).
\item[torp damage]  Torpedo damage
\item[torp range]  Torpedo blast radius.
\item[mine damage] Mine damage
\item[mine range]  Mine blast radius.
\item[phaser damage] Phaser damage.
\item[phaser range] Phaser range.
\item[phaser pulses]  Number of ticks that a phaser will lock on a target.
\item[torp speed]  Torpedo speed
\item[max speed] Maximum ship speed.
\item[shield repair]  Shield repair rate.
\item[max fuel]  Maximum fuel capacity.
\item[torp cost] Torpedo fuel cost.  
\item[mine cost] Mine fuel cost 
\item[phaser cost] Phaser fuel cost
\item[detonate cost] Detonate other torps fuel cost.
\item[warp cost] Warp engine fuel cost per warp factor.
\item[cloak cost] Cloak fuel cost.
\item[recharge]  Fuel recharge rate.
\item[accint]  Acceleration rate.
\item[decint]  Deceleration rate.
\item[max armies]  Maximum army capacity
\item[weapon cool]  Weapon cooling rate.
\item[engine cool]  Engine cooling rate.
\item[max damage] Maximum structural damage sustainable before destruction.
\item[shield max]  Maximum shield damage. 
\item[teleport heat]  Teleport engine heat cost.
\item[teleport cost]  Teleport fuel cost.
\item[teleport range]  Teleport radius.
\item[turbo speed]  Turbo warp speed 
\item[turbo time]  Turbo warp duration 
\item[reload] time between torp salvos
\item[burst] number of shots in a torp salvo
\item[phaser fail]  Probability of phaser failure.  Similarly for 
{\tt torp fail}, {\tt trans fail}, {\tt shield fail}, 
 {\tt cloak fail}, {\tt lrsensor fail}, {\tt srsensor fail}, {\tt lock fail},
{\tt cooling fail} and {\tt warp fail}.
\end{list}

\subsection{Robot Parameters}
\begin{list}{}{ \renewcommand{\makelabel}[1]{{\tt #1 \hfill}} 
\setlength{\leftmargin}{2.5cm} \setlength{\labelwidth}{\leftmargin}
\setlength{\labelsep}{0in}
         \setlength{\itemsep}{.2em}
}
\item[hscruise] speed robots go when going to assist another robot 
\item[cruise] speed robots patrol at
\item[battle] speed robots go in battle
\item[flee] speed robots run away at
\item[cloaked] speed robots go when cloaked
\item[refresh] speed robots go when they want to cool down
\item[engage] distance at which a robot starts firing at a target
\item[disengage] distance at which a robot stops firing at a target
\item[shotdamage] damage a salvo does
\item[circledist] distance at which a robot would like to stand off it's target
\item[sneaky] percent chance that a robot will prefer cloaked approaches to battle
\end{list}

\subsection{Empire Parameters}
\begin{list}{}{
	\renewcommand{\makelabel}[1]{{\tt #1 \hfill}}
         \setlength{\leftmargin}{2.5cm}
         \setlength{\labelwidth}{\leftmargin}
         \setlength{\labelsep}{0in}
         \setlength{\itemsep}{.2em}
}

\item[icon] String parameter specifies bitmap file name.   The bitmap is
            automatically rotated through all 16 orientations.  It 
            should be specified with front pointing upwards.   Some aliasing 
            will result.  

\item[robot name]  String parameter specifies robot name.  
\end{list}

\subsection{Planet Parameters}
\begin{list}{}{
	\renewcommand{\makelabel}[1]{{\tt #1 \hfill}}
         \setlength{\leftmargin}{2.5cm}
         \setlength{\labelwidth}{\leftmargin}
         \setlength{\labelsep}{0in}
}
\item[home]  Boolean flag.  Specifies that planet is the home planet of the
empire.
\item[fuel] Boolean flag.  Specifies that planet is a fuel source; fuel is
replenished at twice normal planetary rates when planet is orbited. 
\item[repair] Boolean flag.  Specifies that planet is a repair yard; damage is
repaired at twice normal planetary rates when planet is orbited. 
\item[(x,  y)]  Specifies planet coordinates. Both {\tt x} and {\tt y} are
integers.
\item[armies] Specifies number of armies on planet at start of game.
Note that a large number of  armies makes the planet an effective obstacle;
any ship passing through such a planet will be destroyed.
\end{list}

\subsection{Global Parameters}
\begin{list}{}{
	\renewcommand{\makelabel}[1]{{\tt #1 \hfill}}
         \setlength{\leftmargin}{1.5in} \setlength{\labelwidth}{\leftmargin}
         \setlength{\labelsep}{0in}
         \setlength{\itemsep}{.2em}
}
\item[death time] Player is dead for this number of seconds.  Defaults to 4. 

\item[torp life min]   Torpedoes live for at least this number of seconds.  
Defaults to 7.  Similarly for {\tt mine life min} (default is 60).

\item[torp life var]  Torpedoes may live for at least this number of seconds 
in addition to the minimum lifetime.  Defaults to 2. Similarly for 
{\tt mine life var} (default is 120).

\item[player explode time] Player explodes for this number of seconds.

\item[weapon lock min] Overheated weapons are unusable for at least
this number of seconds  up   to an additional {\tt weapon lock var}
seconds.  Defaults are 10 and 15. Similarly for {\tt engine lock min}
and {\tt engine lock var}.  Defaults are  identical to those of
weapons.

\item[self destruct time]  Self-destruct countdown time.

\item[robot giveup time] Robots will give up within this number of
seconds if no combat is initiated.

\item[orbit speed]   Maximum warp allowed while entering orbit. 

\item[detonate distance]  Effective range of the counter-torpedo batteries. 

\item[orbit distance]  Range at which a ship can enter orbit. Defaults to 900.

\item[planet fire distance]  Range at which a planet can damage a ship.
Defaults to 1500.

\item[phaser hit angle]  Number of degrees that a phaser will hit.

\item[phaser hit langle]  Number of degrees that a locked phaser will hit.

\item[auto quit]  Auto-logout countdown time.

\item[cool penalty]   Floating point number.  Specifies the increase
in weapon and engine cooling rate when both shields and cloak are inactive.
Defaults to 1.25.

\item[fast destruct]  Boolean flag.  If set, self-destruct time does not
increase with alert level.   Default is to double self-destruct time for
yellow alert, and double again for red alert.

\item[mine detonate]  Boolean flag.   If set, mines detonate when
destroyed.  This feature is potentially very nasty.  It also allows instant
self-destruction, cheating players out of kills.

\item[enable/disable turbo] Boolean flag.  If the game is compiled with
      \verb|-DTURBO_OPTION|, the flag can be used to enable/disable turbo
      mode with the configuration file.  Defaults to disabled.

\item[enable/disable teleport] Boolean flag.  If the game is compiled with
      \verb|-DTELEPORT_OPTION|, the flag can be used to enable/disable teleport
      mode with the configuration file.  Defaults to enabled.

\item[enable/disable destruct] Boolean flag.  Controls whether self-destruct
      is allowed.  Defaults to true. 

\item[enable/disable mine] Boolean flag.  Controls whether mines are allowed,
      for more compatibility with the old xtrek.  Defaults to enabled.

\item[enable/disable cloak] Boolean flag.  Controls whether cloaking is 
     allowed. Defaults to enabled.


\end{list}

\subsection{Sample config file}

\begin{verbatim}

/* The original 4 empire scenario */ 

empire Federation 	F robot name = "M5"	;
empire Romulan 		R robot name = "Colossus";
empire Klingon    	K robot name = "Guardian";
empire Orion 		O robot name = "HAL 9000";

global
	cool penalty     = 1.25
	death time = 4
	phaser fire time = 1
	torp life min = 7
	torp life var = 2
	mine life min = 60
	mine life var = 120
	player explode time = 1
	weapon lock min = 10
	weapon lock var = 15
	engine lock min = 10
	engine lock var = 15
	self destruct time = 5
	robot giveup time = 60
	orbit speed = 20
	detonate distance = 1500
	orbit distance = 900
	planet fire distance = 900
	phaser hit-angle = 5
	phaser hit-langle = 10
	auto-quit = 180
	// fast destruct  
	// mine detonate 
        mine wobble = 128
     ;


/*
 * Define ship parameters.
 */

default ship

               detonate cost 	= 100
               phaser cost 	= 200
               torp cost 	= 100
               teleport cost 	= 2000
               warp cost 	= 7
               cloak cost 	= 30
               phaser heat 	= 500	/* 5 degrees */
               torp heat 	= 200	/* 2 degrees */
               mine heat 	= 600	/* 6 degrees */
               detonate heat 	= 300	/* 3 degrees */
               teleport heat 	= 5000	/* 50 degrees */
               warp heat 	= 32
               weapon 	cool   = 25
               engine   cool = 220	/* warp 7 if shields down */
               srsensor fail = 1000	/* SRS never fail, always see doom */
               turbo speed   = 300
               turbo time    = 10
               teleport range= 6000
               recharge      = 45
               turns         = 27000
               accint        = 250
               decint        = 250
               phaser damage = 30
               phaser pulses = 2
               phaser range  = 6000
               torp damage   = 20
               torp speed    = 160
               shield repair = 21
               max damage    = 200
               shield max    = 200
               reload        = 3
               burst         = 2
               max speed     = 90
               max armies    = 10
               max fuel      = 10000
               cruise        = 50
               battle        = 50
               flee          = 60
               cloaked       = 50
               refresh       = 40
     ;

ship Federation

teleport range		= 0
teleport cost		= 0
teleport heat		= 0
turbo speed		= 0
turbo time		= 0
recharge		= 45 
turns			= 6000
accint			= 250
decint			= 180
phaser damage		= 40
phaser cost		= 400 
phaser pulses		= 2
phaser range		= 6000
mine cost		= 768
mine damage		= 48
mine range		= 1200
torp damage 		= 24
torp cost		= 400
torp speed		= 160
torp range		= 600
detonate cost		= 75 
shield repair		= 21
warp cost		= 6
cloak cost		= 60
max damage		= 200
shield max		= 200
reload			= 4
burst			= 1
weapon cool 		= 25   
engine cool 		= 220 
max speed		= 90
max armies		= 10
max fuel		= 10000
cruise			= 50
battle			= 50
flee			= 60
cloaked			= 50
refresh			= 40
engage			= 10000
disengage		= 20000
sneaky			= 10
cloak min		= 4000
cloak max		= 12000
shotdamage		= 0
circledist		= 7000
;

ship Romulan

teleport range		= 0
teleport cost		= 0
teleport heat		= 0
turbo speed		= 0
turbo time		= 0
recharge		= 30
turns			= 5500
accint			= 230
decint			= 150
phaser damage		= 20
phaser cost		= 200 
phaser pulses		= 3
phaser range		= 6000
mine cost		= 1024
mine damage		= 64
mine range		= 1600
torp damage		= 32
torp cost		= 512
torp speed		= 145
torp range		= 650
detonate cost		= 125 
shield repair		= 18 
warp cost		= 4
cloak cost		= 20
max damage		= 180
shield max		= 200
reload			= 4
burst			= 0
weapon cool		= 40 
engine cool		= 100 
max speed		= 90
max armies		= 8
max fuel		= 10000
cruise			= 50
battle			= 40
flee			= 90
cloaked			= 30
refresh			= 45
engage			= 10000
disengage		= 20000
sneaky			= 75
cloak min		= 4000
cloak max		= 18000
shotdamage		= 0
circledist		= 5000
;

ship Klingon

teleport range		= 0
teleport cost		= 0
teleport heat		= 0
turbo speed		= 0
turbo time		= 0
recharge		= 35 
turns			= 7500
accint			= 280
decint			= 190
phaser damage		= 50
phaser cost		= 500 
phaser pulses		= 2
phaser range		= 8000
mine cost		= 512
mine damage		= 32
mine range		= 1000
torp damage		= 16
torp cost		= 256 
torp speed		= 170
torp range		= 625
detonate cost		= 100 
shield repair		=  20 
warp cost		= 3
cloak cost		= 30
max damage		= 160
shield max		= 160
reload			= 2
burst			= 1
weapon cool		= 32 
engine cool		= 90 
max speed		= 80
max armies		= 8
max fuel		= 10000
cruise			= 40
battle	        	= 50
flee			= 80
cloaked			= 55
refresh			= 40
engage			= 10000
disengage		= 20000
sneaky			= 30
cloak min		= 4000
cloak max		= 13000
shotdamage		= 0
circledist		= 7000
;

ship Orion 

teleport range		= 0
teleport cost		= 0
teleport heat		= 0
turbo speed		= 0
turbo time		= 0
recharge		= 40
turns			= 9100
accint			= 330
decint			= 230
phaser damage		= 10
phaser cost		= 100 
phaser pulses		= 6
phaser range		= 5000
mine cost		= 512
mine damage		= 32
mine range		= 1000
torp damage		= 20
torp cost		= 275
torp speed		= 190
torp range		= 550
detonate cost		= 100 
shield repair		= 19 
warp cost		= 2
cloak cost		= 40
max damage		= 160
shield max		= 200
reload			= 1
burst			= 3
weapon cool		= 25 
engine cool		= 70 
max speed		= 110
max armies		= 6
max fuel		= 8000
cruise			= 80
battle			= 60
flee			= 90
cloaked			= 60
refresh			= 45
engage			= 10000
disengage		= 20000
sneaky			= 50
cloak min		= 4000
cloak max		= 15000
shotdamage		= 0
circledist		= 9000
;

/* Define all the planets. */

default planet armies=30;

planet "Earth"      Federation  (20000, 80000) home fuel repair;
planet "Sasus"      Federation  (10000, 60000);
planet "Candeleron" Federation  (25000, 60000);
planet "Beta III"   Federation  (44000, 81000);
planet "Janus"      Federation  (33000, 55000);
planet "Deneb VI"   Federation  (30000, 90000);
planet "Ceti IV"    Federation  (45000, 66000);
planet "Altar"      Federation  (11000, 75000);
planet "Dekar"      Federation  (8000, 93000);
planet "Daltus"     Federation  (32000, 74000);
planet "Romulus"    Romulan  (20000, 20000) home fuel repair;
planet "Ethen"      Romulan  (45000 , 7000);
planet "Amur"       Romulan  (4000, 12000);
planet "Remus"      Romulan  (42000, 44000);
planet "Bal"        Romulan  (13000, 45000);
planet "Tahndar"    Romulan  (28000 , 8000);
planet "Dact"       Romulan  (28000, 23000);
planet "Sirius II"  Romulan  (40000, 25000);
planet "Rakhir"     Romulan  (25000, 44000);
planet "Rus"        Romulan  (8000, 29000);
planet "Klin"       Klingon  (80000, 20000) home fuel repair;
planet "Malatrakir" Klingon  (70000, 40000);
planet "Amakron"    Klingon  (60000, 10000);
planet "Laltir"     Klingon  (54000, 40000);
planet "Khartair"   Klingon  (93000 , 8000);
planet "Monpur III" Klingon  (90000, 37000);
planet "Sectus"     Klingon  (69000, 31000);
planet "Makus"      Klingon  (83000, 48000);
planet "Gendus"     Klingon  (73000 , 5000);
planet "Jakar"      Klingon  (54000, 21000);
planet "Orion"      Orion  (80000, 80000) home fuel repair;
planet "Amterion"   Orion  (92000, 59000);
planet "Lumecis"    Orion  (65000, 55000);
planet "Bitar V"    Orion  (52000, 60000);
planet "Prastan"    Orion  (72000, 69000);
planet "Sorlen"     Orion  (64000, 80000);
planet "Zetus"      Orion  (56000, 89000);
planet "Jih"        Orion  (91000, 94000);
planet "Markus II"  Orion  (70000, 93000);
planet "Oren"       Orion  (85000, 70000);
\end{verbatim}

\section{Default Scenario}

This is a very informal description of the scenario described
in the {\bold default.config} shipped with xtrek.

Klingon : heavy phasers, low torps. normal engines, but they
can't slow down very fast, so turning is slower than normal. low fuel
gain-back, but they make up for that by toasting anything that gets
near with the phasers. best cloaking costwise ...

Romulan : nickname 'garbage scow', pretty accurate. slowest
turn rate in the game. heavy torps that will rail an opponent that
gets caught in a stream. best shields, and in terms of how much energy
they chew, best cloaking since they also have the best energy regen
rate. hull strength also highest.

Orion : tend to spit torps, which are generally weak but very
cheap and fast to fire. best acel/decel in the game, and turn on a
dime. hull pts and shield are the lowest, so they need those engines
of theirs often. regen is also low, so they need accuracy on those
torp shots. phasers are also weak, and don't have a very good range.
mostly useful for annoyance shots and close in fighting.

Federation : this is basically the standard xtrek ship of old
(we had to keep one of them around :-). All around average ship. High
cloak cost, but also have a decent regen rate. Best repair rate
(Scotty factor), and their weapons are both pretty potent. Can get
out-paced if they don't watch out, though - especially by Orions.


\section{Credits}

\begin{itemize}
\item Original By Chris Guthrie (chris@ic.berkeley.edu) and Ed James (edjames@ic.berkeley.edu)
\item Later X11R3 Mods by Dan Dickey (ddickey@lotus.cray.com)  
\item Overhaul and killer robots by Jon Bennett (jcrb@cs.cmu.edu)  
\item Cleanup/Speedup/Fixes by James Anderson (no email)
\item Wizardry by John Myers (jgm@cs.cmu.edu)  
\item Config and more by Mike Bolotski and Dave Gagne (mikeb@ee.ubc.ca and daveg@ee.ubc.ca)
\item More cleanup by Joe Keane (jgk@osc.osc.com)
\item Floating point fixes, integer trig, robots fixes, and more cleanup by Rob Ryan (rr2b+@andrew.cmu.edu)
\item New damage code by Mario Goertzel (mg2p+@andrew.cmu.edu)
\end{itemize}

\section{Bug Reports}

Bug reports should be sent to \tty{dl2n+xtrek@andrew.cmu.edu}.

\end{document}
