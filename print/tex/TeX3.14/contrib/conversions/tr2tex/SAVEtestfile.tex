% -*-LaTeX-*-
% Converted automatically from troff to LaTeX by tr2tex on Wed Feb 11 14:10:36 1987
% tr2tex was written by Kamal Al-Yahya at Stanford University
% (Kamal%Hanauma@SU-SCORE.ARPA)


\documentstyle[troffms,11pt]{article}
\begin{document}
%
% input file: testfile
%
\par\noindent
\def\dC{\delta C}
\def\xx{x^x}
\def\B{{\bf B}}
\def\ov{\over  }

\footer{\rm\thepage}
\lefthead{Al-Yahya}
\righthead{troff to TeX translator}
\title{Testing the troff-to-tex translator}
\author{Kamal Al-Yahya}
\authoraff{Stanford University}
\begin{abstract}
This file demonstrates the use of  %
\bf  tr2tex %
\rm  which translates
documents written in troff to a LaTeX style.
Examples are given to show what the translator can do.
\end{abstract}
\par
First let's test some equation written for the
{\bf eqn}
preprocessor:
$$
2 \left( 1 \ +\  \sqrt{\omega_{i+1} + \zeta -{x+1  \over \Theta +1} y + 1} \right)
\ \ \ =\ \ \  1
\eqno (1)$$
$$
\left[
\matrix {
   \matrix { e_1 \cr . \cr . \cr e_i \cr . \cr . \cr e_N }
}
\right]_{n+1} \ \ =\ \  y + 1
$$
$$
{\bf X} + {\rm a} \ \ge\ 
\under{\hat a} \kern0.20em \sum_i^N \lim_{{x} \to k} \dC
$$
$$
\left( \ \it\hbox{speed} \times \ \it\hbox{time} = \ \it\hbox{distance travelled} \right)
$$
$$
\tilde \beta i \ge \zeta \dC
$$
\begin{itemize}
\item[{}]
In-line math like $\beta +1$ which is surrounded by math delimiters, as
defined by
{\it delim}
is also translated.
\rm
\end{itemize}
Only simple tables are translated. Translation of more complicated tables
is painful and I won't do it now. Here is an example:
\par
\begin{tabular}{lrcc}
name	 &  type		 &  color	 &  value\\
$\alpha$	 &  real		 &  red	 &  2.3\\
$x$	 &  imaginary	 &  green	 &  -1.2\\
$a + 2$	 &  real		 &  white	 &  0.0
\end{tabular}
\par
% this is a commented text
\par
Now we start a figure.
\begin{figure}
\par\vspace{3.0in}
\caption{This is the caption of the figure.
}\end{figure}
\par
Try some floating objects.
\nwl
{\nobreak
This text should be kept in one page. i.e. a page break is discouraged here.
}
\nwl
Now a floating text.
\begin{figure}
This text should be kept in one page even if we had to move it around,
since it is a
{\it floating}
object. This is a boxed
\fbox{word.}
\end{figure}
\nwl
These characters are special in TeX, so they need to be escaped
in the translation \% \& \# \_, while these characters have to be
printed in math mode $<$ $>$ $|$. 
\footnote{
This is a footnote
}
\par
\ACK
Thanks to all of those who contributed by reporting bugs and suggesting
some improvements. Special thanks go to Nelson Beebe for all of his valuable
contributions and suggestions. He is responsible for making the translator
portable to other computers. He also made significant improvements to
the translation of manuals.
\REF
\reference{Knuth, D.E., 1984, The TeXBook, Addison-Wesley Publishing Company.}
\reference{Lamprt, L., 1986, LaTeX: user's guide \& reference manual, Addison-Wesley Publishing Company.}
\reference{Lesk, M.E., 1978, Typing documents on the UNIX system: using the -ms macros with troff and nroff, UNIX programmer's manual, v. 2B, sec. 3.}
\reference{McGilton, H. and Morgan, R., 1983, Introducing the UNIX system, McGraw-Hill Book Company.}
\end{document}
