% WEB user manual -- last updated by D E Knuth on 4 Dec 89
\input webmac
\parskip 0pt plus 1pt
\def\RA{\char'31 } % right arrow
\def\hang{\hangindent 4em\ignorespaces}
\font\ninerm=cmr9
\font\ninett=cmtt9
\font\eighttt=cmtt8
\let\mc=\ninerm % medium caps for names like SAIL
\def\PASCAL{Pascal}
\font\quoterm=cmssq8
\font\quoteit=cmssqi8
\def\pb{\.{|...|}}
\def\lpile{\def\cr{\hfill\endline}\matrix} % I only use \lpile by itself

\outer\def\section #1.{\penalty-50\vskip 12pt plus 3pt minus 3pt
  \noindent{\bf #1.}\quad\ignorespaces}

\def\lheader{\mainfont\the\pageno\hfill\sc\runninghead\hfill}
\def\rheader{\hfill\sc\runninghead\hfill\mainfont\the\pageno}
\def\runninghead{{\tentt WEB} USER MANUAL}

% This verbatim mode assumes that no ? appears in the text being copied.
\def\verbatim{\begingroup
  \def\do##1{\catcode`##1=12 } \dospecials
  \parskip 0pt \parindent 0pt
  \catcode`\ =13 \catcode`\^^M=13
  \tt \catcode`\?=0 \verbatimdefs \verbatimgobble}
{\catcode`\^^M=13{\catcode`\ =13\gdef\verbatimdefs{\def^^M{\ \par}\let =\ }} %
  \gdef\verbatimgobble#1^^M{}}
\centerline{\titlefont The {\ttitlefont WEB} System
  of Structured Documentation}

\vskip 15pt plus 3pt minus 3pt
\noindent This memo describes how to write programs in the
\.{WEB} language; and it also includes the full \.{WEB} documentation for
\.{WEAVE} and \.{TANGLE}, the programs that read \.{WEB} input and produce
\TeX\ and \PASCAL\ output, respectively. The philosophy behind \.{WEB} is
that an experienced system programmer, who wants to provide the best
possible documentation of software products, needs two things
simultaneously:  a language like \TeX\ for formatting, and a language like
\PASCAL\ for programming. Neither type of language can provide the best
documentation by itself. But when both are appropriately combined, we
obtain a system that is much more useful than either language separately.

The structure of a software program may be thought of as a ``web'' that is
made up of many interconnected pieces. To document such a program, we want
to explain each individual part of the web and how it relates to its
neighbors. The typographic tools provided by \TeX\ give us an opportunity
to explain the local structure of each part by making that structure
visible, and the programming tools provided by \PASCAL\ make it possible
for us to specify the algorithms formally and unambiguously. By combining
the two, we can develop a style of programming that maximizes our ability
to perceive the structure of a complex piece of software, and at the same
time the documented programs can be mechanically translated into a working
software system that matches the documentation.

Since \.{WEB} is an experimental system developed for internal use within
the \TeX\ project at Stanford, this report is rather terse, and it assumes
that the reader is an experienced programmer who is highly motivated to
read a detailed description of \.{WEB}'s rules. Furthermore, even if a
less terse manual were to be written, the reader would have to be warned
in advance that \.{WEB} is not for beginners and it never will be: The
user of \.{WEB} must be familiar with both \TeX\ and \PASCAL. When one
writes a \.{WEB} description of a software system, it is possible to make
mistakes by breaking the rules of \.{WEB} and/or the rules of \TeX\ and/or
the rules of \PASCAL. In practice, all three types of errors will occur,
and you will get different error messages from the different language
processors. In compensation for the sophisticated expertise needed to cope
with such a variety of languages, however, experience has shown that
reliable software can be created quite rapidly by working entirely in
\.{WEB} from the beginning; and the documentation of such programs seems
to be better than the documentation obtained by any other known method.
Thus, \.{WEB} users need to be highly qualified, but they can get some
satisfaction and perhaps even a special feeling of accomplishment when
they have successfully created a software system with this method.

To use \.{WEB}, you prepare a file called \.{COB.WEB} (say), and then you
apply a system program called \.{WEAVE} to this file, obtaining an output
file called \.{COB.TEX}.  When \TeX\ processes \.{COB.TEX}, your output
will be a ``pretty printed'' version of \.{COB.WEB} that takes appropriate
care of typographic details like page layout and the use of indentation,
italics, boldface, etc.; this output will contain extensive cross-index
information that is gathered automatically. You can also submit the same
file \.{COB.WEB} to another system program called \.{TANGLE}, which will
produce a file \.{COB.PAS} that contains the \PASCAL\ code of your \.{COB}
program. The \PASCAL\ compiler will convert \.{COB.PAS} into
machine-language instructions corresponding to the algorithms that were so
nicely formatted by \.{WEAVE} and \TeX. Finally, you can (and should)
delete the files \.{COB.TEX} and \.{COB.PAS}, because \.{COB.WEB} contains
the definitive source code. Examples of the behavior of \.{WEAVE} and
\.{TANGLE} are appended to this manual.

Besides providing a documentation tool, \.{WEB} enhances the \PASCAL\
language by providing a rudimentary macro capability together with the
ability to permute pieces of the program text, so that a large system can
be understood entirely in terms of small modules and their local
interrelationships.  The \.{TANGLE} program is so named because it takes a
given web and moves the modules from their web structure into the order
required by \PASCAL; the advantage of programming in \.{WEB} is that the
algorithms can be expressed in ``untangled'' form, with each module
explained separately.  The \.{WEAVE} program is so named because it takes
a given web and intertwines the \TeX\ and \PASCAL\ portions contained in
each module, then it knits the whole fabric into a structured document.
(Get it? Wow.)  Perhaps there is some deep connection here with the fact
that the German word for ``weave'' is ``{\it web\/}'', and the
corresponding Latin imperative is ``{\it texe\/}''!

It is impossible to list all of the related work that has influenced the
design of \.{WEB}, but the key contributions should be mentioned
here.\quad (1)~Myrtle Kellington, as executive editor for ACM
publications, developed excellent typographic standards for the
typesetting of Algol programs during the 1960s, based on the original
designs of Peter Naur; the subtlety and quality of this influential work
can be appreciated only by people who have seen what happens when other
printers try to typeset Algol without the advice of ACM's copy
editors.\quad(2)~Bill McKeeman introduced a program intended to automate
some of this task [Algorithm 268, ``Algol~60 reference language editor,''
{\sl CACM \bf8} (1965), 667--668]; and a considerable flowering of such
programs has occurred in recent years [see especially Derek Oppen,
``Prettyprinting,'' {\sl ACM TOPLAS \bf2} (1980), 465--483; G.~A. Rose and
J. Welsh, ``Formatted programming languages,'' {\sl SOFTWARE Practice
\char`\&\ Exper.\ \bf11} (1981), 651--669].\quad(3)~The top-down style of
exposition encouraged by \.{WEB} was of course chiefly influenced by Edsger
Dijkstra's essays on structured programming in the late 1960s. The less
well known work of Pierre-Arnoul de Marneffe [``Holon programming: A
survey,'' Univ.\ de Liege, Service Informatique, Liege, Belgium, 1973; 135
pp.\null] also had a significant influence on the author as \.{WEB} was
being formulated.\quad(4)~Edwin Towster has proposed a similar style of
documentation in which the programmer is supposed to specify the relevant
data structure environment in the name of each submodule [``A convention
for explicit declaration of environments and top-down refinement of
data,'' {\sl IEEE Trans.\ on Software Eng.\ \bf SE--5} (1979), 374--386];
this requirement seems to make the documentation a bit too verbose,
although experience with \.{WEB} has shown that any unusual control
structure or data structure should definitely be incorporated into the
module names on psychological grounds.\quad(5)~Discussions with Luis
Trabb~Pardo in the spring of 1979 were extremely helpful for setting up a
prototype version of \.{WEB} that was called \.{DOC}.\quad (6)~Ignacio
Zabala's extensive experience with \.{DOC}, in which he created a full
implementation of \TeX\ in \PASCAL\ that was successfully transported to
many different computers, was of immense value while \.{WEB} was taking
its present form.\quad(7)~David~R. Fuchs made several crucial suggestions
about how to make \.{WEB} more portable; he and Arthur~L. Samuel
coordinated the initial installations of \.{WEB} on dozens of computer
systems, making changes to the code so that it would be acceptable to
a wide variety of \PASCAL\ compilers.\quad(8)~The name \.{WEB} itself
was chosen in honor of my wife's mother, Wilda Ernestine Bates.

The appendices to this report contain complete \.{WEB} programs for the
\.{WEAVE} and \.{TANGLE} processors. A study of these examples, together
with an attempt to write \.{WEB} programs by yourself, is the best way
to understand why \.{WEB} has come to be like it is.
\section General rules.
A \.{WEB} file is a long string of text that has been divided into
individual lines. The exact line boundaries are not terribly crucial, and
a programmer can pretty much chop up the \.{WEB} file in whatever way seems
to look best as the file is being edited; but string constants and control
texts must end on the same line on which they begin, since this convention
helps to keep errors from propagating.  The end of a line means
the same thing as a blank space.

Two kinds of material go into \.{WEB} files: \TeX\ text and \PASCAL\ text.
A programmer writing in \.{WEB} should be thinking both of the
documentation and of the \PASCAL\ program that he or she is creating;
i.e., the programmer should be instinctively aware of the different
actions that \.{WEAVE} and \.{TANGLE} will perform on the \.{WEB} file.
\TeX\ text is essentially copied without change by \.{WEAVE}, and it is
entirely deleted by \.{TANGLE}, since the \TeX\ text is ``pure
documentation.'' \PASCAL\ text, on the other hand, is formatted by
\.{WEAVE} and it is shuffled around by \.{TANGLE}, according to rules that
will become clear later. For now the important point to keep in mind is
that there are two kinds of text. Writing \.{WEB} programs is something
like writing \TeX\ documents, but with an additional ``\PASCAL\ mode''
that is added to \TeX's horizontal mode, vertical mode, and math mode.

A \.{WEB} file is built up from units called {\sl modules\/} that are more
or less self-contained.  Each module has three parts:

\yskip\item{1)} A \TeX\ part, containing explanatory material about what
is going on in the module.

\item{2)} A definition part, containing macro definitions that serve as
abbreviations for \PASCAL\ constructions that would be less comprehensible
if written out in full each time.

\item{3)} A \PASCAL\ part, containing a piece of the program that
\.{TANGLE} will produce. This \PASCAL\ code should ideally be about a
dozen lines long, so that it is easily comprehensible as a unit and so
that its structure is readily perceived.

\yskip\noindent The three parts of each module must appear in this order;
i.e., the \TeX\ commentary must come first, then the definitions, and
finally the \PASCAL\ code. Any of the parts may be empty.

\eject % page break inserted Dec 88

A module begins with the pair of symbols `\.{@\ }' or `\.{@*}', where
`\.{\ }' denotes a blank space. A module ends
at the beginning of the next module (i.e., at the next
`\.{@\ }' or `\.{@*}'), or at the end of the file, whichever comes first.
The \.{WEB} file may also contain material that is not part of any module
at all, namely the text (if any) that occurs before the first module.
Such text is said to be ``in limbo''; it is ignored by \.{TANGLE}
and copied essentially verbatim by \.{WEAVE}, so its function is to
provide any additional formatting instructions that may be desired in the
\TeX\ output. Indeed, it is customary to begin a \.{WEB} file with
\TeX\ code in limbo that loads special fonts, defines special macros,
changes the page sizes, and/or produces a title page.

Modules are numbered consecutively, starting with 1; these numbers appear
at the beginning of each module of the \TeX\ documentation, and they appear
as bracketed comments at the beginning of the code generated by that
module in the \PASCAL\ program.

Fortunately, you never mention these numbers yourself when you are writing
in \.{WEB}. You just say `\.{@\ }' or `\.{@*}' at the beginning of each
new module, and the numbers are supplied automatically by \.{WEAVE} and
\.{TANGLE}. As far as you are concerned, a module has a
{\sl name\/} instead of a number; such a name is specified by writing
`\.{@<}' followed by \TeX\ text followed by `\.{@>}'. When \.{WEAVE}
outputs a module name, it replaces the `\.{@<}' and `\.{@>}' by
angle brackets and inserts the module number in small type. Thus, when you
read the output of \.{WEAVE} it is easy to locate any module that is
referred to in another module.

For expository purposes, a module name should be a good description of the
contents of that module; i.e., it should stand for the abstraction
represented by the module. Then the module can be ``plugged into'' one or
more other modules in such a way
that unimportant details of its inner workings
are suppressed.  A module name therefore ought to be long enough to convey
the necessary meaning. Unfortunately, however, it is laborious to type
such long names over and over again, and it is also difficult to specify a
long name twice in exactly the same way so that \.{WEAVE} and \.{TANGLE}
will be able to match the names to the modules. To ameliorate this difficulty,
\.{WEAVE} and \.{TANGLE} let you abbreviate a module name
after its first appearance in the \.{WEB} file; you can type simply
`\.{@<$\alpha$...@>}', where $\alpha$ is any string that is a prefix of
exactly one module name appearing in the file. For example, `\.{@<Clear
the arrays@>}' can be abbreviated to `\.{@<Clear...@>}' if no other module
name begins with the five letters `\.{Clear}'. Module names must otherwise
match character for character, except that consecutive blank spaces and/or
tab marks are treated as equivalent to single spaces, and such spaces are
deleted at the beginning and end of the name. Thus, `\.{@< Clear { }the
arrays @>}' will also match the name in the previous example.

We have said that a module begins with `\.{@\ }' or `\.{@*}', but we
didn't say how it gets divided up into a \TeX\ part, a definition part,
and a \PASCAL\ part. The definition part begins with the first appearance
of `\.{@d}' or `\.{@f}' in the module, and the \PASCAL\ part begins with
the first appearance of `\.{@p}' or `\.{@<}'. The latter option `\.{@<}'
stands for the beginning of a module name, which is the name of the module
itself. An equals sign (\.=) must follow the `\.{@>}' at the end of this
module name; you are saying, in effect, that the module name stands for
the \PASCAL\ text that follows, so you say `$\langle\,$module
name$\,\rangle=\null$\PASCAL\ text'. Alternatively, if the \PASCAL\ part
begins with `\.{@p}' instead of a module name, the current module is said
to be {\sl unnamed}. Note that module names cannot appear in the
definition part of a module, because the first `\.{@<}' in a module
signals the beginning of its \PASCAL\ part.  But any number of module names
might appear in the \PASCAL\ part, once it has started.

The general idea of \.{TANGLE} is to make a \PASCAL\ program out of these
modules in the following way: First all the \PASCAL\ parts of unnamed
modules are copied down, in order; this constitutes the initial
approximation $T_0$ to the text of the program. (There should be at least
one unnamed module, otherwise there will be no program.) Then all module
names that appear in the initial text $T_0$ are replaced by the \PASCAL\
parts of the corresponding modules, and this substitution process
continues until no module names remain. Then all defined macros are
replaced by their equivalents, according to certain rules that are
explained later. The resulting \PASCAL\ code is ``sanitized'' so that it
will be acceptable to an average garden-variety \PASCAL\ compiler; i.e.,
lowercase letters are converted to uppercase, long identifiers are
chopped, and the lines of the output file are constrained to be at most 72
characters long.  All comments will have been removed from this \PASCAL\
program except for the meta-comments delimited by `\.{@\{}' and
`\.{@\}}', as explained below, and except for the module-number comments
that point to the source location where each piece of the program text
originated in the \.{WEB} file.

If the same name has been given to more than one module, the \PASCAL\ text
for that name is obtained by putting together all of the \PASCAL\ parts in
the corresponding modules. This feature is useful, for example, in a
module named `Global variables in the outer block', since one can then
declare global variables in whatever modules those variables are
introduced. When several modules have the same name, \.{WEAVE} assigns the
first module number as the number corresponding to that name, and it
inserts a note at the bottom of that module telling the reader to `See
also sections so-and-so'; this footnote gives the numbers of all the other
modules having the same name as the present one. The \PASCAL\ text
corresponding to a module is usually formatted by \.{WEAVE} so that the
output has an equivalence sign in place of the equals sign in the \.{WEB}
file; i.e., the output says `$\langle\,$module
name$\,\rangle\equiv\null$\PASCAL\ text'. However, in the case of the second
and subsequent appearances of a module with the same name, this `$\equiv$'
sign is replaced by `$\mathrel+\equiv$', as an indication that the \PASCAL\
text that follows is being appended to the \PASCAL\ text of another
module.

The general idea of \.{WEAVE} is to make a \.{TEX} file from the \.{WEB}
file in the following way: The first line of the \.{TEX} file will be
`\.{\\input webmac}'; this will cause \TeX\ to read in the macros that
define \.{WEB}'s documentation conventions. The next lines of the file
will be copied from whatever \TeX\ text is in limbo before the first
module.  Then comes the output for each module in turn, possibly
interspersed with end-of-page marks.  Finally, \.{WEAVE} will generate a
cross-reference index that lists each module number in which each \PASCAL\
identifier appears, and it will also generate an alphabetized list
of the module names, as well as a table of contents that
shows the page and module numbers for each ``starred'' module.

What is a ``starred'' module, you ask? A module that begins with `\.{@*}'
instead of `\.{@\ }' is slightly special in that it denotes a new major
group of modules. The `\.{@*}' should be followed by the title of this
group, followed by a period. Such modules will always start on a new page
in the \TeX\ output, and the group title will appear as a running headline
on all subsequent pages until the next starred module. The title will also
appear in the table of contents, and in boldface type at the beginning of
its module. Caution:  Do not use \TeX\ control sequences in such titles,
unless you know that the \.{webmac} macros will do the right thing with
them. The reason is that these titles are converted to uppercase when
they appear as running heads, and they are converted to boldface when they
appear at the beginning of their modules, and they are also written out to
a table-of-contents file used for temporary storage while \TeX\ is
working; whatever control sequences you use must be meaningful in all
three of these modes.

The \TeX\ output produced by \.{WEAVE} for each module consists of
the following: First comes the module number (e.g., `\.{\\M123.}'
at the beginning of module 123, except that `\.{\\N}' appears in place of
`\.{\\M}' at the beginning of a starred module). Then comes the
\TeX\ part of the module, copied almost verbatim except as noted
below. Then comes the definition part and the \PASCAL\ part, formatted
so that there will be a little extra space between them if both are
nonempty. The definition and \PASCAL\ parts are obtained by inserting
a bunch of funny looking \TeX\ macros into the \PASCAL\ program; these
macros handle typographic details about fonts and proper math spacing,
as well as line breaks and indentation.

When you are typing \TeX\ text, you will probably want to make frequent
reference to variables and other quantities in your \PASCAL\ code, and you
will want those variables to have the same typographic treatment
when they appear in your text as when they appear in your
program.  Therefore the \.{WEB} language allows you to get the effect of
\PASCAL\ editing within \TeX\ text, if you place `\.|' marks before and
after the \PASCAL\ material. For example, suppose you want to say something
like this:
$$\hbox{The characters are placed into \\{buffer}, which is a
\&{packed} \&{array} $[1\to\|n]$ \&{of} \\{char}.}$$
The \TeX\ text would look like this in your \.{WEB} file:
$$\.{The characters are placed into |buffer|, which is a |packed
array [1..n] of char|.}$$
And \.{WEAVE} translates this into something you are glad you didn't have
to type:
$$\lpile{\.{The characters are placed into \\\\\{buffer\},}\cr
  \.{which is a \\\&\{packed\}{ }\\\&\{array\}{ }\$
    [1\\to\\|n]\${ }\\\&\{of\}{ }\\\\\{char\}.}\cr}$$
Incidentally, the cross-reference index that \.{WEAVE} would make, in
the presence of a comment like this, would include
the current module number as one of the index entries for \\{buffer}
\vadjust{\eject}% page break inserted Dec 88
and \\{char}, even though \\{buffer} and \\{char}
might not appear in the \PASCAL\ part of
this module. Thus, the index covers references to identifiers in
the explanatory comments as well as in the program itself; you will
soon learn to appreciate this feature. However, the identifiers
\&{packed} and \&{array} and \|n\ and \&{of\/} would not be indexed,
because \.{WEAVE} does not make index entries for reserved words or
single-letter identifiers. Such identifiers are felt to be so ubiquitous
that it would be pointless to mention every place where they occur.

Speaking of identifiers, the author of \.{WEB} thinks that
\\{IdentifiersSeveralWordsLong} look terribly ugly when they mix
uppercase and lowercase letters. He recommends that
\\{identifiers\_several\_words\_long} be written with underline characters
to get a much better effect. The actual identifiers sent to the \PASCAL\
compiler by \.{TANGLE} will have such underlines removed, and \.{TANGLE}
will check to make sure that two different identifiers do not become
identical when this happens. (In fact, \.{TANGLE} even checks that
the first seven characters of identifiers are unique, when lowercase
letters have been converted to uppercase; the number seven in this
constraint is more strict than \PASCAL's eight, and it can
be changed if desired.) The \.{WEAVE} processor will properly
alphabetize identifiers that have embedded underlines
when it makes the index.

Although a module begins with \TeX\ text and ends with \PASCAL\ text, we
have noted that the dividing line isn't sharp, since \PASCAL\ text can be
included in \TeX\ text if it is enclosed in `\pb'.  Conversely, \TeX\ text
also appears frequently within \PASCAL\ text, because everything in
comments (i.e., between left and right braces) is treated as \TeX\ text.
Furthermore, a module name consists of \TeX\ text; thus, a \.{WEB} file
typically involves constructions like `\.{if} \.x \.= \.0 \.{then}
\.{@<Empty} \.{the} \.{|buffer|} \.{array@>}' where we go back and forth
between \PASCAL\ and \TeX\ conventions in a natural way.
\section Macros.
A \.{WEB} programmer can define three kinds of macros to make the programs
shorter and more readable:

\yskip\hang`\.{@d} \\{identifier} \.= \\{constant}' defines a {\sl numeric\/}
macro, allowing \.{TANGLE} to do rudimentary arithmetic.

\yskip\hang`\.{@d} \\{identifier} \.{==} \PASCAL\ text' defines a {\sl
simple\/} macro, where the identifier will be replaced by the \PASCAL\ text
when \.{TANGLE} produces its output.

\yskip\hang`\.{@d} \\{identifier}\.{(\#) ==} \PASCAL\ text' defines a
{\sl parametric\/} macro, where the identifier will be replaced by the \PASCAL\
text and where occurrences of \.{\#} in that \PASCAL\ text will be
replaced by an argument.

\yskip\noindent In all three cases, the identifier must have length greater
than one; it must not be a single letter.

Numeric macros are subject to the following restrictions:\quad
(1)~The identifier must
be making its first appearance in the \.{WEB} file;
a numeric macro must be defined before it is used.\quad
(2)~The right-hand side of the numeric definition must be made entirely from
integer constants, numeric macros, preprocessed strings (see below), and
plus~signs or minus signs.  No other operations or symbols are allowed,
not even parentheses, except that \PASCAL-like comments (enclosed in
braces) can appear. Indeed, comments are recommended, since it is usually
wise to give a brief explanation of the significance of each identifier as
it is defined.\quad
(3)~The numeric value must be less than $2^{15}=32768$ in absolute value.
(For larger values, you can use `\.{==}' in place of~`\.=', thus making use
of a simple macro instead of a numeric one. Note, however, that simple
macros sometimes have a different effect. For example, consider the three
definitions `\.{@d n1=2 @d n2=2+n1 @d n3==2+n1}'; then `\.{x-n2}' will
expand into `\.{x-4}', while `\.{x-n3}' will expand into `\.{x-2+2}' which
is quite different! It is wise to include parentheses in non-numeric
macros, e.g., `\.{@d n3==(2+n1)}', to avoid such errors.)

When constants are connected by plus signs or minus
signs in a \PASCAL\ program, \.{TANGLE} does the arithmetic before putting
the constant into the output file. Therefore it is permissible to say, for
example, `\&{array} $[0\,.\,.\,\\{size}-1]$' if \\{size} has been declared
as a macro; note that \PASCAL\ doesn't allow this kind of compile-time
arithmetic if \\{size} is a \&{constant} quantity in the program. Another
use of \.{TANGLE}'s arithmetic is to make \&{case} statement labels such
as `$\\{flag}+1$' legitimate. Of course, it is improper to change \.{2+2}
into \.4 without looking at the surrounding context; many counterexamples
exist, such as the phrases `\.{-2+2}', `\.{x/2+2}', and `\.{2+2E5}'.  The
program for \.{TANGLE}, in the appendix, gives precise details about this
conversion, which \.{TANGLE} does only when it is safe.

The right-hand sides of simple and parametric macros
are required to have balanced parentheses, and the \PASCAL\ texts of
modules must have balanced parentheses too. Therefore when the argument
to a para\-metric macro appears in parentheses, both parentheses
will belong to the same \PASCAL\ text.

The appendices to this report contain hundreds of typical examples of the
usefulness of \.{WEB} macros, so it is not necessary to dwell on the
subject here. However, the reader should know that \.{WEB}'s apparently
primitive macro capabilities can actually do a lot of rather surprising
things. Here is a construction that sheds further light on what is
possible: After making the definitions
$$\catcode`\#=12
\lpile{\.{@d two\_cases(#)==case j of 1:#(1); 2:#(2); end}\cr
\.{@d reset\_file(#)==reset(input\_file@\&#)}\cr}$$
one can write `\.{two\_cases(reset\_file)}' and the resulting \PASCAL\
output will be
$$\.{case j of 1:reset(input\_file1); 2:reset(input\_file2); end}$$
(but in uppercase letters and with \.\_'s removed).
The `\.{@\&}' operation used here joins together two adjacent tokens
into a single token, as explained later; otherwise the \PASCAL\ file would
contain a space between \.{input\_file} and the digit that followed it.
This trick can be used to provide the effect of an array of files, if you
are unfortunate enough to have a \PASCAL\ compiler that doesn't allow such
arrays.  Incidentally, the cross-reference index made by \.{WEAVE} from
this example would contain the identifier \\{input\_file} but it would not
contain \\{input\_file1} or \\{input\_file2}. Furthermore, \.{TANGLE}
would not catch the error that \.{INPUTFILE1} and \.{INPUTFILE2} both
begin with the same nine letters; one should be more careful when using
`\.{@\&}'! But such aspects of the construction in this trick are
peripheral to our main point, which is that a parametric macro name without
arguments can be used as an argument to another parametric macro.

Although \.{WEB}'s macros are allowed to have at most one parameter, the
following example shows that this is not as much of a restriction as it
may seem at first. Let \\{amac} and \\{bmac} be any parametric macros, and
suppose that we want to get the effect of
$$\catcode`\#=12
\.{@d cmac(#1,#2) == amac(#1) bmac(#2)}$$
which \.{WEB} doesn't permit. The solution is to make the definitions
$$\catcode`\#=12
\lpile{\.{@d cmac(#) == amac(#) dmac}\cr
\.{@d dmac(#) == bmac(#)}\cr}$$
and then to say `\.{cmac(x)(y)}'.

There is one restriction in the generality of \.{WEB}'s parametric
macros, however: the argument to a para\-metric macro must not come from
the expansion of a macro that has not already been ``started.'' For
example, here is one of the things \.{WEB} cannot handle:
$$\catcode`\#=12
\lpile{\.{@d arg == (p)}\cr
\.{@d identity(#) == #}\cr
\.{@p identity arg}\cr}$$
In this case \.{TANGLE} will complain that the \.{identity} macro is not
followed by an argument in parentheses.

The \.{WEB} language has another feature that is somewhat similar to a
numeric macro. A {\sl preprocessed string\/} is a string that is like
a \PASCAL\ string but delimited by double-quote marks (\.") instead of
single-quotes. Double-quote marks inside of such strings are indicated by
giving two double-quotes in a row. If a preprocessed string is
of length one (e.g., \.{"A"} or \.{""""}), it will be treated by \.{TANGLE}
as equivalent to the corresponding ASCII-code integer (e.g., \.{65} or
\.{34}). And if a preprocessed string is not of length one, it will be
converted into an integer equal to 256 or more. A {\sl string pool\/}
containing all such strings will be written out by the \.{TANGLE}
processor; this string pool file consists of string 256, then string 257,
etc., where each string is followed by an end-of-line and prefixed by two
decimal digits that define its length.  Thus, for example, the empty string
\.{""} would be represented in the string pool file by a line containing
the two characters `\.{00}', while the string \.{"""String"""} would be
represented by `\.{08"String"}'.  A given string appears at most once in
the string pool; the use of such a pool makes it easier to cope with
\PASCAL's restrictions on string manipulation. The string pool ends with
`\.{*nnnnnnnnn}', where \.{nnnnnnnnn} is a decimal number
called the {\sl string pool check sum}. If any string changes, the check
sum almost surely changes too; thus, the `\.{@\$}' feature
described below makes it possible for a program to assure itself that it
is reading its own string pool.

Here is a simple example that combines numeric macros with preprocessed
strings of length one:
$$\lpile{\.{@d upper\_case\_Y = "Y"}\cr
\.{@d case\_difference = -"y"+upper\_case\_Y}\cr}$$
The result is to define
$\\{upper\_case\_Y}=89$, $\\{case\_difference}=-32$.
\section Control codes.
We have seen several magic uses of `\.{@}' signs in \.{WEB} files, and it
is time to make a systematic study of
these special features. A \.{WEB} {\sl control code\/}
is a two-character combination of which the first is `\.@'.

Here is a complete list of the legal control codes. The letters $L$, $T$,
$P$, $M$, $C$, and/or $S$ following each code indicate whether or not that
code is allowable in limbo, in \TeX\ text, in \PASCAL\ text, in module
names, in comments, and/or in strings.  A bar over such a letter means
that the control code terminates the present part of the \.{WEB} file; for
example, $\overline L$ means that this control code ends the limbo material
before the first module.

\def\@#1[#2] {\yskip\hangindent 2em\noindent\.{@#1\unskip
  \spacefactor1000{ }}$[#2]$\quad}
\def\oP{\overline P}
\def\oT{\overline T\mskip1mu}

\@@ [C,L,M,P,S,T] A double \.@ denotes the single character `\.@'. This is
the only control code that is legal in limbo, in comments, and in strings.

\@\ [\overline L,\oP,\oT] This denotes the beginning of a new
(unstarred) module. A tab mark or end-of-line (carriage return)
is equivalent to a space when it follows an \.@ sign.

\@* [\overline L,\oP,\oT] This denotes the beginning of a new starred
module, i.e., a module that begins a new major group. The title of the new
group should appear after the \.{@*}, followed by a period. As explained
above, \TeX\ control sequences should be avoided in such titles unless
they are quite simple. When \.{WEAVE} and \.{TANGLE} read a \.{@*}, they
print an asterisk on the terminal
followed by the current module number, so that the user
can see some indication of progress. The very first module should be starred.

\@d [\oP,\oT] Macro definitions begin with \.{@d} (or \.{@D}), followed by
the \PASCAL\ text for one of the three kinds of macros, as explained
earlier.

\@f [\oP,\oT] Format definitions begin with \.{@f} (or \.{@F}); they cause
\.{WEAVE} to treat identifiers in a special way when they appear in
\PASCAL\ text. The general form of a format definition is `\.{@f} \|l \.{==}
\|r', followed by an optional comment enclosed in braces, where \|l and \|r
are identifiers; \.{WEAVE} will subsequently treat identifier \|l as it
currently treats \|r. This feature allows a \.{WEB} programmer to invent
new reserved words and/or to unreserve some of \PASCAL's reserved
identifiers. The definition part of each module consists of any number of
macro definitions (beginning with \.{@d}) and format definitions (beginning
with \.{@f}), intermixed in any order.

\@p [\oP,\oT] The \PASCAL\ part of an unnamed module begins with \.{@p}
(or \.{@P}). This causes \.{TANGLE} to append the following \PASCAL\ code
to the initial program text $T_0$ as explained above. The \.{WEAVE}
processor does not cause a `\.{@p}' to appear explicitly in the \TeX\
output, so if you are creating a \.{WEB} file based on a \TeX-printed
\.{WEB} documentation you have to remember to insert \.{@p} in the
appropriate places of the unnamed modules.

\@< [P,\oT] A module name begins with \.{@<} followed by \TeX\ text followed
by \.{@>}; the \TeX\ text should not contain any \.{WEB} control codes
except \.{@@}, unless these control codes appear in \PASCAL\ text that
is delimited by \pb. The module name may be abbreviated, after its first
appearance in a \.{WEB} file, by giving any unique prefix followed by \.{...},
where the three dots immediately precede the closing \.{@>}. No module name
should be a prefix of another. Module names may not appear in \PASCAL\
text that is enclosed in \pb, nor may they appear in the definition part
of a module (since the appearance of a module name ends the definition
part and begins the \PASCAL\ part).

\@\' [P,T] This denotes an octal constant, to be formed from the
succeeding digits. For example, if the \.{WEB} file contains `\.{@\'100}',
the \.{TANGLE} processor will treat this an equivalent to `\.{64}';
the constant will be formatted as ``\O{100}'' in the \TeX\ output
produced via \.{WEAVE}. You should use octal notation only for positive
constants; don't try to get, e.g., $-1$ by saying `\.{@\'777777777777}'.

\@" [P,T] A hexadecimal constant; `\.{@"D0D0}' tangles to \.{53456} and
weaves to `\H{D0D0}'.

\@\$ [P] This denotes the string pool check sum.

\@\{ [P] The beginning of a ``meta comment,'' i.e., a comment
that is supposed to appear in the \PASCAL\ code, is indicated by
\.{@\{} in the \.{WEB} file. Such delimiters can be used as
isolated symbols in macros or modules, but they should be properly nested
in the final \PASCAL\ program. The \.{TANGLE} processor will convert
`\.{@\{}' into `\.\{' in the \PASCAL\ output file, unless
the output is already part of a meta-comment; in the latter case
`\.{@\{}' is converted into `\.[', since \PASCAL\ does not allow
nested comments. Incidentally, module numbers are automatically inserted
as meta-comments into the \PASCAL\ program, in order to help correlate the
outputs of \.{WEAVE} and \.{TANGLE} (see Appendix~C\null). Meta-comments
can be used to put conditional text into a \PASCAL\ program; this helps to
overcome one of the limitations of \.{WEB}, since the simple macro
processing routines of \.{TANGLE} do not include the dynamic evaluation of
boolean expressions.

\@\} [P] The end of a ``meta comment'' is indicated by `\.{@\}}'; this is
converted either into `\.\}' or `\.{]}' in the \PASCAL\ output, according
to the conventions explained for \.{@\{} above.

\@\& [P] The \.{@\&} operation causes whatever is on its left to be
adjacent to whatever is on its right, in the \PASCAL\ output. No spaces or
line breaks will separate these two items. However, the thing on the left
should not be a semicolon, since a line break might occur after a semicolon.

\@\^ [P,T] The ``control text'' that follows, up to the next
`\.{@>}', will be entered into the index together with the identifiers of
the \PASCAL\ program; this text will appear in roman type. For example, to
put the phrase ``system dependencies'' into the index, you can type
`\.{@\^system dependencies@>}' in each module
that you want to index as system dependent. A control text, like a string,
must end on the same line of the \.{WEB} file as it began.  Furthermore,
no \.{WEB} control codes are allowed in a control text, not even
\.{@@}. (If you need an \.{@} sign you can get around this restriction by
typing `\.{\\AT!}'.)

\@. [P,T] The ``control text'' that follows will be entered into the index
in \.{typewriter} \.{type}; see the rules for `\.{@\^}', which is analogous.

\@: [P,T] The ``control text'' that follows will be entered into the index
in a format controlled by the \TeX\ macro `\.{\\9}', which the user
should define as desired; see the rules for `\.{@\^}', which is analogous.

\@t [P] The ``control text'' that follows, up to the next `\.{@>}', will
be put into a \TeX\ \.{\\hbox} and formatted along with the neighboring
\PASCAL\ program. This text is ignored by \.{TANGLE}, but it can be used
for various purposes within \.{WEAVE}. For example, you can make comments
that mix \PASCAL\ and classical mathematics, as in `$\\{size}<2^{15}$', by
typing `\.{|size < @t\$2\^\{15\}\$@>|}'.  A control text must end on the
same line of the \.{WEB} file as it began, and it may not contain any
\.{WEB} control codes.

\@= [P] The ``control text'' that follows, up to the next `\.{@>}', will
be passed verbatim to the \PASCAL\ program.

\@\\ [P] Force end-of-line here in the \PASCAL\ program file.

\@! [P,T] The module number in an index entry will be underlined if `\.{@!}'
immediately precedes the identifier or control text being indexed. This
convention is used to distinguish the modules where an identifier is
defined, or where it is explained in some special way, from the modules
where it is used. A~reserved word or an identifier of length one will not
be indexed except for underlined entries. An `\.{@!}' is implicitly inserted
by \.{WEAVE} just after the reserved words \&{function}, \&{procedure},
\&{program}, and \&{var}, and just after \.{@d} and \.{@f}. But you should
insert your own `\.{@!}' before the definitions of types, constants,
variables, parameters, and components of records and enumerated types that
are not covered by this implicit convention, if you want to improve the
quality of the index that you get.

\@? [P,T] This cancels an implicit (or explicit) `\.{@!}', so that the next
index entry will not be underlined.

\@, [P] This control code inserts a thin space in \.{WEAVE}'s output; it is
ignored by \.{TANGLE}. Sometimes you need this extra space if you are using
macros in an unusual way, e.g., if two identifiers are adjacent.

\@/ [P] This control code causes a line break to occur within a \PASCAL\
program formatted by \.{WEAVE}; it is ignored by \.{TANGLE}. Line breaks
are chosen automatically by \TeX\ according to a scheme that works 99\%\
of the time, but sometimes you will prefer to force a line break so that
the program is segmented according to logical rather than visual criteria.
Caution: `\.{@/}' should be used only after statements or clauses, not in
the middle of an expression; use \.{@|} in the middle of expressions, in
order to keep \.{WEAVE}'s parser happy.

\@| [P] This control code specifies an optional line break in the midst of
an expression. For example, if you have a long condition between \&{if} and
\&{then}, or a long expression on the right-hand side of an assignment
statement, you can use `\.{@|}' to specify breakpoints more logical than
the ones that \TeX\ might choose on visual grounds.

\@\# [P] This control code forces a line break, like \.{@/} does,
and it also causes a little extra white space to appear between the lines at
this break. You might use it, for example, between procedure definitions or
between groups of macro definitions that are logically separate but within
the same module.

\@+ [P] This control code cancels a line break that might otherwise be
inserted by \.{WEAVE}, e.g., before the word `\&{else}', if you want to
put a short if-then-else construction on a single line. It is ignored by
\.{TANGLE}.

\@; [P] This control code is treated like a semicolon, for formatting
purposes, except that it is invisible. You can use it, for example, after
a module name when the \PASCAL\ text represented by that module name ends
with a semicolon.

\yskip\noindent
The last six control codes (namely `\.{@,}', `\.{@/}', `\.{@|}',
`\.{@\#}', `\.{@+}', and `\.{@;}') have no effect on the \PASCAL\
program output by \.{TANGLE}; they merely help to improve the readability
of the \TeX-formatted \PASCAL\ that is output by \.{WEAVE}, in unusual
circumstances. \.{WEAVE}'s built-in formatting method is fairly good, but
it is incapable of handling all possible cases, because it must deal with
fragments of text involving macros and module names; these fragments do
not necessarily obey \PASCAL's syntax. Although \.{WEB} allows you to
override the automatic formatting, your best strategy is not to worry
about such things until you have seen what \.{WEAVE} produces automatically,
since you will probably need to make only a few corrections when you are
touching up your documentation.

Because of the rules by which every module is broken into three parts,
the control codes `\.{@d}', `\.{@f}', and `\.{@p}' are not allowed to occur
once the \PASCAL\ part of a module has begun.
\section Additional features and caveats.

1. The character pairs `\.{(*}', `\.{*)}', `\.{(.}', and `\.{.)}' are
converted automatically in \PASCAL\ text as though they were
`\.{@\{}', `\.{@\}}', `\.[', and `\.]', respectively, except
of course in strings. Furthermore in certain installations of \.{WEB} that
{\def\\#1#2{`{\tentex\char'#1#2}'}%
have an extended character set, the characters \\32, \\34, \\35, \\30,
\\36, \\04, \\37, \\05, and \\06}
can be typed as abbreviations for
`\.{<>}', `\.{<=}', `\.{>=}', `\.{:=}', `\.{==}', `\.{and}', `\.{or}',
`\.{not}', and `\.{in}', respectively. However, the latter abbreviations
are not used in the standard versions of \.{WEAVE.WEB} and \.{TANGLE.WEB}
that are distributed to people who are installing \.{WEB} on other
computers, and the programs are designed to produce only standard ASCII
characters as output if the input consists entirely of ASCII characters.

2. If you have an extended character set, all of the characters listed
in Appendix C of {\sl The \TeX book\/} can be used in strings. But you should
stick to standard ASCII characters if you want to write programs that will
be useful to all the poor souls out there who don't have extended
character sets.

3. The \TeX\ file output by \.{WEAVE} is broken into lines having at most
80 characters each. The algorithm that does this line breaking is unaware
of \TeX's convention about comments following `\.\%' signs on a line. When
\TeX\ text is being copied, the existing line breaks are copied as well,
so there is no problem with `\.\%' signs unless the original \.{WEB} file
contains a line more than eighty characters long or a line with \PASCAL\
text in \pb\ that expands to more than eighty characters long. Such lines
should not have `\.\%' signs.

4. \PASCAL\ text is translated by a ``bottom up'' procedure that
identifies each token as a ``part of speech'' and combines parts of speech
into larger and larger phrases as much as possible according to a special
grammar that is explained in the documentation of \.{WEAVE}. It is easy to
learn the translation scheme for simple constructions like single
identifiers and short expressions, just by looking at a few examples of
what \.{WEAVE} does, but the general mechanism is somewhat complex because
it must handle much more than \PASCAL\ itself. Furthermore the output
contains embedded codes that cause \TeX\ to indent and break lines as
necessary, depending on the fonts used and the desired page width. For
best results it is wise to adhere to the following restrictions:

\yskip\itemitem{a)}Comments in \PASCAL\ text should appear only after
statements or clauses; i.e., after semicolons, after reserved words like
\&{then} and \&{do}, or before reserved words like \&{end} and \&{else}.
Otherwise \.{WEAVE}'s parsing method may well get mixed up.

\itemitem{b)}Don't enclose long \PASCAL\ texts in \pb, since the
indentation and line breaking codes are omitted when the \pb\ text is
translated from \PASCAL\ to \TeX. Stick to simple expressions or
statements.

\yskip
5. Comments and module names are not permitted in \pb\ text. After a `\.|'
signals the change from \TeX\ text to \PASCAL\ text, the next `\.|' that is
not part of a string or control text ends the \PASCAL\ text.

6. A comment must have properly nested occurrences of left and right
braces, otherwise \.{WEAVE} and \.{TANGLE} will not know where the comment
ends. However, the character pairs `\.{\\\{}' and `\.{\\\}}' do not count
as left and right braces in comments, and the character pair `\.{\\|}'
does not count as a delimiter that begins \PASCAL\ text. (The actual rule
is that a character after `\.\\' is ignored; hence in `\.{\\\\\{}' the
left brace {\sl does\/} count.) At present, \.{TANGLE} and \.{WEAVE} treat
comments in slightly different ways, and it is necessary to satisfy both
conventions: \.{TANGLE} ignores `\.|' characters entirely, while \.{WEAVE}
uses them to switch between \TeX\ text and \PASCAL\ text. Therefore, a
comment that includes a brace in a string in \pb---e.g., `\.{\{{ }look at
this |"\{"| \}}'---will be handled correctly by \.{WEAVE}, but \.{TANGLE}
will think there is an unmatched left brace. In order to satisfy both
processors, one can write `\.{\{{ }look at this \\leftbrace\\{ }\}}', after
setting up`\.{\\def\\leftbrace\{|"\{"|\}}'.

7. Reserved words of \PASCAL\ must appear entirely in lowercase letters
in the \.{WEB} file; otherwise their special nature will not be recognized
by \.{WEAVE}. You could, for example, have a macro named \\{END} and it
would not be confused with \PASCAL's \&{end}.

However, you may not want to capitalize macro names just to distinguish them
from other identifiers.  Here is a way to unreserve \PASCAL's reserved word
`\&{type}' and to substitute another word `\&{mtype}' in the \.{WEB} file.
$$\vbox{\halign{\tt #\hfil\cr
@d type(\char'43) == mem[\char'43].t\cr
@d mtype == t \char'100\char'46{} y \char'100\char'46{} p
  \char'100\char'46{} e\cr
@f mtype == type\cr
@f type == true\cr}}$$
In the output of \.{TANGLE}, the macro \.{mtype} now produces `\.{TYPE}'
and the macro \.{type(x)} now produces `\.{MEM[X].T}'. In the output of
\.{WEAVE}, these same inputs produce \&{mtype} and \\{type}(\|x),
respectively.

8. The \.{@f} feature allows you to define one identifier to act like
another, and these format definitions are carried out sequentially, as the
example above indicates. However, a given identifier has only one printed format
throughout the entire document (and this format will even be used before
the \.{@f} that defines it). The reason is that \.{WEAVE} operates in two
passes; it processes \.{@f}'s and cross-references on the first pass and
it does the output on the second.

9. You may want some \.{@f} formatting that doesn't correspond to any
existing reserved word. In that case, \.{WEAVE} could be extended in a
fairly obvious way to include new ``reserved words'' in its vocabulary.
The identifier `\&{xclause}' has in fact been included already as a
reserved word, so that it can be used to format the `\&{loop}' macro,
where `\&{loop}' is defined to be equivalent to `\&{while \\{true} do}'.

10. Sometimes it is desirable to insert spacing into \PASCAL\ code that is
more general than the thin space provided by `\.{@,}'. The \.{@t} feature
can be used for this purpose; e.g., `\.{@t\\hskip 1in@>}' will
leave one inch of blank space. Furthermore, `\.{@t\\4@>}' can be
used to backspace by one unit of indentation, since the control sequence
\.{\\4} is defined in \.{webmac} to be such a backspace. (This
control sequence is used, for example, at the beginning of lines that
contain labeled statements, so that the label will stick out a little at
the left.)

11. \.{WEAVE} and \.{TANGLE} are designed to work with two input files,
called \\{web\_file} and \\{change\_file}, where \\{change\_file} contains
data that overrides selected portions of \\{web\_file}. The resulting merged
text is actually what has been called the \.{WEB} file elsewhere in this
report.

Here's how it works: The change file consists of zero or more ``changes,''
where a change has the form `\.{@x}$\langle$old lines$\rangle$\.{@y}$\langle$%
new lines$\rangle$\.{@z}'. The special control codes \.{@x}, \.{@y}, \.{@z},
which are allowed only in change files, must appear at the beginning of a line;
the remainder of such a line is ignored.
The $\langle$old lines$\rangle$ represent material that exactly matches
consecutive lines of the \\{web\_file}; the $\langle$new lines$\rangle$
represent zero or more lines that are supposed to replace the old. Whenever
the first ``old line'' of a change is found to match a line in the
\\{web\_file}, all the other lines in that change must match too.

Between changes, before the first change, and after the last change,
the change file can have any number of lines that do not begin with
`\.{@x}', `\.{@y}', or~`\.{@z}'. Such lines are bypassed and not used for
matching purposes.

This dual-input feature is useful when working with a master \.{WEB} file
that has been received from elsewhere (e.g., \.{TANGLE.WEB} or
\.{WEAVE.WEB} or \.{TEX.WEB}), when changes are desirable to customize the
program for your local computer system. You will be able to debug your
system-dependent changes without clobbering the master web file; and once
your changes are working, you will be able to incorporate them readily
into new releases of the master web file that you might receive from time
to time.
\section Appendices.
The basic ideas of \.{WEB} can be understood most easily by looking at
examples of ``real'' programs. Appendix~A shows the \.{WEB} input that
generated modules 55--59 of the \.{WEAVE} program; Appendix~B shows the
corresponding \TeX\ code output by \.{WEAVE}; and Appendix~C shows excerpts
from the corresponding \PASCAL\ code output by \.{TANGLE}.

The complete webs for \.{WEAVE} and \.{TANGLE} appear as the bulk of this
report, in Appendices D and~E. The reader should first compare Appendix~A
to the corresponding portion of Appendix~D; then the same material should
be compared to Appendices B and~C. Finally, if time permits, the reader may
enjoy studying the complete programs in Appendices D and~E, since \.{WEAVE}
and \.{TANGLE} contain several interesting aspects, and since an attempt
has been made in these appendices to evolve a style of programming that
makes good use of the \.{WEB} language.

Finally, Appendix F is the `\.{webmac}' file that sets \TeX\ up to accept
the output of \.{WEAVE}; Appendix~G discusses how to use some of its macros
to vary the output formats; and Appendix~H discusses what needs to be done
when \.{WEAVE} and \.{TANGLE} are installed in a new operating environment.
\section Performance statistics.
The programs in Appendices D and E will optionally keep statistics on
how much memory they require. Here is what they once printed out when
processing themselves:

\def\pstat#1#2#3
#4{\yskip\noindent\.{#1} applied to \.{#2} (cpu time #3 sec)\par
\halign{\quad\tt##\hfil\cr#4}}

\pstat{TANGLE}{TANGLE}{15}
{Memory usage statistics:\cr
456 names, 215 replacement texts;\cr
3396+3361 bytes, 6685+7329+5805 tokens.\cr}

\pstat{TANGLE}{WEAVE}{30}
{Memory usage statistics:\cr
692 names, 339 replacement texts;\cr
4576+4294 bytes, 10184+9875+9150 tokens.\cr}

\pstat{WEAVE}{TANGLE}{45}
{Memory usage statistics: 478 names, 2045 cross references, 4159+3729 bytes;\cr
parsing required 684 scraps, 1300 texts, 3766 tokens, 119 levels;\cr
sorting required 34 levels.\cr}

\pstat{WEAVE}{WEAVE}{65}
{Memory usage statistics: 737 names, 3306 cross references, 4896+4962 bytes;\cr
parsing required 684 scraps, 1300 texts, 3766 tokens, 119 levels;\cr
sorting required 73 levels.\cr}

\yskip\noindent The cpu time for \PASCAL\ to process \.{TANGLE.PAS} was
approximately 13 seconds, and \.{WEAVE.PAS} took approximately 26 seconds;
thus the tangling time was slightly more than the compiling time.  The cpu
time for \TeX\ to process \.{TANGLE.TEX} was approximately 500 seconds,
and \.{WEAVE.TEX} took approximately 750 seconds (i.e., about 7
seconds per printed page, where these pages are substantially larger than
the pages in a normal book). All cpu times quoted are for a DECsystem-10.

\def\K{{\mc K}}
The file \.{TANGLE.WEB} is about 125\K\ characters long; \.{TANGLE}
reduces it to a file \.{TANGLE.PAS} whose size is about 42\K\ characters,
while \.{WEAVE} expands it to a file \.{TANGLE.TEX} of about 185\K\null.
The corresponding file sizes for \.{WEAVE.WEB}, \.{WEAVE.PAS}, and
\.{WEAVE.TEX} are 180\K, 89\K, and 265\K.

The much larger file \.{TEX.WEB} led to the following numbers:

\pstat{TANGLE}{TEX}{110}
{Memory usage statistics:\cr
3750 names, 1768 replacement texts;\cr
41895+41053 bytes, 42378+45074+41091 tokens.\cr}

\pstat{WEAVE}{TEX}{270}
{Memory usage statistics: 3412 names, 19699 cross references,
  37900+40232 bytes;\cr
parsing required 685 scraps, 1303 texts, 3784 tokens, 104 levels;\cr
sorting required 52 levels.\cr}

\yskip\noindent
\PASCAL\ did \.{TEX.PAS} in about 75 seconds; \TeX\ did \.{TEX.TEX}
in about 3600.
% Here is a quotation that could not really be omitted

\vfill

{\baselineskip9pt
\halign to\hsize{\hfil\quoteit#\tabskip 0pt plus 100pt&
  \hfil\quoteit#\tabskip 0pt\cr
O, what a tangled web we weave&
  O, what a tangled WEB we weave\cr
When first we practise to deceive!&
  When \TeX\ we practise to conceive!\cr
\noalign{\vskip 2pt}
\quoterm ---SIR WALTER SCOTT, {\quoteit Marmion} 6:17 (1808)&
  \quoterm ---RICHARD PALAIS (1982)\cr
}}
\eject
\def\runninghead{APPENDIX A --- {\tentt WEB} FILE FORMAT}
\section Appendix A.
This excerpt from \.{WEAVE.WEB} produced modules 55--59 in Appendix~D.
Note that some of the lines are indented to show the program structure.
The indentation is ignored by \.{WEAVE} and \.{TANGLE}, but users find
that \.{WEB} files are quite readable if they have some such indentation.

\vskip 6pt
\begingroup \def\tt{\eighttt} \baselineskip9pt
% Note to myself: I had to remove SAIL characters from the file here!
% Also tabs replaced by double-space. The changes were made in WEAVE source.
\verbatim
@* Searching for identifiers.
The hash table described above is updated by the |id_lookup| procedure,
which finds a given identifier and returns a pointer to its index in
|byte_start|. The identifier is supposed to match character by character
and it is also supposed to have a given |ilk| code; the same name may be
present more than once if it is supposed to appear in the index with
different typesetting conventions.
If the identifier was not already present, it is inserted into the table.

Because of the way \.{WEAVE}'s scanning mechanism works, it is most convenient
to let |id_lookup| search for an identifier that is present in the |buffer|
array. Two other global variables specify its position in the buffer: the
first character is |buffer[id_first]|, and the last is |buffer[id_loc-1]|.

@<Glob...@>=
@!id_first:0..long_buf_size; {where the current identifier begins in the buffer}
@!id_loc:0..long_buf_size; {just after the current identifier in the buffer}
@#
@!hash:array [0..hash_size] of sixteen_bits; {heads of hash lists}

@ Initially all the hash lists are empty.

@<Local variables for init...@>=
@!h:0..hash_size; {index into hash-head array}

@ @<Set init...@>=
for h:=0 to hash_size-1 do hash[h]:=0;

@ Here now is the main procedure for finding identifiers (and index
entries).  The parameter |t| is set to the desired |ilk| code. The
identifier must either have |ilk=t|, or we must have
|t=normal| and the identifier must be a reserved word.

@p function id_lookup(@!t:eight_bits):name_pointer; {finds current identifier}
label found;
var i:0..long_buf_size; {index into |buffer|}
@!h:0..hash_size; {hash code}
@!k:0..max_bytes; {index into |byte_mem|}
@!w:0..ww-1; {row of |byte_mem|}
@!l:0..long_buf_size; {length of the given identifier}
@!p:name_pointer; {where the identifier is being sought}
begin l:=id_loc-id_first; {compute the length}
@<Compute the hash code |h|@>;
@<Compute the name location |p|@>;
if p=name_ptr then @<Enter a new name into the table at position |p|@>;
id_lookup:=p;
end;

@ A simple hash code is used: If the sequence of
ASCII codes is $c_1c_2\ldots c_m$, its hash value will be
$$(2^{n-1}c_1+2^{n-2}c_2+\cdots+c_n)\,\bmod\,|hash_size|.$$

@<Compute the hash...@>=
h:=buffer[id_first]; i:=id_first+1;
while i<id_loc do
  begin h:=(h+h+buffer[i]) mod hash_size; incr(i);
  end
?endgroup % end of verbatim mode
\endgroup
\vfill\eject
\def\runninghead{APPENDIX B --- TRANSLATION BY {\tentt WEAVE}}
\section Appendix B.
This excerpt from \.{WEAVE.TEX} corresponds to Appendix A.

% I've inserted \vfill's here at the blank lines, to squeeze this on one page!
\begingroup \def\tt{\eighttt} \baselineskip9pt
\verbatim
?vfill?verbatimgobble
\N55.  Searching for identifiers.
The hash table described above is updated by the \\{id\_lookup} procedure,
which finds a given identifier and returns a pointer to its index in
\\{byte\_start}. The identifier is supposed to match character by character
and it is also supposed to have a given \\{ilk} code; the same name may be
present more than once if it is supposed to appear in the index with
different typesetting conventions.
If the identifier was not already present, it is inserted into the table.
?vfill?verbatimgobble
Because of the way \.{WEAVE}'s scanning mechanism works, it is most convenient
to let \\{id\_lookup} search for an identifier that is present in the %
\\{buffer}
array. Two other global variables specify its position in the buffer: the
first character is $\\{buffer}[\\{id\_first}]$, and the last is $\\{buffer}[%
\\{id\_loc}-1]$.
?vfill?verbatimgobble
\Y\P$\4\X9:Globals in the outer block\X\mathrel{+}\S$\6
\4\\{id\_first}: \37$0\to\\{long\_buf\_size}$;\C{where the current identifier
begins in the buffer}\6
\4\\{id\_loc}: \37$0\to\\{long\_buf\_size}$;\C{just after the current
identifier in the buffer}\7
\4\\{hash}: \37\&{array} $[0\to\\{hash\_size}]$ \1\&{of}\5
\\{sixteen\_bits};\C{heads of hash lists}\2\par
\fi
?vfill?verbatimgobble
\M56. Initially all the hash lists are empty.
?vfill?verbatimgobble
\Y\P$\4\X16:Local variables for initialization\X\mathrel{+}\S$\6
\4\|h: \37$0\to\\{hash\_size}$;\C{index into hash-head array}\par
\fi
?vfill?verbatimgobble
\M57. \P$\X10:Set initial values\X\mathrel{+}\S$\6
\&{for} $\|h\K0\mathrel{\&{to}}\\{hash\_size}-1$ \1\&{do}\5
$\\{hash}[\|h]\K0$;\2\par
\fi
?vfill?verbatimgobble
\M58. Here now is the main procedure for finding identifiers (and index
entries).  The parameter \|t is set to the desired \\{ilk} code. The
identifier must either have $\\{ilk}=\|t$, or we must have
$\|t=\\{normal}$ and the identifier must be a reserved word.
?vfill?verbatimgobble
\Y\P\4\&{function}\1\  \37$\\{id\_lookup}(\|t:\\{eight\_bits})$: \37\\{name%
\_pointer};\C{finds current identifier}\6
\4\&{label} \37\\{found};\6
\4\&{var} \37\|i: \37$0\to\\{long\_buf\_size}$;\C{index into \\{buffer}}\6
\|h: \37$0\to\\{hash\_size}$;\C{hash code}\6
\|k: \37$0\to\\{max\_bytes}$;\C{index into \\{byte\_mem}}\6
\|w: \37$0\to\\{ww}-1$;\C{row of \\{byte\_mem}}\6
\|l: \37$0\to\\{long\_buf\_size}$;\C{length of the given identifier}\6
\|p: \37\\{name\_pointer};\C{where the identifier is being sought}\2\6
\&{begin} \37$\|l\K\\{id\_loc}-\\{id\_first}$;\C{compute the length}\6
\X59:Compute the hash code \|h\X;\6
\X60:Compute the name location \|p\X;\6
\&{if} $\|p=\\{name\_ptr}$ \1\&{then}\5
\X62:Enter a new name into the table at position \|p\X;\2\6
$\\{id\_lookup}\K\|p$;\6
\&{end};\par
\fi
?vfill?verbatimgobble
\M59. A simple hash code is used: If the sequence of
ASCII codes is $c_1c_2\ldots c_m$, its hash value will be
$$(2^{n-1}c_1+2^{n-2}c_2+\cdots+c_n)\,\bmod\,\\{hash\_size}.$$
?vfill?verbatimgobble
\Y\P$\4\X59:Compute the hash code \|h\X\S$\6
$\|h\K\\{buffer}[\\{id\_first}]$;\5
$\|i\K\\{id\_first}+1$;\6
\&{while} $\|i<\\{id\_loc}$ \1\&{do}\6
\&{begin} \37$\|h\K(\|h+\|h+\\{buffer}[\|i])\mathbin{\&{mod}}\\{hash\_size}$;\5
$\\{incr}(\|i)$;\6
\&{end}\2\par
\U section~58.\fi
?endgroup
\endgroup
\eject
\def\runninghead{APPENDIX C --- TRANSLATION BY {\tentt TANGLE}}
\section Appendix C.
The \.{TANGLE} processor converts \.{WEAVE.WEB} into a syntactically
correct (but not very pretty) \PASCAL\ program \.{WEAVE.PAS}.
The first three and last two lines of output are shown here, together with the
lines of code generated by modules 55--62 and the environments of
those lines. There are 1559 lines in all; the notation
`\.. \\. \\.' stands for portions that are not shown.

Note that, for example, the code corresponding to
module 55 begins with `\.{\{55:\}}' and ends with `\.{\{:55\}}';
the code from modules 59--62 has been tangled into the code from module 58.

\vskip6pt
\verbatim
{2:}{4:}{$C-,A+,D-}{[$C+,D+]}{:4}
PROGRAM WEAVE(WEBFILE,CHANGEFILE,TEXFILE);LABEL 9999;CONST{8:}
MAXBYTES=45000;MAXNAMES=5000;MAXMODULES=2000;HASHSIZE=353;BUFSIZE=100;
     . . .
TOKPTR:0..MAXTOKS;{MAXTOKPTR,MAXTXTPTR:0..MAXTOKS;}{:53}{55:}
IDFIRST:0..LONGBUFSIZE;IDLOC:0..LONGBUFSIZE;
HASH:ARRAY[0..HASHSIZE]OF SIXTEENBITS;{:55}{63:}CURNAME:NAMEPOINTER;
     . . .
PROCEDURE INITIALIZE;VAR{16:}I:0..127;{:16}{40:}WI:0..1;{:40}{56:}
H:0..HASHSIZE;{:56}{247:}C:ASCIICODE;{:247}BEGIN{10:}HISTORY:=0;{:10}
     . . .
TOKPTR:=1;TEXTPTR:=1;TOKSTART[0]:=1;TOKSTART[1]:=1;{MAXTOKPTR:=1;
MAXTXTPTR:=1;}{:54}{57:}FOR H:=0 TO HASHSIZE-1 DO HASH[H]:=0;{:57}{94:}
SCANNINGHEX:=FALSE;{:94}{102:}MODTEXT[0]:=32;{:102}{124:}OUTPTR:=1;
     . . .
IF R=0 THEN XREF[P]:=XREFPTR ELSE XMEM[R].XLINKFIELD:=XREFPTR;END;{:51}
{58:}FUNCTION IDLOOKUP(T:EIGHTBITS):NAMEPOINTER;LABEL 31;
VAR I:0..LONGBUFSIZE;H:0..HASHSIZE;K:0..MAXBYTES;W:0..1;
L:0..LONGBUFSIZE;P:NAMEPOINTER;BEGIN L:=IDLOC-IDFIRST;{59:}
H:=BUFFER[IDFIRST];I:=IDFIRST+1;
WHILE I<IDLOC DO BEGIN H:=(H+H+BUFFER[I])MOD HASHSIZE;I:=I+1;END{:59};
{60:}P:=HASH[H];
WHILE P<>0 DO BEGIN IF(BYTESTART[P+2]-BYTESTART[P]=L)AND((ILK[P]=T)OR((T
=0)AND(ILK[P]>3)))THEN{61:}BEGIN I:=IDFIRST;K:=BYTESTART[P];W:=P MOD 2;
WHILE(I<IDLOC)AND(BUFFER[I]=BYTEMEM[W,K])DO BEGIN I:=I+1;K:=K+1;END;
IF I=IDLOC THEN GOTO 31;END{:61};P:=LINK[P];END;P:=NAMEPTR;
LINK[P]:=HASH[H];HASH[H]:=P;31:{:60};IF P=NAMEPTR THEN{62:}
BEGIN W:=NAMEPTR MOD 2;
IF BYTEPTR[W]+L>MAXBYTES THEN BEGIN WRITELN(TERMOUT);
WRITE(TERMOUT,'! Sorry, ','byte memory',' capacity exceeded');ERROR;
HISTORY:=3;JUMPOUT;END;
IF NAMEPTR+2>MAXNAMES THEN BEGIN WRITELN(TERMOUT);
WRITE(TERMOUT,'! Sorry, ','name',' capacity exceeded');ERROR;HISTORY:=3;
JUMPOUT;END;I:=IDFIRST;K:=BYTEPTR[W];
WHILE I<IDLOC DO BEGIN BYTEMEM[W,K]:=BUFFER[I];K:=K+1;I:=I+1;END;
BYTEPTR[W]:=K;BYTESTART[NAMEPTR+2]:=K;NAMEPTR:=NAMEPTR+1;ILK[P]:=T;
XREF[P]:=0;END{:62};IDLOOKUP:=P;END;{:58}{66:}
FUNCTION MODLOOKUP(L:SIXTEENBITS):NAMEPOINTER;LABEL 31;VAR C:0..4;
     . . .
WRITE(TERMOUT,'(That was a fatal error, my friend.)');END;END{:263};
END.{:261}
?endgroup
\vfill\eject
\pageno=197 % take account of the page numbers for App's D and E.
\def\runninghead{APPENDIX F --- MACROS FOR FORMATTING}
\section Appendix F: The \.{webmac.tex} file.
This is the file that extends ``plain \TeX'' format in order to support the
features needed by the output of \.{WEAVE}.

\vskip6pt
\verbatim
% standard macros for WEB listings (in addition to PLAIN.TEX)
\xdef\fmtversion{\fmtversion+WEBMAC4.0} % identifies current set of macros
\parskip 0pt % no stretch between paragraphs
\parindent 1em % for paragraphs and for the first line of Pascal text

\font\eightrm=cmr8 \let\sc=\eightrm % NOT a caps-and-small-caps font!
\let\mainfont=\tenrm
\font\titlefont=cmr7 scaled\magstep4 % title on the contents page
\font\ttitlefont=cmtt10 scaled\magstep2 % typewriter type in title
\font\tentex=cmtex10 % TeX extended character set (used in strings)

\def\\#1{\hbox{\it#1\/\kern.05em}} % italic type for identifiers
\def\|#1{\hbox{$#1$}} % one-letter identifiers look a bit better this way
\def\&#1{\hbox{\bf#1\/}} % boldface type for reserved words
\def\.#1{\hbox{\tentex % typewriter type for strings
  \let\\=\BS % backslash in a string
  \let\'=\RQ % right quote in a string
  \let\`=\LQ % left quote in a string
  \let\{=\LB % left brace in a string
  \let\}=\RB % right brace in a string
  \let\~=\TL % tilde in a string
  \let\ =\SP % space in a string
  \let\_=\UL % underline in a string
  \let\&=\AM % ampersand in a string
  #1}}
\def\#{\hbox{\tt\char`\#}} % parameter sign
\def\${\hbox{\tt\char`\$}} % dollar sign
\def\%{\hbox{\tt\char`\%}} % percent sign
\def\^{\ifmmode\mathchar"222 \else\char`^ \fi} % pointer or hat
% circumflex accents can be obtained from \^^D instead of \^
\def\AT!{@} % at sign for control text

\chardef\AM=`\& % ampersand character in a string
\chardef\BS=`\\ % backslash in a string
\chardef\LB=`\{ % left brace in a string
\def\LQ{{\tt\char'22}} % left quote in a string
\chardef\RB=`\} % right brace in a string
\def\RQ{{\tt\char'23}} % right quote in a string
\def\SP{{\tt\char`\ }} % (visible) space in a string
\chardef\TL=`\~ % tilde in a string
\chardef\UL=`\_ % underline character in a string

\newbox\bak \setbox\bak=\hbox to -1em{} % backspace one em
\newbox\bakk\setbox\bakk=\hbox to -2em{} % backspace two ems

\newcount\ind % current indentation in ems
\def\1{\global\advance\ind by1\hangindent\ind em} % indent one more notch
\def\2{\global\advance\ind by-1} % indent one less notch
\def\3#1{\hfil\penalty#10\hfilneg} % optional break within a statement
\def\4{\copy\bak} % backspace one notch
\def\5{\hfil\penalty-1\hfilneg\kern2.5em\copy\bakk\ignorespaces}% optional break
\def\6{\ifmmode\else\par % forced break
  \hangindent\ind em\noindent\kern\ind em\copy\bakk\ignorespaces\fi}
\def\7{\Y\6} % forced break and a little extra space

\let\yskip=\smallskip
\def\to{\mathrel{.\,.}} % double dot, used only in math mode
\def\note#1#2.{\Y\noindent{\hangindent2em\baselineskip10pt\eightrm#1~#2.\par}}
\def\lapstar{\rlap{*}}
\def\startsection{\Q\noindent{\let\*=\lapstar\bf\modstar.\quad}}
\def\defin#1{\global\advance\ind by 2 \1\&{#1 }} % begin `define' or `format'
\def\A{\note{See also section}} % crossref for doubly defined section name
\def\As{\note{See also sections}} % crossref for multiply defined section name
\def\B{\mathopen{\.{@\{}}} % begin controlled comment
\def\C#1{\ifmmode\gdef\XX{\null$\null}\else\gdef\XX{}\fi % Pascal comments
  \XX\hfil\penalty-1\hfilneg\quad$\{\,$#1$\,\}$\XX}
\def\D{\defin{define}} % macro definition
\def\E{\cdot10^} % exponent in floating point constant
\def\ET{ and~} % conjunction between two section numbers
\def\ETs{, and~} % conjunction between the last two of several section numbers
\def\F{\defin{format}} % format definition
\let\G=\ge % greater than or equal sign
\def\H#1{\hbox{\rm\char"7D\tt#1}} % hexadecimal constant
\let\I=\ne % unequal sign
\def\J{\.{@\&}} % TANGLE's join operation
\let\K=\gets % left arrow
\let\L=\le % less than or equal sign
\outer\def\M#1.{\MN#1.\ifon\vfil\penalty-100\vfilneg % beginning of section
  \vskip12ptminus3pt\startsection\ignorespaces}
\outer\def\N#1.#2.{\MN#1.\vfil\eject % beginning of starred section
  \def\rhead{\uppercase{\ignorespaces#2}} % define running headline
  \message{*\modno} % progress report
  \edef\next{\write\cont{\Z{#2}{\modno}{\the\pageno}}}\next % to contents file
  \ifon\startsection{\bf\ignorespaces#2.\quad}\ignorespaces}
\def\MN#1.{\par % common code for \M, \N
  {\xdef\modstar{#1}\let\*=\empty\xdef\modno{#1}}
  \ifx\modno\modstar \onmaybe \else\ontrue \fi \mark{\modno}}
\def\O#1{\hbox{\rm\char'23\kern-.2em\it#1\/\kern.05em}} % octal constant
\def\P{\rightskip=0pt plus 100pt minus 10pt % go into Pascal mode
  \sfcode`;=3000
  \pretolerance 10000
  \hyphenpenalty 10000 \exhyphenpenalty 10000
  \global\ind=2 \1\ \unskip}
\def\Q{\rightskip=0pt % get out of Pascal mode
  \sfcode`;=1500 \pretolerance 200 \hyphenpenalty 50 \exhyphenpenalty 50 }
\let\R=\lnot % logical not
\let\S=\equiv % equivalence sign
\def\T{\mathclose{\.{@\}}}} % terminate controlled comment
\def\U{\note{This code is used in section}} % crossref for use of a section
\def\Us{\note{This code is used in sections}} % crossref for uses of a section
\let\V=\lor % logical or
\let\W=\land % logical and
\def\X#1:#2\X{\ifmmode\gdef\XX{\null$\null}\else\gdef\XX{}\fi % section name
  \XX$\langle\,$#2{\eightrm\kern.5em#1}$\,\rangle$\XX}
\def\Y{\par\yskip}
\let\Z=\let % now you can \send the control sequence \Z
\def\){\hbox{\.{@\$}}} % sign for string pool check sum
\def\]{\hbox{\.{@\\}}} % sign for forced line break
\def\=#1{\kern2pt\hbox{\vrule\vtop{\vbox{\hrule
        \hbox{\strut\kern2pt\.{#1}\kern2pt}}
      \hrule}\vrule}\kern2pt} % verbatim string
\let\~=\ignorespaces
\let\*=*

\def\onmaybe{\let\ifon=\maybe} \let\maybe=\iftrue
\newif\ifon \newif\iftitle \newif\ifpagesaved
\def\lheader{\mainfont\the\pageno\eightrm\qquad\rhead\hfill\title\qquad
  \tensy x\mainfont\topmark} % top line on left-hand pages
\def\rheader{\tensy x\mainfont\topmark\eightrm\qquad\title\hfill\rhead
  \qquad\mainfont\the\pageno} % top line on right-hand pages
\def\page{\box255 }
\def\normaloutput#1#2#3{\ifodd\pageno\hoffset=\pageshift\fi
  \shipout\vbox{
    \vbox to\fullpageheight{
      \iftitle\global\titlefalse
      \else\hbox to\pagewidth{\vbox to10pt{}\ifodd\pageno #3\else#2\fi}\fi
      \vfill#1}} % parameter #1 is the page itself
  \global\advance\pageno by1}

\def\rhead{\.{WEB} OUTPUT} % this running head is reset by starred sections
\def\title{} % an optional title can be set by the user
\def\topofcontents{\centerline{\titlefont\title}
  \vfill} % this material will start the table of contents page
\def\botofcontents{\vfill} % this material will end the table of contents page
\def\contentspagenumber{0} % default page number for table of contents
\newdimen\pagewidth \pagewidth=6.5in % the width of each page
\newdimen\pageheight \pageheight=8.7in % the height of each page
\newdimen\fullpageheight \fullpageheight=9in % page height including headlines
\newdimen\pageshift \pageshift=0in % shift righthand pages wrt lefthand ones
\def\magnify#1{\mag=#1\pagewidth=6.5truein\pageheight=8.7truein
  \fullpageheight=9truein\setpage}
\def\setpage{\hsize\pagewidth\vsize\pageheight} % use after changing page size
\def\contentsfile{CONTENTS} % file that gets table of contents info
\def\readcontents{\input CONTENTS}

\newwrite\cont
\output{\setbox0=\page % the first page is garbage
  \openout\cont=\contentsfile
  \global\output{\normaloutput\page\lheader\rheader}}
\setpage
\vbox to \vsize{} % the first \topmark won't be null

\def\ch{\note{The following sections were changed by the change file:}
  \let\*=\relax}
\newbox\sbox % saved box preceding the index
\newbox\lbox % lefthand column in the index
\def\inx{\par\vskip6pt plus 1fil % we are beginning the index
  \write\cont{} % ensure that the contents file isn't empty
  \closeout\cont % the contents information has been fully gathered
  \output{\ifpagesaved\normaloutput{\box\sbox}\lheader\rheader\fi
    \global\setbox\sbox=\page \global\pagesavedtrue}
  \pagesavedfalse \eject % eject the page-so-far and predecessors
  \setbox\sbox\vbox{\unvbox\sbox} % take it out of its box
  \vsize=\pageheight \advance\vsize by -\ht\sbox % the remaining height
  \hsize=.5\pagewidth \advance\hsize by -10pt
    % column width for the index (20pt between cols)
  \parfillskip 0pt plus .6\hsize % try to avoid almost empty lines
  \def\lr{L} % this tells whether the left or right column is next
  \output{\if L\lr\global\setbox\lbox=\page \gdef\lr{R}
    \else\normaloutput{\vbox to\pageheight{\box\sbox\vss
        \hbox to\pagewidth{\box\lbox\hfil\page}}}\lheader\rheader
    \global\vsize\pageheight\gdef\lr{L}\global\pagesavedfalse\fi}
  \message{Index:}
  \parskip 0pt plus .5pt
  \outer\def\:##1, {\par\hangindent2em\noindent##1:\kern1em} % index entry
  \let\ttentry=\. \def\.##1{\ttentry{##1\kern.2em}} % give \tt a little room
  \def\[##1]{$\underline{##1}$} % underlined index item
  \rm \rightskip0pt plus 2.5em \tolerance 10000 \let\*=\lapstar
  \hyphenpenalty 10000 \parindent0pt}
\def\fin{\par\vfill\eject % this is done when we are ending the index
  \ifpagesaved\null\vfill\eject\fi % output a null index column
  \if L\lr\else\null\vfill\eject\fi % finish the current page
  \parfillskip 0pt plus 1fil
  \def\rhead{NAMES OF THE SECTIONS}
  \message{Section names:}
  \output{\normaloutput\page\lheader\rheader}
  \setpage
  \def\note##1##2.{\hfil\penalty-1\hfilneg\quad{\eightrm##1 ##2.}}
  \linepenalty=10 % try to conserve lines
  \def\U{\note{Used in section}} % crossref for use of a section
  \def\Us{\note{Used in sections}} % crossref for uses of a section
  \def\:{\par\hangindent 2em}\let\*=*\let\.=\ttentry}
\def\con{\par\vfill\eject % finish the section names
  \rightskip 0pt \hyphenpenalty 50 \tolerance 200
  \setpage
  \output{\normaloutput\page\lheader\rheader}
  \titletrue % prepare to output the table of contents
  \pageno=\contentspagenumber \def\rhead{TABLE OF CONTENTS}
  \message{Table of contents:}
  \topofcontents
  \line{\hfil Section\hbox to3em{\hss Page}}
  \def\Z##1##2##3{\line{\ignorespaces##1
    \leaders\hbox to .5em{.\hfil}\hfil\ ##2\hbox to3em{\hss##3}}}
  \readcontents\relax % read the contents info
  \botofcontents \end} % print the contents page(s) and terminate
?endgroup
\vfill\eject
\def\runninghead{APPENDIX G --- NOTES ON FORMATTING}
\section Appendix G: How to use \.{WEB} macros.
The macros in \.{webmac} make it possible to produce a variety of formats
without editing the output of \.{WEAVE}, and the purpose of this appendix
is to explain some of the possibilities.

\def\point#1.{\yskip\indent#1.\quad\ignorespaces}

\point 1. Three fonts have been declared in addition to the standard fonts of
\.{PLAIN} format: You can say `\.{\{\\sc STUFF\}}' to get {\sc STUFF}
in small caps; and you can select the largish fonts \.{\\titlefont}
and \.{\\ttitlefont} in the title of your document, where \.{\\ttitlefont}
is a typewriter style of type.

\point 2. When you mention an identifier in \TeX\ text, you normally call
it `\.{|identifier|}'. But you can also say `\.{\\\\\{identifier\}}'. The
output will look the same in both cases, but the second alternative
doesn't put \\{identifier} into the index, since
it bypasses \.{WEAVE}'s translation from \PASCAL\ mode.

\point 3. To get typewriter-like type, as when referring to `\.{WEB}', you
can use the `\.{\\.}' macro (e.g., `\.{\\.\{WEB\}}'). In the argument to
this macro you should insert an additional backslash before the symbols
listed as `special string characters' in the index to \.{WEAVE}, i.e.,
before backslashes and dollar signs and the like.
A `\.{\\\ }' here will result in the visible space symbol; to get an
invisible space following a control sequence you can say `\.{\{\ \}}'.

\point 4. The three control sequences \.{\\pagewidth}, \.{\\pageheight},
and \.{\\fullpageheight} can be redefined in the limbo section at the
beginning of your \.{WEB} file, to change the dimensions of each page.
The standard settings
$$\lpile{\.{\\pagewidth=6.5in}\cr
  \.{\\pageheight=8.7in}\cr
  \.{\\fullpageheight=9in}\cr}$$
were used to prepare the present report; \.{\\fullpageheight} is
\.{\\pageheight} plus room for the additional heading and page numbers at
the top of each page. If you change any of these quantities, you should
call the macro \.{\\setpage} immediately after making the change.

\point 5. The \.{\\pageshift} macro defines an amount by which right-hand
pages (i.e., odd-numbered pages) are shifted right with respect to
left-hand (even-numbered) ones. By adjusting this amount you may be
able to get two-sided output in which the page numbers line up on
opposite sides of each sheet.

\point 6. The \.{\\title} macro will appear at the top of each page
in small caps. For example, Appendix~D was produced after saying
`\.{\\def\\title\{WEAVE\}}'.

\point 7. The first page usually is assigned page number 1.
To start on page 16, with contents
on page 15, say this: `\.{\\def\\contentspagenumber\{15\}}
\.{\\pageno=\\contentspagenumber} \.{\\advance\\pageno by 1}'. (Appendix~D
was generated that way.)

\point 8. The macro \.{\\iftitle} will suppress the header line if it is
defined by `\.{\\titletrue}'. The normal value is \.{\\titlefalse}
except for the table of contents; thus, the contents
page is usually unnumbered.

Two macros are provided to give flexibility to the table of
contents: \.{\\topofcontents} is invoked just before the contents
info is read, and \.{\\botofcontents} is invoked just after.
For example, Appendix~D was produced with the following definitions:
$$\lpile{\.{\\def\\topofcontents\{\\null\\vfill}\cr
  \.{ { }\\titlefalse \% include headline on the contents page}\cr
  \.{ { }\\def\\rheader\{\\mainfont Appendix D\\hfil 15\}}\cr
  \.{ { }\\centerline\{\\titlefont The \{\\ttitlefont WEAVE\}{ }processor\}}\cr
  \.{ { }\\vskip 15pt \\centerline\{(Version 4)\}{ }\\vfill\}}\cr}$$
Redefining \.{\\rheader}, which is the headline for right-hand pages,
suffices in this case to put the desired information at the top of the
contents page.

\point 9. Data for the table of contents is written to a file that
is read after the indexes have been \TeX ed; there's one line of data
for every starred module. For example, when Appendix~D was being generated,
a file \.{CONTENTS.TEX} containing
$$\lpile{\.{\\Z \{{ }Introduction\}\{1\}\{16\}}\cr
  \.{\\Z \{{ }The character set\}\{11\}\{19\}}\cr}$$
and similar lines was created. The \.{\\topofcontents} macro could
redefine \.{\\Z} so that the information appears in another format.

\point 10. Sometimes it is necessary or desirable to divide the output of
\.{WEAVE} into subfiles that can be processed separately. For example,
the listing of \TeX\ runs to more than 500 pages, and that is enough to
exceed the capacity of many printing devices and/or their software.
When an extremely large job isn't cut into smaller pieces, the entire
process might be spoiled by a single error of some sort, making it
necessary to start everything over.

Here's a safe way to break a woven file into three parts:
Say the pieces are $\alpha$,
$\beta$, and $\gamma$, where each piece begins with a starred module.
All macros should be defined in the opening limbo section of $\alpha$,
and copies of this \TeX\ code should be placed at the
beginning of $\beta$ and of $\gamma$. In order to process the parts
separately, we need to take care of two things: The starting page
numbers of $\beta$ and $\gamma$ need to be set up properly, and
the table of contents data from all three runs needs to be
accumulated.

The \.{webmac} macros include two control sequences \.{\\contentsfile} and
\.{\\readcontents} that facilitate the necessary processing.  We include
`\.{\\def\\contentsfile\{CONT1\}}' in the limbo section of $\alpha$, and
we include `\.{\\def\\contentsfile\{CONT2\}}' in the limbo section of
$\beta$; this causes \TeX\ to write the contents data for $\alpha$ and $\beta$
into \.{CONT1.TEX} and \.{CONT2.TEX}. Now in $\gamma$ we say
$$\.{\\def\\readcontents\{\\input CONT1 \\input CONT2 \\input CONTENTS\}};$$
this brings in the data from all three pieces, in the proper order.

However, we still need to solve the page-numbering problem. One way to
do it is to include the following in the limbo material for $\beta$:
$$\lpile{\.{\\message\{Please type the last page number of part 1: \}}\cr
  \.{\\read -1 to \\temp \\pageno=\\temp \\advance\\pageno by 1}\cr}$$
Then you simply provide the necessary data when \TeX\ requests
it; a similar construction is used at the beginning of $\gamma$.

This method can, of course, be used to divide a woven file into
any number of pieces.

\point 11. Sometimes it is nice to include things in the index that are
typeset in a special way. For example, we might want to have an
index entry for `\TeX'. \.{WEAVE} provides two simple ways to
typeset an index entry (unless the entry is an identifier or a reserved word):
`\.{@\^}' gives roman type, and `\.{@.}' gives typewriter type.
But if we try to typeset `\TeX' in roman type by saying, e.g.,
`\.{@\^\\TeX@>}', the backslash character gets in the way,
and this entry wouldn't appear in the index with the T's.

The solution is to use the `\.{@:}' feature, declaring a macro that
simply removes a sort key as follows:
$$\.{\\def\\9\#1\{\}}$$
Now you can say, e.g., `\.{@:TeX\}\{\\TeX@>}' in your \.{WEB} file; \.{WEAVE}
puts it into the index alphabetically, based on the sort key, and
produces the macro call `\.{\\9\{TeX\}\{\\TeX\}}' which will ensure that
the sort key isn't printed.

A similar idea can be used to insert hidden material into module
names so that they are alphabetized in whatever way you might wish.
Some people call these tricks ``special refinements''; others call
them ``kludges''.

\point 12. The control sequence \.{\\modno} is set to the number of the
module being typeset.

\point 13. If you want to list only the modules that have changed,
together with the index, put the command `\.{\\let\\maybe=\\iffalse}' in
the limbo section before the first module of your \.{WEB} file. It's
customary to make this the first change in your change file.

\point 14. To get output in languages other than English, redefine the
macros \.{\\A}, \.{\\As}, \.{\\ET}, \.{\\ETs}, \.{\\U}, \.{\\Us},
\.{\\ch}, \.{\\fin}, and \.{\\con}. \.{WEAVE} itself need not be changed.

\vfill\eject
\def\runninghead{APPENDIX H --- GETTING STARTED}
\section Appendix H: Installing the \.{WEB} system.
Suppose you want to use the \.{WEB} programs on your computer, and suppose
that you can't simply borrow them from somebody else who has the same
kind of machine. Here's what to do:

\yskip
\def\step(#1){\par\hangindent 2em\noindent\hbox to 2em{\hfil(#1) }\ignorespaces}
\step(1) Get a tape that contains the files \.{WEAVE.WEB}, \.{TANGLE.WEB},
\.{TANGLE.PAS}, and \.{WEBMAC.TEX}. The tape will probably also contain an
example change file \.{TANGLE.CH}.
\step(2) Look at the sections of \.{TANGLE} that are listed under ``system
dependencies'' in the index of Appendix~E above, and figure out what changes
(if any) will be needed for your system.
\step(3) Make a change file \.{TANGLE.CH} that contains the changes of~(2);
do not change your copy of \.{TANGLE.WEB}, leave it intact. (The
rules for change files are explained at the end of the manual just before
the appendices; you may want to look at the example change file that
arrived with your copy of \.{TANGLE.WEB}. It's also a good idea to
define all the ``switches'' like \&{debug} and \&{gubed} to be null in your
first change files; then you can sure that your compiler will handle
all of the code.)
\step(4) Make the changes of (2) in your copy of \.{TANGLE.PAS}. (If these
changes are extensive, you might be better off finding some computer that
that already has \.{TANGLE} running, and making the new \.{TANGLE.PAS}
from \.{TANGLE.WEB} and your \.{TANGLE.CH}.)
\step(5) Use your \PASCAL\ compiler to convert your copy of \.{TANGLE.PAS}
to a running program \.{TANGLE}.
\step(6) Check your changes as follows: Run \.{TANGLE} on \.{TANGLE.WEB}
and your \.{TANGLE.CH}, yielding $\.{TANGLE.PAS}'$; make a running
program $\.{TANGLE}'$ by applying \PASCAL\ to
$\.{TANGLE.PAS}'$; run $\.{TANGLE}'$ on \.{TANGLE.WEB} and
your \.{TANGLE.CH}, yielding $\.{TANGLE.PAS}''$; and check
that $\.{TANGLE.PAS}''$ is identical to
$\.{TANGLE.PAS}'$. Once this test has been passed, you have got a
working \.{TANGLE} program.
\step(7) Make a change file \.{WEAVE.CH} analogous to (3), but this time
consider the system-dependent parts of \.{WEAVE} that are listed in
the index to Appendix~D.
\step(8) Run \.{TANGLE} on \.{WEAVE.WEB} and your \.{WEAVE.CH}, obtaining
\.{WEAVE.PAS}.
\step(9) Use \PASCAL\ on \.{WEAVE.PAS} to make a running \.{WEAVE} program.
\step(10) Run \.{WEAVE} on \.{TANGLE.WEB} and \.{TANGLE.CH} to produce
\.{TANGLE.TEX}.
\step(11) Run \TeX\ on \.{TANGLE.TEX}, obtaining a listing analogous to
Appendix~E. This listing will incorporate your changes.
\step(12) Run \.{WEAVE} on \.{WEAVE.WEB} and your \.{WEAVE.CH} to produce
\.{WEAVE.TEX}.
\step(13) Run \TeX\ on \.{WEAVE.TEX}, obtaining a listing analogous to
Appendix~D that incorporates your changes.

\yskip\noindent
This description assumes that you already have a working \TeX82 system.
But what if you don't have \TeX82? Then you start with a tape that also
contains \.{TEX.WEB} and \.{plain.tex}, and you refer to a hardcopy
listing of the \TeX82 program corresponding to \.{TEX.WEB}. Between steps
(10) and (11) you do the following:

\yskip
\def\substep(10.#1){\par\hangindent 4em\noindent
  \hbox to 4em{\hfil(10.#1) }\ignorespaces}
\substep(10.1) Make a change file \.{TEX.CH} to fix the system dependent
portions of \.{TEX.WEB}, in a manner analogous to step~(2). Since \TeX\ is
a much more complex program than \.{WEAVE} or \.{TANGLE}, there are more
system-dependent features to think about, but by now you will be good at
making such modifications. Do not make any changes to \.{TEX.WEB}.
\substep(10.2) Make an almost-copy of your \.{TEX.CH} called \.{INITEX.CH};
this one will have the `\&{init}' and `\&{tini}' macros redefined in order
to make the initialization version of \TeX. It also might have smaller
font memory and dynamic memory areas, since \.{INITEX} doesn't need as
much memory for such things; by setting the memory smaller in \.{INITEX},
you guarantee that the production system will have a ``cushion.''
\substep(10.3) Run \.{TANGLE} on \.{TEX.WEB} and \.{INITEX.CH}, obtaining
\.{INITEX.PAS} and \.{TEX.POOL}.
\substep(10.4) Run \PASCAL\ on \.{INITEX.PAS}, obtaining \.{INITEX}.
\substep(10.5) Run \.{INITEX} on \.{TEX.POOL}, during which run you type
`\.{plain}' and `\.{\\dump}'. This will produce a file \.{plain.fmt}
containing the data needed to initialize \TeX's memory.
\substep(10.6) Run \.{TANGLE} on \.{TEX.WEB} and the \.{TEX.CH} of (10.1),
obtaining \.{TEX.PAS}.
\substep(10.7) Run \PASCAL\ on \.{TEX.PAS}, obtaining \.{VIRTEX}.
\substep(10.8) If your operating system supports programs whose core images
have been saved, run \.{VIRTEX}, type `\.{\&plain}', then save the core image
and call it \TeX. Otherwise, \.{VIRTEX} will be your \TeX, and it will
read `\.{plain.fmt}' (or some other \.{fmt} file) each time you run.

\yskip
This 21-step process may seem long, but it is actually an oversimplification,
since you also need fonts and a way to print the device-independent files
that \TeX\ spews out. On the other hand, the total number of steps is not
quite so large when you consider that \.{TANGLE}-followed-by-\PASCAL\ and
\.{WEAVE}-followed-by-\TeX\ may be regarded as single operations.

If you have only the present report, not a tape, you will have to prepare
files \.{WEAVE.WEB} and \.{TANGLE.WEB} by hand, typing them into the
computer by following Appendices D and E. Then you have to simulate the
behavior of \.{TANGLE} by converting \.{TANGLE.WEB} manually into
\.{TANGLE.PAS}; with a good text editor this takes about six hours. Then
you have to correct errors that were made in all this hand work; but still
the whole project is not impossibly difficult, because in fact the entire
development of \.{WEAVE} and \.{TANGLE} (including the writing of the
programs and this manual) took less than two months of work.
\vfill\end
