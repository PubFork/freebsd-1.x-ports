\Section{Running \TeX\ under \Unix}
\TeX82 is written in |PASCAL| and the user interface is
nearly identical on the various operating systems to which
it has been ported.  The \TeX\ source code is placed
in a file with a name like `{\it filename\/\tt.tex}'.
The input should end with a |\bye| macro, which automatically
fills out the last page with |\vfil| glue, outputs it,
and terminates \TeX.
Assuming
a Bourne shell, the command to invoke \TeX\ will be
$${\hbox{\tt\$ tex \it filename}}$$
Assuming that  |tex| is found, \TeX\ will print
something like
\begintt
This is TeX, Version 1.0 for Berkeley UNIX (preloaded format=plain 83.12.9)
(filename.tex [1])
Output written to filename.dvi (1 page, 2092 bytes)
Transcript written on filename.log.
$
\endtt
\TeX\ lists the files as it reads them, and displays the numbers
of the pages as they are produced (in brackets).  It writes
error messages and warnings,
and some occasionally cryptic information,
in {\it filename\/\tt.log} and
produces a device-independent (`dvi') representation
of the result in {\it filename\/\tt.dvi}.  This file can be
sent to  your typesetting device by giving it as input to
the appropriate device driver program.  Consult your local \TeX\
guru for information on the device drivers available at your
site.

If \TeX\ detects errors in the input, and it has a terminal input
device, it will attempt to interact with the user and allow
him or her to correct the error.  The help facilities built
into \TeX\ allow the user to receive an explanation of the
error, and provide some guidance as to possible remedies.
Type a question-mark and hit the |return| key if you're in doubt.
One
can, as an alternative, add the command |\batchmode| to the input file to
suppress this interaction. In this case, \TeX\ can be run
in background mode (redirecting output to |/dev/null|); the
output for errors and warnings will still be placed in the
appropriate |.log| file for later examination.
