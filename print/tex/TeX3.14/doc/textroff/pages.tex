\Section{Page Composition}
Although plain \TeX\ is not as complete a document composition
system as a typical \Troff/ macro package, it does provide
several convenient features for composing technical documents.
\SubSection{Section Headings}
Plain \TeX\ does not provide the sophisticated numbered
headings (with several levels of section number) that
are found in the |-mm| and |-me| macro packages for
\Troff/, or the |Article| document type for |Scribe|.
However, there is a |\beginsection| macro that
can be used to set
a bold heading for a paragraph.  For example:
\begintt
\beginsection 1. An Example
|medskip
This paragraph isn't indented, even
though subsequent paragraphs would be.  A more
extensive example of this appears in Appendix~B
of ``{\sl The \TeX book}.''
\endtt
produces the following result:
\bigskip
\leftline{\bf 1. An Example}\nobreak\smallskip\vskip-\parskip\noindent
This paragraph isn't indented, even
though subsequent paragraphs would be.  A more
extensive example of this appears in Appendix~B
of ``{\sl The \TeX book}.''

Plain \TeX\ also includes
a |\proclaim| macro that can be used to
present theorems or similarly assertive statements.
\begintt
\proclaim Rule 225(a).  All proclamations must end their
labels with periods, and will appear in slanted type.
\endtt
\proclaim Rule 225(a).  All proclamations must end their
labels with periods, and will appear in slanted type.

The text of the proclamation comprises an entire paragraph, and
thus ends at the next empty line (or |\par|).
\SubSection{Page Headers and Footers}
Plain \TeX\ normally produces pages with no header lines and
a page number centered at the bottom.  If something else is wanted,
macros are provided to change them.  |\footline| defines
the footer line.  For example, a paper might set
up a footer line
with the command:
\nobreak
\begintt
\footline={\tenrm Rogue for Pacifists\hfil\folio}
\endtt
to produce a footer line like:
\nobreak\vskip 12pt
\line{\tenrm Rogue for Pacifists\hfil\folio}
\par\noindent
As with \Troff/, it is a good idea to explicitly set the
font to ten-point roman type.  Otherwise, the footer
will be set in whatever font happens to be in effect
when the page break occurs.
The alert reader will have noted the use of |\folio| for
the page number.  This useful (if somewhat
cryptically named) macro produces the Arabic
representation of the page number if it is positive, and
the Roman numeral representation of the number if it is
negative.  The page number begins at 1 and is incremented
after every page (or decremented if it is negative).  If
you want to adjust it yourself, you can issue the command
|\pageno=|$n$, where $n$ is the (possibly negative) new page number.

The header line can be set up the same way, using
|\headline=|.  In fact, the
document {\sl Using Nroff and --ME\/}, by Eric Allman, uses
a header line exactly analogous to the footer line
of the previous example.

Three-part headers and footers are trickier,
because just putting |\hfil| glue between the parts will center
the middle section between the left and right sections, rather
than centering it on the page.  The control sequences
|\rlap| and |\llap|
can be used to good effect here, since their results behave as if they
are of zero width.  Think of this as ``allowing the |\hfil|
springs to push against the sides of the page,'' and
you get the idea.  For example, the footer line of this page was
set with the input
\begintt
\footline= {
   \tenrm                % Ten-point roman
   \rlap{\TeX\ For {\it Troff\/} Users} % Takes no space, types to right
   \hfil\folio\hfil      % Centered Page number
   \llap{Sample}         % Also no space, types to left
}
\endtt
\footline= {
   \tenrm                % Ten-point roman
   \rlap{\TeX\ For {\it Troff\/} Users} % Takes no space
   \hfil\folio\hfil % Centered Page number
   \llap{Sample}         % Also no space
   \global\footline={\hss\tenrm\folio\hss}% Sneaky
}
\SubSection{Footnotes and Figures}
Most documents require some kind of ``movable'' text,
i.e. text that is to be set somewhere other than where it appears
in the input.  In general, \TeX\ calls such items {\sl inserts} and
provides powerful, if complex, primitives for the effective manipulation
of such material.  Fortunately, plain \TeX\ provides fairly handy
macros for doing some of the more familiar types of inserts.

The most common insert is a footnote, which consists of text
to go at the bottom of a page.  The plain \TeX\ |\footnote| macro
can be used within a paragraph\footnote{*}{Like this.}; for example,
the footnote in the present sentence was typed in the following
way:
\nobreak
\begintt
... within a paragraph\footnote{*}{Like this.}; for example, ...
\endtt
There are two parameters to |\footnote|: the reference mark (which
will appear in the footnote and in the paragraph), and the text of
the footnote.  If necessary, long footnotes are automatically
split between pages,
and footnotes normally appear in normal sized type as in this
example\footnote{**}{This paper uses a
private macro (not part of plain \TeX) to automatically
number footnotes and display them in eight-point type; this
footnote was produced with plain \TeX.}.  Note that
footnotes {\sl must} appear in
the context of a paragraph, or anything else ``horizontal''
that contributes to the page, e.g. in a |\centerline|.    If
you insist on using a footnote in a more exotic
context (e.g. the caption of a ``floating keep''), the esoteric
macro |\vfootnote| must be used instead of |\footnote|.

The second kind of insert that many people use is exemplified by
the ``floating keep'' construct of the |-me| macro package.  In this
case, an illustration or figure is to be kept together and set
onto the page at the earliest convenient time.  There are three
kinds of floating inserts in plain \TeX.  The first, |\topinsert|,
places its material, which ends with |\endinsert|,
at the top of the current page, if possible,
otherwise the material will be placed at the top of the next page.
For example:
\nobreak
\begintt
\topinsert
   \centerline{\vbox{
      \hbox{$\spadesuit$ A K 8 4}
      \hbox{$\heartsuit$ K 3 2}
      \hbox{$\diamondsuit$ 5}
      \hbox{$\clubsuit$ A K J 9 6}
   }}
\vskip 12pt
\centerline{{\bf Figure 1.} Example of a figure}
\endinsert
\endtt
|\topinsert| automatically adds some extra space to separate
the caption from the following text.
%\topinsert\vskip 3in
\topinsert
   \centerline{\vbox{
      \hbox{$\spadesuit$ A K 8 4}
      \hbox{$\heartsuit$ K 3 2}
      \hbox{$\diamondsuit$ 5}
      \hbox{$\clubsuit$ A K J 9 6}
   }}
\vskip 12pt
\centerline{{\bf Figure 1.} Example of a figure}
\endinsert

The second floating insert, |\midinsert| is much more like the |.(z|
construct of the |-me| package; it tries to insert the material in place
in the middle of the current page.  If it succeeds, the material
is placed in a vertical box with about 12 points of space above and below.
If this can't be done, the effect is that of a |\topinsert|.

Finally, the macro |\pageinsert| justifies its material to the size
of a full page and places the result on the following page.  For
example, the following produces a familiar result:
\nobreak
\begintt
\pageinsert
   \vfil % vertical centering 
   \centerline{This page intentionally left blank}
   \vfil % vertical centering
\endinsert
\endtt
\pageinsert
   \vfil % vertical centering
   \centerline{This page intentionally left blank}
   \vfil % vertical centering
\endinsert
\SubSection{Other Keeps}
Plain \TeX\ doesn't provide all the macros that one might want
to use for keeps, displays, and the like.  However, it's fairly
easy to tell \TeX\ to try to avoid breaking up a group.  Simply
set the cryptic parameter |\interlinepenalty=150| within that
group, and this will make line breaks within the group into very poor
candidates for page breaking.  Another strategy is to use
the control word |\filbreak|.  |\filbreak| causes the remainder
of the current page to be filled with vertical glue, {\bf unless}
it can be filled with additional text that is itself followed
by |\filbreak|.  If every paragraph had |\filbreak| after it, 
every page of a document would have only complete paragraphs on the
page.

Similarly, while there is no specific ``display'' macro
to correspond to the |.(l| macro of |-me| or
the |.DS| macro of |-mm|, one
can specify |\obeylines| within a group
to cause each newline to begin a new
paragraph.  One might also wish to set |\parindent|
to an appropriate value for the lines within the group.  You
can even have an entire group of centered lines by specifying |\obeylines|,
setting |\parindent| to zero, and |\leftskip| and |\rightskip|
(i.e. the left and right margins) to
be infinitely extensible glue\note{i.e. {\tt \\leftskip=0pt
plus 1fil} and {\tt \\rightskip=0pt plus 1fil}}.  There
may be some prepackaged
macro facilities available on your system for doing these sorts of
things.
