%What is TeX and METAfont all about?    C.G van der Laan, cgl@risc1.rug.nl
\documentstyle[bezier]{article} %Version 1.1  Jan  94
\def\Dash{---}
\def\dash{--}
\def\address#1{#1}\def\netaddress#1{}\def\network#1{}

%Needed files: abr.tex, btable.tex, ds.pic, icon.tex, lit.dat, lit.sel,
%              lus.pic,  math.tex, pic.pic            (tugboat.sty/cmn)
%In final version all these files have been included, however.
%\input{abr.tex}   %Abbreviations from TUGboat.cmn
%\input{icon.tex}  %Icon macros
%% Macros for The TeXbook

\catcode`@=11 % borrow the private macros of PLAIN (with care)

\font\tentex=cmtex10

\font\inchhigh=cminch
\font\titlefont=cmssdc10 at 40pt

\font\ninerm=cmr9
\font\eightrm=cmr8
\font\sixrm=cmr6

\font\ninei=cmmi9
\font\eighti=cmmi8
\font\sixi=cmmi6
\skewchar\ninei='177 \skewchar\eighti='177 \skewchar\sixi='177

\font\ninesy=cmsy9
\font\eightsy=cmsy8
\font\sixsy=cmsy6
\skewchar\ninesy='60 \skewchar\eightsy='60 \skewchar\sixsy='60

\font\eightss=cmssq8

\font\eightssi=cmssqi8

\font\ninebf=cmbx9
\font\eightbf=cmbx8
\font\sixbf=cmbx6

\font\ninett=cmtt9
\font\eighttt=cmtt8

\hyphenchar\tentt=-1 % inhibit hyphenation in typewriter type
\hyphenchar\ninett=-1
\hyphenchar\eighttt=-1

\font\ninesl=cmsl9
\font\eightsl=cmsl8

\font\nineit=cmti9
\font\eightit=cmti8

\font\tenu=cmu10 % unslanted text italic
\font\magnifiedfiverm=cmr5 at 10pt
\font\manual=manfnt % font used for the METAFONT logo, etc.
\font\cmman=cmman % font used for miscellaneous Computer Modern variations

\newskip\ttglue
\def\tenpoint{\def\rm{\fam0\tenrm}%
  \textfont0=\tenrm \scriptfont0=\sevenrm \scriptscriptfont0=\fiverm
  \textfont1=\teni \scriptfont1=\seveni \scriptscriptfont1=\fivei
  \textfont2=\tensy \scriptfont2=\sevensy \scriptscriptfont2=\fivesy
  \textfont3=\tenex \scriptfont3=\tenex \scriptscriptfont3=\tenex
  \def\it{\fam\itfam\tenit}%
  \textfont\itfam=\tenit
  \def\sl{\fam\slfam\tensl}%
  \textfont\slfam=\tensl
  \def\bf{\fam\bffam\tenbf}%
  \textfont\bffam=\tenbf \scriptfont\bffam=\sevenbf
   \scriptscriptfont\bffam=\fivebf
  \def\tt{\fam\ttfam\tentt}%
  \textfont\ttfam=\tentt
  \tt \ttglue=.5em plus.25em minus.15em
  \normalbaselineskip=12pt
  \def\MF{{\manual META}\-{\manual FONT}}%
  \let\sc=\eightrm
  \let\big=\tenbig
  \setbox\strutbox=\hbox{\vrule height8.5pt depth3.5pt width\z@}%
  \normalbaselines\rm}

\def\ninepoint{\def\rm{\fam0\ninerm}%
  \textfont0=\ninerm \scriptfont0=\sixrm \scriptscriptfont0=\fiverm
  \textfont1=\ninei \scriptfont1=\sixi \scriptscriptfont1=\fivei
  \textfont2=\ninesy \scriptfont2=\sixsy \scriptscriptfont2=\fivesy
  \textfont3=\tenex \scriptfont3=\tenex \scriptscriptfont3=\tenex
  \def\it{\fam\itfam\nineit}%
  \textfont\itfam=\nineit
  \def\sl{\fam\slfam\ninesl}%
  \textfont\slfam=\ninesl
  \def\bf{\fam\bffam\ninebf}%
  \textfont\bffam=\ninebf \scriptfont\bffam=\sixbf
   \scriptscriptfont\bffam=\fivebf
  \def\tt{\fam\ttfam\ninett}%
  \textfont\ttfam=\ninett
  \tt \ttglue=.5em plus.25em minus.15em
  \normalbaselineskip=11pt
  \def\MF{{\manual hijk}\-{\manual lmnj}}%
  \let\sc=\sevenrm
  \let\big=\ninebig
  \setbox\strutbox=\hbox{\vrule height8pt depth3pt width\z@}%
  \normalbaselines\rm}

\def\eightpoint{\def\rm{\fam0\eightrm}%
  \textfont0=\eightrm \scriptfont0=\sixrm \scriptscriptfont0=\fiverm
  \textfont1=\eighti \scriptfont1=\sixi \scriptscriptfont1=\fivei
  \textfont2=\eightsy \scriptfont2=\sixsy \scriptscriptfont2=\fivesy
  \textfont3=\tenex \scriptfont3=\tenex \scriptscriptfont3=\tenex
  \def\it{\fam\itfam\eightit}%
  \textfont\itfam=\eightit
  \def\sl{\fam\slfam\eightsl}%
  \textfont\slfam=\eightsl
  \def\bf{\fam\bffam\eightbf}%
  \textfont\bffam=\eightbf \scriptfont\bffam=\sixbf
   \scriptscriptfont\bffam=\fivebf
  \def\tt{\fam\ttfam\eighttt}%
  \textfont\ttfam=\eighttt
  \tt \ttglue=.5em plus.25em minus.15em
  \normalbaselineskip=9pt
  \def\MF{{\manual opqr}\-{\manual stuq}}%
  \let\sc=\sixrm
  \let\big=\eightbig
  \setbox\strutbox=\hbox{\vrule height7pt depth2pt width\z@}%
  \normalbaselines\rm}

\def\tenmath{\tenpoint\fam-1 } % use after $ in ninepoint sections
\def\tenbig#1{{\hbox{$\left#1\vbox to8.5pt{}\right.\n@space$}}}
\def\ninebig#1{{\hbox{$\textfont0=\tenrm\textfont2=\tensy
  \left#1\vbox to7.25pt{}\right.\n@space$}}}
\def\eightbig#1{{\hbox{$\textfont0=\ninerm\textfont2=\ninesy
  \left#1\vbox to6.5pt{}\right.\n@space$}}}

% Page layout
\newdimen\pagewidth \newdimen\pageheight \newdimen\ruleht
\hsize=29pc  \vsize=44pc  \maxdepth=2.2pt  \parindent=3pc
\pagewidth=\hsize \pageheight=\vsize \ruleht=.5pt
\abovedisplayskip=6pt plus 3pt minus 1pt
\belowdisplayskip=6pt plus 3pt minus 1pt
\abovedisplayshortskip=0pt plus 3pt
\belowdisplayshortskip=4pt plus 3pt

%\newinsert\footins
\def\footnote#1{\edef\@sf{\spacefactor\the\spacefactor}#1\@sf
      \insert\footins\bgroup\eightpoint
      \interlinepenalty100 \let\par=\endgraf
        \leftskip=\z@skip \rightskip=\z@skip
        \splittopskip=10pt plus 1pt minus 1pt \floatingpenalty=20000
        \smallskip\item{#1}\bgroup\strut\aftergroup\@foot\let\next}
\skip\footins=12pt plus 2pt minus 4pt % space added when footnote is present
%\count\footins=1000 % footnote magnification factor (1 to 1)
\dimen\footins=30pc % maximum footnotes per page

\newinsert\margin
\dimen\margin=\maxdimen
%\count\margin=0 \skip\margin=0pt % marginal inserts take up no space

\newif\iftitle
\def\titlepage{\global\titletrue} % for pages without headlines
\def\rhead{} % \rhead contains the running headline

\def\leftheadline{\hbox to \pagewidth{%
    \vbox to 10pt{}% strut to position the baseline
    \llap{\tenbf\folio\kern1pc}% folio to left of text
    \tenit\rhead\hfil% running head flush left
    }}
\def\rightheadline{\hbox to \pagewidth{%
    \vbox to 10pt{}% strut to position the baseline
    \hfil\tenit\rhead\/% running head flush right
    \rlap{\kern1pc\tenbf\folio}% folio to right of text
    }}

\def\onepageout#1{\shipout\vbox{ % here we define one page of output
    \offinterlineskip % butt the boxes together
    \vbox to 3pc{ % this part goes on top of the 44pc pages
      \iftitle % the next is used for title pages
        \global\titlefalse % reset the titlepage switch
        \setcornerrules % for camera alignment
      \else\ifodd\pageno \rightheadline\else\leftheadline\fi\fi
      \vfill} % this completes the \vbox to 3pc
    \vbox to \pageheight{
      \ifvoid\margin\else % marginal info is present
        \rlap{\kern31pc\vbox to\z@{\kern4pt\box\margin \vss}}\fi
      #1 % now insert the main information
      \ifvoid\footins\else % footnote info is present
        \vskip\skip\footins \kern-3pt
        \hrule height\ruleht width\pagewidth \kern-\ruleht \kern3pt
        \unvbox\footins\fi
      \boxmaxdepth=\maxdepth
      } % this completes the \vbox to \pageheight
    }
  \advancepageno}

\def\setcornerrules{\hbox to \pagewidth{\vrule width 1pc height\ruleht
    \hfil \vrule width 1pc}
  \hbox to \pagewidth{\llap{\sevenrm(page \folio)\kern1pc}%
    \vrule height1pc width\ruleht depth\z@
    \hfil \vrule width\ruleht depth\z@}}

\output{\onepageout{\unvbox255}}

\newbox\partialpage
\def\begindoublecolumns{\begingroup
  \output={\global\setbox\partialpage=\vbox{\unvbox255\bigskip}}\eject
  \output={\doublecolumnout} \hsize=14pc \vsize=89pc}
\def\enddoublecolumns{\output={\balancecolumns}\eject
  \endgroup \pagegoal=\vsize}

\def\doublecolumnout{\splittopskip=\topskip \splitmaxdepth=\maxdepth
  \dimen@=44pc \advance\dimen@ by-\ht\partialpage
  \setbox0=\vsplit255 to\dimen@ \setbox2=\vsplit255 to\dimen@
  \onepageout\pagesofar
  \unvbox255 \penalty\outputpenalty}
\def\pagesofar{\unvbox\partialpage
  \wd0=\hsize \wd2=\hsize \hbox to\pagewidth{\box0\hfil\box2}}
\def\balancecolumns{\setbox0=\vbox{\unvbox255} \dimen@=\ht0
  \advance\dimen@ by\topskip \advance\dimen@ by-\baselineskip
  \divide\dimen@ by2 \splittopskip=\topskip
  {\vbadness=10000 \loop \global\setbox3=\copy0
    \global\setbox1=\vsplit3 to\dimen@
    \ifdim\ht3>\dimen@ \global\advance\dimen@ by1pt \repeat}
  \setbox0=\vbox to\dimen@{\unvbox1}
  \setbox2=\vbox to\dimen@{\unvbox3}
  \pagesofar}

% To produce only a subset of pages, put the page numbers on separate
% lines in a file called pages.tex
\let\Shipout=\shipout
\newread\pages \newcount\nextpage \openin\pages=pages
\def\getnextpage{\ifeof\pages\else
 {\endlinechar=-1\read\pages to\next
  \ifx\next\empty % in this case we should have eof now
  \else\global\nextpage=\next\fi}\fi}
\ifeof\pages\else\message{OK, I'll ship only the requested pages!}
 \getnextpage\fi
\def\shipout{\ifeof\pages\let\next=\Shipout
 \else\ifnum\pageno=\nextpage\getnextpage\let\next=\Shipout
  \else\let\next=\Tosspage\fi\fi \next}
\newbox\garbage \def\Tosspage{\deadcycles=0\setbox\garbage=}

% Chapter formatting
% The preface and table of contents are formatted in place, not here

\newcount\exno % for the number of exercises in the current chapter
\newcount\subsecno % for the number of subsections in the current chapter

\def\beginchapter#1 #2#3. #4\par{\global\exno=0
  \subsecno=0
  \def\chapno{#2#3}
  \ifodd\pageno
    \errmessage{You had too much text on that last page; I'm backing up}
    \advance\pageno by-1 \fi
  \titlepage
  \def\\{ } % \\'s in the title will be treated as spaces
  \message{#1 #2#3:} % show the chapter title on the terminal
  \def\MF{{\manual 89:;<=>:}} % slant the logo
  \xdef\rhead{#1 #2#3: #4\unskip}
  {\def\TeX{T\kern-.2em\lower.5ex\hbox{E}\kern-.06em X}
    \def\MF{{\vbox to30pt{}\manual ()*+,-.*}}
    \def\\{#3}
    \ifx\empty\\ \rightline{\inchhigh #2\kern-.04em}
    \else\rightline{\inchhigh #2\kern-.06em#3\kern-.04em}\fi
    \vskip 1.75pc
    \baselineskip 36pt \lineskiplimit \titlelsl \lineskip 12pt
    \let\\=\cr % now the \\'s are line dividers
    \halign{\line{\titlefont\hfil##}\\#4\unskip\\}
    \vfill\eject} % output the chapter title page
  \tenpoint
  \noindent\ignorespaces} % the first paragraph of a chapter is not indented
\newdimen\titlelsl \titlelsl=1pt

\outer\def\endchapter{\ifodd\pageno \else\vfill\eject\null\fi
  \begingroup\bigskip\vfill % beginning of the quotes
  \def\eject{\endgroup\eject}
  \def\par{\ifhmode\/\endgraf\fi}\obeylines
  \def\TeX{T\kern-.2em\lower.5ex\hbox{E}\kern-.000em X}
  \def\MF{{\manual opqr}\-{\manual stuq}}
  \eightpoint \let\tt=\ninett
  \baselineskip 10pt
  \parfillskip \z@
  \interlinepenalty 10000
  \leftskip \z@ plus 40pc minus \parindent
  \let\rm=\eightss \let\sl=\eightssi
  \everypar{\sl}}
\def\author#1(#2){\smallskip\noindent\rm--- #1\unskip\enspace(#2)}

\def\dbend{{\manual\char127}} % dangerous bend sign
\def\d@nger{\medbreak\begingroup\clubpenalty=10000
  \def\par{\endgraf\endgroup\medbreak} \noindent\hang\hangafter=-2
  \hbox to0pt{\hskip-\hangindent\dbend\hfill}\ninepoint}
\outer\def\danger{\d@nger}
\def\dd@nger{\medbreak\begingroup\clubpenalty=10000
  \def\par{\endgraf\endgroup\medbreak} \noindent\hang\hangafter=-2
  \hbox to0pt{\hskip-\hangindent\dbend\kern1pt\dbend\hfill}\ninepoint}
\outer\def\ddanger{\dd@nger}
\def\enddanger{\endgraf\endgroup} % omits the \medbreak

\outer\def\subsection#1. {\medbreak\advance\subsecno by 1
  \noindent{\it \the\subsecno.\enspace#1.\enspace}}
\def\ansno#1.#2:{\medbreak\noindent
  \hbox to\parindent{\bf\hss#1.#2.\enspace}\ignorespaces}

% Composition macros
\hyphenation{man-u-script man-u-scripts ap-pen-dix xscaled}

\def\AmSTeX{$\cal A\kern-.1667em\lower.5ex\hbox{$\cal M$}\kern-.075em
  S$-\TeX}
\def\bull{\vrule height .9ex width .8ex depth -.1ex } % square bullet
\def\SS{{\it SS}} % scriptscript style
\def\|{\leavevmode\hbox{\tt\char`\|}} % vertical line
\def\dn{\leavevmode\hbox{\tt\char'14}} % downward arrow
\def\up{\leavevmode\hbox{\tt\char'13}} % upward arrow
\def\]{\leavevmode\hbox{\tt\char`\ }} % visible space

\def\pt{\,{\rm pt}} % units of points, in math formulas
\def\em{\,{\rm em}} % units of ems, in math formulas
\def\<#1>{\leavevmode\hbox{$\langle$#1\/$\rangle$}} % syntactic quantity
\def\oct#1{\hbox{\rm\'{}\kern-.2em\it#1\/\kern.05em}} % octal constant
\def\hex#1{\hbox{\rm\H{}\tt#1}} % hexadecimal constant
\def\cstok#1{\leavevmode\thinspace\hbox{\vrule\vtop{\vbox{\hrule\kern1pt
        \hbox{\vphantom{\tt/}\thinspace{\tt#1}\thinspace}}
      \kern1pt\hrule}\vrule}\thinspace} % control sequence token

{\obeyspaces\gdef {\ }}
\def\parbreak{\hfil\break\indent\strut}
\def\stretch{\nobreak\hskip0pt plus2pt\relax}

% macros for non-centered displays
\outer\def\begindisplay{\obeylines\startdisplay}
{\obeylines\gdef\startdisplay#1
  {\catcode`\^^M=5$$#1\halign\bgroup\indent##\hfil&&\qquad##\hfil\cr}}
\outer\def\enddisplay{\crcr\egroup$$}

% (the following \begin...\end-type macros do not appear in Appendix E)
% macros for demonstrating math constructions
\outer\def\beginmathdemo{$$\advance\baselineskip by2pt
  \halign\bgroup\indent\hbox to 160pt{##\hfil}&$##$\hfil\cr\noalign{\vskip-2pt}}
\outer\def\begindisplaymathdemo {$$\advance\baselineskip by15pt
  \halign\bgroup\indent\hbox to 160pt{##\hfil}&$\displaystyle{##}$\hfil\cr
  \noalign{\vskip-15pt}}
\outer\def\beginlongmathdemo{$$\advance\baselineskip by2pt
  \halign\bgroup\indent\hbox to 210pt{##\hfil}&$##$\hfil\cr\noalign{\vskip-2pt}}
\outer\def\beginlongdisplaymathdemo {$$\advance\baselineskip by15pt
  \halign\bgroup\indent\hbox to 210pt{##\hfil}&$\displaystyle{##}$\hfil\cr
  \noalign{\vskip-15pt}}
\outer\def\endmathdemo{\egroup$$}

% macros for font tables
\def\oddline#1{\cr
  \noalign{\nointerlineskip}
  \multispan{19}\hrulefill&
  \setbox0=\hbox{\lower 2.3pt\hbox{\hex{#1x}}}\smash{\box0}\cr
  \noalign{\nointerlineskip}}
\def\evenline{\cr\noalign{\hrule}}
\def\chartstrut{\lower4.5pt\vbox to14pt{}}
\def\beginchart#1{$$\postdisplaypenalty=-10000 \global\count@=0 #1
  \halign to\hsize\bgroup
    \chartstrut##\tabskip0pt plus10pt&
    &\hfil##\hfil&\vrule##\cr
    \lower6.5pt\null
    &&&\oct0&&\oct1&&\oct2&&\oct3&&\oct4&&\oct5&&\oct6&&\oct7&\evenline}
\def\endchart{\raise11.5pt\null&&&\hex 8&&\hex 9&&\hex A&&\hex B&
  &\hex C&&\hex D&&\hex E&&\hex F&\cr\egroup$$}
\def\:{\setbox0=\hbox{\char\count@}%
  \ifdim\ht0>7.5pt\reposition
  \else\ifdim\dp0>2.5pt\reposition\fi\fi
  \box0\global\advance\count@ by1 }
\def\reposition{\setbox0=\hbox{$\vcenter{\kern2pt\box0\kern2pt}$}}
\def\normalchart{%
  &\oct{00x}&&\:&&\:&&\:&&\:&&\:&&\:&&\:&&\:&&\oddline0
  &\oct{01x}&&\:&&\:&&\:&&\:&&\:&&\:&&\:&&\:&\evenline
  &\oct{02x}&&\:&&\:&&\:&&\:&&\:&&\:&&\:&&\:&&\oddline1
  &\oct{03x}&&\:&&\:&&\:&&\:&&\:&&\:&&\:&&\:&\evenline
  &\oct{04x}&&\:&&\:&&\:&&\:&&\:&&\:&&\:&&\:&&\oddline2
  &\oct{05x}&&\:&&\:&&\:&&\:&&\:&&\:&&\:&&\:&\evenline
  &\oct{06x}&&\:&&\:&&\:&&\:&&\:&&\:&&\:&&\:&&\oddline3
  &\oct{07x}&&\:&&\:&&\:&&\:&&\:&&\:&&\:&&\:&\evenline
  &\oct{10x}&&\:&&\:&&\:&&\:&&\:&&\:&&\:&&\:&&\oddline4
  &\oct{11x}&&\:&&\:&&\:&&\:&&\:&&\:&&\:&&\:&\evenline
  &\oct{12x}&&\:&&\:&&\:&&\:&&\:&&\:&&\:&&\:&&\oddline5
  &\oct{13x}&&\:&&\:&&\:&&\:&&\:&&\:&&\:&&\:&\evenline
  &\oct{14x}&&\:&&\:&&\:&&\:&&\:&&\:&&\:&&\:&&\oddline6
  &\oct{15x}&&\:&&\:&&\:&&\:&&\:&&\:&&\:&&\:&\evenline
  &\oct{16x}&&\:&&\:&&\:&&\:&&\:&&\:&&\:&&\:&&\oddline7
  &\oct{17x}&&\:&&\:&&\:&&\:&&\:&&\:&&\:&&\:&\evenline}

% (now Appendix E resumes again)
% macros for verbatim scanning
\chardef\other=12
\def\ttverbatim{\begingroup
  \catcode`\\=\other
  \catcode`\{=\other
  \catcode`\}=\other
  \catcode`\$=\other
  \catcode`\&=\other
  \catcode`\#=\other
  \catcode`\%=\other
  \catcode`\~=\other
  \catcode`\_=\other
  \catcode`\^=\other
  \obeyspaces \obeylines \tt}

\outer\def\begintt{$$\let\par=\endgraf \ttverbatim \parskip=\z@
  \catcode`\|=0 \rightskip-5pc \ttfinish}
{\catcode`\|=0 |catcode`|\=\other % | is temporary escape character
  |obeylines % end of line is active
  |gdef|ttfinish#1^^M#2\endtt{#1|vbox{#2}|endgroup$$}}

\catcode`\|=\active
{\obeylines \gdef|{\ttverbatim \spaceskip\ttglue \let^^M=\  \let|=\endgroup}}

% macros for syntax rules (again, not in Appendix E)
\def\[#1]{\silenttrue\xref|#1|\thinspace{\tt#1}\thinspace} % keyword in syntax
\def\beginsyntax{\endgraf\nobreak\medskip
  \begingroup \catcode`<=13 \catcode`[=13
  \let\par=\endsyntaxline \obeylines}
\def\endsyntaxline{\futurelet\next\syntaxswitch}
\def\syntaxswitch{\ifx\next\<\let\next=\syntaxrule
  \else\ifx\next\endsyntax\let\next=\endgroup
  \else\let\next=\continuerule\fi\fi \next}
\def\continuerule{\hfil\break\indent\qquad}
\def\endsyntax{\medbreak\noindent}
{\catcode`<=13 \catcode`[=13
  \global\let<=\< \global\let[=\[
  \gdef\syntaxrule<#1>{\endgraf\indent\silentfalse\xref\<#1>}}
\def\is{\ $\longrightarrow$ }
\def\alt{\ $\vert$ }

% macros to demarcate lines quoted from TeX source files
\def\beginlines{\par\begingroup\nobreak\medskip\parindent\z@ \obeylines
  \hrule\kern1pt\nobreak \everypar{\strut}}
\def\endlines{\kern1pt\hrule\endgroup\medbreak\noindent}
\def\weakendlines{\kern1pt\hrule\endgroup\medskip\noindent}
\def\finalendlines{\kern1pt\hrule\endgroup\medbreak}

\outer\def\exercise{\medbreak
  \global\advance\exno by 1
  \noindent\llap{\manual\char'170\rm\kern.15em}% triangle in margin
  {\ninebf EXERCISE \bf\chapno.\the\exno}\par\nobreak\noindent}
\def\dexercise{\global\advance\exno by 1
  \llap{\manual\char'170\rm\kern.15em}% triangle in indented space
  {\eightbf EXERCISE \bf\chapno.\the\exno}\hfil\break}
\outer\def\dangerexercise{\d@nger \dexercise}
\outer\def\ddangerexercise{\dd@nger \dexercise}

\newwrite\ans
\immediate\openout\ans=answers % file for answers to exercises
\outer\def\answer{\par\medbreak
  \immediate\write\ans{}
  \immediate\write\ans{\string\ansno\chapno.\the\exno:}
  \copytoblankline}
\def\copytoblankline{\begingroup\setupcopy\copyans}
\def\setupcopy{\def\do##1{\catcode`##1=\other}\dospecials
  \catcode`\|=\other \obeylines}
{\obeylines \gdef\copyans#1
  {\def\next{#1}%
  \ifx\next\empty\let\next=\endgroup %
  \else\immediate\write\ans{\next} \let\next=\copyans\fi\next}}

% Editorial notes: some things to watch for.

% f |\ and f ^|\ => insert \/  [e.g., if\/ |\hbox|...]
% appendi => check for \null  [e.g., Appendix~B\null.]
% ly- => the hyphen is probably omittable
% ''. and '', => transpose to .'' and ,''
% dgement => dgment
% in MFbook: f@' and \MF, and \MF.

% Macros for drawing figures (not in Appendix E)
\def\hidehrule#1#2{\kern-#1\hrule height#1 depth#2 \kern-#2 }
\def\hidevrule#1#2{\kern-#1{\dimen0=#1
    \advance\dimen0 by#2\vrule width\dimen0}\kern-#2 }
% \makeblankbox puts rules at the edges of a blank box
% whose dimensions are those of \box0 (assuming nonnegative wd,ht,dp)
% #1 is rule thickness outside, #2 is rule thickness inside
\def\makeblankbox#1#2{\hbox{\lower\dp0\vbox{\hidehrule{#1}{#2}%
    \kern-#1% overlap the rules at the corners
    \hbox to\wd0{\hidevrule{#1}{#2}%
      \raise\ht0\vbox to #1{}% set the vrule height
      \lower\dp0\vtop to #1{}% set the vrule depth
      \hfil\hidevrule{#2}{#1}}%
    \kern-#1\hidehrule{#2}{#1}}}}
\def\maketypebox{\makeblankbox{0pt}{1pt}}
\def\makelightbox{\makeblankbox{.2pt}{.2pt}}

% \box\bigdot is a null box with a bullet at its reference point
\newbox\bigdot \newbox\smalldot
\setbox0=\hbox{$\vcenter{}$} % \ht0 is the axis height
\setbox1=\hbox to\z@{$\hss\bullet\hss$} % bullet is centered on the axis
\setbox\bigdot=\vbox to\z@{\kern-\ht1 \kern\ht0 \box1 \vss}
\setbox1=\hbox to\z@{$\hss\cdot\hss$} % cdot is centered on the axis
\setbox\smalldot=\vbox to\z@{\kern-\ht1 \kern\ht0 \box1 \vss}

% \arrows makes things like <--- text --->
\def\arrows#1#2{% #1=width, #2=text
  {\setbox0=\hbox{$\mkern-2mu\mathord-\mkern-2mu$}
    \hbox to #1{\kern-.055556em$\leftarrow\mkern-6mu$%
      \cleaders\copy0\hfil
      \kern.4em #2\kern.4em
      \cleaders\copy0\hfil
      $\mkern-6mu\rightarrow$\kern-.055556em}}}

% \samplebox makes the outline of a box, with big dot at reference point
\def\samplebox#1#2#3#4{% #1=ht, #2=dp, #3=wd, #4=text
  {\setbox0=\vtop{\vbox to #1{\hbox to #3{}\vss}
      \nointerlineskip
      \vbox to #2{}}% now \box0 has the desired ht, dp, and wd
    \hbox{\copy\bigdot
      \vrule height.2pt depth.2pt width#3%
      \kern-#3%
      \makelightbox
      \kern-#3%
      \raise#1\vbox{\hbox to #3{\hss#4\hss}
        \kern 3pt}}}}

% \sampleglue makes glue between sample boxes
\newdimen\varunit
\varunit=\hsize \advance\varunit by-2\parindent
\divide\varunit by 58 % illustrations in Chapter 12
\def\sampleglue#1#2{% #1=width, #2=text
  \vtop{\hbox to #1{\xleaders\hbox to .5\varunit{\hss\copy\smalldot\hss}\hfil}
    \kern3pt
    \tabskip \z@ plus 1fil
    \halign to #1{\hfil##\cr#2\cr}}}

% Indexing macros
\newif\ifproofmode
\proofmodetrue % this should be false when making camera-ready copy
\newwrite\inx
\immediate\openout\inx=index % file for index reminders
\newif\ifsilent
\def\specialhat{\ifmmode\def\next{^}\else\let\next=\beginxref\fi\next}
\def\beginxref{\futurelet\next\beginxrefswitch}
\def\beginxrefswitch{\ifx\next\specialhat\let\next=\silentxref
  \else\silentfalse\let\next=\xref\fi \next}
\catcode`\^=\active \let ^=\specialhat
\def\silentxref^{\silenttrue\xref}

\def\marginstyle{\vrule height6pt depth2pt width\z@ \sevenrm}

\chardef\bslash=`\\
\def\xref{\futurelet\next\xrefswitch}
\def\xrefswitch{\begingroup
  \ifx\next|\aftergroup\vxref % case 1 or 2, |arg| or |\arg|
  \else\ifx\next\<\aftergroup\anglexref % case 3, "\<arg>" means angle brackets
    \else\aftergroup\normalxref \fi\fi\endgroup} % case 0, "{arg}"
\def\vxref|{\catcode`\\=\active \futurelet\next\vxrefswitch}
\def\vxrefswitch#1|{\catcode`\\=0
  \ifx\next\empty\def\xreftype{2}%
    \def\next{{\tt\bslash\text}}% type 2, |\arg|
  \else\def\xreftype{1}\def\next{{\tt\text}}\fi % type 1, |arg|
  \edef\text{#1}\makexref}
{\catcode`\|=0 \catcode`\\=\active |gdef\{}}
\def\anglexref\<#1>{\def\xreftype{3}\def\text{#1}%
  \def\next{\<\text>}\makexref}
\def\normalxref#1{\def\xreftype{0}\def\text{#1}\let\next=\text\makexref}
\def\makexref{\ifproofmode\insert\margin{\hbox{\marginstyle\text}}%
   \xdef\writeit{\write\inx{\text\space!\xreftype\space
     \noexpand\number\pageno.}}\writeit
   \else\ifhmode\kern\z@\fi\fi
  \ifsilent\ignorespaces\else\next\fi}
% the \insert (which is done in proofmode only) suppresses hyphenation,
% so the \kern\z@ is put in to give the same effect in non-proofmode.

% Internal cross references that may change
\def\sesame{61} % page number for Sesame Street quote
\def\bmiexno{20} % exercise number for bold math italic
\def\punishexno{1} % exercise number for `punishment'
\def\fracexno{6} % exercise number for `\frac'
\def\vshippage{31} % error message from `\vship'
\def\storypage{24} % listing of story.tex
\def\metaT{4} % exercise number for T of METAFONT
\def\xwhat{2} % exercise number for x3:=whatever
\def\Xwhat{2} % exercise number for whatever itself

\def\checkequals#1#2{\ifnum#1=#2\else
  \errmessage{Redefine \string#1 to be \the#2}\fi}

% Things for The METAFONTbook only
\ifx\MFmanual\!\else\endinput\fi

\def\!{\kern-.03em\relax}

\def\frac#1/#2{\leavevmode\kern.1em
  \raise.5ex\hbox{\the\scriptfont0 #1}\kern-.1em
  /\kern-.15em\lower.25ex\hbox{\the\scriptfont0 #2}}

\outer\def\displayfig #1 (#2){$$\advance\abovedisplayskip by 3pt
  \leftline{\indent\figbox{#1}{3in}{#2}\vbox}$$}
\def\rightfig #1 (#2 x #3) ^#4 {% #2 wide and #3 deep, raised #4
  \strut\vadjust{\setbox0=\vbox to 0pt{\vss
      \hbox to\pagewidth{\hfil
        \raise #4\figbox{#1}{#2}{#3}\vtop \quad}}
    \dp0=0pt \box0}}
\def\figbox#1#2#3#4{#4to#3{ % makes a box #2 wide and #3 deep
    \ifproofmode\kern0pt\hrule\vfill
    \hsize=#2 \baselineskip 6pt \fiverm\noindent\raggedright
    (Figure #1 will be inserted here; too bad you can't see it now.)
    \endgraf\vfill\hrule
    \else\vfill\hbox to#2{}\fi}}

\def\endsyntax{\begingroup\let\par=\endgraf\medbreak\endgroup\noindent}

\let\BEGINCHAPTER=\beginchapter
\def\beginchapter{\titlelsl=1pt \BEGINCHAPTER}
\def\beginChapter{\titlelsl=2pt \BEGINCHAPTER}

\def\decreasehsize #1 {\advance\hsize-#1}
\def\restorehsize{\hsize=\pagewidth}

\catcode`\@=\active
\catcode`\"=\active
\def\ttverbatim{\begingroup \catcode`\@=\other \catcode`\"=\other
  \catcode`\\=\other
  \catcode`\{=\other
  \catcode`\}=\other
  \catcode`\$=\other
  \catcode`\&=\other
  \catcode`\#=\other
  \catcode`\%=\other
  \catcode`\~=\other
  \catcode`\_=\other
  \catcode`\^=\other
  \obeyspaces \obeylines \tt}
\def\setupcopy{\def\do##1{\catcode`##1=\other}\dospecials
  \catcode`\|=\other \catcode`\@=\other \catcode`\"=\other \obeylines}
\def\_{\leavevmode \kern.06em \vbox{\hrule width.3em}}
\def@#1@{\begingroup\def\_{\kern.04em
    \vbox{\hrule width.3em height .6pt}\kern.08em}%
  \ifmmode\mathop{\bf#1}\else\hbox{\bf#1\/}\fi\endgroup}
\def"#1"{\hbox{\it#1\/\kern.05em}} % italic type for identifiers
\def\xrefswitch{\begingroup
  \ifx\next|\aftergroup\vxref % case 1, |arg| or |\arg|
  \else\ifx\next@\aftergroup\boldxref % case 2, "@arg@" means boldface
  \else\ifx\next"\aftergroup\italxref % case 4, ""arg"" means boldface
  \else\ifx\next\<\aftergroup\anglexref % case 3, "\<arg>" means angle brackets
    \else\aftergroup\normalxref \fi\fi\fi\fi\endgroup} % case 0, "{arg}"
\def\boldxref@#1@{\def\xreftype{2}\def\text{#1}%
  \def\next{@\text@}\makexref}
\def\italxref"#1"{\def\xreftype{4}\def\text{#1}%
  \def\next{"\text"}\makexref}

\def\pyth+{\mathbin{++}}
\def\0{\raise.7ex\hbox{$\scriptstyle\#$}}
\def\to{\mathrel{\ldotp\ldotp}}
\def\dashto{\mathrel{\hbox{-\thinspace-\kern-.05em}}}
\def\ddashto{\mathrel{\hbox{-\thinspace-\thinspace-\kern-.05em}}}
\def\round{\mathop{\rm round}}
\def\angle{\mathop{\rm angle}}
\def\rmsqrt{\mathop{\rm sqrt}}
\def\reverse{\mathop{\rm reverse}}
\def\curl{\mathop{\rm curl}}
\def\tension{\mathop{\rm tension}}
\def\atleast{\mathop{\rm atleast}}
\def\controls{\mathop{\rm controls}}
\def\and{\,{\rm and}\,}
\def\cycle{{\rm cycle}}
\def\pickup{@pickup@ \thinspace}
\def\penpos#1{\hbox{\it penpos}_{#1}}
\def\pentaper#1{\hbox{\it pentaper}_{#1}}

\chardef\hexa=1  % first hex
\chardef\hexb=2 % top and bot adjusted
\chardef\hexc=3 % same, bold
\chardef\hexd=4 % same, confined to box
\chardef\hexe=5 % penstroked hex
\chardef\Aa=6 % stick-figure A, golden ratio
\def\sevenAs{\char7\char8\char9\char10\char11\char12\char13} % same, variants
\chardef\Az=14 % same, with crooked bar
\chardef\Ab=15 % \Aa with rectilinear elliptical pen
\chardef\Ac=16 % same, with the ellipse tilted
\chardef\beana=17 % kidney bean, default pen
\chardef\beanb=18 % same, twice as bold
\chardef\beanc=19 % same, rectilinear elliptical pen
\chardef\beand=20 % same, with the ellipse tilted
\chardef\niba=21 % 10x rectilinear ellipse
\chardef\nibb=22 % same, with the ellipse tilted
\chardef\nibc=23 % same, 90 degrees titled
\chardef\IOT=24 % Ionian T
\chardef\IOS=25 % Ionian S
\chardef\IOO=26 % Ionian O
\chardef\IOI=27 % Ionian I
\chardef\cubea=28 % possible cube
\chardef\cubeb=29 % impossible cube
\chardef\bicentennial=30 % star with overlapping strokes
\chardef\oneu=31 % 1/4 of uuuu ornament
\chardef\circa=32 % quartercircle
\chardef\circb=33 % filled quartercircle
\chardef\circc=34 % rotated quartercircle
\chardef\circd=35 % cone
\chardef\circe=36 % concentric circles
\chardef\circf=37 % concentric diamonds
\chardef\fouru=38 % uuuu ornament
\chardef\fourc=39 % same, rotated
\chardef\seventh='140 % 1/7, to go with cmssqi8

\newdimen\apspix
\apspix=31448sp % 8 APS pixels = 52413.64sp, and I'm taking 60% of this
% to crude approximation, there are about 2\apspix per pt
\newdimen\blankpix \newdimen\Blankpix
\setbox0=\hbox{\manual P} \blankpix=\wd0 % approximately 1pt blank pixel
\setbox0=\hbox{\manual R} \Blankpix=\wd0 % approximately 3pt blank pixel

\def\leftheadline{\hbox to \pagewidth{%
    \vbox to 10pt{}% strut to position the baseline
    \llap{\tenbf\folio\kern1pc}% folio to left of text
    \def\MF{{\manual 89:;<=>:}}% slanted 10pt
    \tenit\rhead\hfil% running head flush left
    }}
\def\rightheadline{\hbox to \pagewidth{%
    \vbox to 10pt{}% strut to position the baseline
    \def\MF{{\manual 89:;<=>:}}% slanted 10pt
    \hfil\tenit\rhead\/% running head flush right
    \rlap{\kern1pc\tenbf\folio}% folio to right of text
    }}
\def\ttok#1{\leavevmode\thinspace\hbox{\vrule\vtop{\vbox{\hrule\kern1pt
        \hbox{\vphantom{\tt(j}\thinspace{\tt#1}\thinspace}}
      \kern1pt\hrule}\vrule}\thinspace} % token

\newdimen\tinypix \setbox0=\hbox{\sixrm0} \tinypix=5pt
\newdimen\pixcorr \pixcorr=\tinypix \advance\pixcorr by-\wd0
\def\pixpat#1#2#3#4{\vcenter{\sixrm\baselineskip=\tinypix
  \hbox{#1\kern\pixcorr#2}\hbox{#3\kern\pixcorr#4}}}

\font\rand=random
%Some boxing macros from manmac
%Input at the appropriate place
%   ds.pic    , picture inspired by David Salomon
%   lus.pic   , picture from lustrum paper
%   pic.pic   , picture from Furuta
%Input along the way  (within  \begingroup ... \endgroup)
%   btable.tex, bordered tabl macro (redefines \multispan!!!)
%   math.tex  , multipositioning of \eqalign (redefines centering needed)
%   lit.dat   , literature database (at the end)
%   lit.sel   , literature reference names pointing to the database.
%%%%%%%%%%%%%%%%%%%%%%%%%%%%%%%%%%%%%%%%%%%%%%%%%%%%%%%%%%%%%%%%%%%%%%%%%%%%%%%
\newcount\TestCount
\def\smc{\tensmc}
\def\SMC{\ninerm}
\font\tensmc=cmcsc10
%
%     *****  abbreviations and logos  *****
%

\def\AllTeX{(\La)\TeX}

\def\AMS{American Mathematical Society}

\def\AmS{{\the\textfont2 A}\kern-.1667em\lower.5ex\hbox
        {\the\textfont2 M}\kern-.125em{\the\textfont2 S}}
\def\AmSTeX{\AmS-\TeX}

\def\aw{A\kern.1em-W}
\def\AW{Addison\kern.1em-\penalty\z@\hskip\z@skip Wesley}

\def\BibTeX{{\rm B\kern-.05em{\smc i\kern-.025emb}\kern-.08em\TeX}}

\def\CandT{{\sl Computers \& Typesetting}}

\def\DVItoVDU{DVIto\kern-.12em VDU}

\def\ISBN{{\SMC ISBN} }

%       Japanese TeX
\def\JTeX{\leavevmode\hbox{\lower.5ex\hbox{J}\kern-.18em\TeX}}

\def\JoT{{\sl The Joy of \TeX}}

\def\LAMSTeX{L\raise.42ex\hbox{\kern-.3em\the\scriptfont2 A}%
    \kern-.2em\lower.376ex\hbox{\the\textfont2 M}\kern-.125em
    {\the\textfont2 S}-\TeX}

%       note -- \LaTeX definition is from LATEX.TEX 2.09 of 7 Jan 86,
%               adapted for additional flexibility in TUGboat
%\def\LaTeX{\TestCount=\the\fam \leavevmode L\raise.42ex
%       \hbox{$\fam\TestCount\scriptstyle\kern-.3em A$}\kern-.15em\TeX}
%       note -- broken in two parts, to permit separate use of La,
%               as in (La)TeX
\def\La{\TestCount=\the\fam \leavevmode L\raise.42ex
        \hbox{$\fam\TestCount\scriptstyle\kern-.3em A$}}
\def\LaTeX{\La\kern-.15em\TeX}

%       for Robert McGaffey
\def\Mc{\setbox\TestBox=\hbox{M}M\vbox to\ht\TestBox{\hbox{c}\vfil}}

\font\manual=logo10 % font used for the METAFONT logo, etc.
\def\MF{{\manual META}\-{\manual FONT}}
\def\mf{{\smc Metafont}}
\def\MFB{{\sl The \slMF book}}

%       multilingual (INRS) TeX
\def\mtex{T\kern-.1667em\lower.5ex\hbox{\^E}\kern-.125emX}

\def\pcMF{\leavevmode\raise.5ex\hbox{p\kern-.3ptc}MF}
\def\PCTeX{PC\thinspace\TeX}
\def\pcTeX{\leavevmode\raise.5ex\hbox{p\kern-.3ptc}\TeX}

\def\Pas{Pascal}

\def\PiC{P\kern-.12em\lower.5ex\hbox{I}\kern-.075emC}
\def\PiCTeX{\PiC\kern-.11em\TeX}

\def\plain{{\tt plain}}

\def\POBox{P.\thinspace O.~Box }
\def\POBoxTUG{\POBox\unskip~9506, Providence, RI~02940}

\def\PS{{Post\-Script}}

\def\SC{Steering Committee}

\def\SGML{{\SMC SGML}}

\def\SliTeX{{\rm S\kern-.06em{\smc l\kern-.035emi}\kern-.06em\TeX}}

\def\slMF{\MF}
%       Use \font\manualsl=logosl10 instead, if it's available,
%       for \def\slMF{{\manualsl META}\-{\manualsl FONT}}

%       Atari ST (Klaus Guntermann)
\def\stTeX{{\smc st\rm\kern-0.13em\TeX}}

\def\TANGLE{{\tt TANGLE}}

\def\TB{{\sl The \TeX book}}
\def\TP{{\sl \TeX\/}: {\sl The Program\/}}

\def\TeX{T\hbox{\kern-.1667em\lower.424ex\hbox{E}\kern-.125emX}}

\def\TeXhax{\TeX hax}

%       Don Hosek
\def\TeXMaG{\TeX M\kern-.1667em\lower.5ex\hbox{A}\kern-.2267emG}

%\def\TeXtures{\TestCount=\the\fam
%       \TeX\-\hbox{$\fam\TestCount\scriptstyle TURES$}}
\def\TeXtures{{\it Textures}}
\let\Textures=\TeXtures

\def\TeXXeT{\TeX--X\kern-.125em\lower.5ex\hbox{E}\kern-.1667emT}

\def\ttn{{\sl TTN}}
\def\TTN{{\sl \TeX{} and TUG NEWS}}

\def\tubfont{\sl}               % redefined in other situations
\def\TUB{{\tubfont TUGboat\/}}

\def\TUG{\TeX\ \UG}

\def\UG{Users Group}

\def\UNIX{{\SMC UNIX}}

\def\VAX{\leavevmode\hbox{V\kern-.12em A\kern-.1em X}}
\def\VorTeX{V\kern-2.7pt\lower.5ex\hbox{O\kern-1.4pt R}\kern-2.6pt\TeX}

\def\XeT{\leavevmode\hbox{X\kern-.125em\lower.424ex\hbox{E}\kern-.1667emT}}

\def\WEB{{\tt WEB}}
\def\WEAVE{{\tt WEAVE}}
%********************************************************************
\def\icmat#1#2{%ICon MATrix(rectangular)
%#1 is ht of icon matrix, e.g. 4
%#2 is wd of icon matrix, e.g. 2
\vbox to#1\unitlength{\hrule
   \hbox to#2\unitlength{\vrule
     height#1\unitlength\hfil\vrule}%
           \hrule}%
}%end icmat
%
\def\icurt#1#2{%IConUpperRightTriangle
%#1 is ht of icon matrix, with UT
%the upper triangular part, e.g. 4
%#2 is wd of icon (upper triangular)
%matrix, e.g. 2
\vbox to #1\unitlength{\hrule
   \hbox{\picture(#2,#2)%
    \put(0,#2){\line(1,-1){#2}}%
    \endpicture\vrule}%
   \vfil}%
}%end icurt
%
\def\icllt#1#2{%IConLowerLeftTriangle
%#1 is ht of icon matrix, with LT
%the lower triangular part, e.g. 4
%#2 is wd of icon (lower triangular)
%matrix, e.g. 2
\vbox to #1\unitlength{\vfil
   \hbox{\vrule\picture(#2,#1)%
     \put(0,#2){\line(1,-1){#2}}%
     \endpicture}%
   \hrule}%
}%end icllt
%
\def\icuh#1#2#3{%IConUpperHessenberg
%#1 is size of icon matrix, with UH
% the upper Hessenberg part, e.g. 4
%#2 is wd of icon (upper Hesenberg)
% matrix, e.g. 1
%#3 is size Lower Left triangular part,
% #1-#2 (for simplicity the latter is added,
% could have been calculated, perhaps some
% inconsistency test could be incorporated)
\vbox to #1\unitlength{\offinterlineskip
   \hrule
   \hbox to#1\unitlength{\vrule height%
       #2\unitlength depth0pt\relax
       \hfil\vrule}%
   \hbox to#1\unitlength{\picture(#3,#3)%
    \put(0,#3){\line(1,-1){#3}}\endpicture
    \hfil\vrule}%
   \hbox to#1\unitlength{\hfil\vrule
     width#2\unitlength height.2pt\relax}%
   }%
}%end icuh
%
\def\hidehrule#1#2{\kern-#1\hrule
 height#1 depth#2 \kern-#2 }
\def\hidevrule#1#2{\kern-#1{\dimen0=#1
 \advance\dimen0 by#2\vrule width\dimen0}%
 \kern-#2 }
% \makeblankbox puts rules at the edges of
% a blank box whose dimensions are those
% of \box0 (assuming nonnegative wd,ht,dp)
% #1 is rule thickness outside,
% #2 is rule thickness inside
\def\makeblankbox#1#2{\hbox{\lower\dp0
 \vbox{\hidehrule{#1}{#2}%
  \kern-#1% overlap the rules at the corners
  \hbox to\wd0{\hidevrule{#1}{#2}%
  \raise\ht0\vbox to #1{}% set the vrule height
  \lower\dp0\vtop to #1{}% set the vrule depth
  \hfil\hidevrule{#2}{#1}}%
  \kern-#1\hidehrule{#2}{#1}}}}
\def\maketypebox{\makeblankbox{0pt}{1pt}}
\def\makelightbox{\makeblankbox{.2pt}{.2pt}}
\def\<#1>{$\langle#1\rangle$}
\def\cs#1{{\tt\char92#1}}
%
\def\mm{{\tt manmac}}
\def\mmt{{\tt manmac.sty}}
%For abstracting and customizing
%\def\head*#1*{\chapter*{#1}}
%\def\subhead*#1*{\section*{#1}}
%\def\subsubhead*#1*{\subsection*{#1}}
%\def\ftn#1{\footnote{#1}}
\let\ea=\expandafter \let\ag=\aftergroup \let\nx=\noexpand
%Customize footer
\def\pfoottext{NLUUG meeting Fall '93}
%
\begin{document}

\title{What is \TeX{} and METAfont all about?}
%\thanks{Paper to be presented at NLUUG meeting of 2 November, 1993.}
\author{Kees van der Laan}
\address{Hunzeweg 57, 9893 PB\\
        Garnwerd, Groningen (NL)\\
        +31 5941 1525}
\netaddress[\network{Internet}]{cgl@risc1.rug.nl}
\overfullrule0pt
%\begin{abstract}
%A survey of
%   \TeX,
%   its flavours, and
%   its twin sister \MF{},
%within the context of Electronic Publishing,
%is given.
%\end{abstract}
\maketitle
%\paragraph*{Keywords:}{\small \AmSTeX,
%education, electronic publishing, (La)\TeX, METAfont, (encapsulated) \PS,
% SGML, hypertext.}
\section*{Contents}%
\begingroup\small
Introduction\\
-- \TeX{} etc.{} tools\\
-- Importance\\
-- \TeX's flavours, drivers, and fonts\\
-- Descriptive mark-up\\
-- \TeX{} its author, users, and publishers\\
-- \TeX\ and other EP tools\\
-- Trends\\
-- Examples: generic format, and\\
\phantom{--} in the small math, tables, and graphics\\
-- Front \& back matter \\
-- Guidelines for choosing\\
Acknowledgements, Conclusions, References.
\endgroup
\section*{Introduction}
This work about computer-assisted typesetting by \AllTeX{} and \MF{}
in context, is aimed at a broad audience.
Novice users \`a la
BLU\footnote{BLU is Knuth's nickname for the innocent user, the so-called
    Ben Lee User of the \TeX book fame, with BLUe its cousin, adopted by me.
    Nowadays we would say Beginning \LaTeX\ User.}
who like to become informed what it is all about,
advanced \LaTeX\ users who hardly have heard of \mm,
and mathematicians and publishers who will find the offerings
of the \AMS{} interesting.

There have been published many notes, articles and books about \TeX.
Advanced ones exploring \TeX's limits, and also contributions at
the survey and introductory level.
The latter deal with
the macroscopic mark-up features as well as
the microscopics of automatic kerning,
   for example with A and V in AV,
the automatic handling of ligatures,
the automatic justification and hyphenation
supported by hyphenation tables, and the formatting of
math, tables and graphics.
They also boast of the quality which can be
obtained when formatting the typographic teasers:
math, tables and graphics.

In the \TeX niques series we have the
tutorials:
A gentle introduction to \TeX, by Michael Doob, and
First grade \TeX, by Arthur Samuel.
For \LaTeX\ there is: An introduction to \LaTeX, by Michael Urban,
and---for the Dutch speaking community---Publiceren met \LaTeX, by
de Bruin.
Also noteworthy is Hoenig's \TeX\ for new users, and
the introduction chapter in Salomon's courseware Insights and Hindsights.
For \MF{} see Henderson's An introduction to \MF{},
Tobin's \MF{} for beginners, and Knuth's
introductory article on the issue in TUGboat.
A survey with respect to EP tools (Electronic Publishing) is
Document Formatting Systems:
Survey, Concepts and Issues, by Furuta and co-authors.

For trying it out and working with it, the user groups
distribute PD versions of (La)\TeX{} as well as
integrated working environments for PCs,
with all kinds of bells-and-whistles added.
Ubiquitous is Mattes' PD em\TeX, and the working environments
 As\TeX{} (apart from Framework it is
 in the Public Domain), next to the Dutch 4\TeX{} (which is shareware).

This paper  relates \TeX{} and \MF{} to EP,
SGML and the like, as a helicopter view, and accounts for the many
activities of its users.
At the end an annotated bibliography has been supplied.

\paragraph*{Conventions and notations.}
I adhered to the historical development of \TeX\ et cetera,
and did not order the tools with respect to perceived importance.
The latter is a matter of taste and definitely time-dependent.

The Contents list is not a one-to-one mapping of the section titles.
It is used to stress the main items and their treatment within a logical
hierarchy.
I clustered some section titles and subsection titles, whenever
convenient, to enhance readability.
The aim was to convey the contents and not so much the form,
to paraphrase Marvin Minsky.

Because it is a `helicopter' view I need to refer to other work.
This has been done a little loose via the name of the
(first) author and the title, or keywords form the title.
The reader can easily spot from the supplied list of references
which work is hinted at.
Just start by the author name and look for the matching title.
I also did not bother about traditions which require that book titles are set
in italics or so. In my opinion to find out whether it is a book, a report or
a journal article follows easily from the ISBN number if provided,
respectively the journal name.
Hereby I assume that readers are familiar with some
journal names, for example TUGboat, the journal of the \TeX\ Users Group.

For common words in the \TeX\ arcana\Dash like \TeX, \LaTeX,
\AMS, et cetera\Dash I adopted the TUGboat
typesetting conventions by using their macros for formatting these names.
File names are set in the \cs{tt} font.

\section{\TeX{} etc.{} tools}
First of all \TeX\ etc.\ has been around  for some  fifteen years,
and many of its users have contributed to the components
and to the porting to many platforms,
with the result that it is
not easy to really survey the whole complex.

Going back to the roots we can say that
\TeX\ is a program for formatting documents,
born as a twin with its sister \MF, for creating fonts.
\TeX\ and \MF{} have been designed to facilitate the
high-quality computer-assisted production of books.
A more modern way of talking is that \TeX\ is
a mark-up language with \MF{} the accompanying
tool for designing the needed graphics, starting with the fonts.

A nice survey of the most important components and files when working with
\TeX\ is supplied by the accompanying diagram,\footnote{Inspired by
   Salomon's diagram as supplied in his courseware: Insights and Hindsights.}
which illustrates the two main fields: font design and typesetting, with
the relations between the components and files, all in one, and abstracting
from details.

\noindent
%Version Aug 93    cgl@risc1.rug.nl
\begingroup%Basically Salomon's diagram
%\Large\setlength{\unitlength}{3ex}
       \setlength{\unitlength}{3.8ex}
\begin{picture}(14,16)(-.5, -3)
%1st column
\put(1, 0){\line(0, 1){1.5}}
\put(1, 2){\oval(2, 1)}
\put(1, 2){\makebox(0, 0){.pk}}
\put(1, 4){\vector(0, -1){1.5}}
\put(-.5, 4){\framebox(3, 1){METAfont}}
\put(1, 6.5){\vector(0, -1){1.5}}
\put(1, 7){\oval(2, 1)}
\put(1, 7){\makebox(0, 0){.mf}}
%second column
\put(7.5, -.5){\framebox(2, 1){driver}}
\put(9.6, .6){\line(-1, 0){.5}}
\put(9.6, .6){\line( 0, -1){.5}}
\put(9.7, .7){\line(-1, 0){.5}}
\put(9.7, .7){\line( 0, -1){.5}}
\put(8.5, 1.25){\vector(0, -1){.75}}
\put(8.5, 2){\oval(2, 1)}
\put(8.5, 2){\makebox(0, 0){.dvi}}
\put(8.5, 4){\vector(0, -1){1.5}}
\put(7.5, 4){\framebox(2, 1){\TeX}}
%Manmac
\put(9.75, 5.25){\line(-1, 0){1.25}}
\put(9.75, 5.25){\line( 0, -1){.5}}
\put(9.75, 4.75){\line(-1, 0){.25}}
\put(9.8, 4.75){{\tiny manmac}}
%LaTeX
\put(10, 5.5){\line(-1, 0){1.25}}
\put(10, 5.5){\line( 0, -1){.5}}
\put(10, 5){\line(-1, 0){.25}}
\put(8.75, 5.5){\line( 0, -1){.25}}
\put(10.1, 5){{\tiny \LaTeX}}
\put(10.25, 5.75){\line(-1, 0){1.25}}
\put(10.25, 5.75){\line( 0, -1){.5}}
\put(10.25, 5.25){\line(-1, 0){.25}}
\put(9, 5.75){\line( 0, -1){.25}}
\put(10.35, 5.35){{\tiny AMS-(La)\TeX}}
%
\put(10.35, 5.85){\hbox{.}\kern.1ex
\raise.5ex\hbox{.}\kern.1ex\raise1ex\hbox{.}}
%
\put(8.5, 6.5){\vector(0, -1){1.5}}
\put(8.5, 7){\oval(2, 1)}
\put(8.5, 7){\makebox(0, 0){.tex}}
%
\multiput(8.5, 9)(0, -.415){3}{\line(0, -1){.25}}
\put(8.5, 8.7){\vector(0, 1){.3}}
\put(8.5, 7.8){\vector(0, -1){.3}}
\put(7.5, 9){\framebox(2, 1){editor}}
%Spelling checker
\put(9.75, 10.25){\line(-1, 0){1.25}}
\put(9.75, 10.25){\line( 0, -1){.5}}
\put(9.75, 9.75){\line(-1, 0){.25}}
\put(9.85, 9.75){{\tiny spell}}
%Style checker
\put(10, 10.5){\line(-1, 0){1.25}}
\put(10, 10.5){\line( 0, -1){.5}}
\put(10, 10){\line(-1, 0){.25}}
\put(8.75, 10.5){\line( 0, -1){.25}}
\put(10.1, 10){{\tiny style}}
%
\put(10.1, 10.6){\hbox{.}\kern.1ex
\raise.5ex\hbox{.}\kern.1ex\raise1ex\hbox{.}}
%
\put(8.5, 11.5){\vector(0, -1){1.5}}
\put(8.5, 12){\oval(2, 1)}
\put(8.5, 12){\makebox(0, 0){copy}}
%basis
\put(1, 0){\vector(1, 0){6.5}}
\put(9.5, 0){\vector(1, 0){1.5}}
\put(11, -.75){\framebox(2, 1.5){}}
\put(11.25, -.4){\shortstack{\small printer\\\small screen}}
%middle
\multiput(5, 5.25)(0, 1){3}{\line(0,1){.5}}
\multiput(5,  .25)(0, 1){4}{\line(0,1){.5}}
\multiput(5, -1.75)(0, 1.3){2}{\line(0,1){.2}}
%
\put(3.5, -1.250){\dashbox{.25}(3, .5){{\tiny \PS}}}
\multiput(6.5, -1)(.45, 0){4}{\line(1,0){.25}}
\put(8.5, -1){\line( -1, 0){.2}}
%
\put(8.5, -1){\vector(0, 1){.5}}
\put(2.5, 4.5){\vector(1, 0){1.5}}
\put(5, 4.5){\oval(2, 1)}
\put(5, 4.5){\makebox(0, 0){.tfm}}
\put(6, 4.5){\vector(1, 0){1.5}}
\put(9.5, 4.5){\vector(1, 0){1.5}}
\put(12, 4.5){\oval(2, 1)}
\put(12, 4.5){\makebox(0, 0){.log}}
%base line
\put(.51,-1.75){\vector(-1, 0){1}}
\put(2.25, -1.75){\makebox(0, 0){Fonts}}
\put(3.9,-1.75){\vector( 1, 0){1}}
\put(6.1,-1.75){\vector(-1, 0){1}}
\put(9, -1.75){\makebox(0, 0){Typesetting}}
\put(11.9,-1.75){\vector(1, 0){1}}
\end{picture}
\endgroup

\noindent That is
\begin{itemize}
\item the flow from copy to printed results
\item  where the editor and its associated tools come in
\item the location of \TeX\Dash its flavours, and add-ons\Dash at the heart
\item what is used from \MF{} and where
\item the printer independence via various drivers
\item at what level \PS{} can be included.
\end{itemize}
The important files are indicated by their extensions and are
depicted within ovals. What holds for creating the \verb|.tex| file
holds also for the \verb|.mf| file.\footnote{Not mentioned are vir\TeX\
   and ini\TeX. Erik-Jan Vens communicated the following functionalities
   on the TeX-nl network: `Ini\TeX\ allows preparing and fast loading
   of {\tt .fmt} files. Vir\TeX\ is a program that can accept fast
   your macros and then do the typesetting job proper.'}

\subsection{Working environments.}
The needed tools are nowadays embedded in
computer-assisted (scientific) working environments.
At first sight this seems trivial, but it is really handy that the tools
are integrated, also with non-formatting applications per se, such
as email, database applications and the old running of C or FORTRAN programs.
A model of thinking is that, for example, a thesis is prepared and all
the simulations and calculations are done as a side-step of the main work:
publishing! That is document preparation, formatting, typesetting, and
dissemination.
The graphics-oriented PCs like Macintosh and Atari paved the way.
Nowadays the 486-based PCs with their (graphics) window facilities allow
this way of working too.

\subsection{Installation.}
The products are usually accompanied by their installation documentation.
Famous, and top class, are the AMS installation Guides.
With the PD PC versions the idea is to supply turn-key scripts so that
the installation goes automatically.
>From those distributed by the TUG/LUGs the only nice one
I have seen is the GUTenberg  PD PC set and installation guide,
prepared by Lavaud. Installation of the working environments is more
complicated, because of the many components.

\subsection{Lifetime.}
The kernel \TeX\ and \MF{}
programs have been designed with flexibility and portability in mind.
Knuth envisioned that the two could be used a hundred years from now,
just as we do today, with the same
input and  results!\footnote{Or better.}
In order to make this possible Knuth
\begin{itemize}
\item invented the \WEB{} literate programming way of working
\item documented the programs (open system) well
\item worked hard on making the systems error free
\item delivered the twins into the public domain, and
\item froze the kernels.
\end{itemize}
Because of these goodies the user community could port the systems
to any conceivable platform, and add layers on top
to adjust for  users' wishes and demands. All-in-all one can say
that the twins are  portable in place and time, are powerful, useful,
and will serve a lifetime.

The working environments suffer from a much shorter lifetime.
Read: need continuous maintenance and that is something, especially in
a volunteer-based world.
It is always
a matter of the right balance: how fast do I need to do the day-to-day
work and how often do I wish to upgrade the working environment.

\section{Importance}
>From the computer science point of view
\TeX\ and \MF{} are big research achievements
in how software engineering should be done,
if not for the literate programming way of software design and creation.
Top-class algorithms for line-breaking, hyphenation and page make-up
have been incorporated.
It is designed to be device-independent.
That Knuth succeeded so well in his basic research can be witnessed
by the many publications which
have been built upon his Computer and Typesetting works,
and the many honorary degrees he has received.

>From the users' point of view \TeX\ etc.\
is relevant because of the quality which
can be obtained when used as a formatter.
\TeX\ is an open and freely available system.
It has been frozen, and delivered into the public domain to serve
for a lifetime.
That Knuth succeeded here so well can be distilled from the many organized
users of \AllTeX\ world-wide, and perhaps the tenfold more who
just use the systems.

Its weakness is that \TeX\ proper does {\em not\/} have easy user guides.
This weakness has been compensated for by efforts like \LaTeX,
\AmSTeX/\LaTeX, and the styles from publishing houses and their user and
installation guides.
%\begin{quote}
Perhaps an unexpected side-effect of \TeX\ is that it is so heavily used
with alphabets different from Latin, and even with scripts
which run from right to left (Hebrew) or scripts which run vertically
(Japanese), not to mention specific hyphenation patterns.
%\end{quote}
That \TeX\ allows for these usages might give an idea of its power.

>From the publishers' point of view \TeX\ has the potential of being used
for producing complex scientific documents cost-effectively.
This is the current practice of the \AMS,
and the American Physical Society, APS for short.
They  supply authors with
\begin{itemize}
\item user and installation guides
\item fonts
\item style files
\item templates, and
\item support, in general.
\end{itemize}

\paragraph*{The advantages}can be summarized as
\begin{itemize}
\item high-quality craftsman tool
\item lingua franca for exchange of typographically complex documents
\item stability (\TeX{} kernel has been frozen)
\item open system
\item available for nearly all platforms
\item in the public domain
\item portable, flexible, extensible, \ldots
\item 7.5--10k organized users world-wide
\item cost-effective production tool.
\end{itemize}

\paragraph*{Disadvantages}are there any?
Of course there are. But it is questionable
whether one should talk about disadvantages.
Perhaps one should talk more in terms of incompleteness.
\begin{quote}
What is felt like an omission can be added,
because it is an extensible system.
\end{quote}
I for one miss that \mmt---Knuth's macros for formatting his books---doesn't
take a user guide, nor does plain \TeX.
Of course there is the \TeX book---the bible for the \TeX ies---but that does
{\em not\/} hide the details---it is all there, for the beginner as
well as for the advanced macro writer---which is confusing
and simply too much for a novice. In summary
\begin{itemize}
\item \AllTeX\ is not
      WYSIWYG-like\footnote{Usually commercial.}
\item unusual macro language\footnote{It is always a matter of education,
      and after that the {\em un\/}usual issues metamorphose into paradigms.}
\item complex: $\approx$ 1k commands, parameters, \ldots\footnote{Abstraction,
   subsetting and user guides\Dash like those of \AmSTeX\Dash are needed.
   Tools which concentrate on the publishing goal and not so much on
   understanding and learning the formatting language per se.}
\end{itemize}
So its incompleteness is a challenge to all of us, to fill it up.

It is true, however, that professionals have found some niches which deserve
further research and development. Surveys on these items are provided in
the E-\TeX\ paper by Mittelbach, and the New Typesetting System efforts
initiated by the German-speaking users group DANTE.
Also noteworthy is the effort to improve
\LaTeX\ via the so-called \LaTeX3 (better known as lxiii) project.

One can also argue that delving into these details is sub-optimization,
concentrating too much on the mapping onto paper. Bigger issues are
related to the multi-media aspects, let us say to represent information
in a flexible way such that it can be processed by various technologies,
into forms suited for various users, their circumstances and their
tastes, limited only by their senses.
I like to call this {\em real\/} applied information technology:
information to be accessed by the masses.

\section{\TeX's flavours}
\TeX\ has gotten its children already, like \mm, \LaTeX, and
\AmSTeX/\LaTeX, to name but a few.
As usual with children they live their own lives.
For \TeX\ this means that they have the confusing
side-effect of not being completely compatible.
In spite of this incompatibility reality has it that authors and publishers
make their choice---\TeX-based, or \LaTeX-oriented---and therefore
the incompatibilities don't hinder most of us.

\paragraph*{\mmt} is a set of macros
written and used by Knuth to format his magnum opus:
The Art of Computer Programming,
his Computers and Typesetting series,
and so on. For an account see my Manmac BLUes.

\paragraph*{\LaTeX}stresses the higher-level approach of descriptive mark-up
and hides the formatting details as much as possible from an author.
Because of the rigorous way this has been implemented,
it is  hard to customize the prefab styles.

Leslie Lamport's manual, \LaTeX, A Document Preparation System,
exhibits the functionalities
\begin{itemize}
\item prefab styles: article, book, letter, report, slides
\item automatic (symbolic) numbering and cross-referencing
\item multi-column formatting, with its embedded 1-column occasionally
      for tables and figures
\item automatic generation of ToC, LoT, LoF
\item picture environment
\item bibliography environment.
\end{itemize}

\paragraph*{\AmSTeX/\LaTeX}are the tools of the
pace-setting American Mathematical Society. This publisher adopted
and supported the \TeX\ development from the beginning. (See below
under \TeX\ and its publishers.)

\paragraph*{\LAMSTeX} reimplemented in a flexible way
  the descriptive \LaTeX\ approach, next to
  a general automatic numbering and symbolic referencing scheme,
  advanced table macros, and
  sophisticated commutative diagram macros.
  See my review of Spivak's \oe uvre
  for more details about the Joy of \TeX\ and \LAMSTeX---The Synthesis.

\paragraph*{In summary}
\begin{itemize}
\item \mmt, Knuth's format
\item \LaTeX, descriptive mark-up, and user's guide
\item \AmSTeX/\LaTeX\ styles and fonts, with support
\item \LAMSTeX
\item TUGboat styles
\item PD software and working environments
\end{itemize}

\section{\TeX's drivers}
Normally the drivers come with your \TeX\ when you buy it.
With the PD versions, users have to be aware of the PD available drivers,
for the various PCs and printers,
unless your user group provides you with an
integrated working environment which contains all.
For a survey of the available `Output device
drivers' see Hosek's paper in TUG's resource directory.
He details drivers for
\begin{itemize}
\item laser xerographic and electron-erosion printers
\item impact printers and miscellaneous output devices
\item phototypesetters
\item screen previewers
\end{itemize}
\noindent and ends up with supplier information.
Joachim Schrod reported in TUGboat 13, 1,
(early 1992) from the TUG DVI driver standards committee.

Well-known is the PD Beebe driver family. em\TeX\ comes with some
drivers for dot matrix printers and the HP LaserJets.

At the TUG '92 meeting the attendees were surprised by Raman's paper
`An audio view of (La)\TeX\ documents.' It has all to do with
representing the contents of a publication for the blind.

With respect to \PS\ the \verb|dvitops| driver is important. Formerly,
I also used \verb|dvitodvi| in order to print out selected pages.
Now I use \mm's facility to do that which is essential
simpler for that purpose because it ships out only the required pages.

\section{\TeX{} and fonts}
>From the beginning Knuth provided \TeX\ with the computer modern family of
fonts. These fonts can be generated, and varied via \MF,
by adjusting some parameters.
Since the introduction of the virtual font concept, in revision '89
better known as \TeX{} version 3,
many industrial fonts can be used as well.
Via this mechanism, font elements can be combined at the driver level.
The need for handling in a flexible way the positioning of diacritical marks
was the incentive for adding the virtual font concept, to make it
feasible to handle languages with their own special placements of
diacritical marks without the need to
regenerate complete new fonts.
The other way is to generate complete font tables for every language,
which is a perfectly acceptible way of doing it,
but will entail many font tables and
of larger size.\footnote{Reality has it that the \TeX\ community standardized
   on the 256-character DC font tables, to allow for some special characteres,
   like the use of the ij in Dutch. See Haralambous' paper in TTN 1, 4.
   An entirely different approach is needed for the Japanese ideograms,
   that is symbols representing things or ideas. At present there are some
   6,353 kanji characters available on various types of computers known
   as JIS level 1 or 2 (Japanese Industrial Standard is akin to ASCII.)
}

However, since \TeX\ is used for more and more applications
the need for more fonts\Dash different shapes, sizes and so on\Dash
has emerged.
Using standard bitmap technology much computer memory is needed.
Reality has it that scaling fonts linearly does not yield
pleasing results.
To compensate for this the intelligent scalable fonts technology emerged%
---near-linear and intelligent, that is with some enhancements---%
as opposed to the classical memory-consuming bitmap fonts, extended by
the linear scaling as such.

Also the mark-up for fonts has gotten a new dimension: the linear space of
available fonts is seen as a 4-dimensional space governed by the coordinates
family, serie, shape, and size. The approach goes with the buzzword NFSS,
New Font Selection Scheme (See Goossens, Mittlebach and Samarin).

\paragraph*{Which fonts can be used with \TeX?}
The following classes of {\em text\/} fonts can be used with \TeX
\begin{itemize}
\item CM, the native Computer Modern
\item 14,000 fonts in industry standard Adobe type 1
\item several hundreds in formats such as TrueType.
\end{itemize}
(Very) few fonts can be used with math,
because of the specialities of the
font characteristics \TeX\ assumes.
However, the following fonts can be used with math
\begin{itemize}
\item CM math, the native Computer Modern
\item lucida math
\item lucida newmath
\item mathtimes.
\end{itemize}
For more details see Horn's Scalable outline fonts paper, and for Japanese
Fujiura in TTN 1, 2.

\section{Descriptive mark-up}
Since the start of computer-assisted typography attention has been paid
to abstraction from details, to the principle of the
{\em separation of concerns}.

Leading in this area is the SGML approach.\footnote{The relation
   between SGML and \TeX\ will be discussed later.}
It is argued that
\begin{quote}
authors should concentrate on the contents\Dash and inherently on the
structure\Dash of their documents, leaving the details for formatting
to the publisher.
\end{quote}

\paragraph*{Example:}(Call for papers, Furuta)
\begingroup\small\begin{verbatim}
\input cfp.tex%contains format and macros
%next copy proper
The aim of this paper...

Paper are solicited on ...
\lstitm Picture editing
\lstitm Text processing
\lstitm Algorithms and software...

Detailed abstracts should not ...

Duration of presentation...
\bye
\end{verbatim}\endgroup
The above example is a mixture of natural input, where blank lines
have an intuitive but context-dependent meaning, and of
handling trivia automatically behind the scenes.
An example of a default is the heading.

For this format the heading is always the same,
so there is no need for a user to provide it each time the format
is used. It comes along with the format.
So do the fonts used and the shortcuts
like \verb|\def\lstitm{\item{--} }|.

My approach looks simpler than Furuta's\Dash in that paper all the low-level
  formatting details were there\Dash
because I applied the principle of the separation of concerns
and abstracted from the low-level formatting details.
The point I'd like to make is that it is possible to hide
formatting details, to account for these separately and at a lower level.
I like to call this approach generic, because the mark-up is customized
at a lower level to the suited tool.

\section{\TeX\ and its author}
Don Knuth started the design of \TeX\ in 1978.
The first major revison  dates back to  1982.
The final version is dated 1989, and called \TeX\ version $\pi$.\footnote{%
   Essentially version 3, but because reality has it that even Knuth
   `makes errors' he allows for adjusted versions denoted by the decimals of
   $\pi$: 3.1, 3.14, 3.141, et cetera.}
It is all a side-step(!) of his magnus opus: The Art of Computer Programming,
of which three volumes have appeared of the envisioned seven.
Because of the rapid development in computer science volume four consists of
three books already.

In designing and developing \TeX,
Knuth adhered to several software engineering paradigms like:
portability, flexibility, robustness, and not to
forget correctness and documentation.\footnote{The software crises of the
   seventies suffered much from inadequate documentation.}
In order to do this gracefully
he coined the words {\em literate programming},
and provided en-passant tools for practical use!
In fact \TeX\ can be seen as
a real-life and significant example of literate programming.

In designing \TeX\ he adopted and developed the following
\begin{itemize}
\item boxes, glue and penalties as building blocks
\item paragraph-wise searching for line-breaks
\item page mapping via the OTR,\footnote{A buzzword to denote the
      output routine which performs this task.}
      optimizing for least penalties
\item device-independent output, to be printed, typeset, or viewed,
      by independent driver programs
\item virtual fonts.
\end{itemize}
\noindent \TeX\ was developed as a side-step. \MF{} can be seen as
an off-off-spring.

\section{\TeX{} and its users}
It is unknown how many people use \AllTeX, and for what purposes.
We know, however, that it is used all over the world, to typset
\begin{itemize}
\item scientific documents, exchange
      and publish such documents\footnote{For an impression of
   published books formatted via \TeX\ see Beebe's bibliography
   in the TUG resource directory.}
\item documents which require special fonts and layout
      conventions, like Japanese, Arabic, Hebrew and so on
\item transparencies and slides
\item material associated with a
      hobby (bridge, chess, crosswords, go, music, and add yours).
\end{itemize}
\noindent A great virtue of the users' action is
\begin{itemize}
\item the porting to various platforms
\item to provide macros, fonts and formats
\item to maintain \LaTeX
\item to ponder about and develop New Typsetting Systems
\item to develop and maintain integrated working environments.
\end{itemize}

\paragraph*{The user groups.}We also know that many users have
organized themselves into user groups,
to start with the original \TeX\ Users Group (TUG),
and more recently into so-called LUGs---language-oriented
local user groups.
The Dutchies are organized since 1988 as the NTG,
Nederlandstalige \TeX\ Gebruikersgroep, that is Dutch language-oriented
\TeX\ Users' Group. We enjoy some 225 members of whom are 30 institutions.
\\
World-wide some  7.5--10k users are organized.
\\
The benefits of being organized, apart from those which come
from cooperation and sharing in general, are
\begin{itemize}
\item meetings
\item TUGboat, newsletter, casu quo bulletins, `specials'
\item resource directory (information about the (La)\TeX\ working environments
      of members, their addresses and similar things)
\item TUGboat styles
\item assistance\\
      -- archives\\
      -- BBS  (Bulletin Board Services)   \\
      -- digests \\
      -- FAQs (Frequently Asked QuestionS)
\item courses
\item PD sets (Public Domain)
\item distributing point books (tutorials), software.
\end{itemize}
\noindent
Moreover, the user groups stimulate and support research and development,
such as  the projects: \TeX HaX, \BibTeX,
and more recently \LaTeX3, and NTS.
>From the social side we have the TUG  bursary fund,
to grant attendence for a TUG meeting for those TUG members who can't
afford it, next to the Knuth Scholarship award. The latter is a competition
which rewards the winner with attending a meeting for free.

\paragraph*{Some addresses?}
\begin{quote}
TUG: Balboa Building, Room 307, 735 State Street, Santa Barbara, Ca 93101, USA,
     {\tt tug@tug.org}\\[1ex]
NTG: Postbus 394, 1740 AJ Schagen,  {\tt ntg@nic.surfnet.nl}.
\end{quote}
For other addresses consult the resource directory of TUG, or
your friendly NTG around the corner.

\subsection{Add-ons}have been provided by the user communities.
They have also supplied mutual support, and have provided logistic facilities.
The latter is not restricted to \AllTeX\ proper.
It is about the general use of the electronic networks
\begin{itemize}
\item exchange via e-mail
\item electronic digests and list servers
\item the file servers, which store all the macro and style files.
\end{itemize}
Really, very nice goodies! The proper add-ons concern
\begin{itemize}
\item porting the complex to every system, especially the affordable
      and widespread PCs
\item macro and style files\footnote{A survey of what is provided is contained
   in the so-called Jones' index, and Beebe's TUGlib.}
\item extra fonts, casu quo font couplings via virtual font scripts
\item WYSIWYG user interfaces (commercial)
\item \TeX-based PD/shareware working environments
\item language-specific issues (hyphenation patterns, reserved words, \ldots)
\item drivers for new printers
\item \PS{} etc.{} inclusion at the dvi level.
\end{itemize}
And the end is not yet in sight.


\section{\TeX{} and the publishers}
The importance of the \AMS{} effort is that the AMS is leading in how (La)\TeX\
can be used cost-effectively as a
high-quality tool in a production environment:
publishers cooperating with authors.

As I understand it the American Physical Society is following
the AMS approach.

At the TUG '91 meeting at Boston, it was estimated that commercial publishers
handle some 5 to 10\% of their (scientific) production via (La)\TeX.

And in the CIS---Commonwealth of Independent States, the former Russia---MIR
has adopted the AMS approach as well.
And then there is the Ukraine group to be founded officially this fall,
and undoubtedly more to follow.
%
\paragraph*{The \AMS}do their
complete production via \TeX: $\approx$100,000 pages/year,
 and provide authors with
\begin{itemize}
\item (generic) styles
\item macros, and fonts
\item user guides
\item support (keyboarding, mark-up, fine-tuning).
\end{itemize}
The approach can be depicted by the following scheme
$$\hbox{\vbox{\lineskip.5\lineskip
\hbox to15ex{\hss author(\TeX)\hss}
\hbox to15ex{\hss$\downarrow$\hss}
\hbox to15ex{\hss\tt amsppt.sty\hss}
\hbox to15ex{\hss$\downarrow$\hss}
\hbox to15ex{\hss\tt amstex.tex\hss}
\hbox to15ex{\hss$\downarrow$\hss}
\hbox to15ex{\hss\TeX\hss}
}\qquad\qquad\qquad\qquad\vbox{\lineskip.5\lineskip
\hbox to15ex{\hss author(\LaTeX)\hss}
\hbox to15ex{\hss$\downarrow$\hss}
\hbox to15ex{\hss\tt amsart.sty\hss}
\hbox to15ex{\hss$\downarrow$\hss}
\hbox to15ex{\hss\llap{{\tt amstex.sty}$\,%
\rightarrow\;$}\LaTeX\hss}
\hbox to15ex{\hss$\downarrow$\hss}
\hbox to15ex{\hss\TeX\hss}
}}$$
They also supply fonts: Euler, Fraktur, \ldots

For more details
consult the AMS sources or see my AMS BLUes paper on the issue.

\paragraph*{The American Physical Society}handle some 20\%
of their production via \LaTeX. They cooperate with The Optical
Society of America and the American Institute of Physics.
Their style is called REV\TeX.

\paragraph*{MIR}publishers Moscow---the driving force
behind CyrTUG, the Cyrillic language-oriented \TeX\
users group\footnote{See also `News about CyrTUG and Russian \TeX\ Users'
   in TTN 2, 1.}---translated Spivak's The Joy of \TeX\
into Russian among others.
I would not be surprised to hear that they do the
production of their scientific documents with \TeX\ too, completely.
They have the knowledge and \TeX nology. And \TeX- and \MF-based
technology does not require much hard currency for investment.

\paragraph*{JTUG?}And what is going on in Japan?
The JTUG has at least .5k members.\footnote{See also `Update of \TeX\ in Japan'
   TTN 1, 2.} They have translated among others the \TeX book and the
   \LaTeX\ manual into Japanese.
Some years ago I received a Japanese newspaper set by J\TeX!

\section{\TeX\ and other EP tools}
Furuta gives a good account of the history and early tools
in relation with computer-assisted typesetting. However, since the appearance
of that paper
\begin{itemize}
\item the laser printer technology has taken off
\item many computer-based fonts have emerged
\item thinking in structures has gotten more interest (SGML)
\item the DTP (Desktop Publishing) credo has come into existence, and
\item hardware prices have continued to spiral down.
\end{itemize}
Everybody can afford a PC, a laser(jet) printer, and some software (especially
Word{\em whatever\/} or the PD \AllTeX). % and publish (or perish). ;-)))

\subsection{\TeX\ and intelligent editors.}
Keyboarding compuscripts in (La)\TeX\ can be assisted by editors which
are (La)\TeX\ intelligent, and next, to use templates as `fill-in' forms.
An example is Beebe's \LaTeX-intelligent emacs.
This approach can prevent errors like the level 1 or so endings, or
non-matching braces and the like.
At this level we can also make use of spelling checkers and style
assistants.

\subsection{Word{\it whatever\/} and \TeX?}
It is true that Word-you-name-it, has made the use of computers more popular.
They replaced the typewriters, don't forget that. And of course that was a
step forward. These are the tools the masses are using because of the
sufficient and improved quality which can be obtained.
This must be seen in context of course: most of the publications
are just in-house reports, memos and the like.

\begin{quote}
For high-quality typesetting a \TeX-like tool,
high-resolution fonts and ipso facto printer, or viewer,
are needed.
\end{quote}
\noindent
Because wordprocessors are so widespread and heavily used, it can be
anticipated that users start from there and
need \TeX's formatting capabilities now and then.
For that group there exist conversion software:
the public domain DRILCON
and the commercial K-Talk.
Simpler, and better when it concerns complex structured copy,
is to
\begin{quote}
output in ASCII from Word{\it whatever\/}
and insert \AllTeX\ mark-up.
\end{quote}
\noindent And, of course, the wordprocessor can always be used as an editor
for \TeX, with taking advantage of the integrated spelling checker.

\subsection{Troff or \TeX?}
Troff preceded \TeX. It comes with UNIX.
Both have been in use for the last decade.
To begin with Knuth built upon troff, scribe and similar tools.
On the other hand the troff add-ons have learned from \TeX.
So there has been mutual influence.

With respect to the functionality the tools are comparable. Both aim at
computer-assisted typography. But there is also a world of difference.
Basically the difference is that troff is a program which can be
extended by independent preprocessors, and \TeX\ is an extensible
language itself, with plain \TeX---the kernel program---device independent,
that is the mapping on the media has to be done by independent drivers.
That the latter was not in troff
can be discerned from the subsequent nroff\Dash with accompanying neqn\Dash
and finally, di-roff, device-independent roff.
Furthermore, remember that \TeX\ is just one of the twins.

Rumour has it that interest in troff weakened because
the early PCs did not come with UNIX, and that
the kernel has remained undocumented (Its author Ossanna
died in an accident.)
The following table is supplied to indicate roughly the differences.

\begingroup\tiny
%\input{btable.tex}
%C.G. van der Laan, Hunzeweg 57, 9893PB, Garnwerd. Holland. 05941-1525.
%btable.tex version 1, 17/7/92                 author: cgl@risc1.rug.nl
\newbox\tbl\let\ea=\expandafter
%Cell vertical size, row height and depth (separation implicit),
\newdimen\cvsize\newdimen\tsht\newdimen\tsdp\newdimen\tvsize\newdimen\thsize
%Parameter setting macros:   Rules
\def\hruled{\def\lineglue{\hrulefill}\def\colsep{}      \def\rowsep{\hrule}
   \let\rowstbsep=\colsep\let\headersep=\rowsep}
\def\vruled{\def\lineglue{\hfil}     \def\colsep{\vrule}\def\rowsep{}
   \let\rowstbsep=\colsep\let\headersep=\hrule}
\def\ruled {\def\lineglue{\hrulefill}\def\colsep{\vrule}\def\rowsep{\hrule}
   \let\rowstbsep=\colsep\let\headersep=\rowsep}
\def\nonruled{\def\lineglue{\hfil}   \def\colsep{}      \def\rowsep{}
   \def\rowstbsep{\vrule}\def\headersep{\hrule}}
\def\dotruled{\def\lineglue{\dotfill}\def\rowsep{\hbox to\thsize{\dotfill}}
\def\colsep{\lower1.5\tsdp\vbox to\cvsize{%
\leaders\hbox to0pt{\vrule height2pt depth2pt width0pt\hss.\hss}\vfil}}
\let\rowstbsep=\colsep\let\headersep=\rowsep}
%Parameter setting macros:   Controling positioning
\def\ctr{\def\lft{\hfil}\def\rgt{\hfil}}%Centered
\def\fll{\def\lft{}     \def\rgt{\hfil}}%Flushed left
\def\flr{\def\lft{\hfil}\def\rgt{}}     %Flushed right
%Parameter setting macros:   Framing
\def\framed{\let\frameit=\boxit}
\def\nonframed{\def\frameit##1{##1}}
\def\dotframed{\let\frameit=\dotboxit}
%
\def\btable#1{\vbox{\let\rsl=\rowstblst%Copy
\ifx\empty\template\ifx\empty\rowstblst
    \def\template{\colsepsurround\lft####\rgt&&\lft####\rgt\cr}
    \else\def\template{\colsepsurround####\hfil&&\lft####\rgt\cr}\fi
   \fi
\tsht=.775\cvsize\tsdp=.225\cvsize
\def\tstrut{\vrule height\tsht depth\tsdp width0pt}
%Logical mark up of column and row separators, via use of
\def\cs{&\colsepsurround\colsep\colsepsurround&}
\def\prs{&\colsepsurround\lineglue&}   \def\srp{&\lineglue\colsepsurround&}
\def\rs{\colsepsurround\tstrut\cr
        \ifx\empty\rowsep\else\noalign{\rowsep}\fi
        \ifx\empty\rowstblst\else\ea\nxtrs\fi}
\def\grs{\colsepsurround\tstrut\cr\ghostrow}
\def\rss{&\colsepsurround\rowstbsep\colsepsurround&}
\def\hs{\colsepsurround\tstrut\cr
       \ifx\empty\headersep\else\noalign{\headersep}\fi
       \ifx\empty\rowstblst\else\ea\nxtrs\fi}
\preinsert
\setbox\tbl=\vbox{\tabskip=0pt\relax\offinterlineskip
\halign{\span\template\ifx\empty\first\ifx\empty\rowstblst\else
\ifx\empty\header\else\ea\rss\fi\fi\else\first\ea\rss\fi
\ifx\empty\header\ifx\empty\first\if\empty\rsl\else\ea\nxtrs\fi
                 \else\ea\hs\fi
\else\header\ea\hs\fi
#1\colsepsurround\tstrut\crcr}    }                              %end \setbox
\postinsert
\ifx\caption\empty\else\hbox to\thsize{\strut\hfil\caption\hss}\captionsep\fi
\frameit{\copy\tbl}
\ifx\footer\empty\else\footersep\hbox{\vtop{\noindent\hsize=\thsize%
\footer}}\fi                     }}                              %end \btable
%Defaults
\cvsize=4ex\tsht=.775\cvsize\tsdp=.225\cvsize\def\colsepsurround{\kern.5em}
\def\caption{}\def\first{}\def\header{}\def\rowstblst{}\def\footer{}\def\data{}
\def\captionsep{\medskip}    \def\headersep{\hrule}
\def\footersep{\smallskip}   \def\rowstbsep{\vrule}
\def\preinsert{}
\def\postinsert{\global\thsize=\wd\tbl
                \global\tvsize=\ht\tbl\global\advance\tvsize by\dp\tbl}
\ctr\nonruled\nonframed\def\template{}\def\ghostrow{}            %end Defaults
%Auxiliaries
\def\boxit#1{\vbox{\hrule\hbox{\vrule\vbox{#1}\vrule}\hrule}}
\def\dotboxit#1{\vbox{\offinterlineskip\hbox to\thsize{\dotfill}%
\hbox{\lower\tsdp\vbox to\tvsize{%
\leaders\hbox to0pt{\hss\vrule height2pt depth2pt width0pt.\hss}\vfil}%
\vbox{#1}\lower\tsdp\vbox to\tvsize{%
\leaders\hbox to0pt{\hss\vrule height2pt depth2pt width0pt.\hss}\vfil}}%
\hbox to\thsize{\dotfill}}}
%And to account for logical columns with \multispan
\def\spicspan{\span\omit}
\def\multispan#1{\omit\mscount=#1\multiply\mscount by2 \advance\mscount by-1
\loop\ifnum\mscount>1 \spicspan\advance\mscount by-1 \repeat}
%To process FIFO, an improved version is available
\def\bfifo#1{\ifx\efifo#1\else\def\nxt{\process#1\bfifo}\ea\nxt\fi}
\def\process#1{\hbox to0pt{\hss#1\hss}\kern.5ex}
%To handle the row stub list: \rsl
\def\nxtrs{\ifx\empty\rsl%\let\nxtel=\relax
\else\def\nxtel{\ea\nrs\rsl\srn}\ea\nxtel\fi}%next Row Stub
\def\nrs#1#2\srn{\gdef\rsl{#2}#1\rss}                        %end btable.tex
%%%%btable end%%%%
%
\def\data{%Costs\cs
PD                   \cs licensed via UNIX\rs
%Availability\cs
all platforms        \cs under UNIX       \rs
%Documentation\cs
\TeX book  (also on-line)\cs On-line manual   \rs
%Fonts        \cs
METAfont's CM, virtual fonts  \cs ?\rs
%Design       \cs
open system          \cs kernel undocumented\rs
%Printers     \cs
device independent   \cs di-roff approach\rs
%Flexibility   \cs
complete             \cs ?\rs
%Extensibility \cs
macros               \cs preprocessors\rs
%Mark-up\cs
formats and styles\cs ms macros\rs
%Coding        \cs
uniform in \WEB{}       \cs C\rs
%Future        \cs
kernel frozen, users augment   \cs frozen\rs
%Acceptance    \cs
users, AMS, APS, \ldots     \cs users, ?\rs
}
\def\header{\AllTeX\cs T/Di-roff}
\def\rowstblst{{%
Costs}{%
Availability}{%
Documentation}{%
Fonts        }{%
Design       }{%
Printers     }{%
Flexibility   }{%
Extensibility }{%
Mark-up       }{%
Coding        }{%
Future        }{{%
Acceptance    }}}
$$\fll\btable\data$$
\endgroup

\subsection{SGML and \TeX?}
SGML stands for Standardized Generalized Mark-up Language.
It is an effort to formalize mark-up, and is defined as a meta-language
to define the mark-up language of each publication series into
so-called Document Type Definitions, DTDs for short.

SGML is part of a huge standardization effort supported by the
US military via the CALS initiative. Other components are: FOSI---Formatted
Output Specification Instance\footnote{See Dobrowolski's paper.}---and
DSSSL.\footnote{See Bryan's paper.}
It is not so much a question of
\begin{quote}\TeX\ {\em or\/} SGML,
but more \TeX\ {\em and\/} SGML.
\end{quote}
\noindent
\TeX\ formats can learn a lot from  the SGML approach and on the other hand
SGML needs a formatter when it is used to  print documents.
This cooperative approach is known as
\begin{quote}
SGML the front-end,
\AllTeX\ the back-end.
\end{quote}
\noindent A diagram about the SGML-\TeX\ relation is
supplied in the accompanying picture.

\newcount\leg
\begin{figure}[hbt]
\begin{center}
\begingroup\small
\setlength{\unitlength}{2.5ex}
\begin{picture}(20,23)(-10, 0)
\put(-7,\the\leg){\framebox(4,2){(La)\TeX}}
\put(-2,\the\leg){\framebox(4,2){TROFF}}
\put(3,\the\leg){\framebox(4,2){\ldots}}
\advance \leg by 4
%hark
\put(-5,\the\leg){\vector(0,-1){1.75}}
\put(0,\the\leg){\vector(0,-1){1.75}}
\put(5,\the\leg){\vector(0,-1){1.75}}
\put(-5,\the\leg){\vector(1,0){5}} %backarrow
\put( 0,\the\leg){\line(1,0){5}}
%converters
\put(5.5,\the\leg){\vtop to 0pt{\hbox{Specific}
                                \hbox{format file}\vss
                                }
                  }
\put(0,\the\leg){\vector(0, 1){1}} %backarrow
\advance \leg by 1
\put(-4,\the\leg){\framebox(8,2){%
   \shortstack[c]{%\footnotesize
                  Generic markup\ \\
                  %\footnotesize
                  $\Rightarrow$\  procedural}    }
                 }
\advance \leg by 2
\put(4.2,\the\leg.2){\line(0, -1){1}}
\put(4.2,\the\leg.2){\line(-1, 0){1}}
\put(4.4,\the\leg.4){\line(0, -1){1}}
\put(4.4,\the\leg.4){\line(-1, 0){1}}
\put(4.6,\the\leg.2){{\tiny Formats}}
%
\advance \leg by 1
\put(0,\the\leg.5){\vector(0,-1){1.5}}
\put(0,\the\leg.5){\vector(0, 1){0}} %backarrow head
%applications
\advance \leg by 1
\put(-10,\the\leg){\makebox(0,0){Exchange}}
\put(-5,\the\leg){\makebox(0,0){Storage}}
\put(0,\the\leg){\makebox(0,0){Publication}}
\put(5,\the\leg){\makebox(0,0){Database}}
\put(10,\the\leg){\makebox(0,0){\vtop to 0pt{\hbox{(Text-)}
                                             \hbox{analysis}\vss}
                                }
                  }
%hark
\advance \leg by 2
\put(0,\the\leg){\vector(0,-1){1.25}}
\put(-5,\the\leg){\vector(0,-1){1.25}}
\put(-10,\the\leg){\vector(0,-1){1.25}}
\put(5,\the\leg){\vector(0,-1){1.25}}
\put(10,\the\leg){\vector(0,-1){1.25}}
\put(-10.5,\the\leg){\line(1,0){21}}
\put(11.25,\the\leg){\makebox(0,0){\dots}}
\put(-11.25,\the\leg){\makebox(0,0){\dots}}
\advance \leg by 2
\put(0,\the\leg){\line(0,-1){2}}
\put(0,\the\leg){\vector(0,1){0}}  %back (up) arrow head
\put(-7.5,\the\leg){\framebox(15,1){Complete, correct SGML
document}}
\advance \leg by 2
\put(.5,\the\leg.25){Parser}
\advance \leg by 2
\put(0,\the\leg){\vector(0,-1){3}}
%\advance \leg by 1
\put(-7.5,\the\leg){\framebox(15,3){
  \shortstack[l]{\verb=<!=SGML     - -declaration - -\verb=>=\\
              \verb=<!=DOCTYPE - - declaration - -\verb=>=\\
              \verb=<!= - - Markup copy - -\verb=>= }
                                    }
                    }
\advance \leg by 3
%\corners
\put(7.7,\the\leg.2){\line(0,-1){2}}
\put(7.7,\the\leg.2){\line(-1,0){2}}
\put(7.9,\the\leg.4){\line(0,-1){2}}
\put(7.9,\the\leg.4){\line(-1,0){2}}
\put(8.1,\the\leg.2){{\tiny DTDs}}
%
\advance \leg by 1
\put(.5,\the\leg){Editor}
\advance \leg by 1
\put(0,\the\leg.5){\vector(0,-1){2.5}}
\advance \leg by 1
\put(0,\the\leg){\framebox(0,0){\strut ``Copy''}}
\end{picture}
\endgroup% \small/\Large
\end{center}
%\caption{Relation SGML and (La)\TeX}
\end{figure}
%
\subsection{SGML and Hypermedia?} The following has been contributed
by Gerard van Nes (from SGML FAQs and
Personal Computer Word, March 1992)
\begin{quote}
`HyTime---Hypermedia/Time-based Structuring Language (ISO/IEC 10744).
HyTime is a standard neutral markup language for representing hypertext,
multimedia, hypermedia and time- and space-based documents in terms of their
logical structure. Its purpose is to make hyperdocuments interoperable
and maintainable over the long term. HyTime can be used to represent
documents containing any combination of digital notations. HyTime is
parsable as Standard Generalized Markup Language.
HyTime was accepted as a full International Standard in spring 1992.

SGML's hypermedia capabilities have been beefed up in the SGML standard
extension HyTime. Although it started out in life as a specific set of
standards for representing music, it was soon realised that these could
be generalised for multimedia. HyTime provides
\begin{itemize}
\item SGML itself
\item Extended Hyperdocument management facilities, including support for
  various types of hyperlink
\item A Coordinate Addressing Facility which positions and synchronises
  on-screen events. This allows authors to specify how hypermedia
  documents are to be rendered
\item Better version-control of comments and activity-tracking policy support.
\end{itemize}
HyTime has been adapted as the basis for hyperlinking in the US
Department of Defense's Interactive Electronic Technical Manual project.
HyTime is an extension of SGML, providing a set of syntactic constructs:
it doesn't specify a processing system.'
\end{quote}
\noindent Sounds very promising!


\subsection{\TeX\ within the context of EP.}
When we think about Electronic Publishing we can't avoid being
aware of the life-cycle of publications.
This obeys the biological invariant: produce, consume and reuse.
\subsection*{Life-cycle: producing.}
The production process has all to do with the dimensions

\begingroup
\setlength{\unitlength}{1ex}
\begin{picture}(18,15)(2, -1)
\put(5,5){\vector(1,0){5}}
\put(5,5){\vector(0,1){5}}
\put(5,5){\vector(-1,-2){2.5}}
%Text
\put(11,5){Place}
\put(6,10){Representation}
\put(4,0){Time}
\end{picture}
\endgroup

\noindent and with the characteristics
\begin{itemize}
\item representation of the contents, that is the typesetting proper aspects
\item logistics, that is distribution and selling points---the place dimension
\item reuse, that is the time aspect, when
      (parts of) document are reused.
\end{itemize}
The flow can be depicted via

$$\vbox{\halign{&\enspace\hfil#\hfil\cr
Produce&$\rightarrow$&Distribute&$\rightarrow$&Consume\cr
$\uparrow$&&$\uparrow$&&$\downarrow$\cr
reuse&$\leftarrow$&retrieve&$\leftarrow$&store\cr}}$$

\noindent
The big features are the unambiguous mark-up of copy via  \AllTeX\
and the lifetime of the \TeX\ kernel. Therefore storing documents
formatted by \TeX, leaves the reuse aspect open. Reality has it that
documents formatted via \TeX\ are easily redistributed via the electronic
networks, because it is all in ASCII, and \TeX\ is everywhere, so are its
drivers.

My day-to-day reuse is transforming reports into articles and these
into transparencies.
In this work it is the other way round I'm recollecting elements I have set
earlier. Similarly with the book I'm working on Publishing with \TeX.
Actually my first work in the document preparation area,
in the early eighties,  was called `Van rapport naar tranparant.'
%
\subsection*{Life-cycle: consuming.}
\TeX's drivers have not paid attention to other representations as yet,
although an exception is a driver for the blind.
Difficulties in formatting languages different from English have been
exercised in recent years. Undoubtedly research will be devoted
to the aspects hinted at in the diagram given below with the dimensions

\begingroup
\hbox{\setlength{\unitlength}{1ex}
\begin{picture}(18,14)(2, -1)
\put(5,5){\vector(1,0){5}}
\put(5,5){\vector(0,1){5}}
\put(5,5){\vector(2,-1){5}}
\put(5,5){\vector(-1,-2){2.5}}
%Text
\put(11,5){Level}
\put(11,2.5){Media}
\put(6,10){Senses}
\put(4,0){Language}
\end{picture}%\quad
\vbox to3.5\baselineskip{\halign{#\hfil:&\enspace#\hfil\cr
Senses& eyes, ears, tactile\cr
Level&abridged, full, \ldots\cr
Language&English, Dutch, \ldots\cr}
\vss}}%
\endgroup

\noindent and with the characteristics
\begin{itemize}
\item choice of consumer language independent of the submitted language,
      that is automatic translation
\item choice of representation, that is for example voice
      output from written submission.
\end{itemize}
\noindent
Of course the above aspects will
keep research busy for some time to come.
This is the direction multi-media development will go.

\section{Trends}
Adobe has been the trendsetter of the last decade with respect to new
EP technologies. Recently, I heard about their
PDF---Portable Document Format---which is at the heart of their
Acrobat. Very promising, if not for the tools which come along
with this product.

I believe that the multi-media information technology will take off in
the next century.
Much is known under the buzzword hypertext.
See the special issue of the Communications of the ACM for
an introductory survey.
As a \TeX ie it is fun to ponder about what niche
there will be for \TeX. At the various TUG meetings people are concerned
about the future of \TeX\ and share their doubts and optimisms.
>From that the following anthology
\begin{itemize}
\item \LaTeX\ is the future, forget about \TeX
\item make \AllTeX\ available on low-cost machines
\item embed \TeX\ etc. in working environments
\item improve \TeX, in short keep it alive
\item provide WYSIWYG user interfaces
\item increase the number of (organized) \AllTeX\ users
\item get \AllTeX\ accepted by publishers (formats, support, fonts,
      and the like)
\item get \AllTeX\ accepted by other communities: SGML,
      scientific societies
\item provide user guides and templates
\item education is paramount
\item keep it simple and small is beautiful.
\end{itemize}
\noindent and so on.
\paragraph*{Prophecy.}
The demand on IT will be that
\begin{quote}
people can access cost-effectively, and easily,
from their homes  the information they need in a representation they wish.
\end{quote}
I envision that the following technologies will influence each other
in realizing the stated prophecy
\begin{itemize}
\item \TeX's role? Embedded in a Hypertext approach?
\item Increased self-publishing
\item Electronic Production \& Consumption \\
      + Photography\\
      + CD\\
      + TV/Radio, video\\
      + PC       \\
      + Phone, fax, email \\
      + Holography \\
      + \ldots
\item Involvement of linguists and behaviourists
\end{itemize}
\noindent with the functionalities
\begin{itemize}
\item Various inputs (o.a. voice, photography, \ldots)
\item Diverse outputs
(language, level, media and representation,\ldots)
\end{itemize}
Some years ago I day-dreamed about holographic-based true 3-D `displays,'
as a generalization of computer-assisted interactive TV.
Science-fiction? Wait and see, or better hang on and make it happen!

\section{Examples}
With a publication we have two main issues:  macroscopic and microscopic.
With the first I mean the aspects which govern the total outer level of
a publication, let us say to look upon it as a tree consisting of
\begin{itemize}
\item front matter (front pages (title etc.), publication characteristics,
                    foreword, table of contents and the like)
\item copy proper  (the chapters and their substructures),
      and
\item back matter  (appendixes with references,  index, and other special
                   items).
\end{itemize}
\noindent These macroscopic aspects are accounted for in so-called formats or
style files.

The microscopic aspects deal with formatting in the small within paragraphs,
the complex mark-up of math, tables and graphics.


Another basic way to look at the matter is that it has all to do with
\begin{quote}
positioning of typographical elements on pages.
\end{quote}
\noindent The following examples,
biased by my own (scientific) needs, are in the main about
\begin{itemize}
\item formats, generic and special
\end{itemize}
\noindent and deal in the small with
\begin{itemize}
\item special texts like programs
\item (displayed) math (formulae, matrices, \ldots)
\item tables
\item graphics
\item bibliographies, and
\item indexes.
\end{itemize}
\noindent So nothing in here about the use of \TeX\ for
non-Latin languages and the design and generation of the needed fonts,
simply because I don't speak them.
I also refrained from including examples about the hobby use---games---%
without a serious reason.
See NTG's PR set for the latter.
See the works of Haralambous with respect to non-Latin languages, and
the work of Horak for (math) \MF{} examples.

\subsection{Examples: formats.}
In this section some detailed formatting examples are provided.

I will consider \LaTeX\ as formatter for a rudimentary
house-style, followed by a generic approach customized to \mm{} and \LaTeX's
report style.

\paragraph*{House-style.}
\LaTeX\  is heavily used for this {\em as-is}
\begingroup\small\begin{verbatim}
\documentstyle[options]{house}
%preamble
\begin{document}
%front matter
\title{...}
...
\begin{abstract}
...
\end{abstract}
\tableofcontents
\listoffigures
\listoftables
%copy proper
%\section, \subsection structuring with
%paragraphs with (displayed) math, tables
%and graphics.
%back matter
\begin{thebibliography}{xxx}
\bibitem{dek84} Knuth, D.E (1984):...
...
\end{thebibliography}
%Index material (\makeindex tool)
\end{document}
\end{verbatim}\endgroup
Options are, for example,
the number of columns,\footnote{It is not true in general that switching from
   1-column into 2-column format can be done without altering the mark-up
   of displayed math, tables or figures.
   At least one must change locally back into
   1-column format, or one has to scale the document element into smaller
   size as was done in this paper.}
the size of the used fonts,
the paper size, and the like.

As style files there are next to report,
the styles book, article, letter, and so on.\footnote{\LaTeX's \SliTeX\
is a bit different. One can't simply switch from report into slides.}


\paragraph*{Generic mark-up.}
Many users start nowadays via \LaTeX. Sooner or later the demand for
a generic approach pops up.
Then the user wishes to
abstract from the concrete formatter and
use some higher-level mark-up for the global structuring commands, customizable
to a concrete formatter of choice.\footnote{This sounds like SGML, but without
   its generality and its overhead. I like to call this `SGML on your mind
   and \TeX\ in your hands.'}

The idea is that the {\em user\/} mark-up at the outer level is
as independent as possible from the concrete formatter.
\begin{quote}
A generic approach is needed because of
the variety of environments we live in
and because of their rapid change.
\end{quote}
For the generic approach to become realistic, and to handle it gracefully,
I assume that
\begin{itemize}
\item the opening part is available for the various formats as
      templates
\item the copy proper uses as structuring commands \verb|\head| and
      the like
\item for the detailed formatting plain \TeX\ is used,
      so that this can be used in
      \AllTeX\ (math, tables, and graphics)
\item for the end matter a generic approach for the bibliography---see
      my  BLUe's Bibliography paper---is used
\item for index preparation a non-specific tool is used.
\end{itemize}
With the above a generic approach for a house-style is
\begingroup\small\begin{verbatim}
%Front matter
\opening%To be replaced by template
%Copy proper
% Structured via \head{...} and the like
% with detailed plain mark-up: math, tables,
% line diagrams,...
%Back matter
 \bibliography
 \index
\closing
\end{verbatim}\endgroup
\vskip1ex
\noindent Customization to \mm.\\
\mm{} is flexible, and alas too much overlooked,
because it lacks a user guide.
Customization of the generic approach to manmac
goes along the following lines to give you an
impression. (Not tested!)
\begingroup\small\begin{verbatim}
 \input manmac
 \input manmac.cus%manmac customization
 \input man.tem   %manmac template
 \input toc       %table of contents
 \input cover     %see my manmac blues
 %Copy proper
 %Back matter
 \closing
\end{verbatim}\endgroup
\noindent with in \verb|manmac.cus|
\begingroup\small\begin{verbatim}
 %Customization of manmac
 %Redefine \beginchapter also non-outer
 \def\beginchapter#1 #2#3.#4\par{%
   \def\hl{\gdef\hl{\issue\hfil\it\rhead}}
   \headline{\hl}
   \def\\{ }\xdef\rhead{#4}
   {\let\\\cr\halign{\line{\titlefont
    \hfil##\hfil}\\#1 #2#3 #4\unskip\\}}
   \bigskip\tenpoint\noindent\ignorespaces}
 \def\endchapter{\vfill\eject}
 %
 \newcnt\chpcnt \newcnt\seccnt
 \def\head#1{\endchapter\beginchapter
     \advance\chpcnt1 \seccnt0
     {} {}\the\chpcnt. #1\endgraf}
 \def\subhead#1{\beginsection\advance\seccnt1
     \the\seccnt. #1\endgraf}
 \def\bibliography{\beginchapter Bibliography
   {}{}.{}\endgraf}
 \def\closing{\bye}
\end{verbatim}\endgroup
\noindent and with in \verb|manmac.tem|
\begingroup\small\begin{verbatim}
\def\opening{
 \def\issue{%
 MAPS Special 93.x            %issue
 }\def\title{%
 MAPS Special Template        %title
 }\def\abstract{%
 A template for MAPS Special is provided.
 }\def\keywords{%
 manmac, MAPS, NTG            %keywords
 }
}
\end{verbatim}\endgroup
\noindent In my Manmac BLUes paper I have worked out a prototype, directed
to customization of \mm.
\begin{quote}
Actually there it was the other way round:
I started from Manmac formatting and abstracted
into independent structures.
\end{quote}
In Manmac BLUes I also worked out \verb|cover|.
Too much detail here.
\vskip1ex
\noindent Customization to \LaTeX.\\
The `title part'-template is  inserted instead of \verb|\opening|,
edited to suit the publication at hand.
In \verb|latex.cus| the macros are supplied to customize the generic
mark-up to \LaTeX.
\begingroup\small\begin{verbatim}
%Begin LaTeX report \opening template
\documentstyle{report}
% LATEX VERSION 2.09 <25 March 1992>
% Copyright (C) 1992 by Leslie Lamport
 
\everyjob{\typeout{LaTeX Version 2.09 <25 March 1992>}}
\immediate\write10{LaTeX Version 2.09 <25 March 1992>}
 
%                 TABLE OF CONTENTS
% COMMAND LIST .........................................  2
% GENERAL CONVENTIONS ..................................  6
% COUNTERS, ETC. .......................................  7
% USEFUL HACKS .........................................  8
% ERROR HANDLING ....................................... 12
% \par AND \everypar ................................... 15
% SPACING / LINE AND PAGE BREAKING ..................... 17
% PROGRAM CONTROL STRUCTURE MACROS ..................... 21
% FILE HANDLING ........................................ 24
% ENVIRONMENT COUNTER MACROS ........................... 27
% PAGE NUMBERING ....................................... 30
% CROSS REFERENCING MACROS  ............................ 31
% ENVIRONMENTS ......................................... 33
% MATH ENVIRONMENTS .................................... 36
% CENTER, FLUSHRIGHT, FLUSHLEFT, ETC. .................. 39
% VERBATIM ............................................. 40
% THE LIST ENVIRONMENT ................................. 41
% ITEMIZE AND ENUMERATE ................................ 49
% BOXES ................................................ 51
% THE TABBING ENVIRONMENT .............................. 57
% ARRAY AND TABULAR ENVIRONMENTS ....................... 63
% THE PICTURE ENVIRONMENT .............................. 72
% THEOREM ENVIRONMENTS ................................. 86
% LENGTHS .............................................. 88
% THE TITLE .............................................89
% SECTIONING ........................................... 90
% TABLE OF CONTENTS, ETC. .............................. 94
% INDEX COMMANDS ....................................... 97
% BIBLIOGRAPHY ......................................... 98
% FLOATS .............................................. 100
% FOOTNOTES ........................................... 106
% INITIAL DECLARATION COMMANDS ........................ 110
% OUTPUT .............................................. 113
% DEBUGGING AND TEST INITIALIZATIONS  ................. 137
 
 
\catcode`\~=13 \def~{\penalty\@M \ }
 
 
%     ****************************************
%     *           COMMAND LIST               *
%     ****************************************
%
% DECLARATIONS:
%  PREAMBLE:     \nofiles \documentstyle \includeonly
%                \makeindex \makeglossary
%  IN DOCUMENT :
%    FONT SELECTION:
%         SIZE: \normalsize \small \footnotesize \scriptsize \tiny
%               \large \Large \LARGE \huge \Huge
%         STYLE: \bf \it \rm \sl \ss \tt \mit[math mode only]
%    STYLE:
%         PAGE: [all global] \pagestyle \thispagestyle \pagenumbering \head
%         MISC: \raggedright \thicklines \thinlines
%    PARAMETER: \setlength \settowidth \addtolength \setcounter \addtocounter
%    NEW:       \newlength \newtheorem \newcommand
%    MISC:      \savebox \sbox \obeycr \restorecr
%
% ENVIRONMENTS:
%    ?   -> PAR:  document
%   PAR  -> PAR:  list enumerate itemize description
%                 center flushright flushleft
%                 verbatim picture float
%   PAR  -> BOX:  tabular tabbing
%   PAR  -> MATH: math displaymath equation
%   MATH -> MATH: array
%   ANY  -> PAR:  minipage
%   ANY  -> BOX:  stack
%
% TEXT-PRODUCING:
%   WITH TEXT ARGUMENT:
%      ANY -> BOX: \makebox \mbox \framebox \fbox \dashbox
%                  \shortstack \footnotemark \cite[] \raisebox
%      ANY -> PAR: \parbox[inner]
%      PAR -> PAR: \chapter \section ... \footnote \footnotetext
%                  \topnewpage \verb
%      MATH:       \sqrt \underline \overline
%      PICTURE:    \put \multiput
%      LIST:       \item
%   WITHOUT TEXT ARGUMENT:
%     ANY MODE:
%          SYMBOLS: \$ \{ \} \_ \@ \& \#
%          ACCENTS: See TeXbook
%          OTHER:   \rule \ref \pageref \today \usebox \typein \input \cite
%     MATH:     \over
%     PAR MODE: \include \bibliography \tableofcontents \listoffigures ...
%     LIST:     \item \arabic \roman \Roman \alph \Alph
%     PICTURE:  \line \vector \circle \oval
%     ARRAY & TABULAR: \hline \vline
%
% SPACING & BREAKING:
%    ANY       : \hfill \hspace
%    PAR       : \newpage \newpage \vspace \noindent
%    PAR & INNER MATH
%              : \newpage \clearpage \cleardoublepage
%              : \pagebreak \nopagebreak \linebreak \nolinebreak \newline
%    MATH      : \over \; \, \!
%    MULTILINE : \\
%    TABBING   : \pushtab \poptab \> \< \+ \- \kill ...
%    ARRAY & TABULAR
%              : \multicolumn \noalign
%
% NO DIRECT CHANGES TO DOCUMENT:
%    \index \glossary \typeout \label \tableentry \stop \protect
%
% PARAMETERS:
%
% \columnsep         \skip\footin        \intextsep
% \columnseprule                         \oddsidemargin
% \columnwidth                           \textfloatsep
% \evensidemargin    \footsep            \textheight
% \floatsep          \headheight         \textwidth
% \headsep            \topmargin
 
 
%   ALPHABETIZED LIST:
%
% ORDINARY COMMANDS:
%
% \Alph                \include             \parbox
% \Roman               \index               \put
% \\                   \item                \raisebox
% \alph                \label               \ref
% \appendix            \line                \roman
% \arabic              \linebreak           \rule
% \bibliography        \listoffigures       \section
% \chapter             \listoftables        \shortstack
% \circle              \makebox             \stop
% \cite                \mbox                \subsection
% \cite                \multicolumn         \subsubsection
% \cleardoublepage     \multiput            \tableentry
% \clearpage                                \tableofcontents
% \dashbox             \newline             \today
% \fbox                \newpage             \typein
% \footnotemark        \noindent            \typeout
% \footnotetext        \nolinebreak         \usebox
% \framebox            \nopagebreak         \vector
% \glossary            \oval                \vline
% \hline               \pagebreak           \vspace
% \hspace              \pageref             \protect
%
%
% ENVIRONMENTS & DECLARATIONS:
%
% For each of these commands, the same command name prefixed by 'end'
% is also reserved--e.g., \enddocument.
%
% \BIG               \footnotesize       \pagestyle
% \Big               \head               \picture
% \addtocounter      \includeonly        \raggedright
% \addtolength       \itemize            \restorecr
% \array             \list               \savebox
% \big               \makeglossary       \sbox
% \center            \makeindex          \scriptscriptsize
% \description       \math               \scriptsize
% \displaymath       \minipage           \setcounter
% \document          \newcommand         \setlength
% \documentstyle     \newlength          \settowidth
% \enumerate         \newtheorem         \small
% \equation          \nofiles            \shortstack\tabbing
%                    \normalsize             \tabular
% \float             \obeycr             \thicklines
% \flushleft         \pagelayout         \thinlines
% \flushright        \pagenumbering      \thispagestyle
%                                        \verb, \verbatim
%
% PARAMETERS :
%
% \columnsep         \footinsertskip     \intextsep
% \columnseprule                         \oddsidemargin
% \columnwidth                           \textfloatsep
% \evensidemargin    \footsep            \textheight
% \floatsep          \headheight         \textwidth
% \headsep            \topmargin
%
%
% TABBING COMMANDS:
%
% These commannds are defined only within a tabbing environment.
%
% \kill      \>    \-
% \pushtab   \<    \=
% \poptab    \+
 
 
% COMPLETE LIST :
% Below is a complete list of every command starting with `\' that
% appears in LATEX.TEX.
 
% \
% \!
% \#
% \$
% \&
% \'
% \(
% \)
% \+
% \,
% \-
% \.
% \:
% \;
% \<
% \=
% \>
% \@
% \@@
% \@@end
% \@@endpbox
% \@@eqncr
% \@@hyph
% \@@input
% \@@par
% \@@sqrt
% \@@startpbox
% \@@underline
% \@@warning
% \@acci
% \@accii
% \@acciii
% \@acol
% \@acolampacol
% \@addamp
% \@addfield
% \@addmarginpar
% \@addtobot
% \@addtocurcol
% \@addtodblcol
% \@addtonextcol
% \@addtopreamble
% \@addtoreset
% \@addtotoporbot
% \@afterheading
% \@afterindentfalse
% \@afterindenttrue
% \@Alph
% \@alph
% \@ampacol
% \@arabic
% \@argarraycr
% \@argdef
% \@argrsbox
% \@argtabularcr
% \@array
% \@arrayacol
% \@arrayclassiv
% \@arrayclassv
% \@arrayclassz
% \@arraycr
% \@arrayparboxrestore
% \@arrayrule
% \@arstrut
% \@arstrutbox
% \@auxout
% \@badcrerr
% \@badend
% \@badlinearg
% \@badmath
% \@badpoptabs
% \@badtab
% \@beginparpenalty
% \@begintheorem
% \@bibitem
% \@biblabel
% \@bitor
% \@botlist
% \@botnum
% \@botroom
% \@bsphack
% \@caption
% \@captype
% \@car
% \@carcube
% \@cclv
% \@cdr
% \@centercr
% \@centering
% \@cfla
% \@cflb
% \@charlb
% \@charrb
% \@chclass
% \@checkend
% \@chnum
% \@circ
% \@circle
% \@circlefnt
% \@cite
% \@citea
% \@citeb
% \@citex
% \@cla         % counter used in \cline
% \@classi
% \@classii
% \@classiii
% \@classiv
% \@classv
% \@classz
% \@clb         % counter used in \cline
% \@cline
% \@clnht
% \@clnwd
% \@clubpenalty
% \@colht
% \@colnum
% \@colroom
% \@combinedblfloats
% \@combinefloats
% \@comdblflelt
% \@comflelt
% \@cons
% \@contfield
% \@ctrerr
% \@curfield
% \@curline
% \@currbox
% \@currentlabel
% \@currentreference
% \@currenvir
% \@currlist
% \@currtype
% \@curtab
% \@curtabmar
% \@dascnt
% \@dashbox
% \@dashcnt
% \@dashdim
% \@dblarg
% \@dbldeferlist
% \@dblfloat
% \@dblfloatplacement
% \@dblfloatsep
% \@dblfpbot
% \@dblfpsep
% \@dblfptop
% \@dblmaxsep
% \@dbltextfloatsep
% \@dbltoplist
% \@dbltopnum
% \@dbltoproom
% \@deferlist
% \@definecounter
% \@defpar
% \@depth
% \@dischyph
% \@doclearpage
% \@documentstyle
% \@doendpe
% \@donoparitem
% \@dot
% \@dotsep
% \@dottedtocline
% \@downline
% \@downvector
% \@eha
% \@ehb
% \@ehc
% \@ehd
% \@elt
% \@empty
% \@endparenv
% \@endparpenalty
% \@endpbox
% \@endpefalse
% \@endpetrue
% \@endtabbing
% \@endtheorem
% \@enumctr
% \@enumdepth
% \@enumspacing
% \@eqncr
% \@eqnnum
% \@eqnsel
% \@eqnswtrue
% \@esphack
% \@Esphack
% \@evenfoot
% \@evenhead
% \@expast
% \@failedlist
% \@fcolmadefalse
% \@filesw
% \@fileswfalse
% \@fileswtrue
% \@firstampfalse
% \@firstamptrue
% \@firstcolumntrue
% \@firsttab
% \@flfail
% \@float
% \@floatpenalty
% \@floatplacement
% \@floatsep
% \@flsucceed
% \@fltovf
% \@flushglue
% \@fnsymbol
% \@footnotemark
% \@footnotetext
% \@for
% \@forloop
% \@fornoop
% \@fpbot
% \@fpmin
% \@fpsep
% \@fptop
% \@framebox
% \@framepicbox
% \@freelist
% \@getcirc
% \@getlarrow
% \@getlinechar
% \@getpen
% \@getrarrow
% \@glossaryfile
% \@gobble
% \@gobblecr
% \@gobbletwo
% \@gtempa
% \@halfwidth
% \@halignto
% \@hangfrom
% \@height
% \@highpenalty
% \@hightab
% \@hline
% \@holdpg
% \@hspace
% \@hspacer
% \@hvector
% \@icentercr
% \@iden
% \@ifatmargin
% \@ifdefinable
% \@ifnch
% \@ifnextchar
% \@iforloop
% \@iframebox
% \@iframepicbox
% \@ifstar
% \@ifundefined
% \@iinput           % used in \input
% \@iirsbox
% \@imakebox
% \@imakepicbox
% \@iminipage
% \@index
% \@indexfile
% \@inlabelfalse
% \@input
% \@inputcheck
% \@insertfalse
% \@inserttrue
% \@iparbox
% \@irsbox
% \@isavebox
% \@isavepicbox
% \@ishortstack
% \@istackcr
% \@itabcr
% \@item
% \@itemdepth
% \@itemfudge
% \@itemitem
% \@itemlabel
% \@itempenalty
% \@itemspacing
% \@iwhiledim
% \@iwhilenum
% \@iwhilesw
% \@ixstackcr
% \@killglue
% \@labels
% \@lastchclass
% \@latexbug
% \@latexerr
% \@lbibitem
% \@leftcolumn
% \@leftmarginskip
% \@leftmark
% \@lhead
% \@linechar
% \@linefnt
% \@linelen
% \@list
% \@listctr
% \@listdepth
% \@listi
% \@listii
% \@listvi
% \@lnbk
% \@lowpenalty
% \@lquote
% \@ltab
% \@M
% \@m
% \@mainaux
% \@mainout
% \@makebox
% \@makecaption
% \@makecol
% \@makefcolumn
% \@makefnmark
% \@makefntext
% \@makeonecolumn
% \@makeother
% \@makepicbox
% \@maketwocolumn
% \@marbox
% \@markright
% \@maxdepth
% \@maxsep
% \@maxtab
% \@medpenalty
% \@Mi
% \@midlist
% \@Mii
% \@Miii
% \@minipagefalse
% \@minipagerestore
% \@Miv
% \@mkboth
% \@mklab
% \@mkpream
% \@MM
% \@mparbottom
% \@mparswitchfalse
% \@mpfn
% \@mpfnnumber
% \@mpfootins
% \@mpfootnotetext
% \@mplistdepth
% \@multicnt
% \@namedef
% \@nameuse
% \@nbitem
% \@ne
% \@negargfalse
% \@negargtrue
% \@newctr
% \@newenv
% \@newline
% \@newlist
% \@newlistfalse
% \@next
% \@nextchar
% \@nextwhile
% \@nil
% \@nmbrlistfalse
% \@nmbrlisttrue
% \@nnil
% \@nobreakfalse
% \@nocnterr
% \@nodocument
% \@nofonterror
% \@noitemargfalse
% \@noitemargtrue
% \@noitemerr
% \@noligs
% \@nolnbk
% \@nolnerr
% \@noparitemfalse
% \@noparitemtrue
% \@noparlistfalse
% \@noparlisttrue
% \@nopgbk
% \@normalcr
% \@normalsize
% \@noskipsecfalse
% \@notdefinable
% \@notprerr
% \@nthm
% \@nxttabmar
% \@oddfoot
% \@oddhead
% \@opargbegintheorem
% \@opcol
% \@optionfiles
% \@optionlist
% \@options
% \@othm
% \@outerparskip
% \@outputbox
% \@outputdblcol
% \@outputpage
% \@oval
% \@ovbtrue
% \@ovdx
% \@ovdy
% \@ovhorz
% \@ovltrue
% \@ovri
% \@ovro
% \@ovrtrue
% \@ovttrue
% \@ovvert
% \@ovxx
% \@ovyy
% \@pagedp
% \@pageht
% \@par
% \@parboxrestore
% \@parmoderr
% \@partaux
% \@partlist
% \@partout
% \@partsw
% \@partswfalse
% \@partswtrue
% \@pboxswfalse
% \@pboxswtrue
% \@pgbk
% \@picbox
% \@picht
% \@picture
% \@pnumwidth
% \@preamble
% \@preamblecmds
% \@preamerr
% \@put
% \@qend
% \@qrelax
% \@reargdef
% \@renewenv
% \@restorepar
% \@reversemarginfalse
% \@reversemargintrue
% \@rhead
% \@rightmark
% \@rightskip
% \@Roman
% \@roman
% \@rsbox
% \@rtab
% \@rule
% \@sanitize
% \@savebox
% \@savemarbox
% \@savepicbox
% \@savsf
% \@savsk
% \@scolelt
% \@sdblcolelt
% \@secpenalty
% \@sect
% \@setpar
% \@settab
% \@sharp
% \@shortstack
% \@sline
% \@spaces
% \@specialoutput
% \@specialpagefalse
% \@specialstyle
% \@sptoken
% \@sqrt
% \@ssect
% \@startcolumn
% \@startdblcolumn
% \@startfield
% \@startline
% \@startpbox
% \@startsection
% \@starttoc
% \@stopfield
% \@stopline
% \@stpelt
% \@svector
% \@sverb
% \@svsec
% \@svsechd
% \@tabacol
% \@tabarray
% \@tabclassiv
% \@tabclassz
% \@tabcr
% \@tablab
% \@tabminus
% \@tabplus
% \@tabpush
% \@tabrj
% \@tabular
% \@tabularcr
% \@temp
% \@tempa
% \@tempb
% \@tempbox
% \@tempboxa
% \@tempc
% \@tempcnta
% \@tempcntb
% \@tempd
% \@tempdima
% \@tempdimb
% \@tempe
% \@tempskipa
% \@tempskipb
% \@tempswa
% \@tempswafalse
% \@tempswatrue
% \@temptokena
% \@testdef
% \@testfp
% \@testpach
% \@textbottom
% \@textfloatsep
% \@textmin
% \@texttop
% \@tfor
% \@tforloop
% \@thanks
% \@thefnmark
% \@thefoot
% \@thehead
% \@themargin
% \@themark
% \@thm
% \@thmcounter
% \@thmcountersep
% \@tocrmarg
% \@toodeep
% \@toplist
% \@topnewpage
% \@topnum
% \@toproom
% \@topsep
% \@topsepadd
% \@totalleftmargin
% \@trivlist
% \@tryfcolumn
% \@trylist
% \@twocolumnfalse
% \@twoside
% \@twosidefalse
% \@typein
% \@upline
% \@upordown
% \@upvector
% \@verb
% \@verbatim
% \@vline
% \@vobeyspaces
% \@vspace
% \@vspacer
% \@vtryfc
% \@vvector
% \@warning
% \@wckptelt
% \@whiledim
% \@whilenoop
% \@whilenum
% \@whilesw
% \@whileswnoop
% \@wholewidth
% \@width
% \@wrindex
% \@writeckpt
% \@writefile
% \@wtryfc
% \@x@sf
% \@xarg
% \@xargarraycr
% \@xarraycr
% \@xbitor
% \@xcentercr
% \@xdblarg
% \@xdblfloat
% \@xdim
% \@xeqncr
% \@xexnoop
% \@xexpast
% \@xfloat
% \@xfootnote
% \@xfootnotemark
% \@xfootnotenext
% \@xhead
% \@xifnch
% \@xmpar
% \@xnewline
% \@xnthm
% \@xobeysp
% \@xsect
% \@xstartcol
% \@xtabcr
% \@xtabularcr
% \@xthm
% \@xtryfc
% \@xtypein
% \@xverbatim
% \@xxxii
% \@xympar
% \@yarg
% \@yargarraycr
% \@ydim
% \@yeqncr
% \@yhead
% \@ympar
% \@ynthm
% \@ythm
% \@ytryfc
% \@yyarg
% \@ztryfc
% \a
% \active
% \addcontentsline
% \addpenalty
% \addtocontents
% \addtocounter
% \addtolength
% \addvspace
% \advance
% \alloc@
% \allocationnumber
% \Alph
% \alph
% \and
% \appendix
% \arabic
% \array
% \arraycolsep
% \arrayrulewidth
% \arraystretch
% \author
% \bar
% \baselineskip
% \begin
% \begingroup
% \bf
% \bgroup
% \bibcite
% \bibdata
% \bibitem
% \bibliography
% \bibliographystyle
% \bibstyle
% \BIG
% \Big
% \big
% \bigskip
% \botfigrule
% \botmark
% \botnum
% \bottomfraction
% \box
% \boxmaxdepth
% \buildrel
% \bullet
% \c@bottomnumber
% \c@chapter
% \c@dbltopnumber
% \c@equation
% \c@eval
% \c@footnote
% \c@mpfootnote
% \c@page
% \c@secnumdepth
% \c@section
% \c@tocdepth
% \c@topnumber
% \c@totalnumber
% \caption
% \catcode
% \catcoded
% \center
% \centering
% \chapter
% \chaptermark
% \char
% \chardef
% \circle
% \cite
% \cl@@ckpt
% \cleardoublepage
% \clearpage
% \cline
% \closeout
% \clubpenalty
% \columnsep
% \columnseprule
% \columnwidth
% \contentsline
% \copy
% \count
% \countdef
% \cr
% \crcr
% \csname
% \dag
% \dagger
% \dashbox
% \date
% \dblfigrule
% \dblfloatpagefraction
% \dblfloatsep
% \dbltexfloatsep
% \dbltextfloatsep
% \dbltopfraction
% \ddagger
% \deadcycles
% \def
% \description
% \dimen
% \dimen@
% \discretionary
% \displaymath
% \displaystyle
% \displaywidth
% \divide
% \do
% \document
% \documentstyle
% \dospecials
% \doublerulesep
% \dp
% \edef
% \egroup
% \else
% \end
% \end@dblfloat
% \end@float
% \endarray
% \endcsname
% \enddocument
% \endenumerate
% \endequation
% \endfigure
% \endgroup
% \enditemize
% \endlist
% \endpicture
% \endsloppypar
% \endtabbing
% \endtabular
% \endthebibliography
% \endtrivlist
% \enumerate
% \eqnarray
% \eqno
% \equation
% \errmessage
% \errorstopmode
% \eval
% \evensidemargin
% \everyjob
% \everypar
% \expandafter
% \extracolsep
% \fbox
% \fboxrule
% \fboxsep
% \fi
% \figure
% \fill
% \firstmark
% \float
% \floatingpenalty
% \floatpagefraction
% \floatsep
% \flushbottom
% \flushleft
% \flushright
% \fnsymbol
% \footins
% \footinsertskip
% \footnote
% \footnotemark
% \footnoterule
% \footnotesep
% \footnotesize
% \footnotetext
% \footsep
% \footskip
% \frac
% \frame
% \framebox
% \frenchspacing
% \fussy
% \futurelet
% \gdef
% \global
% \glossary
% \halfwidth
% \halign
% \hangindent
% \hbox
% \head
% \headheight
% \headsep
% \hfil
% \hfill
% \hfuzz
% \hline
% \hrule
% \hsize
% \hskip
% \hspace
% \hss
% \ht
% \Huge
% \huge
% \hyphenchar
% \if
% \if@afterindent
% \if@eqnsw
% \if@endpe
% \if@fcolmade
% \if@filesw
% \if@firstamp
% \if@firstcolumn
% \if@ignore
% \if@inlabel
% \if@insert
% \if@minipage
% \if@mparswitch
% \if@negarg
% \if@newlist
% \if@nmbrlist
% \if@nobreak
% \if@noitemarg
% \if@noparitem
% \if@noparlist
% \if@noskipsec
% \if@ovb
% \if@ovl
% \if@ovr
% \if@ovt
% \if@pboxsw
% \if@reversemargin
% \if@rjfield
% \if@specialpage
% \if@tempswa
% \if@test
% \if@twocolumn
% \if@twoside
% \ifcase
% \ifdim
% \ifeof
% \ifhmode
% \ifinner
% \ifmmode
% \ifnum
% \ifodd
% \ifvmode
% \ifvoid
% \ifx
% \ignorespaces
% \immediate
% \include
% \includeonly
% \indent
% \index
% \indexentry
% \input
% \insc@unt
% \insert
% \interdisplaylinepenalty
% \interfootnotelinepenalty
% \interlinepenalty
% \intextsep
% \it
% \item
% \itemindent
% \itemize
% \itemsep
% \jobname
% \kern
% \kill
% \label
% \labelenumi
% \labelenumiv
% \labelitemi
% \labelitemii
% \labelitemiii
% \labelitemiv
% \labelsep
% \labelwidth
% \LARGE
% \Large
% \large
% \lastbox
% \lastskip
% \LaTeX
% \lbrace
% \leaders
% \leavevmode
% \lefteqn
% \leftmargin
% \leftmargini
% \leftmarginvi
% \leftmark
% \leftskip
% \let
% \limits
% \line
% \linebreak
% \lineskip
% \linethickness
% \linewidth
% \list
% \listoffigures
% \listoftables
% \listparindent
% \llap
% \long
% \lower
% \m@ne
% \m@th
% \makeatletter
% \makeatother
% \makebox
% \makeglossary
% \makeindex
% \makelabel
% \maketitle
% \marginpar
% \marginparpush
% \marginparsep
% \marginparwidth
% \mark
% \markboth
% \markright
% \math
% \mathchar
% \mathchardef
% \mathop
% \mathrel
% \maxdeadcycles
% \maxdepth
% \maxdimen
% \mb@b
% \mb@eval
% \mb@l
% \mb@r
% \mb@t
% \mbox
% \medskip
% \message
% \minipage
% \mit
% \mkern
% \moveright
% \mskip
% \multicolumn
% \multiply
% \multiput
% \multispan
% \newbox
% \newcommand
% \newcount
% \newcounter
% \newdimen
% \newenvironment
% \newif
% \newinsert
% \newlabel
% \newlength
% \newline
% \newlinechar
% \newpage
% \newsavebox
% \newskip
% \newswitch
% \newtheorem
% \newtoks
% \newwrite
% \noalign
% \nobreak
% \nocite
% \noexpand
% \nofiles
% \noindent
% \nointerlineskip
% \nolinebreak
% \nonumber
% \nopagebreak
% \normalbaselineskip
% \normallineskip
% \normalmarginpar
% \normalsize
% \nullfont
% \number
% \numberline
% \obeycr
% \obeylines
% \obeyspaces
% \oddsidemargin
% \of
% \on@line
% \onecolumn
% \openin
% \or
% \outer
% \output
% \outputpenalty
% \oval
% \over
% \overfullrule
% \overline
% \p@
% \pagebreak
% \pagelayout
% \pagenumbering
% \pageref
% \pagestyle
% \par
% \paragraph
% \parbox
% \parfillskip
% \parindent
% \parsep
% \parshape
% \parskip
% \partopsep
% \partsw
% \penalty
% \picture
% \poptab
% \poptabs
% \postdisplaypenalty
% \prevdepth
% \protect
% \ps@empty
% \ps@plain
% \pushtab
% \pushtabs
% \put
% \quotation
% \raggedbottom
% \raggedleft
% \raggedright
% \raise
% \raisebox
% \rbrace
% \read
% \ref
% \refstepcounter
% \relax
% \renewcommand
% \renewenvironment
% \reset@font
% \restorecr
% \reversemarginpar
% \right
% \rightmargin
% \rightmark
% \rightskip
% \rlap
% \rm
% \Roman
% \roman
% \romannumeral
% \root
% \rule
% \samepage
% \savebox
% \sbox
% \sc
% \scriptscriptsize
% \scriptsize
% \secdef
% \section
% \sectionmark
% \setbox
% \setcounter
% \setlength
% \settowidth
% \shipout
% \shortstack
% \showboxbreadth
% \showboxdepth
% \sixt@@n
% \skip
% \sl
% \SLiTeX
% \sloppy
% \sloppypar
% \small
% \smallskip
% \space
% \spacefactor
% \splitmaxdepth
% \splittopskip
% \sqrt
% \ss
% \stackrel
% \stepcounter
% \stop
% \stretch
% \string
% \strut
% \subsection
% \subsubsection
% \tabalign
% \tabbing
% \tabbingsep
% \tabcolsep
% \tableentry
% \tableofcontents
% \tabskip
% \tabular
% \tencirc
% \tencircw
% \tenln
% \tenlnw
% \textfloatsep
% \textfraction
% \textheight
% \textwidth
% \thanks
% \the
% \thebibliography
% \theenumi
% \theenumii
% \theequation
% \thefigure
% \thefootnote
% \thempfn
% \thempfootnote
% \thepage
% \thesection
% \thicklines
% \thinlines
% \thinspace
% \thispagestyle
% \tiny
% \title
% \today
% \tolerance
% \topfigrule
% \topfraction
% \topmargin
% \topnewpage
% \topnum
% \topsep
% \topskip
% \tracingonline
% \tracingoutput
% \tracingstats
% \trivlist
% \tt
% \tw@
% \twocolumn
% \typein
% \typeout
% \unbox
% \underline
% \unhbox
% \unitlength
% \unskip
% \unvbox
% \usebox
% \usecounter
% \vadjust
% \value
% \vbox
% \vcenter
% \vector
% \verb
% \verbatim
% \vfil
% \vfuzz
% \vline
% \vrule
% \vsize
% \vskip
% \vspace
% \vsplit
% \vss
% \vtop
% \wd
% \write
% \writes
% \xdef
% \z@
% \[
% \\
% \]
% \^
% \_
% \`
% \{
% \|
% \}
% \~
 
 
 
 
%      ****************************************
%      *         GENERAL CONVENTIONS          *
%      ****************************************
%
% THE \LaTeX AND \SLiTeX LOGOS ARE DEFINED HERE.
%
%% RmS 91/09/29: \reset@font added to \LaTeX logo.
\def\p@LaTeX{{\reset@font\rm L\kern-.36em\raise.3ex\hbox{\sc a}\kern-.15em%
    T\kern-.1667em\lower.7ex\hbox{E}\kern-.125emX}}
 
%% RmS 91/09/29: \SLiTeX logo added.
\def\p@SLiTeX{{\reset@font\rm S\kern-.06em{\sc l\kern-.035emi}\kern-.06emT\kern
   -.1667em\lower.7ex\hbox{E}\kern-.125emX}}

%% RmS 91/10/17: \protect'ed the logos
\def\LaTeX{\protect\p@LaTeX}
\def\SLiTeX{\protect\p@SLiTeX}


% SAVED VERSIONS OF TeX PRIMITIVES:
%
%  The TeX primitive \foo is saved as \@@foo .  The following primitives
%  are handled in this way:
 
\let\@@par=\par
%\let\@@relax=\relax  % This was needed at one time, but seems to be obsolete.
\let\@@input=\input
\let\@@end=\end
 
% The following was added 19 April 1986:
% The \- command is redefined to allow it to work in the \tt type style,
% where automatic hyphenation is suppressed by setting \hyphenchar to -1.
% The original definition is saved as \@@hyph just in case anyone needs it.
 
\let\@@hyph=\-        % Original defin
\def\-{\discretionary{-}{}{}}
 
% SAVED VERSIONS OF TeX PARAMETERS
%
%  \normalbaselineskip and \normallineskip hold the
%  normal values of \baselineskip and \lineskip
 
% Any font-changing commands that change the normal value of \lineskip
% and \baselineskip should change their saved values.
 
% The following definitions save token space.  E.g., using \@height
% instead of height saves 5 tokens at the cost in time of one macro
% expansion.
 
\def\@height{height}
\def\@depth{depth}
\def\@width{width}
 
% The following implements the LaTeX \{ and \} commands.
% Changed 21 Apr 87 to make them robust.
 
\def\{{\protect\@lb}
\def\@lb{\relax\ifmmode\lbrace\else$\m@th\lbrace$\fi}
\def\}{\protect\@rb}
\def\@rb{\relax\ifmmode\rbrace\else$\m@th\rbrace$\fi}
 
 \message{counters,}
%      ****************************************
%      *          COUNTERS, ETC.              *
%      ****************************************
%
% THE FOLLOWING ARE FROM PLAIN:
% \z@         : A zero dimen or number.  It's more efficient to write
%               \parindent\z@ than \parindent 0pt.
% \@ne        : The number 1.
% \m@ne       : The number -1.
% \tw@        : The number 2.
% \sixt@@n    : The number 16.
% \@m         : The number 1000.
% \@xxxii     : The number 32
% \@M         : The number 10000.
% \@Mi        : The number 10001.
% \@Mii       : The number 10002.
% \@Miii      : The number 10003.
% \@Miv       : The number 10004.
% \@MM        : The number 20000.
%
% \@flushglue : Glue used for \right- & \leftskip to = 0pt plus 1fil
 
\chardef\@xxxii=32
\mathchardef\@Mi=10001
\mathchardef\@Mii=10002
\mathchardef\@Miii=10003
\mathchardef\@Miv=10004
 
% Redefine PLAIN.TEX macros not to be \outer
 
\def\newcount{\alloc@0\count\countdef\insc@unt}
\def\newdimen{\alloc@1\dimen\dimendef\insc@unt}
\def\newskip{\alloc@2\skip\skipdef\insc@unt}
\def\newbox{\alloc@4\box\chardef\insc@unt}
\def\newwrite{\alloc@7\write\chardef\sixt@@n}
 
\newwrite\@unused
\newcount\@tempcnta
\newcount\@tempcntb
\newif\if@tempswa\@tempswatrue
 
\newdimen\@tempdima
\newdimen\@tempdimb
 
\newbox\@tempboxa
 
\newskip\@flushglue \@flushglue = 0pt plus 1fil
\newskip\@tempskipa
\newskip\@tempskipb
\newtoks\@temptokena
 
 \message{hacks,}
%      ****************************************
%      *            USEFUL HACKS              *
%      ****************************************
%
%  \@namedef{NAME}   : Expands to \def\NAME , except name can contain any
%                      characters.
%  \@nameuse{NAME}   : Expands to \NAME .
%
%  \@ifnextchar X{YES}{NO}
%                    : Expands to YES if next character is an 'X',
%                      and to NO otherwise. (Uses temps a-c.)
%                      NOTE: GOBBLES ANY SPACE FOLLOWING IT.
%
%  \@ifstar{YES}{NO} : Gobbles following spaces and then tests if next the
%                      character is a '*'.  If it is, then it gobbles the
%                      '*' and expands to YES, otherwise it expands to NO.
%
%  \@dblarg{CMD}{ARG}  : \@dblarg{CMD}{ARG} expands to CMD[ARG]{ARG}.  Use
%                        \@dblarg\CS when \CS takes arguments [ARG1]{ARG2},
%                        where default is ARG1 = ARG2.
%
%  \@ifundefined{NAME}{YES}{NO}
%                    : If \NAME is undefined then it executes YES,
%                      otherwise it executes NO.  More precisely,
%                      true if \NAME either undefined or = \relax.
%  \@ifdefinable \NAME {YES}
%                    : Executes YES if the user is allowed to define \NAME,
%                      otherwise it gives an error.  The user can define \NAME
%                      if \@ifundefined{NAME} is true, 'NAME' /= 'relax'
%                      and the first three letters of 'NAME' are not
%                      'end'.
%  \newcommand{\FOO}[i]{TEXT}
%                    : User command to define \FOO to be a macro with
%                      i arguments (i = 0 if missing) having the definition
%                      TEXT.  Produces an error if \FOO already defined.
%
%  \renewcommand{\FOO}[i]{TEXT} : Same as \newcommand, except it
%                      checks if \FOO already defined.
%
%  \newenvironment{FOO}[i]{DEF1}{DEF2}
%         equivalent to
%         \newcommand{\FOO}[i]{DEF1} \def{\endFOO}{DEF2}
%
%  \renewenvironment : obvious companion to \newenvironment
%
%  \@cons : See description of \output routine.
%
%  \@car T1 T2 ... Tn\@nil == T1  (unexpanded)
%
%  \@cdr T1 T2 ... Tn\@nil == T2 ... Tn     (unexpanded)
%
%  \typeout{message} : produces a warning message on the terminal
%
%  \@warning{message}: prints 'LaTeX Warning: message.'
%                      With TeX 3.x, it also prints line number.
%                      (Changed 24 Jun 91 RmS)
%  \@@warning{message}: like \@warning, except that it never prints
%                      the line number (added 24 Jun 91 RmS).
%
%  \typein{message}  : Types message, asks the user to type in a command, then
%                      executes it
%
%  \typein[\CS]{MSG} : Same as above, except defines \CS to be the input
%                      instead of executing it.

%% RmS 92/03/18: changed input channel from 0 to \@inputcheck to avoid conflicts
%%               with other channels allocated by \newread
\newread\@inputcheck
\def\typein{\let\@typein\relax\@ifnextchar[{\@xtypein}{\@xtypein[\@typein]}}
\def\@xtypein[#1]#2{\typeout{#2}\read\@inputcheck to#1\ifx #1\@defpar \def#1{}\else
   \@iden{\expandafter\@strip\expandafter
   #1#1\@gobble\@gobble} \@gobble\fi\@typein}
\def\@strip#1#2 \@gobble{\def #1{#2}}
\def\@defpar{\par}
\def\@iden#1{#1}
 
\ifx\inputlineno\undefined
  \let\on@line\empty
\else
  \ifnum\inputlineno=\m@ne
    \let\on@line\empty
  \else
    \def\on@line{ on input line \the\inputlineno}
  \fi
\fi

\def\typeout#1{{\let\protect\string\immediate\write\@unused{#1}}}
\def\@@warning#1{\typeout{LaTeX Warning: #1.}}
\def\@warning#1{\@@warning{#1\on@line}}
\def\@namedef#1{\expandafter\def\csname #1\endcsname}
\def\@nameuse#1{\csname #1\endcsname}
 
\def\@cons#1#2{\begingroup\let\@elt\relax\xdef#1{#1\@elt #2}\endgroup}
 
\def\@car#1#2\@nil{#1}
\def\@cdr#1#2\@nil{#2}
 
% \@carcube T1 ... Tn\@nil = T1 T2 T3 , n > 3
\def\@carcube#1#2#3#4\@nil{#1#2#3}
 
\def\newcommand#1{\@ifnextchar [{\@argdef#1}{\@argdef#1[0]}}
 
\def\renewcommand#1{\edef\@tempa{\expandafter\@cdr\string
  #1\@nil}\@ifundefined{\@tempa}{\@latexerr{\string#1\space undefined}\@ehc
    }{}\@ifnextchar [{\@reargdef#1}{\@reargdef#1[0]}}
 
\def\newenvironment#1{\@ifnextchar
     [{\@newenv{#1}}{\@newenv{#1}[0]}}
 
\long\def\@newenv#1[#2]#3#4{\expandafter\newcommand
     \csname #1\endcsname[#2]{#3}\long
     \expandafter\def\csname end#1\endcsname{#4}}
 
\def\renewenvironment#1{\@ifnextchar
     [{\@renewenv{#1}}{\@renewenv{#1}[0]}}
 
\long\def\@renewenv#1[#2]#3#4{\expandafter\renewcommand
     \csname #1\endcsname[#2]{#3}\long
     \expandafter\def\csname end#1\endcsname{#4}}
 
\long\def\@argdef#1[#2]#3{\@ifdefinable #1{\@reargdef#1[#2]{#3}}}
 
% Absolutely untypable control sequence \@?@?  substituted for \@tempb in
% definition of \@reargdef because it (and therefore \newcommand and
% \renewcommand) leaves the control sequence dangerously \let to #.
% (Change made 23 November 87.)
%
\catcode`\?=11\relax
\long\def\@reargdef#1[#2]#3{\@tempcnta#2\relax\let#1\relax
\edef\@tempa{\long\def#1}\@tempcntb \@ne
\let\@?@?\relax\@whilenum\@tempcnta>\z@
\do{\edef\@tempa{\@tempa\@?@?\the\@tempcntb}\advance\@tempcntb \@ne \advance
\@tempcnta \m@ne}\let\@?@?##\@tempa{#3}}
\catcode`\?=12\relax
 
 
% 9 Jan 90 : Missing % added to following definition.
\long\def\@ifdefinable #1#2{\edef\@tempa{\expandafter\@cdr\string #1\@nil}%
\@ifundefined{\@tempa}{\edef\@tempb{\expandafter\@carcube \@tempa xxxx\@nil}%
\ifx \@tempb\@qend \@notdefinable\else
\ifx \@tempa\@qrelax \@notdefinable\else  #2\fi\fi}{\@notdefinable}}
 
\long\def\@ifundefined#1#2#3{\expandafter\ifx\csname
  #1\endcsname\relax#2\else#3\fi}
 
 
% The following define \@qend and \@qrelax to be the strings 'end' and
% 'relax' with the characters \catcoded 12.
 
\edef\@qend{\expandafter\@cdr\string\end\@nil}
\edef\@qrelax{\expandafter\@cdr\string\relax\@nil}
 
% \@ifnextchar X{YES}{NO}
%  BEGIN
%    \@tempe := X  % uses \let
%    \@tempa := YES
%    \@tempb := NO
%    \futurelet\@tempc
%    \@ifnch
%  END
%
% \@ifnch ==
%   BEGIN
%     if  \@tempc = blank space
%       then  \@tempd := def(\@xifnch)
%       else  if  \@tempc = \@tempe
%                then  \@tempd := def(\@tempa)
%                else  \@tempd := def(\@tempb)
%             fi
%     fi
%     \@tempd
%   END
%
% \@xifnch ==
%  BEGIN
%    gobble blanks
%    \futurelet\@tempc
%    \@ifnch
%  END
%
\def\@ifnextchar#1#2#3{\let\@tempe #1\def\@tempa{#2}\def\@tempb{#3}\futurelet
    \@tempc\@ifnch}
\def\@ifnch{\ifx \@tempc \@sptoken \let\@tempd\@xifnch
      \else \ifx \@tempc \@tempe\let\@tempd\@tempa\else\let\@tempd\@tempb\fi
      \fi \@tempd}
 
% NOTE: the following hacking must precede the definition of \:
%  as math medium space.
 
\def\:{\let\@sptoken= } \:  % this makes \@sptoken a space token
 
\def\:{\@xifnch} \expandafter\def\: {\futurelet\@tempc\@ifnch}
 
\def\@ifstar#1#2{\@ifnextchar *{\def\@tempa*{#1}\@tempa}{#2}}
 
\long\def\@dblarg#1{\@ifnextchar[{#1}{\@xdblarg{#1}}}
\long\def\@xdblarg#1#2{#1[{#2}]{#2}}
 
% The command \@sanitize changes the catcode of all special characters
% except for braces to 'other'.  It can be used for commands like
% \index that want to write their arguments verbatim.  Needless to
% say, this command should only be executed within a group, or chaos
% will ensue.
 
\def\@sanitize{\@makeother\ \@makeother\\\@makeother\$\@makeother\&%
\@makeother\#\@makeother\^\@makeother\_\@makeother\%\@makeother\~}
 
 
 \message{errors,}
%      ****************************************
%      *           ERROR HANDLING             *
%      ****************************************
%
%  \@latexerr{MSG}{HLP}: Types a LaTeX error message MSG and gives an error
%                          halt with error help message HLP.
%
\newlinechar`\^^J
 
% 19 Jun 86, took out the grouping. re: John Hobby
\def\@latexerr#1#2{%
\edef\@tempc{#2}\errhelp\expandafter{\@tempc}%
\typeout{LaTeX error. \space See LaTeX manual for explanation.^^J
 \space\@spaces\@spaces\@spaces Type \space H <return> \space for
 immediate help.}\errmessage{#1}}
 
\def\@spaces{\space\space\space\space}
 
%% error help message pieces.
\def\@eha{Your command was ignored.
^^JType \space I <command> <return> \space to replace it
  with another command,^^Jor \space <return> \space to continue without it.}
\def\@ehb{You've lost some text. \space \@ehc}
\def\@ehc{Try typing \space <return>
  \space to proceed.^^JIf that doesn't work, type \space X <return> \space to
  quit.}
\def\@ehd{You're in trouble here.  \space\@ehc}
 
% Here are all the error message-generating commands of LaTeX.
%
% \@notdefinable : Error message generated in \@ifdefinable from calls
%                  by \newcommand, \newlength, \newtheorem specifying an
%                  already-defined command name.
%
% \@nolnerr      : Generated by \newline and \\ when called in vertical mode.
%
% '\... undefined' : Generated in \renewcommand.
%
% \@nocnterr     : Generated by \setcounter, \addtocounter or \newcounter
%                  for undefined counter.
%
% \@ctrerr       : Called when trying to print the value of a counter
%                  numbered by letters that's greater than 26.
%
% 'Environment --- undefined' : Issued by \begin for undefined environment.
%
% \@badend       : Called by \end that doesn't match its \begin.
%
% \@badmath      : Called by \[, \], \( or \) when used in wrong mode.
%
% \@toodeep      : Called by a list environment nested more than six levels
%                  deep, or an enumerate or itemize nested more than four
%                  levels.
%
% \@badpoptabs   : Called by \endtabbing when not enough \poptabs have
%                  occurred, or by \poptabs when too many have occurred.
%
% \@badtab : Called by \>, \+ , \- or \< when stepping to an undefined tab.
%
% 'tab overflow' : Occurs in \= when maximum number of tabs exceeded.
%
% '\< in mid line' : Occurs in \< when it appears in middle of line.
%
% \@preamerr : Occurs in array or tabular environment, or in \multicolumn
%              command, when error in argument detected.
%
% \@badlinearg : Occurs in \line and \vector command when a bad slope
%                argument is encountered.
%
% \@parmoderr  : Occurs in a float environment or a \marginpar when
%                encountered in inner vertical mode.
%
% \@fltovf     : Occurs in float environment or \marginpar when there
%                are no more free boxes for storing floats.
%
% \@latexbug   :  Occurs in output routine.  This is bad news.
%
% 'Float(s) lost' : In output routine, caused by a float environment or
%                   \marginpar occurring in inner vertical mode.
%
% \@nofonterror   : Typeface not available.  %%% OBSOLETE; DELETED.
%
% \@badcrerr      : A \\ used where it shouldn't be in a centering or flushing
%                   environment.
%
% \@noitemerr     : \addvspace or \addpenalty was called when not in vmode.
%                   Probably caused by a missing \item.
%
% \@notprerr      : A command that can be used only in the preamble
%                   appears after the \begin{document} command.
 
\def\@notdefinable{\@latexerr{Command name '\@tempa' already used}\@eha}
 
\def\@nolnerr{\@latexerr{There's no line here to end}\@eha}
 
\def\@nocnterr{\@latexerr{No such counter}\@eha}
 
\def\@ctrerr{\@latexerr{Counter too large}\@ehb}
 
\def\@nodocument{\@latexerr{Missing \string\begin{document}}\@ehd}
 
\def\@badend#1{\@latexerr{\string\begin{\@currenvir} ended by
      \string\end{#1}}\@eha}
 
\def\@badmath{\@latexerr{Bad math environment delimiter}\@eha}
 
\def\@toodeep{\@latexerr{Too deeply nested}\@ehd}
 
\def\@badpoptabs{\@latexerr{\string\pushtabs \space and \string\poptabs
    \space don't match}\@ehd}
 
\def\@badtab{\@latexerr{Undefined tab position}\@ehd}
 
\def\@preamerr#1{\@latexerr{\ifcase #1 Illegal character\or
     Missing @-exp\or Missing p-arg\fi\space
     in array arg}\@ehd}
 
\def\@badlinearg{\@latexerr{Bad \string\line\space or \string\vector
   \space argument}\@ehb}
 
\def\@parmoderr{\@latexerr{Not in outer par mode}\@ehb}
 
\def\@fltovf{\@latexerr{Too many unprocessed floats}\@ehb}
 
\def\@latexbug{\@latexerr{This may be a LaTeX bug}{Call for help}}
 
% \def\@nofonterror{\@latexerr{Typeface not available}\@eha}
 
\def\@badcrerr {\@latexerr{Bad use of \string\\}\@ehc}
 
\def\@noitemerr{\@latexerr{Something's wrong--perhaps a missing
\string\item}\@ehc}
 
\def\@notprerr {\@latexerr{Can be used only in preamble}\@eha}
 
 \message{par,}
%       ****************************************
%       *          \par AND \everypar          *
%       ****************************************
%
% There are two situations in which \par may be changed:
%
%   - Long-term changes, in which the new value is to remain in effect
%     until the current environment is left.  The environments that
%     change \par in this way are the following:
%
%         * All list environments (itemize, quote, etc.)
%         * Environments that turn \par into a noop:
%              tabbing, array and tabular.
%
%   - Temporary changes, in which \par is restored to its previous value the
%     next time it is executed. The following are all such uses.
%         * \end [when preceded by \@endparenv, which is called by
%                 \endtrivlist]
%         * The mechanism for avoiding page breaks and getting the
%           spacing right after section heads.
%
% To permit the proper interaction of these two situations, long-term
% changes are made by the following command:
%     \@setpar{VAL}      : To set \par. It \def's \par and \@par to VAL.
% Short-term changes are made by the usual \def\par commands.
% The original values are restored after a short-term change
% by the \@restorepar commands.
%
% NOTE: \@@par always is defined to be the original TeX \par.
%
% \everypar is changed only for the short term.  Whenever \everypar
% is set non-null, it should restore itself to null when executed.
% The following commands change \everypar in this way:
%         * \item
%         * \end [when preceded by \@endparenv, which is called by
%                 \endtrivlist]
%         * \minipage
%
% WARNING: Commands that make short-term changes to \par and \everypar
% must take account of the possibility that the new commands and the
% ones that do the restoration may be executed inside a group.  In
% particular, \everypar is executed inside a group whenever a new paragraph
% begins with a left brace.  The \everypar command that restores its
% definition should be local to the current group (in case the command
% is inside a minipage used inside someplace where \everypar has been
% redefined).  Thus, if \everypar is redefined to do an \everypar{}
% it could take several executions of \everypar before
% the restoration 'holds'.  This usually causes no problem.  However, to
% prevent the extra executions from doing harm, use a global switch
% to keep anything harmful in the new \everypar from being done twice.
%
% WARNING: Commands that change \everypar should remember that \everypar
% might be supposed to set the following switches false:
%              @nobreak
%              @minipage
% they should do the setting if necessary.
 
\def\@par{\let\par\@@par\par}
 
\def\@setpar#1{\def\par{#1}\def\@par{#1}}
\def\@restorepar{\def\par{\@par}}
 
 \message{spacing,}
%      **********************************************
%      *     SPACING / LINE AND PAGE BREAKING       *
%      **********************************************
%
% USER COMMANDS:
% \nopagebreak[i] : i = 0,...,4.  Default argument = 4.  Puts a penalty
%                 into the vertical list output as follows:
%                   0 : penalty = 0
%                   1 : penalty = \@lowpenalty
%                   2 : penalty = \@medpenalty
%                   3 : penalty = \@highpenalty
%                   4 : penalty = 10000
% \pagebreak[i]   : same as \nopagebreak except negatives of its penalty
% \linebreak[i], \nolinebreak[i] : analogs of the above
% \samepage : inhibits page breaking most places by setting the following
%             penalties to 10000
%                    \interlinepenalty
%                    \postdisplaypenalty
%                    \interdisplaylinepenalty
%                    \@beginparpenalty
%                    \@endparpenalty
%                    \@itempenalty
%                    \@secpenalty
%                    \interfootnotelinepenalty
%
% \obeycr    : defines <CR> == \\.
% \restorecr : restores <CR> to its usual meaning.
%
% \\         : initially defined to be \newline
% \\[LENGTH] : initially defined to be \vspace{LENGTH}\newline
%              Note: \\* adds a \vadjust{\penalty 10000}
 
\def\nopagebreak{\@ifnextchar[{\@nopgbk}{\@nopgbk[4]}}
\def\@nopgbk[#1]{\ifvmode \penalty \@getpen{#1}\else
\@bsphack\vadjust{\penalty \@getpen{#1}}\@esphack\fi}
 
\def\pagebreak{\@ifnextchar[{\@pgbk}{\@pgbk[4]}}
\def\@pgbk[#1]{\ifvmode \penalty -\@getpen{#1}\else
\@bsphack\vadjust{\penalty -\@getpen{#1}}\@esphack\fi}
 
\def\nolinebreak{\@ifnextchar[{\@nolnbk}{\@nolnbk[4]}}
\def\@nolnbk[#1]{\ifvmode \@nolnerr\else \@tempskipa\lastskip
     \unskip \penalty \@getpen{#1}\ifdim \@tempskipa >\z@
     \hskip\@tempskipa\ignorespaces\fi\fi}
 
\def\linebreak{\@ifnextchar[{\@lnbk}{\@lnbk[4]}}
\def\@lnbk[#1]{\ifvmode \@nolnerr\else
     \unskip\penalty -\@getpen{#1}\fi}
 
\def\samepage{\interlinepenalty\@M
   \postdisplaypenalty\@M
   \interdisplaylinepenalty\@M
   \@beginparpenalty\@M
   \@endparpenalty\@M
   \@itempenalty\@M
   \@secpenalty\@M
   \interfootnotelinepenalty\@M}
 
% \nobreak added to \newline to prevent null lines when \newline
% ends an overfull line.  Change made 24 May 89 as suggested by
% Frank Mittelbach and Rainer Sch\"opf
%
\def\newline{\ifvmode \@nolnerr \else \unskip\nobreak\hfil
  \penalty -\@M\fi}
 
 
\def\@normalcr{\@ifstar{\vadjust{\penalty\@M}\@xnewline}{\@xnewline}}
 
\def\@xnewline{\@ifnextchar[{\@newline}{\newline}}
 
\def\@newline[#1]{\ifhmode\unskip\fi\vspace{#1}\newline}
 
\let\\=\@normalcr
 
\def\@getpen#1{\ifcase #1 0 \or \@lowpenalty\or
         \@medpenalty \or \@highpenalty
         \else \@M \fi}
 
% @nobreak : Switch used to avoid page breaks caused by \label after a section
%            heading, etc. It should be GLOBALLY set true after the \nobreak
%            and GLOBALLY set false by the next invocation of \everypar.
%            Commands that reset \everypar should globally set it false
%            if appropriate.
%
\newif\if@nobreak \@nobreakfalse
 
% \@bsphack ... \@esphack
%     used by macros such as \index and \begin{@float} ... \end{@float}
%     that want to be invisible -- i.e.,
%     not leave any extra space when used in the middle of text.  Such
%     a macro should begin with \@bsphack and end with \@esphack
%     The macro in question should not create any text, nor change the
%     mode.
%
% \@Esphack is a variant of \@esphack that sets the @ignore switch to true
%     (as \@esphack used to do previously). This is currently used only
%     for float and similar environments.
%
% \@bsphack ==
%  BEGIN
%   if not mmode then                 %% Test for math mode added 18 Dec 89
%         \dimen\@savsk := \lastskip
%       if  hmode  then  \@savsf := \spacefactor  fi
%   fi
%  END
%
% \@esphack ==
%  BEGIN
%   if not mmode then                 %% Test for math mode added 18 Dec 89
%    if  hmode
%      then  \spacefactor := \@savsf
%            if \dimen\@savsk > 0pt  then  \ignorespaces fi
%    fi
%   fi
%  END
%
% \@Esphack ==
%  BEGIN
%   if not mmode then
%    if  hmode
%      then  \spacefactor := \@savsf
%            if \dimen\@savsk > 0pt  then  \ignorespaces
%                                          \global\@ignoretrue   fi
%    fi
%   fi
%  END
%
 
\newdimen\@savsk
\newcount\@savsf
 
\def\@bsphack{\relax\ifmmode\else\@savsk\lastskip
    \ifhmode\@savsf\spacefactor\fi\fi}

\def\@esphack{\relax\ifmmode\else\ifhmode\spacefactor\@savsf
     {}\ifdim \@savsk >\z@ \ignorespaces
  \fi \fi\fi}

\def\@Esphack{\relax\ifmmode\else\ifhmode\spacefactor\@savsf
     {}\ifdim \@savsk >\z@ \global\@ignoretrue \ignorespaces
  \fi \fi\fi}

% VERTICAL SPACING:
%
% LaTeX supports the PLAIN TeX commands \smallskip, \medskip and \bigskip.
% However, it redefines them using \vspace instead of \vskip.
%
% Extra vertical space is added by the command command \addvspace{SKIP},
% which adds a vertical skip of SKIP to the document.  The sequence
%         \addvspace{S1} \addvspace{S2}
% is equivalent to
%         \addvspace{maximum of S1, S2}.
% \addvspace should be used only in vertical mode, and gives an error if it's
% not.  The \addvspace command does NOT add vertical space if
% @minipage = T. The minipage environment uses this to inhibit
% the addition of extra vertical space at the beginning.
%
% Penalties are put into the vertical list with the \addpenalty{PENALTY}
% command.  It works properly when \addpenalty and \addvspace commands
% are mixed.
%
% The @nobreak switch is set true used when in vertical mode and no page
% break should occur.  (Right now, it is used only by the section heading
% commands to inhibit page breaking after a heading.)
%
%
% \addvspace{SKIP} ==
%  BEGIN
%   if vmode
%     then if @minipage
%            else if \lastskip =0
%                    then  \vskip SKIP
%                    else  if \lastskip < SKIP
%                             then  \vskip -\lastskip
%                                   \vskip SKIP
%                             else if SKIP < 0 and \lastskip >= 0
%                                    then \vskip -\lastskip
%                                         \vskip \lastskip + SKIP
%          fi      fi       fi      fi
%     else 'missing \item' error.
%   fi
%  END
 
\def\addvspace#1{\ifvmode
     \if@minipage\else
          \ifdim \lastskip =\z@ \vskip #1\relax
             \else \@tempskipb#1\relax\@xaddvskip
     \fi\fi
  \else\@noitemerr\fi}
 
\def\@xaddvskip{\ifdim \lastskip <\@tempskipb\vskip-\lastskip\vskip
             \@tempskipb
        \else  \ifdim \@tempskipb<\z@
                 \ifdim \lastskip <\z@
                    \else \advance\@tempskipb\lastskip
                           \vskip -\lastskip \vskip \@tempskipb
      \fi\fi\fi}
 
\def\addpenalty#1{\ifvmode
   \if@minipage\else\if@nobreak\else
      \ifdim\lastskip=\z@ \penalty#1\relax
         \else \@tempskipb\lastskip
               \vskip -\lastskip \penalty#1\vskip\@tempskipb
      \fi\fi\fi
   \else\@noitemerr\fi}
 
\def\vspace{\@ifstar{\@vspacer}{\@vspace}}
\def\@vspace#1{\ifvmode
    \dimen@\prevdepth \vskip #1\vskip\z@ \prevdepth\dimen@
       \else
        \@bsphack\vadjust{\dimen@\prevdepth
            \vskip #1\vskip\z@ \prevdepth\dimen@}\@esphack\fi}
\def\@vspacer#1{\ifvmode \dimen@\prevdepth
         \hrule \@height\z@ \nobreak \vskip #1\vskip\z@
           \prevdepth\dimen@
        \else
         \@bsphack\vadjust{\dimen@\prevdepth \hrule \@height\z@ \nobreak
            \vskip #1\vskip\z@ \prevdepth\dimen@}\@esphack\fi}
 
\def\smallskip{\vspace\smallskipamount}
\def\medskip{\vspace\medskipamount}
\def\bigskip{\vspace\bigskipamount}
 
 
% See list environment for explanation of the following macros.
 
\def\endtrivlist{\if@newlist\@noitemerr\fi
   \if@inlabel\indent\fi
   \ifhmode\unskip \par\fi
   \if@noparlist \else
      \ifdim\lastskip >\z@ \@tempskipa\lastskip \vskip -\lastskip
         \advance\@tempskipa\parskip \advance\@tempskipa -\@outerparskip
         \vskip\@tempskipa
   \fi\@endparenv\fi}
 
% CHANGES TO \@endparenv:
% Changed  \hskip -\parindent  to  \setbox0=\lastbox  so a \noindent
% becomes a no-op when used before a line immediately following a
% list environment.  (Changed 23 Oct 86)
%
% To suppress the paragraph indentation in text immediately following
% a paragraph-making environment, \everypar is changed to remove the
% space, and \par is redefined to restore \everypar.  Instead of redefining
% \par and \everpar, \@endparenv was changed to set the @endpe switch,
% letting \end redefine \par and \everypar.  This allows paragraph-
% making environments work right when called by other environments.
% (Changed 27 Oct 86)
 
\def\@endparenv{\addpenalty\@endparpenalty\addvspace\@topsepadd\@endpetrue}
 
\def\@doendpe{\@endpetrue
     \def\par{\@restorepar\everypar{}\par\@endpefalse}\everypar
               {\setbox\z@\lastbox\everypar{}\@endpefalse}}
 
\newif\if@endpe
\@endpefalse
 
% HORIZONTAL SPACE
%
% \, : used in paragraph mode produces a \thinspace.  It has the ordinary
%      definition in math mode.  Useful for quotes inside quotes, as in
%      ``\,`Foo', he said.''
%
% \@ : placed before a '.', makes it a sentence-ending period.  Does the
%     right thing for other punctuation marks as well.  Does this by
%     setting spacefactor to 1000.
 
\def\,{\protect\pcomma}
\def\pcomma{\relax\ifmmode\mskip\thinmuskip\else\thinspace\fi}
 
 
\def\@{\spacefactor\@m}
 
\def\hspace{\protect\phspace}
\def\phspace{\@ifstar{\@hspacer}{\@hspace}}
\def\@hspace#1{\leavevmode\hskip #1\relax}
 
\def\@hspacer#1{\leavevmode\vrule \@width\z@\nobreak
                \hskip #1\hskip \z@skip}
                      %% extra \hskip 0pt added 12/17/85 to guard
                      %% against a following \unskip
                      %% \relax added 13 Oct 88 for usual TeX lossage
                      %% replaced both changes by \hskip\z@skip 27 Nov 91
 
% define \fill to = 0pt plus 1fill
\newskip\fill \fill = 0pt plus 1fill
 
% \stretch{N} == 0pt plus N fill
\def\stretch#1{\z@ plus #1fill\relax}
 
{\catcode`\^^M=13 \gdef\obeycr{\catcode`\^^M=13 \def^^M{\\}\@gobblecr}%
\gdef\restorecr{\catcode`\^^M=5 }} %} BRACE MATCHING
 
 
 \message{control,}
%      **********************************************
%      *     PROGRAM CONTROL STRUCTURE MACROS       *
%      **********************************************
%
% \@whilenum TEST \do {BODY}
% \@whiledim TEST \do {BODY}  : These implement the loop
%           while  TEST  do  BODY  od
%     where  TEST  is a TeX \ifnum or \ifdim test, respectively.
%     They are optimized for the normal case of TEST initially false.
%
% \@whilesw SWITCH \fi {BODY} : Implements the loop
%               while SWITCH do BODY od
%     where SWITCH is a command defined by \newswitch.
%     Optimized for normal case of SWITCH initially false.
%
% \@for NAME := LIST \do {BODY} : Assumes that LIST expands to A1,A2, ... ,An .
%      Executes  BODY  n  times, with  NAME = Ai  on the i-th iteration.
%      Optimized for the normal case of n = 1.  Works for n=0.
%
% \@tfor NAME := LIST \do {BODY}
%      if, before expansion, LIST = T1 ... Tn  where each Ti is a
%      token or {...}, then executes  BODY  n  times, with  NAME = Ti
%      on the i-th iteration.  Works for n=0.
%
%  NOTES: 1. These macros use no \@temp sequences.
%         2. These macros do not work if the body contains anything that looks
%            syntactically to TeX like an improperly balanced \if \else \fi.
%
% \@whilenum TEST \do {BODY} ==
%  BEGIN
%    if  TEST
%      then  BODY
%            \@iwhilenum{TEST \relax BODY}
%  END
%
% \@iwhilenum {TEST BODY} ==
%  BEGIN
%    if  TEST
%      then  BODY
%            \@nextwhile = def(\@iwhilenum)
%      else  \@nextwhile = def(\@whilenoop)
%    fi
%    \@nextwhile {TEST BODY}
%  END
%
% \@whilesw SWITCH \fi {BODY} ==
%  BEGIN
%    if SWITCH
%      then BODY
%           \@iwhilesw {SWITCH BODY}\fi
%    fi
%  END
%
% \@iwhilesw {SWITCH BODY} \fi ==
%  BEGIN
%    if SWITCH
%      then BODY
%           \@nextwhile = def(\@iwhilesw)
%      else \@nextwhile = def(\@whileswnoop)
%    fi
%    \@nextwhile {SWITCH BODY} \fi
%  END
 
\def\@whilenoop#1{}
\def\@whilenum#1\do #2{\ifnum #1\relax #2\relax\@iwhilenum{#1\relax
     #2\relax}\fi}
\def\@iwhilenum#1{\ifnum #1\let\@nextwhile\@iwhilenum
         \else\let\@nextwhile\@whilenoop\fi\@nextwhile{#1}}
 
\def\@whiledim#1\do #2{\ifdim #1\relax#2\@iwhiledim{#1\relax#2}\fi}
\def\@iwhiledim#1{\ifdim #1\let\@nextwhile\@iwhiledim
        \else\let\@nextwhile\@whilenoop\fi\@nextwhile{#1}}
 
\long\def\@whileswnoop#1\fi{}
\long\def\@whilesw#1\fi#2{#1#2\@iwhilesw{#1#2}\fi\fi}
\long\def\@iwhilesw#1\fi{#1\let\@nextwhile\@iwhilesw
         \else\let\@nextwhile\@whileswnoop\fi\@nextwhile{#1}\fi}
 
% \@for NAME := LIST \do {BODY} ==
%    BEGIN \@forloop expand(LIST),\@nil,\@nil \@@ NAME {BODY} END
%
% \@forloop CAR, CARCDR, CDRCDR \@@ NAME {BODY} ==
%   BEGIN
%     NAME = CAR
%     if def(NAME) = def(\@nnil)
%       else BODY;
%            NAME = CARCDR
%            if def(NAME) = def(\@nnil)
%              else BODY
%                   \@iforloop CDRCDR \@@ NAME \do {BODY}
%            fi
%     fi
%   END
%
% \@iforloop CAR, CDR \@@ NAME {BODY} =
%     NAME = CAR
%     if def(NAME) = def(\@nnil)
%        then  \@nextwhile = def(\@fornoop)
%        else  BODY ;
%              \@nextwhile = def(\@iforloop)
%     fi
%     \@nextwhile name cdr {body}
%
% \@tfor NAME := LIST \do {BODY}
%    =  \@tforloop LIST \@nil \@@ NAME {BODY}
%
% \@tforloop car cdr \@@ name {body} =
%     name = car
%     if def(name) = def(\@nnil)
%        then  \@nextwhile == \@fornoop
%        else  body ;
%              \@nextwhile == \@forloop
%     fi
%     \@nextwhile name cdr {body}
%
 
\def\@nnil{\@nil}
\def\@empty{}
\def\@fornoop#1\@@#2#3{}
 
\def\@for#1:=#2\do#3{\edef\@fortmp{#2}\ifx\@fortmp\@empty \else
    \expandafter\@forloop#2,\@nil,\@nil\@@#1{#3}\fi}
 
\def\@forloop#1,#2,#3\@@#4#5{\def#4{#1}\ifx #4\@nnil \else
       #5\def#4{#2}\ifx #4\@nnil \else#5\@iforloop #3\@@#4{#5}\fi\fi}
 
\def\@iforloop#1,#2\@@#3#4{\def#3{#1}\ifx #3\@nnil
       \let\@nextwhile\@fornoop \else
      #4\relax\let\@nextwhile\@iforloop\fi\@nextwhile#2\@@#3{#4}}

%%RmS 91/10/17: Corrected bug in \@tfor: \xdef replaced by \def
%% (See FMi's array.doc)
\def\@tfor#1:=#2\do#3{\def\@fortmp{#2}\ifx\@fortmp\@empty \else
    \@tforloop#2\@nil\@nil\@@#1{#3}\fi}
\def\@tforloop#1#2\@@#3#4{\def#3{#1}\ifx #3\@nnil
       \let\@nextwhile\@fornoop \else
      #4\relax\let\@nextwhile\@tforloop\fi\@nextwhile#2\@@#3{#4}}
 
 
 \message{files,}
%     ****************************************
%     *           FILE HANDLING              *
%     ****************************************
%
% THE FOLLOWING USER COMMANDS ARE DEFINED IN THIS PART:
%  \document            : Reads in the .AUX files and \catcode's @ to 12.
%  \nofiles             : Suppresses all file output by setting \@filesw false.
%  \includeonly{NAME1, ... ,NAMEn}
%                    : Causes only parts NAME1, ... ,NAMEn to be read by
%                      their \include commands.  Works by setting \@partsw true
%                      and setting \@partlist to NAME1, ... ,NAMEn.
%  \include{NAME}    : Does an \input NAME unless \partsw is true and
%                      NAME is not in \@partlist.  If \@filesw is true, then
%                      it directs .AUX output to NAME.AUX, including a
%                      checkpoint at the end.
% \input{NAME}       : The same as TeX's \input, except it allows optional
%                      braces around the file name.
%
%  VARIABLES, SWITCHES AND INTERNAL COMMANDS:
%    \@mainaux    : Output file number for main .AUX file.
%    \@partaux    : Output file number for current part's .AUX file.
%    \@auxout     : Either \@mainout or \@partout, depending on which .AUX
%                   file output goes to.
%    \@input{foo} : If file foo exists, then \input's it, otherwise types
%                   a warning message.
%    @filesw       : Switch -- set false if no .AUX, .TOC, .IDX etc files are
%                   to be
%    @partsw      : Set true by a \includeonly command.
%    \@partlist   : Set to the argument of the \includeonly command.
%
%    \cp@FOO      : The checkpoint for \include'd file FOO.TEX, written
%                   by \@writeckpt at the end of file FOO.AUX
%
% \document ==
%   BEGIN
%     \endgroup   % cancels \begingroup generated by \begin command
%     \@colht := \@colroom := \vsize := \textheight
%     \columnwidth := \textwidth
%     \@clubpenalty := \clubpenalty          % \@clubpenalty saves value.
%     IF @twocolumn = T
%       THEN \columnwidth := (\columnwidth - \columnsep)/2
%            @firstcolumn := T
%     FI
%     \hsize  := \linewidth := \columnwidth
%     \begingroup
%        \@floatplacement \@dblfloatplacement
%        \@input{\jobname.aux}
%     \endgroup
%     IF \@filesw = T
%       THEN  open file \@mainaux for writing
%             write ``\relax''on file \@mainaux
%     FI
%     \do{COMMAND} == BEGIN \let COMMAND = \@notprerr END
%     \@preamblecmds
%     \do == \noexpand
%     \@normalsize
%     \everypar{}
%     @noskipsec := F
%   END
%
% \includeonly{FILELIST} ==
%  BEGIN
%   \@partsw   := T
%   \@partlist := FILELIST
%  END
%
% \include{FILE} ==
%  BEGIN
%   \clearpage
%   if \@filesw = T
%     then  \immediate\write\@mainaux{\string\@input{FILE.AUX}}
%   fi
%   if  \@partsw = T
%     then \@tempswa := F
%          \@tempb == FILE
%          for \@tempa := \@partlist
%              do if eval(\@tempa) = eval(\@tempb)
%                   then \@tempswa := T          fi
%              od
%   fi
%
%   if \@tempswa = T
%      then \@auxout := \@partaux
%           if \@filesw = T
%             then  \immediate\openout\@partaux{FILE.AUX}
%                   \immediate\write\@partaux{\relax}
%           fi
%           \@input{FILE.TEX}
%           \clearpage
%           \@writeckpt{FILE}
%           if @filesw then \closeout \@partaux fi
%           \@auxout := \@mainaux
%      else \cp@FILE
%   fi
%  END
%
% \@writeckpt{FILE} ==
%  BEGIN
%    if \@filesw = T
%        \immediate\write on file \@partaux:
%                  \gdef\cp@FILE{                  %% }
%        for \@tempa := \cl@@ckpt
%           do  \immediate\write on file \@partaux:
%                   \global\string\setcounter
%                       {eval(\@tempa)}{eval(\c@eval(\@tempa))}
%           od                                     %% {
%        \immediate\write on file \@partaux:  }
%    fi
%  END
%
%  INITIALIZATION
%    \@tempswa := T
 
\newif\if@filesw \@fileswtrue
\newif\if@partsw \@partswfalse
\newwrite\@mainaux
\newwrite\@partaux
 
\newcount\@clubpenalty

%% FMi & RmS 91/08/26 set @noskipsec switch to true in the preamble
%% and to false by \begin{document} to catch lists in the preamble,
%% i.e., to produce a ``nodocument'' error when things like
%% \maketitle appear before \begin{document}.
%
% \@noskipsectrue %% set below where switch is defined
 
% 91/03/26 FMi: |\process@table| added to support NFSS.
% This will also work with old lfonts if no other style defines
% |\process@table|.
%
\def\document{\endgroup
  \@colht\textheight  \@colroom\textheight \vsize\textheight
   \columnwidth\textwidth \@clubpenalty\clubpenalty
   \if@twocolumn \advance\columnwidth -\columnsep
      \divide\columnwidth\tw@ \hsize\columnwidth \@firstcolumntrue
   \fi
  \hsize\columnwidth \linewidth\hsize
  \begingroup\@floatplacement\@dblfloatplacement
   \makeatletter\let\@writefile\@gobbletwo
   \@input{\jobname.aux}\endgroup
  \if@filesw \immediate\openout\@mainaux=\jobname.aux
    \immediate\write\@mainaux{\relax}\fi
  \csname process@table\endcsname
  \let\glb@currsize\@empty  %% Force \baselineskip initialisation.
  \def\do##1{\let ##1\@notprerr}%
  \@preamblecmds
  \let\do\noexpand
  \@normalsize\everypar{}\@noskipsecfalse}
 
\def\@gobbletwo#1#2{}
 
\def\nofiles{\@fileswfalse \typeout
   {No auxiliary output files.}\typeout{}}

%% RmS 92/03/18: changed input channel from 1 to \@inputcheck to avoid
%%               conflicts with other channels allocated by \newread
\def\@input#1{\openin\@inputcheck #1 \ifeof\@inputcheck \typeout
  {No file #1.}\else\closein\@inputcheck \relax\@@input #1 \fi}
\let\@auxout=\@mainaux
 
\def\includeonly#1{\@partswtrue\edef\@partlist{#1}}
 
% In the definition of \include, \def\@tempb changed to \edef\@tempb to
% be consistent with the \edef in \includeonly.  (Suggested by Rainer
% Sch\"opf & Frank Mittelbach.  Change made 20 Jul 88.)
%
% Changed definition of \include to allow space at end of file name--
% otherwise, typing \include{foo } would cause LaTeX to overwrite
% foo.tex.  Change made 24 May 89, suggested by Rainer Sch\"opf  and
% Frank Mittelbach
 
\def\include#1{\@include#1 }
\def\@include#1 {\clearpage
\if@filesw \immediate\write\@mainaux{\string\@input{#1.aux}}\fi
\@tempswatrue\if@partsw \@tempswafalse\edef\@tempb{#1}\@for
\@tempa:=\@partlist\do{\ifx\@tempa\@tempb\@tempswatrue\fi}\fi
\if@tempswa \if@filesw \let\@auxout\@partaux
\immediate\openout\@partaux #1.aux
\immediate\write\@partaux{\relax}\fi\@input{#1.tex}\clearpage
\@writeckpt{#1}\if@filesw \immediate\closeout\@partaux \fi
\let\@auxout\@mainaux\else\@nameuse{cp@#1}\fi}
 
\def\@writeckpt#1{\if@filesw
\immediate\write\@partaux{\string\global\string\@namedef{cp@#1}\@charlb}%
{\let\@elt\@wckptelt \cl@@ckpt}\immediate\write\@partaux{\@charrb}\fi}
 
\def\@wckptelt#1{\immediate\write\@partaux
{\string\setcounter{#1}{\the\@nameuse{c@#1}}}}
 
\def\input{\@ifnextchar \bgroup{\@iinput}{\@@input }}
\def\@iinput#1{\@@input #1 }
 
% The following defines \@charlb and \@charrb to be { and }, respectively
% with \catcode 11.
{\catcode`[=1 \catcode`]=2
\catcode`{=11 \catcode`}=11
\gdef\@charlb[{]
\gdef\@charrb[}]
]% }brace matching
 
 
 \message{env. counters,}
%          ****************************************
%          *     ENVIRONMENT COUNTER MACROS       *
%          ****************************************
%
%  An environment  foo  has an associated counter defined by the
%  following control sequences:
%    \c@foo  :  Contains the counter's numerical value.  It is defined by
%                   \newcount\foocounter.
%    \thefoo : Macro that expands to the printed value of \foocounter.
%              For example, if sections are numbered within chapters,
%              and section headings look like
%                  Section II-3.  The Nature of Counters
%              then \thesection might be defined by:
%                 \def\thesection{\@Roman{\c@chapter}-\@arabic{\c@section}}
%
%    \p@foo  : Macro that expands to a printed 'reference prefix' of
%              counter foo.  Any \ref to a value created by counter
%              foo will produce the expansion of \p@foo\thefoo  when the
%              the \label command is executed.
%
% NOTE: \thefoo and \p@foo MUST BE DEFINED IN SUCH A WAY THAT
% \edef\bar{\thefoo} OR \edef\bar{\p@foo}
% DEFINES \bar SO THAT IT WILL EVALUATE TO THE COUNTER VALUE AT THE TIME
% OF THE \edef, EVEN AFTER \foocounter AND ANY OTHER COUNTERS HAVE BEEN
% CHANGED.  THIS WILL HAPPEN IF YOU USE THE STANDARD COMMANDS \@arabic,
% \@Roman, ETC.
%
%    \cl@foo : List of counters to be reset when foo stepped.  Has format
%              \@elt{countera}\@elt{counterb}\@elt{counterc}.
%
%  The following commands are used to define and modify counters.
%    \setcounter{FOO}{VAL}  : Globally sets \foocounter equal to VAL.
%    \addtocounter{FOO}{VAL}: Globally increments \foocounter by VAL.
%    \newcounter{NEWCTR}[OLDCTR] : Defines NEWCTR to be a counter, which is
%                             reset when counter OLDCTR is stepped.  If
%                             NEWCTR already defined produces 'c@NEWCTR
%                             already defined' error.
%    \value{CTR}           : produces the value of counter CTR, for use with
%                            a \setcounter or \addtocounter command.
%    \stepcounter{FOO}     : Globally increments counter \c@FOO
%                             and resets all subsidiary counters.
%    \refstepcounter{FOO}  : Same a \stepcounter, but it also defines
%                             \@currentreference so that a subsequent
%                             \label{bar} command causes \ref{bar} to
%                             generate the current value of counter foo.
%    \@definecounter{FOO}   : Initializes counter FOO (with empty reset list),
%                             defines \p@FOO and \theFOO to be null.
%                             Also adds FOO to \cl@@ckpt -- the reset
%                             list of a dummy counter @ckpt used for
%                             taking checkpoints.
%    \@addtoreset{FOO}{BAR} : Adds counter FOO to the list of counters
%                             \cl@BAR to be reset when counter bar is stepped.
%
%   NUMBERING MACROS:
%     \arabic{COUNTER} : Representation of COUNTER as arabic numerals.
%                        Changed 29 Apr 86 to make it print the obvious thing
%                        it COUNTER not positive.
%
%     \roman{COUNTER}  : Representation of COUNTER as lower-case
%                           Roman numerals.
%     \Roman{COUNTER}  : Representation of COUNTER as upper-case
%                           Roman numerals.
%     \alph{COUNTER}   : Representation of COUNTER as a lower-case
%                           letter: 1 = a, 2 = b, etc.
%     \Alph{COUNTER}   : Representation of COUNTER as an upper-case
%                           letter: 1 = A, 2 = B, etc.
%     \fnsymbol{COUNTER} : Representation of COUNTER as a footnote
%                           symbol: 1 = *, 2 = \dagger, etc.  Can be
%                           used only in math mode.
%
%  THE ABOVE ARE IMPLEMENTED IN TERMS OF THE FOLLOWING:
%     \@arabic\FOOcounter : Representation of \FOOcounter as arabic numerals.
%     \@roman\FOOcounter  : Representation of \FOOcounter as lower-case
%                           Roman numerals.
%     \@Roman\FOOcounter  : Representation of \FOOcounter as upper-case
%                           Roman numerals.
%     \@alph\FOOcounter   : Representation of \FOOcounter as a lower-case
%                           letter: 1 = a, 2 = b, etc.
%     \@Alph\FOOcounter   : Representation of \FOOcounter as an upper-case
%                           letter: 1 = A, 2 = B, etc.
%     \@fnsymbol\FOOcounter : Representation of \FOOcounter as a footnote
%                             symbol.  Can be used only in math mode.
 
\def\setcounter#1#2{\@ifundefined{c@#1}{\@nocnterr}%
{\global\csname c@#1\endcsname#2\relax}}
 
\def\addtocounter#1#2{\@ifundefined{c@#1}{\@nocnterr}%
{\global\advance\csname c@#1\endcsname #2\relax}}
 
\def\newcounter#1{\expandafter\@ifdefinable \csname c@#1\endcsname
    {\@definecounter{#1}}\@ifnextchar[{\@newctr{#1}}{}}
 
\def\value#1{\csname c@#1\endcsname}
 
\def\@newctr#1[#2]{\@ifundefined{c@#2}{\@nocnterr}{\@addtoreset{#1}{#2}}}
 
\def\stepcounter#1{\global\advance\csname c@#1\endcsname \@ne
    {\let\@elt\@stpelt \csname cl@#1\endcsname}}
 
\def\@stpelt#1{\global\csname c@#1\endcsname \z@}
 
\def\cl@@ckpt{\@elt{page}}
 
\def\@definecounter#1{\expandafter\newcount\csname c@#1\endcsname
     \setcounter{#1}0 \expandafter\gdef\csname cl@#1\endcsname{}\@addtoreset
     {#1}{@ckpt}\expandafter\gdef\csname p@#1\endcsname{}\expandafter
     \gdef\csname the#1\endcsname{\arabic{#1}}}
 
\def\@addtoreset#1#2{\expandafter\@cons\csname cl@#2\endcsname {{#1}}}
 
% Numbering commands for definitions of \theCOUNTER and \list arguments.
% \fnsymbol produces the standard footnoting symbols: asterisk, dagger, etc.
% They can be used only in math mode.
 
\def\arabic#1{\@arabic{\@nameuse{c@#1}}}
\def\roman#1{\@roman{\@nameuse{c@#1}}}
\def\Roman#1{\@Roman{\@nameuse{c@#1}}}
\def\alph#1{\@alph{\@nameuse{c@#1}}}
\def\Alph#1{\@Alph{\@nameuse{c@#1}}}
\def\fnsymbol#1{\@fnsymbol{\@nameuse{c@#1}}}
 
\def\@arabic#1{\number #1}  %% changed 29 Apr 86
\def\@roman#1{\romannumeral #1}
\def\@Roman#1{\expandafter\uppercase\expandafter{\romannumeral #1}}
\def\@alph#1{\ifcase#1\or a\or b\or c\or d\else\@ialph{#1}\fi}
\def\@ialph#1{\ifcase#1\or \or \or \or \or e\or f\or g\or h\or i\or j\or
   k\or l\or m\or n\or o\or p\or q\or r\or s\or t\or u\or v\or w\or x\or y\or
   z\else\@ctrerr\fi}
\def\@Alph#1{\ifcase#1\or A\or B\or C\or D\else\@Ialph{#1}\fi}
\def\@Ialph#1{\ifcase#1\or \or \or \or \or E\or F\or G\or H\or I\or J\or
   K\or L\or M\or N\or O\or P\or Q\or R\or S\or T\or U\or V\or W\or X\or Y\or
   Z\else\@ctrerr\fi}
\def\@fnsymbol#1{\ifcase#1\or *\or \dagger\or \ddagger\or
   \mathchar "278\or \mathchar "27B\or \|\or **\or \dagger\dagger
   \or \ddagger\ddagger \else\@ctrerr\fi\relax}
 
 
 
 
 \message{page nos.,}
%          ****************************************
%          *          PAGE NUMBERING              *
%          ****************************************
%
% Page numbers are produced by a page counter, used just like any other
% counter.  The only difference is that \c@page contains the number of
% the next page to be output (the one currently being produced), rather
% than one minus it.  Thus, it is normally initialized to 1 rather than
% 0.  \c@page is defined to be \count0, rather than a count assigned by
% \newcount.
%
% The user sets the pagenumber style with the \pagenumbering{FOO}
% command, which sets the page counter to 1 and defines \thepage to be
% \FOO.  For example, \pagenumbering{roman} causes pages to be numbered
% i, ii, etc.
 
 
\countdef\c@page=0 \c@page=1
\def\cl@page{}
\def\pagenumbering#1{\global\c@page \@ne \gdef\thepage{\csname @#1\endcsname
   \c@page}}
 
 
 \message{x-ref,}
%          ****************************************
%          *      CROSS REFERENCING MACROS        *
%          ****************************************
%
%  The user writes  \label{foo}  to define the following cross-references:
%     \ref{foo}     : value of most recently incremented referencable counter.
%                     in the current environment. (Chapter, section, theorem
%                     and enumeration counters counters are referencable,
%                     footnote counters are not.)
%     \pageref{foo} : page number at which \label{foo} command appeared.
%  where  foo  can be any string of characters not containing '\', '{' or '}'.
%
%  Note: The scope of the \label command is delimited by environments, so
%          \begin{theorem} \label{foo} ... \end{theorem} \label{bar}
%  defines \ref{foo} to be the theorem number and \ref{bar} to be
%  the current section number.
%
%  Note: \label does the right thing in terms of spacing -- i.e.,
%  leaving a space on both sides of it is equivalent to leaving
%  a space on either side.
%
%  This is implemented as follows.  A referencable counter  CNT  is
%  incremented by the command  \refstepcounter{CNT} , which sets
%  \@currentlabel == {CNT}{eval(\p@cnt\theCNT)}.   The command
%  \label{FOO} then writes the following on file \@auxout :
%        \newlabel{FOO}{{eval(\@currentlabel)}{eval(\thepage)}}
%
%  \ref{FOO} ==
%    BEGIN
%      if \r@foo undefined
%        then  ??
%              Warning: 'reference foo on page ... undefined'
%        else  \@car \eval(\r@FOO)\@nil
%      fi
%    END
%
%  \pageref{foo} =
%    BEGIN
%      if \r@foo undefined
%        then  ??
%              Warning: 'reference foo on page ... undefined'
%        else  \@cdr \eval(\r@FOO)\@nil
%      fi
%    END
%
 
%% RmS 91/10/25: added a few extra \reset@font,
%%               as suggested by Bernd Raichle
\def\ref#1{\@ifundefined{r@#1}{{\reset@font\bf ??}\@warning
   {Reference `#1' on page \thepage \space
    undefined}}{\edef\@tempa{\@nameuse{r@#1}}\expandafter
    \@car\@tempa \@nil\null}}
 
\def\pageref#1{\@ifundefined{r@#1}{{\reset@font\bf ??}\@warning
   {Reference `#1' on page \thepage \space
    undefined}}{\edef\@tempa{\@nameuse{r@#1}}\expandafter
    \@cdr\@tempa\@nil\null}}
 
\def\newlabel#1#2{\@ifundefined{r@#1}{}{\@warning{Label `#1' multiply
   defined}}\global\@namedef{r@#1}{#2}}
 
% \label and \refstepcounter changed to allow \protect'ed commands to
% work properly.  For example,
%   \def\thechapter{\protect\foo{\arabic{chapter}.\roman{section}}}
% will cause a \label{bar} command to define \ref{bar} to expand to
% something like \foo{4.d}.  Change made 20 Jul 88.
 
\def\label#1{\@bsphack\if@filesw {\let\thepage\relax
   \def\protect{\noexpand\noexpand\noexpand}%
   \edef\@tempa{\write\@auxout{\string
      \newlabel{#1}{{\@currentlabel}{\thepage}}}}%
   \expandafter}\@tempa
   \if@nobreak \ifvmode\nobreak\fi\fi\fi\@esphack}
 
\def\refstepcounter#1{\stepcounter{#1}\let\@tempa\protect
\def\protect{\noexpand\protect\noexpand}%
\edef\@currentlabel{\csname p@#1\endcsname\csname the#1\endcsname}%
\let\protect\@tempa}
 
\def\@currentlabel{} % For \label commands that come before any environment
 
 \message{environments,}
%          ****************************************
%          *            ENVIRONMENTS              *
%          ****************************************
%
%  \begin{foo} and \end{foo} are used to delimit environment foo.
%  \begin{foo} starts a group and calls \foo if it is defined, otherwise
%  it does nothing.  \end{foo} checks to see that it matches the
%  corresponding \begin and if so, it calls \endfoo and does an
%  \endgroup.  Otherwise, \end{foo} does nothing.
%
%  If \end{foo} needs to ignore blanks after it, then \endfoo should
%  globally set the @ignore switch true with \global\@ignoretrue.
%
%  \@currenvir : the name of the current environment.  Initialized to
%                'document' to make \end{document} work right.
%
%  \@preamblecmds : a list of commands that can be used only in the
%                   preamble (before the \begin{document}), in the
%                   form  \do \CMDA \do \CMDB ... .   These commands
%                   are redefined to \@notprerr by \begin{document}
%                   to save space.  They include the following:
%                       \document \documentstyle \@documentstyle
%                       \@options \@preamblecmds \@optionlist
%                       \@optionfiles \nofiles \includeonly \makeindex
%                       \makeglossary
%                   The document style can add any other commands to
%                   this list by
%                      \def\do{\noexpand\do\noexpand}
%                      \edef\@preamblecmds{\@preamblecmds \do ...}
%
%  NOTE: \@@end is defined to be the \end command of TeX82.
%
%  \enddocument is the user's command for ending the manuscript file.
%
%  \stop is a panic button -- to end TeX in the middle.
%
% \enddocument ==
%   BEGIN
%    \@checkend{document}   %% checks for unmatched \begin
%    \clearpage
%    \begingroup
%      if @filesw = true
%        then  close file @mainaux
%              \global \@namedef {ARG1}{ARG2} == null
%              \newlabel{LABEL}{VAL} ==
%                  BEGIN
%                    \@tempa == VAL
%                    if def(\@tempa) = def(\r@LABEL)
%                      else @tempswa := true          fi
%                  END
%              \bibcite{LABEL}{VAL} == null
%                  BEGIN
%                    \@tempa == VAL
%                    if def(\@tempa) = def(\g@LABEL)
%                      else @tempswa := true          fi
%                  END
%              @tempswa := false
%              make @ a letter
%              \input \jobname.AUX
%              if @tempswa = true
%                then LaTeX Warning: 'Label may have changed.
%                                     Rerun to get cross-references right.'
%       fi     fi
%    \endgroup
%    finish up
%   END
%
%  \@writefile{EXT}{ENTRY} ==
%      if tf@EXT undefined
%        else \write\tf@EXT{ENTRY}
%      fi
%
\def\@currenvir{document}
 
\def\@preamblecmds{\do\document \do\documentstyle \do\@documentstyle
   \do\@options \do\@preamblecmds \do\@optionlist \do\@optionfiles
   \do\nofiles \do\includeonly \do\makeindex \do\makeglossary}
 
\newif\if@ignore
 
\def\enddocument{\@checkend{document}\clearpage\begingroup
\if@filesw \immediate\closeout\@mainaux
\def\global\@namedef##1##2{}\def\newlabel{\@testdef r}%
\def\bibcite{\@testdef b}\@tempswafalse \makeatletter\input \jobname.aux
\if@tempswa \@@warning{Label(s) may have changed.  Rerun to get
cross-references right}\fi\fi\endgroup\deadcycles\z@\@@end}
 
\def\@testdef #1#2#3{\def\@tempa{#3}\expandafter \ifx \csname #1@#2\endcsname
 \@tempa  \else \@tempswatrue \fi}
 
\long\def\@writefile#1#2{\@ifundefined{tf@#1}{}{%
   \immediate\write\csname tf@#1\endcsname{#2}}}
 % \long added 8 Feb 90, as suggested by Chris Rowley
 
\def\stop{\clearpage\deadcycles\z@\let\par\@@par\@@end}
 
\everypar{\@nodocument} %% To get an error if text appears before the
\nullfont               %% \begin{document}
 
% \begin, \end, and \@checkend changed so \end{document} will catch
% an unmatched \begin.  Changed 24 May 89 as suggested by
% Frank Mittelbach and Rainer Sch\"opf.
 
 
% \begin{NAME} ==
%  BEGIN
%    IF \NAME undefined  THEN  \@tempa == BEGIN report error END
%                        ELSE  \@tempa ==  (\@currenvir :=L NAME) \NAME
%    FI
%    @ignore :=G F      %% Added 30 Nov 88
%    \begingroup
%    \@currenvir :=L NAME
%    \NAME
%  END
 
% \end{NAME} ==
%  BEGIN
%   \endNAME
%   \@checkend{NAME}
%   IF @endpe = T                  %% @endpe set True by \@endparenv
%     THEN \@gtempa :=G \@doendpe  %% \@doendpe redefines \par and \everypar
%     ELSE \@gtempa :=G \relax     %% to suppress paragraph indentation in
%   FI                             %% immediately following text
%   \endgroup
%   \@gtempa
%   IF @ignore = T
%     THEN @ignore :=G F
%          \ignorespaces
%   FI
%  END
 
% \@checkend{NAME} ==
%  BEGIN
%   IF \@currenvir = NAME
%     ELSE \@badend{NAME}
%   FI
%  END
 
%% RmS 92/03/18: changed \@ignoretrue to \@ignorefalse (as documented)
\def\begin#1{\@ifundefined{#1}{\def\@tempa{\@latexerr{Environment #1
  undefined}\@eha}}{\def\@tempa{\def\@currenvir{#1}%
  \csname #1\endcsname}}\global\@ignorefalse %% \global... added 2 May 90
     \begingroup\@endpefalse\@tempa}
 
\def\end#1{\csname end#1\endcsname\@checkend{#1}%
  \expandafter\endgroup \if@endpe \@doendpe \fi
  \if@ignore \global\@ignorefalse \ignorespaces\fi}
 
\def\@checkend#1{\def\@tempa{#1}\ifx
      \@tempa\@currenvir \else\@badend{#1}\fi}
 
 
 \message{math,}
%      **********************************************
%      *               MATH ENVIRONMENTS            *
%      **********************************************
%
% \( ==  BEGIN  if math mode
%                 then error: '\( in math mode'
%                 else $
%               fi
%        END
%
% \) ==  BEGIN  if math mode
%                 then if inner mode
%                        then $
%                        else error ``\[ closed with \)''
%                 else error 'unmatched \)'
%               fi
%        END
%
% \[ ==  BEGIN  if math mode
%                 then error: '\[ in math mode'
%                 else $$
%               fi
%        END
%
% \] ==  BEGIN  if math mode
%                 then if inner mode
%                        then error '\( closed with \]'
%                        else $$
%                 else error 'unmatched \]'
%               fi
%        END
%
% \equation      ==  BEGIN \refstepcounter{equation} $$ END
%
% \endequation   ==  BEGIN \eqno (\theequation) $$\ignorespaces END
%
% NOTE: The document style must define \theequation etc., and do
% the appropriate \@addtoreset.  It should also redefine \@eqnnum
% if another format for the equation number is desired other than the
% standard (...), or to move the equation numbers to the flushleft.
% (See comment on the \def of \@eqnnum.)
%
% \stackrel{TOP}{BOT} == PLAIN TeX's \buildrel {TOP} \over {BOT}
%
% \frac{TOP}{BOT} == {TOP \over BOT}
%
% \sqrt[N]{EXP} produces an Nth root of EXP formula.
%
%  \: == \>  (medium space)
 
\def\({\relax\ifmmode\@badmath\else$%%$BRACE MATCH HACK
\fi}
 
\def\){\relax\ifmmode\ifinner$\else\@badmath%%$ BRACE MATCH HACK
\fi\else \@badmath\fi}
 
\def\[{\relax\ifmmode\@badmath\else
    \ifvmode \nointerlineskip \makebox[.6\linewidth]\fi$$%%$$ BRACE MATCH HACK
\fi}
 
\def\]{\relax\ifmmode\ifinner\@badmath\else$$\fi%%$$ BRACE MATCH HACK
        \else \@badmath \fi\ignorespaces}
 
\let\math=\(
\let\endmath=\)
\def\displaymath{\[}
\def\enddisplaymath{\]\global\@ignoretrue}
 
\@definecounter{equation}
\def\equation{$$ % $$ BRACE MATCHING HACK
\refstepcounter{equation}}

%% RmS 92/01/10: put \hbox around \@eqnnum to typeset the equation
%%               number in text mode (as in the eqnarray env.).
\def\endequation{\eqno \hbox{\@eqnnum}% $$ BRACE MATCHING HACK
$$\global\@ignoretrue}
 
%  \@eqnnum: Produces the equation number for equation and
%     eqnarray environments.  The following definition is for
%     flushright numbers; for flushleft numbers, see leqno.doc.
%     The {\rm ... } puts the equation number in roman type even if
%     an eqnarray environment appears in an italic environment.
%
%% RmS 91/09/29: \reset@font added.
\def\@eqnnum{{\reset@font\rm (\theequation)}}
 
 
\def\stackrel#1#2{\mathrel{\mathop{#2}\limits^{#1}}}
\def\frac#1#2{{#1\over #2}}
 
\let\@@sqrt=\sqrt
\def\sqrt{\@ifnextchar[{\@sqrt}{\@@sqrt}}
\def\@sqrt[#1]{\root #1\of}
 
\let\:=\>
 
% Here's the eqnarray environment:
%  Default is for left-hand side of equations to be flushleft.
%  To make them flushright, \let\@eqnsel = \hfil
 
\newcount\@eqcnt
\newcount\@eqpen
\newif\if@eqnsw\@eqnswtrue
 
\@centering = 0pt plus 1000pt % Changed 11/4/85 to produce warning message
                              % if line extends into margin.  Doesn't warn
                              % about formula overprinting equation number.
 
\def\eqnarray{\stepcounter{equation}\let\@currentlabel\theequation
\global\@eqnswtrue\m@th
\global\@eqcnt\z@\tabskip\@centering\let\\\@eqncr
$$\halign to\displaywidth\bgroup\@eqnsel\hskip\@centering
  $\displaystyle\tabskip\z@{##}$&\global\@eqcnt\@ne
  \hskip 2\arraycolsep \hfil${##}$\hfil
  &\global\@eqcnt\tw@ \hskip 2\arraycolsep $\displaystyle\tabskip\z@{##}$\hfil
   \tabskip\@centering&\llap{##}\tabskip\z@\cr}
 
\def\endeqnarray{\@@eqncr\egroup
      \global\advance\c@equation\m@ne$$\global\@ignoretrue}
 
\let\@eqnsel=\relax
 
\def\nonumber{\global\@eqnswfalse}
 
\def\@eqncr{{\ifnum0=`}\fi\@ifstar{\global\@eqpen\@M
    \@yeqncr}{\global\@eqpen\interdisplaylinepenalty \@yeqncr}}
 
\def\@yeqncr{\@ifnextchar [{\@xeqncr}{\@xeqncr[\z@]}}
 
\def\@xeqncr[#1]{\ifnum0=`{\fi}\@@eqncr
   \noalign{\penalty\@eqpen\vskip\jot\vskip #1\relax}}
 
\def\@@eqncr{\let\@tempa\relax
    \ifcase\@eqcnt \def\@tempa{& & &}\or \def\@tempa{& &}%
      \else \def\@tempa{&}\fi
     \@tempa \if@eqnsw\@eqnnum\stepcounter{equation}\fi
     \global\@eqnswtrue\global\@eqcnt\z@\cr}
 
% Here's the eqnarray* environment:
 
\let\@seqncr=\@eqncr
\@namedef{eqnarray*}{\def\@eqncr{\nonumber\@seqncr}\eqnarray}
\@namedef{endeqnarray*}{\nonumber\endeqnarray}
 
% \lefteqn{FORMULA} typesets FORMULA in display math style
%  flushleft in a box of width zero.
%
 
\def\lefteqn#1{\hbox to\z@{$\displaystyle #1$\hss}}
 
 
 \message{center,}
%      ************************************************
%      *      CENTER, FLUSHRIGHT, FLUSHLEFT, ETC.     *
%      ************************************************
%
%
% \center, \flushright and \flushleft set
%   \rightskip = 0pt or \@flushglue (as appropriate)
%   \leftskip  = 0pt or \@flushglue (as appropriate)
%   \parindent = 0pt
%   \parfillskip   = 0pt. (except \flushleft)
%   \\         == \par \vskip -\parskip
%   \\[LENGTH] == \\ \vskip LENGTH
%   \\*        == \par \penalty 10000 \vskip -\parskip
%   \\*[LEN]   == \\* \vskip LENGTH
%
% They invoke the trivlist environment to handle vertical spacing before
% and after them.
%
% \centering, \raggedright and \raggedleft are the declaration analogs
% of the above.
%
% \raggedright has a more universal effect, however.  It sets
% \@rightskip := flushglue.  Every environment, like the list environments,
% that set \rightskip to its 'normal' value set it to \@rightskip
 
\def\@centercr{\ifhmode \unskip\else \@badcrerr\fi
       \par\@ifstar{\penalty \@M\@xcentercr}{\@xcentercr}}
 
\def\@xcentercr{\addvspace{-\parskip}\@ifnextchar
    [{\@icentercr}{\ignorespaces}}
 
\def\@icentercr[#1]{\vskip #1\ignorespaces}
 
\def\center{\trivlist \centering\item[]}
\def\centering{\let\\=\@centercr\rightskip\@flushglue\leftskip\@flushglue
\parindent\z@\parfillskip\z@}
\let\endcenter=\endtrivlist
 
\newskip\@rightskip \@rightskip \z@
 
\def\flushleft{\trivlist \raggedright\item[]}
\def\raggedright{\let\\=\@centercr\@rightskip\@flushglue \rightskip\@rightskip
  \leftskip\z@
  \parindent\z@}
\let\endflushleft=\endtrivlist
 
\def\flushright{\trivlist \raggedleft\item[]}
\def\raggedleft{\let\\=\@centercr\rightskip\z@\leftskip\@flushglue
  \parindent\z@\parfillskip\z@}
\let\endflushright=\endtrivlist
 
 \message{verbatim,}
%       ****************************************
%       *              VERBATIM                *
%       ****************************************
%
%  The verbatim environment uses the fixed-width \tt font, turns blanks into
%  spaces, starts a new line for each carrige return (or sequence of
%  consecutive carriage returns), and interprets EVERY character literally.
%  I.e., all special characters \, {, $, etc. are \catcode'd to 'other'.
%
%  The command \verb produces in-line verbatim text, where the argument
%  is delimited by any pair of characters.  E.g., \verb #...# takes
%  '...' as its argument, and sets it verbatim in \tt font.
%
%  The *-variants of these commands is the same, except that spaces
%  print as the TeXbook's space character instead of as blank spaces.
 
{\catcode`\^^M=13 \gdef\@gobblecr{\@ifnextchar
{\@gobble}{\ignorespaces}}}
 
{\catcode`\ =\active\gdef\@vobeyspaces{\catcode`\ \active\let \@xobeysp}}
 
% Definition of \@xobeysp chaned on 19 Nov 86 from
% \def\@xobeysp{\leavevmode{} }
% to prevent line breaks at spaces.  Change suggested by
% Nelson Beebe
%
\def\@xobeysp{\leavevmode\penalty10000\ }
 
 
 
\begingroup \catcode `|=0 \catcode `[= 1
\catcode`]=2 \catcode `\{=12 \catcode `\}=12
\catcode`\\=12 |gdef|@xverbatim#1\end{verbatim}[#1|end[verbatim]]
|gdef|@sxverbatim#1\end{verbatim*}[#1|end[verbatim*]]
|endgroup
 
% \@sverbatim obsolete -- removed 24 May 89, as suggested by
% Rainer Sch\"opf and Frank Mittelbach
% \def\@sverbatim{\obeyspaces\@verbatim}
 
\def\@gobble#1{}
 
% 91/07/24 RmS: added \penalty\interlinepenalty to definition
%               of \par so that \samepage works.

\def\@verbatim{\trivlist \item[]\if@minipage\else\vskip\parskip\fi
\leftskip\@totalleftmargin\rightskip\z@
\parindent\z@\parfillskip\@flushglue\parskip\z@
%%RmS 91/08/26 Added \@@par to clear possible \parshape definition
%%from a surrounding list (the verbatim guru says)
\@@par
\@tempswafalse \def\par{\if@tempswa\hbox{}\fi\@tempswatrue\@@par
\penalty\interlinepenalty}%
\obeylines \tt \catcode``=13 \@noligs \let\do\@makeother \dospecials}
 
\def\verbatim{\@verbatim \frenchspacing\@vobeyspaces \@xverbatim}
\let\endverbatim=\endtrivlist
 
\@namedef{verbatim*}{\@verbatim\@sxverbatim}
\expandafter\let\csname endverbatim*\endcsname =\endtrivlist
 
 
\def\@makeother#1{\catcode`#1=12\relax}
 
\def\verb{\begingroup \catcode``=13 \@noligs
\tt \let\do\@makeother \dospecials
\@ifstar{\@sverb}{\@verb}}
 
% Definitions of \@sverb and \@verb changed so \verb+ foo+  does not lose
% leading blanks when it comes at the beginning of a line.
% Change made 24 May 89. Suggested by Frank Mittelbach and Rainer Sch\"opf.
%
\def\@sverb#1{\def\@tempa ##1#1{\leavevmode\null##1\endgroup}\@tempa}
 
\def\@verb{\@vobeyspaces \frenchspacing \@sverb}
 
 
%% \@noligs prevents ?` and !` from being treated as ligatures
%% added 19 April 86

\begingroup
\catcode``=13
\gdef\@noligs{\let`\@lquote}
\endgroup
 
%% RmS 91/06/21: added \leavevmode to definition of \@lquote
%%  to avoid the \kern being processed in vertical mode
 
\def\@lquote{\leavevmode{\kern\z@}`}
 \message{list,}
%       ****************************************
%       *         THE LIST ENVIRONMENT         *
%       ****************************************
%
% The generic commands for creating an indented environment -- enumerate,
% itemize, quote, etc -- are
%        \list{LABEL}{COMMANDS} ... \endlist
% which can be invoked by the user as the list environment.  The LABEL
% argument specifies item labeling.  COMMANDS contains commands for
% changing the horizontal and vertical spacing parameters.
%
% Each item of the environment is begun by the command \item[ITEMLABEL]
% which produces an item labeled by ITEMLABEL.  If the argument is
% missing, then the LABEL argument of the \list command is used as the
% item label.
%
% The label is formed by putting \makelabel{ITEMLABEL} in an hbox whose
% width is either its natural width or else \labelwidth, whichever is
% larger.  The \list command defines \makelabel to have the default
% definition
%     \makelabel{ARG} == BEGIN \hfil ARG END
% which, for a label of width less than \labelwidth, puts the label
% flushright, \labelsep to the left of the item's text.  However,
% \makelabel can be \let to another command by the \list's COMMANDS
% argument.
%
% A \usecounter{foo} command in the second argument causes the counter
% foo to be initialized to zero, and stepped by every \item command
% without an argument.  (\label commands within the list refer to this
% counter.)
%
% When you leave a list environment, returning either to an enclosing
% list or normal text mode, LaTeX begins a new paragraph if and only if
% you leave a blank line after the \end command.  This is accomplished
% by the \@endparenv command.
%
% Blank lines are ignored every other reasonable place--i.e.:
%     - Between the \begin{list} and the first \item,
%     - Between the \item and the text of that item.
%     - Between the end of the last item and the \end{list}.
%
% For an environment like quotation, in which items are not labeled,
% the entire environment is a single item.  It is defined by
% letting \quotation == \list{}{...}\item[].  (Note the [], there in
% case the first character in the environment is a '['.)  The spacing
% parameters provide a great deal of flexability in designing the
% format, including the ability to let the indentation of the first
% paragraph be different from that of the subsequent ones.
%
% The trivlist environment is equivalent to a list environment
% whose second argument sets the following parameter values:
%    \leftmargin = 0 : causes no indentation of left margin
%    \labelwidth = 0 : see below for precise effect this has.
%    \itemindent = 0 : with a null label, makes first paragraph
%        have no indentation.  Succeeding paragraphs have \parindent
%        indentation.  To give first paragraph same indentation, set
%        \itemindent = \parindent before the \item[].
% Every \item in a trivlist environment must have an argument---in many
% cases, this will be the null argument (\item[]).  The trivlist
% environment is mainly used for paragraphing environments, like
% verbatim, in which there is no margin change.  It provides the same
% vertical spacing as the list environment, and works reasonably well
% when it occurs immediately after an \item command in an enclosing list.
%
% The following variables are used inside a list environment:
%   \@totalleftmargin : The distance that the prevailing left margin is
%                       indented from the outermost left margin,
%   \linewidth        : The width of the current line.  Must be
%                       initialized to \hsize.
%   \@listdepth       : A count for holding current list nesting depth.
%   \makelabel        : A macro with a single argument, used to generate
%                       the label from the argument (given or implied) of the
%                       \item command.  Initialized to \@mklab by the \list
%                       command.  This command must produce some stretch--i.e.,
%                       an \hfil.
%   @inlabel          : A switch that is false except between the time an
%                       \item is encountered and the time that TeX actually
%                       enters horizontal mode.  Should be tested by
%                       commands that can be messed up by the list
%                       environment's use of \everypar.
%   \box\@labels      : When @inlabel = true, it holds the labels
%                       to be put out by \everypar.
%   @noparitem        : A switch set by \list when @inlabel = true.
%                       Handles the case of a \list being the first thing
%                       in an item.
%   @noparlist        : A switch set true for a list that begins an
%                       item.  No \topsep space is added before or after
%                       such a list.
%   @newlist          : Set true by \list, set false by the first \item's
%                       text (by \everypar).
%   @noitemarg        : Set true when executing an \item with no explicit
%                       argument.  Used to save space.  To save time,
%                       make two separate \@item commands.
%   @nmbrlist         : Set true by \usecounter command, causes list to
%                       be numbered.
%   \@listctr         : \def'ed by \usecounter to name of counter.
%   @noskipsec        : A switch set true by a sectioning command when it is
%                       creating an in-text heading with \everypar.
%
% Throughout a list environment, \hsize is the width of the current
% line, measured from the outermost left margin to the outermost right
% margin.  Environments like tabbing should use \linewidth instead of
% \hsize.
%
% Here are the parameters of a list that can be set by commands in
% the \list's COMMANDS argument.  These parameters are all TeX
% skips or dimensions (defined by \newskip or \newdimen), so the usual
% TeX or LaTeX commands can be used to set them.  The commands will
% be executed in vmode if and only if the \list was preceded by a
% \par (or something like an \end{list}), so the spacing parameters
% can be set according to whether the list is inside a paragraph
% or is its own paragraph.
%
%   VERTICAL SPACING (skips):
%
%      \topsep  : Space between first item and preceding paragraph.
%      \partopsep : Extra space added to \topsep when environment starts
%                   a new paragraph (is called in vmode).
%      \itemsep : Space between successive items.
%      \parsep  : Space between paragraphs within an item -- the \parskip
%                 for this environment.
%
%   PENALTIES
%     \@beginparpenalty : put at the beginning of a list
%     \@endparpenalty   : put at end of list
%     \@itempenalty     : put between items.
%
%   HORIZONTAL SPACING (dimens)
%      \leftmargin    : space between left margin of enclosing environment
%                       (or of page if top level list) and left margin of
%                       this list.  Must be nonnegative.
%      \rightmargin   : analogous.
%      \listparindent : extra indentation at beginning of every paragraph
%                       of a list except the one started by the \item
%                       command.  May be negative!  Usually, labeled lists
%                       have \listparindent equal to zero.
%      \itemindent    : extra indentation added right BEFORE an item label.
%      \labelwidth    : nominal width of box that contains the label.
%                       If the natural width of the label < = \labelwidth,
%                       then the label is flushed right inside a box
%                       of width \labelwidth (with an \hfil).  Otherwise,
%                       a box of the natural width is employed, which causes
%                       an indentation of the text on that line.
%      \labelsep      : space between end of label box and text of
%                       first item.
%
%   DEFAULT VALUES:
%      Defaults for the list environment are set as follows.
%      First, \rightmargin, \listparindent and \itemindent are set
%      to 0pt.  Then, one of the commands \@listi, \@listii, ... , \@listvi
%      is called, depending upon the current level of the list.
%      The \@list... commands should be defined by the document
%      style.  A convention that the document style should follow is
%      to set \leftmargin to \leftmargini, ... , \leftmarginvi for
%      the appropriate level.  Items that aren't changed may be left
%      alone, but everything that could possibly be changed must be
%      reset.
%
%  \list{LABEL}{COMMANDS} ==
%   BEGIN
%     if \@listdepth > 5
%       then LaTeX error: 'Too deeply nested'
%       else \@listdepth :=G \@listdepth + 1
%     fi
%     \rightmargin     := 0pt
%     \listparindent   := 0pt
%     \itemindent      := 0pt
%     \eval(@list \romannumeral\the\@listdepth)  %% Set default values:
%     \@itemlabel      :=L LABEL
%     \makelabel       == \@mklab
%     @nmbrlist        :=L false
%     COMMANDS
%
%     \@trivlist                % commands common to \list and \trivlist
%
%     \parskip          :=L \parsep
%     \parindent        :=L \listparindent
%     \linewidth        :=L \linewidth - \rightmargin -\leftmargin
%     \@totalleftmargin :=L \@totalleftmargin + \leftmargin
%     \parshape 1 \@totalleftmargin \linewidth
%     \ignorespaces                    % gobble space up to \item
%    END
%
% \endlist == BEGIN \@listdepth :=G \@listdepth -1
%                   \endtrivlist
%             END
%
% \@trivlist ==
%  BEGIN
%     if @newlist = T then \@noitemerr fi  %% This command removed for some
%                                          %% forgotten reason.
%     \@topsepadd :=L \topsep
%     if @noskipsec then leave vertical mode fi  %% Added 11 Jun 85
%     if vertical mode
%       then \@topsepadd :=L \@topsepadd + \partopsep
%       else \unskip \par                % remove glue from end of last line
%     fi
%     if @inlabel = true
%        then @noparitem :=L true
%             @noparlist :=L true
%        else @noparlist :=L false
%             \@topsep   :=L \@topsepadd
%     fi
%     \@topsep         :=L \@topsep + \parskip  %% Change 4 Sep 85
%     \leftskip        :=L 0pt           % Restore paragraphing parameters
%     \rightskip       :=L \@rightskip
%     \parfillskip     :=L 0pt + 1fil
%
%   NOTE: \@setpar called on every \list in case \par has been temporarily
%         munged before the \list command.
%     \@setpar{if @newlist = false then {\@@par} fi}
%     \@newlist         :=G T
%     \@outerparskip    :=L \parskip
% END
%
% \trivlist  ==
% BEGIN
%  \parsep     := \parskip
%  \@trivlist
%  \labelwidth := 0
%  \leftmargin := 0
%  \itemindent := \parindent
%  \makelabel{LABEL} == LABEL
% END
%
% \endtrivlist ==
%   BEGIN
%     if @inlabel = T then \indent fi
%     if horizontal mode then \unskip \par fi
%     if @noparlist = true
%       else if \lastskip > 0
%               then \@tempskipa := \lastskip
%                    \vskip - \lastskip
%                    \vskip \@tempskipa -\@outerparskip + \parskip
%            fi
%            \@endparenv
%     fi
%   END
%
% \@endparenv ==
%   BEGIN
%    \addpenalty{@endparpenalty}
%    \addvspace{\@topsepadd}
%    \endgroup    %% ends the \begin command's \begingroup
%    \par  ==  BEGIN
%                \@restorepar
%                \everypar{}
%                \par
%              END
%    \everypar == BEGIN remove \lastbox \everypar{} END
%    \begingroup  %% to match the \end commands \endgroup
%   END
%
% \item == BEGIN if  next char = [
%                  then  \@item
%                  else  @noitemarg := true
%                        \@item[@itemlabel]
%          END
%
% \@item[LAB] ==
%    BEGIN
%     if @noparitem = true
%       then @noparitem := false                        % NOTE: then clause
%                                                       % hardly every taken,
%            \box\@labels :=G \hbox{\hskip -\leftmargin % so made a macro
%                                   \box\@labels        % \@donoparitem
%                                   \hskip \leftmargin }
%            if @minipage = false then
%               \@tempskipa := \lastskip
%               \vskip -\lastskip
%               \vskip \@tempskipa + \@outerparskip - \parskip
%            fi
%       else if @inlabel = true
%              then \indent \par   % previous item empty.
%            fi
%            if hmode then 2 \unskip's  % To remove any space at end of prev.
%                     \par              % paragraph that could cause a blank
%            fi                         % line.
%            if @newlist = T
%               then if @nobreak = T   % Kludge if list follows \section
%                      then \addvspace{\@outerparskip - \parskip}
%                      else \addpenalty{\@beginparpenalty}
%                           \addvspace{\@topsep}
%                           \addvspace{-\parskip}   %% added 4 Sep 85
%                    fi
%               else \addpenalty{\@itempenalty}
%                    \addvspace{\itemsep}
%            fi
%            @inlabel :=G true
%     fi
%
%     \everypar{ @minipage :=G F
%                @newlist :=G F
%                if  @inlabel = true
%                  then @inlabel :=G false
%                       \hskip -\parindent
%                       \box\@labels
%                       \penalty 0       %% 3 Oct 85  -- allow line break here
%                       \box\@labels :=G null
%                fi
%                \everypar{} }
%     @nobreak :=G false
%     if  @noitemarg = true
%       then @noitemarg := false
%            if @nmbrlist
%              then \refstepcounter{\@listctr}
%     fi     fi
%     \@tempboxa   :=L \hbox{\makelabel{LAB}}
%     \box\@labels :=G \@labels \hskip \itemindent
%                       \hskip - (\labelwidth + \labelsep)
%                       if \wd \@tempboxa > \labelwidth
%                          then \box\@tempboxa
%                          else \hbox to \labelwidth {\makelabel{LAB}}
%                       fi
%                       \hskip\labelsep
%     \ignorespaces                        %gobble space up to text
%   END
%
%   \usecounter{CTR} == BEGIN  @nmbrlist :=L true
%                              \@listctr == CTR
%                              \setcounter{CTR}{0}
%                       END
%
% DEFINE \dimen's and \count
\newskip\topsep
\newskip\partopsep
\newskip\itemsep
\newskip\parsep
\newskip\@topsep
\newskip\@topsepadd
\newskip\@outerparskip
 
\newdimen\leftmargin
\newdimen\rightmargin
\newdimen\listparindent
\newdimen\itemindent
\newdimen\labelwidth
\newdimen\labelsep
\newdimen\linewidth
\newdimen\@totalleftmargin \@totalleftmargin=\z@
\newdimen\leftmargini
\newdimen\leftmarginii
\newdimen\leftmarginiii
\newdimen\leftmarginiv
\newdimen\leftmarginv
\newdimen\leftmarginvi
 
\newcount\@listdepth \@listdepth=0
\newcount\@itempenalty
\newcount\@beginparpenalty
\newcount\@endparpenalty
 
\newbox\@labels
 
\newif\if@inlabel \@inlabelfalse
\newif\if@newlist   \@newlistfalse
\newif\if@noparitem \@noparitemfalse
\newif\if@noparlist \@noparlistfalse
\newif\if@noitemarg \@noitemargfalse
\newif\if@nmbrlist  \@nmbrlistfalse
 
\def\list#1#2{\ifnum \@listdepth >5\relax \@toodeep
     \else \global\advance\@listdepth\@ne \fi
  \rightmargin \z@ \listparindent\z@ \itemindent\z@
  \csname @list\romannumeral\the\@listdepth\endcsname
  \def\@itemlabel{#1}\let\makelabel\@mklab \@nmbrlistfalse #2\relax
  \@trivlist
  \parskip\parsep \parindent\listparindent
  \advance\linewidth -\rightmargin \advance\linewidth -\leftmargin
  \advance\@totalleftmargin \leftmargin
  \parshape \@ne \@totalleftmargin \linewidth
  \ignorespaces}
 
\def\@trivlist{\@topsepadd\topsep
  \if@noskipsec \leavevmode \fi
  \ifvmode \advance\@topsepadd\partopsep \else \unskip\par\fi
  \if@inlabel \@noparitemtrue \@noparlisttrue
    \else \@noparlistfalse \@topsep\@topsepadd \fi
    \advance\@topsep \parskip
  \leftskip\z@\rightskip\@rightskip \parfillskip\@flushglue
  \@setpar{\if@newlist\else{\@@par}\fi}%
  \global\@newlisttrue \@outerparskip\parskip}

%% RmS 92/03/18 added \@nmbrlistfalse
\def\trivlist{\parsep\parskip\@nmbrlistfalse
  \@trivlist \labelwidth\z@ \leftmargin\z@
  \itemindent\z@ \def\makelabel##1{##1}}
 
\def\endlist{\global\advance\@listdepth\m@ne
    \endtrivlist}
 
% Definition of \endtrivlist moved earlier in file so other commands
% can be \let = to it.
 
\def\@mklab#1{\hfil #1}
 
\def\item{\@ifnextchar [{\@item}{\@noitemargtrue \@item[\@itemlabel]}}
 
\def\@donoparitem{\@noparitemfalse
   \global\setbox\@labels\hbox{\hskip -\leftmargin
                               \unhbox\@labels
                                \hskip \leftmargin}\if@minipage\else
  \@tempskipa\lastskip
  \vskip -\lastskip \advance\@tempskipa\@outerparskip
  \advance\@tempskipa -\parskip \vskip\@tempskipa\fi}
 
\def\@item[#1]{\if@noparitem \@donoparitem
  \else \if@inlabel \indent \par \fi
         \ifhmode \unskip\unskip \par \fi
         \if@newlist \if@nobreak \@nbitem \else
                        \addpenalty\@beginparpenalty
                        \addvspace\@topsep \addvspace{-\parskip}\fi
           \else \addpenalty\@itempenalty \addvspace\itemsep
          \fi
    \global\@inlabeltrue
\fi
\everypar{\global\@minipagefalse\global\@newlistfalse
          \if@inlabel\global\@inlabelfalse \hskip -\parindent \box\@labels
             \penalty\z@ \fi
          \everypar{}}\global\@nobreakfalse
\if@noitemarg \@noitemargfalse \if@nmbrlist \refstepcounter{\@listctr}\fi \fi
\setbox\@tempboxa\hbox{\makelabel{#1}}%
\global\setbox\@labels
 \hbox{\unhbox\@labels \hskip \itemindent
       \hskip -\labelwidth \hskip -\labelsep
       \ifdim \wd\@tempboxa >\labelwidth
                \box\@tempboxa
%% RmS 91/11/22: Changed second call to \makelabel to \unhbox\@tempboxa.
%%               Avoids problems with side effects in \makelabel and is
%%               more efficient.
%          \else \hbox to\labelwidth {\makelabel{#1}}\fi
          \else \hbox to\labelwidth {\unhbox\@tempboxa}\fi
       \hskip \labelsep}\ignorespaces}

%% RmS 91/11/04: added default definition for \makelabel,
%%               to produce an error message.
\def\makelabel#1{\@latexerr{Lonely \string\item--perhaps a missing
        list environment}\@ehc}
 
\def\@nbitem{\@tempskipa\@outerparskip \advance\@tempskipa -\parskip
              \addvspace{\@tempskipa}}
 
\def\usecounter#1{\@nmbrlisttrue\def\@listctr{#1}\setcounter{#1}\z@}
 
 \message{itemize,}
%      ****************************************
%      *        ITEMIZE AND ENUMERATE         *
%      ****************************************
%
%  Enumeration is done with four counters: enumi, enumii, enumiii
%  and enumiv, where enumN controls the numbering of the Nth level
%  enumeration.  The label is generated by the commands
%  \labelenumi ... \labelenumiv, which should be defined by the
%  document style.  Note that \p@enumN\theenumN defines the output
%  of a \ref command.  A typical definition might be:
%     \def\theenumii{\alph{enumii}}
%     \def\p@enumii{\theenumi\theenumii}
%     \def\labelenumii{(\theenumii)}
% which will print the labels as '(a)', '(b)', ... and print a \ref as
% '3a'.
%
% The item numbers are moved to the right of the label box, so they are
% always a distance of \labelsep from the item.
%
% \@enumdepth holds the current enumeration nesting depth.
%
% Itemization is controlled by four commands: \labelitemi, \labelitemii,
% \labelitemiii, and \labelitemiv.  To cause the second-level list to be
% bulleted, you just define \labelitemii to be $\bullet$.  \@itemspacing
% and \@itemdepth are the analogs of \@enumspacing and \@enumdepth.
%
% \enumerate ==
%   BEGIN
%     if \@enumdepth > 3
%       then errormessage: ``Too deeply nested''.
%       else \@enumdepth :=L \@enumdepth + 1
%            \@enumctr :=L eval(enum@\romannumeral\the\@enumdepth)
%            \list{\label(\@enumctr)}
%                 {\usecounter{\@enumctr}
%                  \makelabel{LABEL} ==  \hss \llap{LABEL}}
%     fi
%   END
%
% \endenumerate == \endlist
%
\newcount\@enumdepth \@enumdepth = 0
 
\@definecounter{enumi}
\@definecounter{enumii}
\@definecounter{enumiii}
\@definecounter{enumiv}
 
\def\enumerate{\ifnum \@enumdepth >3 \@toodeep\else
      \advance\@enumdepth \@ne
      \edef\@enumctr{enum\romannumeral\the\@enumdepth}\list
      {\csname label\@enumctr\endcsname}{\usecounter
        {\@enumctr}\def\makelabel##1{\hss\llap{##1}}}\fi}
 
\let\endenumerate =\endlist
 
 
%  \itemize ==
%    BEGIN
%      if \@itemdepth > 3
%        then  errormessage: 'Too deeply nested'.
%        else  \@itemdepth :=L \@itemdepth + 1
%              \@itemitem  == eval(labelitem\romannumeral\the\@itemdepth)
%              \list{\@nameuse{\@itemitem}}
%                   {\makelabel{LABEL} ==  \hss \llap{LABEL}}
%      fi
%    END
%
%  \enditemize ==  \endlist
%
\newcount\@itemdepth \@itemdepth = 0
 
\def\itemize{\ifnum \@itemdepth >3 \@toodeep\else \advance\@itemdepth \@ne
\edef\@itemitem{labelitem\romannumeral\the\@itemdepth}%
\list{\csname\@itemitem\endcsname}{\def\makelabel##1{\hss\llap{##1}}}\fi}
 
\let\enditemize =\endlist
 
 \message{boxes,}
%       *********************************************
%       *                 BOXES                     *
%       *********************************************
%
%  USER COMMANDS:
%
%  \makebox [WID][POS]{OBJ}
%          : puts OBJ in an \hbox of width WID, positioned by POS.
%            POS = l -> flushleft, POS = r -> flushright.
%            Default is centered.
%            If WID is missing, then POS is also missing and OBJ
%            is put in an \hbox of its natural width.
%
%  \mbox{OBJ} == \makebox{OBJ}, and is more efficient.
%
%  \makebox (X,Y)[POS]{OBJ}
%          : puts OBJ in an \hbox of width X * \unitlength
%            and height Y * \unitlength.  POS arguments are
%            l or r for flushleft, flushright and  t or b
%            for top, bottom -- or combinations like  tr or rb.
%            Default for horizontal and vertical are centered.
%
%  \newsavebox{\CMD} : If \CMD is undefined, then defines it
%           to be a TeX box register.
%
%  \savebox {\CMD} ... : \CMD is defined to be a TeX box register,
%            and the '...' are any \makebox arguments.  It is
%            like \makebox, except it doesn't produce text but
%            saves the value in \box \CMD.
%            \sbox N{OBJ} is an efficient abbreviation for
%            \savebox N{OBJ}.
%
%  \framebox ...  : like \makebox, except it puts a 'frame' around
%            the box.  The frame is made of lines of thickness
%            \fboxrule, separated by space \fboxsep from the
%            text -- except for \framebox(X,Y) ... , where the
%            thickness of the lines is as for the picture environment,
%            and there is no separation added.
%            \fbox{OBJ} is an efficient abbreviation for \framebox{OBJ}
%
%  \parbox[POS]{WIDTH}{TEXT} : Makes a box with \hsize TEXT, positioned
%         by POS as follows:
%              c : \vcenter (placed in $...$ if not in math mode)
%              b : \vbox
%              t : \vtop
%         default value is c.
%    Sets \hsize := WIDTH and calls \@parboxrestore, which does
%    the following:
%         Restores the original definitions of:
%              \par
%              \\
%              \- \' \` \=
%         Resets the following parameters:
%              \parindent        = 0pt
%              \parskip          = 0pt           %% added 20 Jan 87
%              \linewidth        = \hsize
%              \@totalleftmargin = 0pt
%              \leftskip         = 0pt
%              \rightskip        = 0pt
%              \@rightskip       = 0pt
%              \parfillskip      = 0pt plus 1fil
%              \lineskip         = \normallineskip
%              \baselineskip     = \normalbaselineskip
%         Calls \sloppy
%
%  Note: \@arrayparboxrestore same as \@parboxrestore
%         but it doesn't restore \\.
%
% \minipage  :  Similar to parbox, except it also
%        makes this look like a page by setting
%              \textwidth == \columnwidth == box width
%        changes footnotes by redefining:
%              \@mpfn         == mpfootnote
%              \thempfn       == \thempfootnote
%              \@footnotetext == \@mpfootnotetext
%        resets the following list environment parameters
%              \@listdepth    == \@mplistdepth
%        where  \@mplistdepth is initialized to zero,
%        and executes \@minipagerestore to allow the document
%        style to reset any other parameters it desires.
%        It sets @minipage := T, and resets \everypar to set
%        it false.  This switch keeps \addvspace from putting space
%        at the top of a minipage.
%
%        Change added 24 May 89: \minipage sets @minipage globally;
%        \endminipage resets it false.
%
%
% \rule [RAISED]{WIDTH}{HEIGHT} : Makes a WIDTH X HEIGHT rule, raised
%        RAISED.
%
% \underline {TEXT} : Makes an underlined hbox with TEXT in it.
%
% \raisebox{DISTANCE}[HEIGHT][DEPTH]{BOX} : Raises BOX up by DISTANCE
%        length (down if DISTANCE negative).  Makes TeX think that
%        the new box extends HEIGHT above the line and DEPTH below, for
%        a total vertical length of HEIGHT+DEPTH.  Default values of
%        HEIGHT & DEPTH = actual height and depth of box in new position.
%
% \makebox ==
%  BEGIN
%    if next char = (
%      then  \@makepicbox
%      else  if  next char = [
%              then  \@makebox
%              else  \mbox     fi
%    fi
%  END
%
% \@makebox[LEN] ==
% BEGIN
%   leave vertical mode
%   if next char '[' then \@imakebox[LEN]
%                    else \@imakebox[LEN][x]  fi
% END
%
% \@imakebox[LEN][POS]{OBJ} ==
%  BEGIN
%    \hbox to LEN
%      { \mb@l :=L \mb@r :=L \hss
%        \let\mb@POS = \relax
%       \mb@l OBJ \mb@r }
%  END
%
% \@makepicbox(X,Y) ==
%  BEGIN
%    leave vertical mode
%    if next char = [  then  \@imakepicbox(X,Y)
%                      else  \@imakepicbox(X,Y)[]  fi
%  END
%
% \@imakepicbox(X,Y)[POS]{OBJ} ==
%  BEGIN
%    \vbox to Y * \unitlength
%       { \mb@l :=L \mb@r :=L \hss
%         \mb@t :=L \mb@b :=L \hss
%         tfor \@tempa := POS    % one iteration for each token in POS
%           do  \mb@eval(\@tempa) :=L null od
%         \mb@t
%         \hbox to X * \unitlength
%           {\mb@l OBJ \mb@r }
%        \mb@b}
%  END
%
 
\def\makebox{\@ifnextchar ({\@makepicbox}{\@ifnextchar
     [{\@makebox}{\mbox}}}
 
\def\mbox#1{\leavevmode\hbox{#1}}
 
\def\@makebox[#1]{\leavevmode\@ifnextchar [{\@imakebox[#1]}{\@imakebox[#1][x]}}
 
\long\def\@imakebox[#1][#2]#3{\hbox to#1{\let\mb@l\hss
\let\mb@r\hss \expandafter\let\csname mb@#2\endcsname\relax
\mb@l #3\mb@r}}
 
\def\@makepicbox(#1,#2){\leavevmode\@ifnextchar
   [{\@imakepicbox(#1,#2)}{\@imakepicbox(#1,#2)[]}}
 
\long\def\@imakepicbox(#1,#2)[#3]#4{\vbox to#2\unitlength
   {\let\mb@b\vss \let\mb@l\hss\let\mb@r\hss
    \let\mb@t\vss
    \@tfor\@tempa :=#3\do{\expandafter\let
        \csname mb@\@tempa\endcsname\relax}%
\mb@t\hbox to #1\unitlength{\mb@l #4\mb@r}\mb@b}}
 
\def\newsavebox#1{\@ifdefinable#1{\newbox#1}}
 
\def\savebox#1{\@ifnextchar ({\@savepicbox#1}{\@ifnextchar
     [{\@savebox#1}{\sbox#1}}}
 
\def\sbox#1#2{\setbox#1\hbox{#2}}
 
\def\@savebox#1[#2]{\@ifnextchar [{\@isavebox#1[#2]}{\@isavebox#1[#2][x]}}
 
\long\def\@isavebox#1[#2][#3]#4{\setbox#1 \hbox{\@imakebox[#2][#3]{#4}}}
 
\def\@savepicbox#1(#2,#3){\@ifnextchar
   [{\@isavepicbox#1(#2,#3)}{\@isavepicbox#1(#2,#3)[]}}
 
\long\def\@isavepicbox#1(#2,#3)[#4]#5{\setbox#1 \hbox{\@imakepicbox
     (#2,#3)[#4]{#5}}}
 
\def\usebox#1{\leavevmode\copy #1\relax}
 
%% The following definition of \frame was written by Pavel Curtis
%% (Extra space removed 14 Jan 88)
\long\def\frame#1{\leavevmode
    \hbox{\hskip-\@wholewidth
     \vbox{\vskip-\@wholewidth
            \hrule \@height\@wholewidth
          \hbox{\vrule \@width\@wholewidth #1\vrule \@width\@wholewidth}\hrule
           \@height \@wholewidth\vskip -\@halfwidth}\hskip-\@wholewidth}}
 
\newdimen\fboxrule
\newdimen\fboxsep

%% (Extra space removed 21 Jun 1991)
\long\def\fbox#1{\leavevmode\setbox\@tempboxa\hbox{#1}\@tempdima\fboxrule
    \advance\@tempdima \fboxsep \advance\@tempdima \dp\@tempboxa
   \hbox{\lower \@tempdima\hbox
  {\vbox{\hrule \@height \fboxrule
          \hbox{\vrule \@width \fboxrule \hskip\fboxsep
          \vbox{\vskip\fboxsep \box\@tempboxa\vskip\fboxsep}\hskip
                 \fboxsep\vrule \@width \fboxrule}%
                 \hrule \@height \fboxrule}}}}
 
\def\framebox{\@ifnextchar ({\@framepicbox}{\@ifnextchar
     [{\@framebox}{\fbox}}}
 
\def\@framebox[#1]{\@ifnextchar [{\@iframebox[#1]}{\@iframebox[#1][x]}}

%% (Extra space removed 21 Jun 1991)
\long\def\@iframebox[#1][#2]#3{\leavevmode
  \savebox\@tempboxa[#1][#2]{\kern\fboxsep #3\kern\fboxsep}\@tempdima\fboxrule
    \advance\@tempdima \fboxsep \advance\@tempdima \dp\@tempboxa
   \hbox{\lower \@tempdima\hbox
  {\vbox{\hrule \@height \fboxrule
          \hbox{\vrule \@width \fboxrule \hskip-\fboxrule
              \vbox{\vskip\fboxsep \box\@tempboxa\vskip\fboxsep}\hskip
                  -\fboxrule\vrule \@width \fboxrule}%
                  \hrule \@height \fboxrule}}}}
 
\def\@framepicbox(#1,#2){\@ifnextchar
   [{\@iframepicbox(#1,#2)}{\@iframepicbox(#1,#2)[]}}
 
\long\def\@iframepicbox(#1,#2)[#3]#4{\frame{\@imakepicbox(#1,#2)[#3]{#4}}}
 
\def\parbox{\@ifnextchar [{\@iparbox}{\@iparbox[c]}}
 
\long\def\@iparbox[#1]#2#3{\leavevmode \@pboxswfalse
   \if #1b\vbox
     \else \if #1t\vtop
              \else \ifmmode \vcenter
                        \else \@pboxswtrue $\vcenter
                     \fi
           \fi
%% RmS 91/11/04 added \m@th
    \fi{\hsize #2\@parboxrestore #3}\if@pboxsw \m@th$\fi}
 
\let\@dischyph=\-
\let\@acci=\'
\let\@accii=\`
\let\@acciii=\=
 
 
\def\@arrayparboxrestore{\let\par\@@par
    \let\-\@dischyph
    \let\'\@acci \let\`\@accii \let\=\@acciii
    \parindent\z@ \parskip\z@
    \everypar{}\linewidth\hsize
    \@totalleftmargin\z@ \leftskip\z@ \rightskip\z@ \@rightskip\z@
    \parfillskip\@flushglue \lineskip\normallineskip
    \baselineskip\normalbaselineskip\sloppy}
 
\def\@parboxrestore{\@arrayparboxrestore\let\\=\@normalcr}
 
\newif\if@minipage \@minipagefalse
 
\def\minipage{\@ifnextchar [{\@iminipage}{\@iminipage[c]}}
 
\def\@iminipage[#1]#2{\leavevmode \@pboxswfalse
   \if #1b\vbox
     \else \if #1t\vtop
              \else \ifmmode \vcenter
                        \else \@pboxswtrue $\vcenter
                     \fi
           \fi
    \fi\bgroup
    \hsize #2\textwidth\hsize \columnwidth\hsize
    \@parboxrestore
    \def\@mpfn{mpfootnote}\def\thempfn{\thempfootnote}\c@mpfootnote\z@
    \let\@footnotetext\@mpfootnotetext
    \let\@listdepth\@mplistdepth \@mplistdepth\z@
    \@minipagerestore\global\@minipagetrue %% \global added 24 May 89
    \everypar{\global\@minipagefalse\everypar{}}}
 
 
\let\@minipagerestore=\relax
 
\def\endminipage{\par\vskip-\lastskip
\ifvoid\@mpfootins\else
\vskip\skip\@mpfootins\footnoterule\unvbox\@mpfootins\fi
\global\@minipagefalse   %% added 24 May 89
\egroup\if@pboxsw \m@th$\fi} %% RmS 91/11/04 added \m@th
 
\newcount\@mplistdepth
\newinsert\@mpfootins
 
%% RmS 91/09/29: added \reset@font
\long\def\@mpfootnotetext#1{\global\setbox\@mpfootins
    \vbox{\unvbox\@mpfootins
    \reset@font\footnotesize
    \hsize\columnwidth \@parboxrestore
   \edef\@currentlabel{\csname p@mpfootnote\endcsname\@thefnmark}\@makefntext
     {\rule{\z@}{\footnotesep}\ignorespaces #1\strut}}}
     % \strut added 27 Mar 89, on suggestion by Don Hosek
 
\newif\if@pboxsw
 
\def\rule{\@ifnextchar[{\@rule}{\@rule[\z@]}}
 
\def\@rule[#1]#2#3{\@tempdima#3\advance\@tempdima #1\leavevmode\hbox{\vrule
  \@width#2 \@height\@tempdima \@depth-#1}}
 
\let\@@underline\underline
\def\underline#1{\relax\ifmmode
  \@@underline{#1}\else $\@@underline{\hbox{#1}}\m@th$\relax\fi}
 
\def\raisebox#1{\@ifnextchar[{\@argrsbox{#1}}{\@rsbox{#1}}}
 
\def\@argrsbox#1[#2]{%
\@ifnextchar[{\@iirsbox{#1}[#2]}{\@irsbox{#1}[#2]}}
 
\long\def\@rsbox#1#2{\leavevmode\hbox{\raise #1\hbox{#2}}}
 
\long\def\@irsbox#1[#2]#3{\setbox\@tempboxa \hbox
   {\raise #1\hbox{#3}}\ht\@tempboxa#2\leavevmode\box\@tempboxa}
 
\long\def\@iirsbox#1[#2][#3]#4{\setbox\@tempboxa \hbox
   {\raise #1\hbox{#4}}\ht\@tempboxa#2\dp\@tempboxa#3\leavevmode\box\@tempboxa}
 
 
 \message{tabbing,}
%       ****************************************
%       *       THE TABBING ENVIRONMENT        *
%       ****************************************
%
%  \dimen(\@firsttab + i) = distance of tab stop i from left margin
%         0 <= i <= 15 (?).
%
%  \dimen\@firsttab is initialized to \@totalleftmargin, so it starts
%      at the prevailing left margin.
%
%  \@maxtab          = number of highest defined tab register
%                       probably = \@firsttab + 12
%  \@nxttabmar = tab stop number of next line's left margin
%  \@curtabmar = tab stop number of current line's left margin
%  \@curtab    = number of the current tab. At start of line,
%                      it equals \@curtabmar
%  \@hightab   = largest tab number currently defined.
%  \@tabpush   = depth of \pushtab's
%
%  \box\@curline     = contents of current line, excluding left margin skip,
%                      and excluding contents of current field
%  \box\@curfield    = contents of current field
%
%  @rjfield          = switch: T iff the last field of the line should be
%                      right-justified at the right margin.
%
%  \tabbingsep          = distance left by the \' command between the current
%                      position and the field that is ``left-shifted''.
%
%  UTILITY MACROS
%   \@stopfield  : closes the current field
%   \@addfield   : adds the current field to the current line.
%   \@contfield  : continues the current field
%   \@startfield : begins the next field
%   \@stopline   : closes the current line and outputs it
%   \@startline  : starts the next line
%   \@ifatmargin : an \if that is true iff the current line.
%                  has width zero
%
% \@startline ==
%  BEGIN
%   \@curtabmar :=G \@nxttabmar
%   \@curtab :=G \@curtabmar
%   \box\@curline :=G null
%   \@startfield
%   \strut
%  END
%
% \@stopline ==
%  BEGIN
%   \unskip
%   \@stopfield
%   if @rjfield = T
%     then  @rjfield :=G F
%           \@tempdima := \@totalleftmargin + \linewidth
%           \hbox to \@tempdima{\@itemfudge
%                               \hskip \dimen\@curtabmar
%                               \box\@curline
%                               \hfil
%                               \box\@curfield}
%     else \@addfield
%          \hbox {\@itemfudge
%                 \hskip \dimen\@curtabmar
%                 \box\@curline}
%   fi
%  END
%
% \@startfield ==
%  BEGIN
%    \box\@curfield :=G \hbox {
%  END
%
% \@stopfield ==
%  BEGIN
%     }
%  END
%
% \@contfield ==
%  BEGIN
%   \box\@curfield :=G \hbox { \unhbox\@currfield  %%} brace matching
%  END
% \@addfield ==
%  BEGIN
%   \box\@curline :=G \unbox\@curline * \unbox\@curfield
%  END
%
% \@ifatmargin ==
%  BEGIN
%   if  dim of box\@curline = 0pt  then
%  END
%
%
% \tabbing ==
%  BEGIN
%   \lineskip :=L 0pt
%   \> == \@rtab
%   \< == \@ltab
%   \= == \@settab
%   \+ == \@tabplus
%   \- == \@tabminus
%   \` == \@tabrj
%   \' == \@tablab
%   \\ == BEGIN \@stopline \@startline END
%   \\[DIST] == BEGIN \@stopline \vskip DIST \@startline\ignorespaces END
%   \\* == BEGIN \@stopline \penalty 10000 \@startline END
%   \\*[DIST] == BEGIN \@stopline \penalty 10000 \vskip DIST
%                      \@startline\ignorespaces               END
%   \@hightab :=G \@nxttabmar :=G \@firsttab
%   \@tabpush :=G 0
%   \dimen\@firsttab := \@totalleftmargin
%   @rjfield :=G F
%   \trivlist  \item[]
%   if @minipage = F then \vskip \parskip fi
%   \box\@tabfbox = \rlap{\indent\the\everypar}  % note: \the\everypar sets
%   \@itemfudge == BEGIN \box\@tabfbox END       %       @inlabel :=G F
%   \@startline
%   \ignorespaces
%  END
%
% \@endtabbing ==
%  BEGIN
%   \@stopline
%   if \@tabpush > 0 then error message: ''unmatched \poptabs'' fi
%   \endtrivlist
%  END
%
% \@rtab ==
%  BEGIN
%   \@stopfield
%   \@addfield
%   if \@curtab < \@hightab
%     then \@curtab :=G \@curtab + 1
%     else error message ``Undefined Tab''   fi
%   \@tempdima := \dimen\@curtab - \dimen\@curtabmar
%                        - width of box \@curline
%   \box\@curline :=G \hbox{\unhbox\@curline + \hskip\@tempdima}
%   \@startfield
%  END
%
% \@settab ==
%  BEGIN
%   \@stopfield
%   \@addfield
%   if \@curtab < \@maxtab
%     then \@curtab :=G \@curtab+1
%     else error message: ``Too many tabs''    fi
%   if \@curtab > \@hightab
%     then \@hightab :=L \@curtab    fi
%   \dimen\@curtab :=L \dimen\@curtabmar + width of \box\@curline
%   \@startfield
%  END
%
% \@ltab ==
%  BEGIN
%   \@ifatmargin
%     then if \@curtabmar > \@firsttab
%            then \@curtab :=G \@curtab - 1
%                 \@curtabmar :=G \@curtabmar - 1
%            else error message ``Too many untabs''      fi
%     else error message ``Left tab in middle of line''
%   fi
%  END
%
% \@tabplus ==
%  BEGIN
%        if  \@nxttabmar < \@hightab
%           then \@nxttabmar :=G \@nxttabmar+1
%           else error message ``Undefined tab''
%        fi
%  END
%
% \@tabminus ==
%  BEGIN
%        if \@nxttabmar > \@firsttab
%           then \@nxttabmar :=G \@nxttabmar-1
%           else error message ``Too many untabs''
%        fi
%  END
%
% \@tabrj ==
%  BEGIN \@stopfield
%        \@addfield
%        @rjfield :=G T
%        \@startfield
%  END
%
% \@tablab ==
%  BEGIN \@stopfield
%        \box\@curline G:= \hbox{ \box\@curline %% `G' added 17 Jun 86
%                                \hskip - width of \box\@curfield
%                                \hskip -\tabbingsep
%                                \box\@curfield
%                                \hskip \tabbingsep }
%        \@startfield
%  END
%
% \pushtabs ==
%   BEGIN
%     \@stopfield
%     \@tabpush :=G \@tabpush + 1
%     \begingroup
%     \@contfield
%   END
%
% \poptabs ==
%  BEGIN
%    \@stopfield
%    if \@tabpush > 0
%      then \endgroup
%           \@tabpush :=G \@tabpush - 1
%      else error message: ``Too many \poptabs''
%    fi
%    \@contfield
%  END
%
% The accents \` , \' , and \= that have been redefined inside a tabbing
% environment can be called by typing \a` , \a' , and \a=.
%
 
\expandafter \let \csname a`\endcsname = \`
\expandafter \let \csname a'\endcsname = \'
\expandafter \let \csname a=\endcsname = \=
\def\a#1{\csname a#1\endcsname}
 
\newif\if@rjfield
\newcount\@firsttab
\newcount\@maxtab
\newdimen\@gtempa \@firsttab=\allocationnumber
\newdimen\@gtempa\newdimen\@gtempa\newdimen\@gtempa\newdimen\@gtempa
\newdimen\@gtempa\newdimen\@gtempa\newdimen\@gtempa\newdimen\@gtempa
\newdimen\@gtempa\newdimen\@gtempa\newdimen\@gtempa\newdimen\@gtempa
\newdimen\@gtempa \@maxtab=\allocationnumber
\dimen\@firsttab=0pt
\newcount\@nxttabmar
\newcount\@curtabmar
\newcount\@curtab
\newcount\@hightab
\newcount\@tabpush
\newbox\@curline
\newbox\@curfield
\newbox\@tabfbox
 
\def\@startline{\global\@curtabmar\@nxttabmar
   \global\@curtab\@curtabmar\global\setbox\@curline\hbox % missing \global
    {}\@startfield\strut}                                 % added 17 Jun 86
 
\def\@stopline{\unskip\@stopfield\if@rjfield \global\@rjfieldfalse
   \@tempdima\@totalleftmargin \advance\@tempdima\linewidth
\hbox to\@tempdima{\@itemfudge\hskip\dimen\@curtabmar
   \box\@curline\hfil\box\@curfield}\else\@addfield
   \hbox{\@itemfudge\hskip\dimen\@curtabmar\box\@curline}\fi}
 
\def\@startfield{\global\setbox\@curfield\hbox\bgroup}%{ BRACE MATCH HACK
\let\@stopfield=}
\def\@contfield{\global\setbox\@curfield\hbox\bgroup\unhbox\@curfield}
\def\@addfield{\global\setbox\@curline\hbox{\unhbox
     \@curline\unhbox\@curfield}}
\def\@ifatmargin{\ifdim \wd\@curline =\z@}
 
\def\@tabcr{\@stopline \@ifstar{\penalty \@M \@xtabcr}{\@xtabcr}}
 
\def\@xtabcr{\@ifnextchar[{\@itabcr}{\@startline\ignorespaces}}
 
\def\@itabcr[#1]{\vskip #1\@startline\ignorespaces}
 
\def\kill{\@stopfield\@startline\ignorespaces}
 
% REMOVE \outer FROM PLAIN'S DEF OF \+
 
\def\+{\tabalign}
 
 
\def\tabbing{\lineskip \z@\let\>\@rtab\let\<\@ltab\let\=\@settab
     \let\+\@tabplus\let\-\@tabminus\let\`\@tabrj\let\'\@tablab
     \let\\=\@tabcr
     \global\@hightab\@firsttab
     \global\@nxttabmar\@firsttab
     \dimen\@firsttab\@totalleftmargin
     \global\@tabpush\z@ \global\@rjfieldfalse
     \trivlist \item[]\if@minipage\else\vskip\parskip\fi
     \setbox\@tabfbox\hbox{\rlap{\indent\hskip\@totalleftmargin
       \the\everypar}}\def\@itemfudge{\box\@tabfbox}\@startline\ignorespaces}
 
\def\endtabbing{\@stopline\ifnum\@tabpush >\z@ \@badpoptabs \fi\endtrivlist}
 
\def\@rtab{\@stopfield\@addfield\ifnum \@curtab<\@hightab
      \global\advance\@curtab \@ne \else\@badtab\fi
      \@tempdima\dimen\@curtab
      \advance\@tempdima -\dimen\@curtabmar
      \advance\@tempdima -\wd\@curline
      \global\setbox\@curline\hbox{\unhbox\@curline\hskip\@tempdima}%
      \@startfield\ignorespaces}
% Omitted \global added to \@rtab 17 Jun 86
 
\def\@settab{\@stopfield\@addfield\ifnum \@curtab <\@maxtab
      \global\advance\@curtab \@ne \else\@latexerr{Tab overflow}\@ehd\fi
      \ifnum\@curtab >\@hightab
      \@hightab\@curtab\fi
      \dimen\@curtab\dimen\@curtabmar
      \advance\dimen\@curtab \wd\@curline\@startfield\ignorespaces}
\def\@ltab{\@ifatmargin\ifnum\@curtabmar >\@firsttab
      \global\advance\@curtab \m@ne \global\advance\@curtabmar \m@ne \else
      \@badtab\fi\else
      \@latexerr{\string\<\space in mid line}\@ehd\fi\ignorespaces}
\def\@tabplus {\ifnum \@nxttabmar <\@hightab
      \global\advance\@nxttabmar \@ne \else
      \@badtab\fi\ignorespaces}
\def\@tabminus{\ifnum\@nxttabmar >\@firsttab
      \global\advance\@nxttabmar \m@ne \else
      \@badtab\fi\ignorespaces}
\def\@tabrj{\@stopfield\@addfield\global\@rjfieldtrue\@startfield\ignorespaces}
 
\def\@tablab{\@stopfield\global\setbox\@curline\hbox{\box\@curline
     \hskip -\wd\@curfield \hskip -\tabbingsep \box\@curfield
      \hskip \tabbingsep}\@startfield\ignorespaces}
% \setbox\@curline made \global in \@tablab. 17 Jun 86
 
\def\pushtabs{\@stopfield\@addfield\global\advance\@tabpush \@ne \begingroup
       \@contfield}
\def\poptabs{\@stopfield\@addfield\ifnum\@tabpush >\z@ \endgroup
     \global\advance\@tabpush \m@ne \else
     \@badpoptabs\fi\@contfield}
 
\newdimen\tabbingsep
 
 \message{array,}
%      ****************************************
%      *    ARRAY AND TABULAR ENVIRONMENTS    *
%      ****************************************
%
% ARRAY PARMETERS:
%  \arraycolsep    : half the width separating columns in an array environment
%  \tabcolsep      : half the width separating columns in a tabular environment
%  \arrayrulewidth : width of rules
%  \doublerulesep  : space between adjacent rules in array or tabular
%  \arraystretch   : line spacing in array and tabular environments is done by
%                    placing a strut in every row of height and depth
%                    \arraystretch times the height and depth of the strut
%                    produced by an ordinary \strut commmand.
%
% PREAMBLE:
%  The PREAMBLE argument of an array or tabular environment can contain
%  the following:
%    l,r,c  : indicate where entry is to be placed.
%    |      : for vertical rule
%    @{EXP} : inserts the text EXP in every column.  \arraycolsep or \tabcolsep
%             spacing is suppressed.
%    *{N}{PRE} : equivalent to writing N copies of PRE in the preamble.  PRE
%                may contain *{N'}{EXP'} expressions.
%    p{LEN} : makes entry in parbox of width LEN.
%
% SPECIAL ARRAY COMMANDS:
%   \multicolumn{N}{FORMAT}{ITEM} : replaces the next N column items by
%       ITEM, formatted according to FORMAT.  FORMAT should contain at most
%       one l,r or c.  If it contains none, then ITEM is ignored.
%
%   \vline : draws a vertical line the height of the current row.  May
%            appear in an array element entry.
%   \hline : draws a horizontal line between rows.  Must appear either
%            before the first entry (to appear above the first row) or right
%            after a \\ command.  If followed by another \hline, then adds
%            a \vskip of \doublerulesep.
%
%   \cline[i-j] : draws horizontal lines between rows covering columns
%                 i through j, inclusive.  Multiple commands may follow
%                 one another to provide lines covering several disjoint
%                 columns
%   \extracolsep{WIDTH} : for use inside an @ in the preamble.  Causes a WIDTH
%                 space to be added between columns for the rest of the
%                 columns.  This is in addition to the ordinary intercolumn
%                 space.
%
%  \array ==
%    BEGIN
%      \@acol    == \@arrayacol
%      \@classz  == \@arrayclassz
%      \@classiv == \@arrayclassiv
%      \\        == \@arraycr
%      \@halignto == NULL
%      \@tabarray
%    END
%
%  \endarray{NAME} ==  BEGIN  \crcr }}  END
%
%  \tabular  ==
%    BEGIN
%      \@halignto == NULL
%      \@tabular
%    END
%
%  \tabular*{WIDTH} ==
%    BEGIN
%      \@halignto == to WIDTH
%      \@tabular
%    END
%
%  \@tabular ==
%    BEGIN
%      \leavevmode
%      \hbox { $
%         \@acol    == \@tabacol
%         \@classz  == \@tabclassz
%         \@classiv == \@tabclassiv
%         \\        == \@tabularcr
%         \@tabarray
%    END
%
%  \endtabular == BEGIN \crcr}} $} END
%
%  \@tabarray == if next char = [ then \@array else \@array[c] fi
%
%  \@array[POS]{PREAMBLE} ==
%    BEGIN
%      define \@arstrutbox to make \@arstrut produce strut of height
%        and depth \arraystretch times the height and
%        depth of a normal strut.
%      \@mkpream{PREAMBLE}
%      \@preamble == \halign \@halignto {\tabskip=0pt\@arstrut
%                              eval{\@preamble}\tabskip = 0pt\cr  %% }
%      \@startpbox == \@@startpbox
%      \@endpbox == \@@endpbox
%      if POS = t then \vtop
%                 else if POS = b then \vbox
%                                 else \vcenter
%      fi              fi
%     {
%      \par          ==L \relax
%      \@sharp       == #
%      \protect      == \relax
%      \lineskip     :=L 0pt
%      \baselineskip :=L 0pt
%      \@preamble
%    END
%
%
%  \@arraycr ==
%   BEGIN
%     $                    %% Prevents extra space at end of row's last entry.
%     if next char = [
%      then  \@argarraycr
%      else  $ \cr         %% Needed to balance $
%   END
%
%  \@argarraycr[LENGTH] ==
%   BEGIN
%     $                    %% Needed to balance $ of \@arraycr
%     if LENGTH > 0
%       then  \@tempdima := depth of \@arstrutbox + LENGTH
%             \vrule height 0pt width 0pt depth \@tempdima
%             \cr
%       else  \cr \noalign{\vskip LENGTH}
%   END
%
%  \@tabularcr and \@argtabularcr  same as \@arraycr and \@argarraycr
%  except without the extra $'s.
 
\def\extracolsep#1{\tabskip #1\relax}
 
\def\array{\let\@acol\@arrayacol \let\@classz\@arrayclassz
 \let\@classiv\@arrayclassiv \let\\\@arraycr\let\@halignto\@empty\@tabarray}
 
\def\endarray{\crcr\egroup\egroup}
\def\endtabular{\crcr\egroup\egroup $\egroup}
\expandafter \let \csname endtabular*\endcsname = \endtabular
 
\def\tabular{\let\@halignto\@empty\@tabular}
 
\expandafter \def\csname tabular*\endcsname #1{\def\@halignto{to#1}\@tabular}
 
\def\@tabular{\leavevmode \hbox \bgroup $\let\@acol\@tabacol
   \let\@classz\@tabclassz
   \let\@classiv\@tabclassiv \let\\\@tabularcr\@tabarray}

%% RmS 91/11/04 added \m@th
\def\@tabarray{\m@th\@ifnextchar[{\@array}{\@array[c]}}
 
\def\@array[#1]#2{\setbox\@arstrutbox\hbox{\vrule
     \@height\arraystretch \ht\strutbox
     \@depth\arraystretch \dp\strutbox
     \@width\z@}\@mkpream{#2}\edef\@preamble{\halign \noexpand\@halignto
\bgroup \tabskip\z@ \@arstrut \@preamble \tabskip\z@ \cr}%
\let\@startpbox\@@startpbox \let\@endpbox\@@endpbox
  \if #1t\vtop \else \if#1b\vbox \else \vcenter \fi\fi
  \bgroup \let\par\relax
  \let\@sharp##\let\protect\relax \lineskip\z@\baselineskip\z@\@preamble}
 
\def\@arraycr{${\ifnum0=`}\fi\@ifstar{\@xarraycr}{\@xarraycr}}
\def\@xarraycr{\@ifnextchar[{\@argarraycr}{\ifnum0=`{\fi}${}\cr}}
 
\def\@argarraycr[#1]{\ifnum0=`{\fi}${}\ifdim #1>\z@ \@xargarraycr{#1}\else
   \@yargarraycr{#1}\fi}
 
\def\@tabularcr{{\ifnum0=`}\fi\@ifstar{\@xtabularcr}{\@xtabularcr}}
\def\@xtabularcr{\@ifnextchar[{\@argtabularcr}{\ifnum0=`{\fi}\cr}}
 
\def\@argtabularcr[#1]{\ifnum0=`{\fi}\ifdim #1>\z@
   \unskip\@xargarraycr{#1}\else \@yargarraycr{#1}\fi}
 
\def\@xargarraycr#1{\@tempdima #1\advance\@tempdima \dp \@arstrutbox
   \vrule \@height\z@ \@depth\@tempdima \@width\z@ \cr}
 
\def\@yargarraycr#1{\cr\noalign{\vskip #1}}
 
 
% \multicolumn{NUMBER}{FORMAT}{ITEM} ==
%  BEGIN
%  \multispan{NUMBER}
%  \begingroup
%  \@addamp == null
%  \@mkpream{FORMAT}
%  \@sharp == ITEM
%  \protect == \relax
%  \@startpbox == \@@startpbox
%  \@endpbox == \@@endpbox
%  \@arstrut
%  \@preamble
%  \endgroup
%  END
 
% The command \def\@addamp{} was removed from \multicolumn on 6 Dec 86
% because it caused embedded array environments not to work.  I think
% that it was included originally to prevent an error message if
% the 2nd argument to the \multicolumn command had two column specifiers.
%
% 8 Feb 89 - \hbox{} added after \@preamble to correct bug that
%            occurred if \multicolumn preceded \\[D] with D > 0,
%            caused by \\[] command doing an \unskip, which removed
%            \tabcolsep glue inserted by \multicolumn
 
\def\multicolumn#1#2#3{\multispan{#1}\begingroup
\@mkpream{#2}%
\def\@sharp{#3}\let\protect\relax
  \let\@startpbox\@@startpbox\let\@endpbox\@@endpbox
  \@arstrut \@preamble\hbox{}\endgroup\ignorespaces}
 
 
% Codes for classes and character numbers of array, tabular and
% multicolumn arguments.
%
%    Character     Class       Number
%    ---------     -----       ------
%        c           0           0
%        l           0           1
%        r           0           2
%
%        |           1           -
%        @           2           -
%        p           3           -
%      {@-exp}       4           -
%      {p-arg}       5           -
%
% \@testpach \foo : expands \foo, which should be an array parameter token,
%                   and sets \@chclass and \@chnum to its class and number.
%                   Uses \@lastchclass to distinguish 4 and 5
%
% Preamble error codes
%    0: 'illegal character'
%    1: 'Missing @-exp'
%    2: 'Missing p-arg'
%
% \@addamp ==
%   BEGIN if @firstamp = true then @firstamp := false
%                             else &                     fi
%   END
%
% \@mkpream TOKENLIST ==
%   BEGIN
%    @firstamp     := T
%    \@lastchclass := 6
%    \@preamble    == null
%    \@sharp       == \relax
%    \protect      == BEGIN \noexpand\protect\noexpand END
%    \@startpbox   == \relax
%    \@endpbox     == \relax
%    \@expast{TOKENLIST}
%    for \@nextchar := expand(\@tempa)
%      do  \@testpach{\@nextchar}
%          case of \@chclass
%            0 -> \@classz
%            1 -> \@classi
%              ...
%            5 -> \@classv
%          end case
%          \@lastchclass := \@chclass
%      od
%      case of \@lastchclass
%         0 -> \hskip \arraycolsep             % lrc
%         1 ->                                  % |
%         2 -> \@preamerr1 % 'Missing @-exp'    % @
%         3 -> \@preamerr2 % 'Missing p-arg'    % p
%         4 ->                                  % @-exp
%         5 -> \hskip \arraycolsep             % p-exp
%      end case
%   END
%
%  \@arrayclassz ==
%    BEGIN
%      \@preamble := \@preamble *
%                    case of \@lastchclass
%                       0 -> \hskip \arraycolsep \@addamp \hskip \arraycolsep
%                       1 -> \@addamp \hskip \arraycolsep
%                       2 ->  % impossible
%                       3 ->  % impossible
%                       4 -> \@addamp
%                       5 -> \hskip \arraycolsep \@addamp \hskip \arraycolsep
%                       6 -> \@addamp \hskip \arraycolsep
%                     end case
%                   * case of \@chnum
%                        0 -> \hfil$\relax\@sharp$\hfil
%                        1 -> $\relax\@sharp$\hfil
%                        2 -> \hfil$\relax\@sharp$
%                     end case
%    END
%
% \@tabclassz == similar to \@arrayclassz
%
% \@classi ==
%  BEGIN
%    \@preamble := \@preamble *
%                  case of \@lastchclass
%                     0 -> \hskip \arraycolsep \@arrayrule
%                     1 -> \hskip \doublerulesep \@arrayrule
%                     2 -> % impossible
%                     3 -> % impossible
%                     4 -> \@arrayrule
%                     5 -> \hskip \arraycolsep \@arrayrule
%                     6 -> \@arrayrule
%                  end case
%  END
%
% \@classii ==
%  BEGIN
%    \@preamble := \@preamble *
%                  case of \@lastchclass
%                     0    ->
%                     1    -> \hskip .5\arrayrulewidth
%                     2    -> % impossible
%                     else ->
%                  end case
%  END
%
% \@classiii ==
%  BEGIN
%    \@preamble := \@preamble *
%                  case of \@lastchclass
%                     0 -> \hskip \arraycolsep \@addamp \hskip \arraycolsep
%                     1 -> \@addamp \hskip \arraycolsep
%                     2 -> % impossible
%                     3 -> % impossible
%                     4 -> \@addamp
%                     5 -> \hskip \arraycolsep \@addamp \hskip \arraycolsep
%                     6 -> \@addamp \hskip \arraycolsep
%                  end case
%  END
%
% \@arrayclassiv  ==  BEGIN  \@preamble := \@preamble * $ \@nextchar$  END
%
% \@tabclassiv   == same as \@arrayclassv except without the $ ... $
%
% \@classv ==
%   BEGIN
%    \@preamble := \@preamble * \@startpbox{\@nextchar}\ignorespaces\@sharp
%                               \@endpbox
%   END
%
% \@expast{S}: Sets \@tempa := S with all instances of *{N}{STRING}
%              replaced by N copies of STRING, where N > 0.  An *
%              appearing inside braces is ignored, but *-expressions
%              inside STRING are expanded, so nested *-expressions are
%              handled properly.
%
% \@expast{S} == BEGIN  \@xexpast S *0x\@@  END
%
% \@xexpast S1 *{N}{S2} S3 \@@ ==
%  BEGIN
%    \@tempa   := S1
%    \@tempcnta := N
%    if \@tempcnta > 0
%      then  while \@tempcnta > 0 do \@tempa   := \@tempa S2
%                                   \@tempcnta := \@tempcnta - 1 od
%            \@tempb == \@xexpast
%      else  \@tempb == \@xexnoop
%    fi
%    \expandafter \@tempb \@tempa S3 \@@
%  END
%
 
\def\@xexnoop #1\@@{}
 
\def\@expast#1{\@xexpast #1*0x\@@}
 
\def\@xexpast#1*#2#3#4\@@{\edef\@tempa{#1}\@tempcnta#2\relax
    \ifnum\@tempcnta >\z@ \@whilenum\@tempcnta >\z@\do
       {\edef\@tempa{\@tempa#3}\advance\@tempcnta \m@ne}\let\@tempb\@xexpast
      \else \let\@tempb\@xexnoop\fi
    \expandafter\@tempb \@tempa #4\@@}
 
 
\newif\if@firstamp
\def\@addamp{\if@firstamp \@firstampfalse \else
    \edef\@preamble{\@preamble &}\fi}
\def\@arrayacol{\edef\@preamble{\@preamble \hskip \arraycolsep}}
\def\@tabacol{\edef\@preamble{\@preamble \hskip \tabcolsep}}
\def\@ampacol{\@addamp \@acol}
\def\@acolampacol{\@acol\@addamp\@acol}
 
\def\@mkpream#1{\@firstamptrue\@lastchclass6
\let\@preamble\@empty\def\protect{\noexpand\protect\noexpand}\let\@sharp\relax
\let\@startpbox\relax\let\@endpbox\relax
\@expast{#1}\expandafter\@tfor \expandafter
  \@nextchar \expandafter:\expandafter=\@tempa\do{\@testpach\@nextchar
  \ifcase \@chclass \@classz \or \@classi \or \@classii \or \@classiii
    \or \@classiv \or\@classv \fi\@lastchclass\@chclass}%
\ifcase \@lastchclass \@acol
    \or \or \@preamerr \@ne\or \@preamerr \tw@\or \or \@acol \fi}
 
\def\@arrayclassz{\ifcase \@lastchclass \@acolampacol \or \@ampacol \or
   \or \or \@addamp \or
   \@acolampacol \or \@firstampfalse \@acol \fi
\edef\@preamble{\@preamble
  \ifcase \@chnum
     \hfil$\relax\@sharp$\hfil \or $\relax\@sharp$\hfil
    \or \hfil$\relax\@sharp$\fi}}
 
%% RmS 91/08/14 inserted extra braces around entry for NFSS
\def\@tabclassz{\ifcase \@lastchclass \@acolampacol \or \@ampacol \or
   \or \or \@addamp \or
   \@acolampacol \or \@firstampfalse \@acol \fi
\edef\@preamble{\@preamble{%
  \ifcase \@chnum
     \hfil\ignorespaces\@sharp\unskip\hfil
     \or \ignorespaces\@sharp\unskip\hfil
     \or \hfil\hskip\z@ \ignorespaces\@sharp\unskip\fi}}}
 
\def\@classi{\ifcase \@lastchclass \@acol \@arrayrule \or
   \@addtopreamble{\hskip \doublerulesep}\@arrayrule\or
   \or \or \@arrayrule \or
   \@acol \@arrayrule \or \@arrayrule \fi}
 
 
\def\@classii{\ifcase \@lastchclass \or
   \@addtopreamble{\hskip .5\arrayrulewidth}\fi}
 
\def\@classiii{\ifcase \@lastchclass \@acolampacol \or
   \@addamp\@acol \or
   \or \or \@addamp \or
   \@acolampacol \or \@ampacol \fi}
 
\def\@tabclassiv{\@addtopreamble\@nextchar}
 
\def\@arrayclassiv{\@addtopreamble{$\@nextchar$}}
 
\def\@classv{\@addtopreamble{\@startpbox{\@nextchar}\ignorespaces
\@sharp\@endpbox}}
 
\def\@addtopreamble#1{\edef\@preamble{\@preamble #1}}
 
\newcount\@chclass
\newcount\@lastchclass
\newcount\@chnum
 
\newdimen\arraycolsep
\newdimen\tabcolsep
\newdimen\arrayrulewidth
\newdimen\doublerulesep
 
\def\arraystretch{1}    % Default value.
 
\newbox\@arstrutbox
\def\@arstrut{\relax\ifmmode\copy\@arstrutbox\else\unhcopy\@arstrutbox\fi}
 
 
\def\@arrayrule{\@addtopreamble{\hskip -.5\arrayrulewidth
   \vrule \@width \arrayrulewidth\hskip -.5\arrayrulewidth}}
 
\def\@testpach#1{\@chclass \ifnum \@lastchclass=\tw@ 4 \else
    \ifnum \@lastchclass=3 5 \else
     \z@ \if #1c\@chnum \z@ \else
                              \if #1l\@chnum \@ne \else
                              \if #1r\@chnum \tw@ \else
          \@chclass \if #1|\@ne \else
                    \if #1@\tw@ \else
                    \if #1p3 \else \z@ \@preamerr 0\fi
  \fi  \fi  \fi  \fi  \fi  \fi
\fi}
 
\def\hline{\noalign{\ifnum0=`}\fi\hrule \@height \arrayrulewidth \futurelet
   \@tempa\@xhline}
 
\def\@xhline{\ifx\@tempa\hline\vskip \doublerulesep\fi
      \ifnum0=`{\fi}}
 
\def\vline{\vrule \@width \arrayrulewidth}
 
\newcount\@cla
\newcount\@clb
 
\def\cline#1{\@cline[#1]}
\def\@cline[#1-#2]{\noalign{\global\@cla#1\relax
\global\advance\@cla\m@ne
\ifnum\@cla>\z@\global\let\@gtempa\@clinea\else
  \global\let\@gtempa\@clineb\fi
\global\@clb#2\relax
\global\advance\@clb-\@cla}\@gtempa
\noalign{\vskip-\arrayrulewidth}}
 
\def\@clinea{\multispan\@cla&\multispan\@clb
\unskip\leaders\hrule \@height \arrayrulewidth \hfill
\cr}
 
\def\@clineb{\multispan\@clb
\unskip\leaders\hrule \@height \arrayrulewidth \hfill
\cr}
 
% \@startpbox{WIDTH} TEXT \egroup == \parbox{WIDTH}{TEXT}
% \@endpbox == \unskip \strut \par \egroup\hfil (Changed 14 Jan 89)
%
 
\def\@startpbox#1{\vtop\bgroup \hsize #1\@arrayparboxrestore}
\def\@endpbox{\unskip\strut\par\egroup\hfil}
 
% 14 Jan 89: Def of \@endpbox changed from
%    \def\@endpbox{\par\vskip\dp\@arstrutbox\egroup\hfil}
% so vertical spacing works out right if the last line of a `p' entry
% has a descender.
 
\let\@@startpbox=\@startpbox
\let\@@endpbox=\@endpbox
 
 \message{picture,}
%      ****************************************
%      *       THE PICTURE ENVIRONMENT        *
%      ****************************************
%
%  \unitlength     = value of dimension argument
%  \@wholewidth    = current line width
%  \@halfwidth     = half of current line width
%  \@linefnt       = font for drawing lines
%  \@circlefnt     = font for drawing circles
%
% \linethickness{DIM} : Sets the width of horizontal and vertical lines
%     in a picture to DIM.  Does not change width of slanted lines
%     or circles.   Width of all lines reset by \thinlines and
%     \thicklines
%
% \picture(XSIZE,YSIZE)(XORG,YORG)
%   BEGIN
%     \@picht :=L YSIZE * \unitlength
%     box \@picbox :=
%          \hbox to XSIZE * \unitlength
%            {\hskip -XORG * \unitlength
%             \lower YORG * \unitlength
%             \hbox{
%             \ignorespaces    %% added 13 June 89
%   END
%
% \endpicture ==
%   BEGIN
%                   } \hss }
%                   heigth of \@picbox := \@picht
%                   depth  of \@picbox := 0
%                   \mbox{\box\@picbox}   %% change 26 Aug 91
%   END
%
% \put(X, Y){OBJ} ==
%   BEGIN
%     \@killglue
%     \raise Y * \unitlength  \hbox to 0pt { \hskip X * \unitlength
%                                              OBJ \hss             }
%     \ignorespaces
%   END
%
% \multiput(X,Y)(DELX,DELY){N}{OBJ} ==
%   BEGIN
%    \@killglue
%    \@multicnt := N
%    \@xdim  := X * \unitlength
%    \@ydim  := Y * \unitlength
%    while \@multicnt > 0
%      do \raise \@ydim \hbox to 0pt { \hskip \@xdim
%                                             OBJ \hss   }
%         \@multicnt := \@multicnt - 1
%         \@xdim     := \@xdim + DELX * \unitlength
%         \@ydim     := \@ydim + DELY * \unitlength
%      od
%    \ignorespaces
%   END
%
%  \shortstack[POS]{TEXT} : Makes a \vbox containing TEXT stacked as
%      a one-column array, positioned l, r or c as indicated by POS.
 
\newdimen\@wholewidth
\newdimen\@halfwidth
\newdimen\unitlength \unitlength =1pt
\newbox\@picbox
\newdimen\@picht
 
\def\picture(#1,#2){\@ifnextchar({\@picture(#1,#2)}{\@picture(#1,#2)(0,0)}}
 
\def\@picture(#1,#2)(#3,#4){\@picht #2\unitlength
\setbox\@picbox\hbox to#1\unitlength\bgroup
\hskip -#3\unitlength \lower #4\unitlength \hbox\bgroup\ignorespaces}
 
%% 91/08/26 RmS & FMi: extra boxing level around \@picbox
%%     to guard against unboxing in math mode
%%     (proposed by John Hobby)

\def\endpicture{\egroup\hss\egroup\ht\@picbox\@picht
\dp\@picbox\z@\mbox{\box\@picbox}}
 
% In the definitions of \put and \multiput, \hskip was replaced by \kern
% just in case arg #3 = ``plus''.  (Bug detected by Don Knuth.
% changed 20 Jul 87).
%
\long\def\put(#1,#2)#3{\@killglue\raise#2\unitlength\hbox to\z@{\kern
#1\unitlength #3\hss}\ignorespaces}
 
\long\def\multiput(#1,#2)(#3,#4)#5#6{\@killglue\@multicnt #5\relax
\@xdim #1\unitlength
\@ydim #2\unitlength
\@whilenum \@multicnt >\z@\do
{\raise\@ydim\hbox to\z@{\kern
\@xdim #6\hss}\advance\@multicnt \m@ne\advance\@xdim
#3\unitlength\advance\@ydim #4\unitlength}\ignorespaces}
 
\def\@killglue{\unskip\@whiledim \lastskip >\z@\do{\unskip}}
 
\def\thinlines{\let\@linefnt\tenln \let\@circlefnt\tencirc
  \@wholewidth\fontdimen8\tenln \@halfwidth .5\@wholewidth}
\def\thicklines{\let\@linefnt\tenlnw \let\@circlefnt\tencircw
  \@wholewidth\fontdimen8\tenlnw \@halfwidth .5\@wholewidth}
 
\def\linethickness#1{\@wholewidth #1\relax \@halfwidth .5\@wholewidth}
 
\def\shortstack{\@ifnextchar[{\@shortstack}{\@shortstack[c]}}
 
\def\@shortstack[#1]{\leavevmode
\vbox\bgroup\baselineskip-\p@\lineskip 3\p@\let\mb@l\hss
\let\mb@r\hss \expandafter\let\csname mb@#1\endcsname\relax
\let\\\@stackcr\@ishortstack}
 
%% RmS 91/08/14 inserted extra braces around entry for NFSS
\def\@ishortstack#1{\halign{\mb@l {##}\unskip\mb@r\cr #1\crcr}\egroup}
 
 
\def\@stackcr{\@ifstar{\@ixstackcr}{\@ixstackcr}}
\def\@ixstackcr{\@ifnextchar[{\@istackcr}{\cr\ignorespaces}}
 
\def\@istackcr[#1]{\cr\noalign{\vskip #1}\ignorespaces}
 
 
% \line(X,Y){LEN} ==
% BEGIN
%  \@xarg    := X
%  \@yarg    := Y
%  \@linelen := LEN * \unitlength
%  if \@xarg = 0
%     then \@vline
%     else if \@yarg = 0
%            then \@hline
%            else \@sline
%          if
%  if
% END
%
% \@sline ==
%  BEGIN
%    if \@xarg < 0
%      then @negarg := T
%           \@xarg  := -\@xarg
%           \@yyarg := -\@yarg
%      else @negarg := F
%           \@yyarg := \@yarg
%    fi
%    \@tempcnta := |\@yyarg|
%    if \@tempcnta > 6
%      then error: 'LATEX ERROR: Illegal \line or \vector argument.'
%           \@tempcnta := 0
%    fi
%    \box\@linechar := \hbox{\@linefnt \@getlinechar(\@xarg,\@yyarg) }
%     if \@yarg > 0 then \@upordown = \raise
%                         \@clnht := 0
%                   else \@upordown = \lower
%                        \@clnht := height of \box\@linechar
%     fi
%     \@clnwd  := width of \box\@linechar
%     if @negarg
%       then \hskip - width of \box\@linechar
%            \@tempa == \hskip - 2* width of box \@linechar
%       else \@tempa == \relax
%     fi
%  %% Put out integral number of line segments
%     while \@clnwd <  \@linelen
%       do  \@upordown \@clnht \copy\@linechar
%           \@tempa
%           \@clnht := \@clnht + ht of \box\@linechar
%           \@clnwd := \@clnwd + width of \box\@linechar
%       od
%
%  %% Put out last segment
%     \@clnht := \@clnht - height of \box\@linechar
%     \@clnwd := \@clnwd - width of \box\@linechar
%     \@tempdima   := \@linelen - \@clnwd
%     \@tempdimb   := \@tempdima - width of \box\@linechar
%     if @negarg  then \hskip -\@tempdimb
%                 else \hskip  \@tempdimb
%     fi
%     \@tempdima   := 1000 * \@tempdima
%     \@tempcnta   := \@tempdima / width of \box\@linechar
%     \@tempdima   := (\@tempcnta * ht of \box\@linechar)/1000
%     \@clnht := \@clnht + \@tempdima
%     if \@linelen < width of box\@linechar
%         then \hskip width of box\@linechar
%         else \hbox{\@upordown \@clnht \copy\@linechar}
%     fi
% END
%
% \@hline ==
%   BEGIN
%     if \@xarg < 0 then  \hskip -\@linelen \fi
%     \vrule height \@halfwidth depth \@halfwidth width \@linelen
%     if \@xarg < 0 then  \hskip -\@linelen \fi
%  END
%
% \@vline == if \@yarg < 0 \@downline else \@upline  fi
%
%
% \@getlinechar(X,Y) ==
%   BEGIN
%     \@tempcnta := 8*X - 9
%     if Y > 0
%       then \@tempcnta := \@tempcnta + Y
%       else \@tempcnta := \@tempcnta - Y + 64
%     fi
%     \char\@tempcnta
%   END
%
% \vector(X,Y){LEN} ==
% BEGIN
%  \@xarg    := X
%  \@yarg    := Y
%  \@linelen := LEN * \unitlength
%  if \@xarg = 0
%     then \@vvector
%     else if \@yarg = 0
%            then \@hvector
%            else \@svector
%          if
%  if
% END
%
% \@hvector ==
%   BEGIN
%     \@hline
%     {\@linefnt if \@xarg < 0 then  \@getlarrow(1,0)
%                              else  \@getrarrow(1,0)
%                 fi}
%   END
%
% \@vvector == if \@yarg < 0 \@downvector else \@upvector  fi
%
% \@svector ==
%  BEGIN
%   \@sline
%   \@tempcnta := |\@yarg|
%     if  \@tempcnta < 5
%        then  \hskip - width of \box\@linechar
%              \@upordown \@clnht \hbox
%                       {\@linefnt
%                        if @negarg then \@getlarrow(\@xarg,\@yyarg)
%                                   else \@getrarrow(\@xarg,\@yyarg)
%                        fi }
%        else  error: 'LATEX ERROR: Illegal \line or \vector argument.'
%     fi
%  END
%
% \@getlarrow(X,Y) ==
%  BEGIN
%   if Y = 0
%     then \@tempcnta := '33
%     else \@tempcnta := 16 * X  -  9
%          \@tempcntb := 2 * Y
%          if \@tempcntb > 0
%            then  \@tempcnta := \@tempcnta  +  \@tempcntb
%            else  \@tempcnta := \@tempcnta  -  \@tempcntb +  64
%          fi
%   fi
%   \char\@tempcnta
%  END
%
% \@getrarrow(X,Y) ==
%  BEGIN
%   \@tempcntb := |Y|
%   case of \@tempcntb
%     0 : \@tempcnta := '55
%     1 : if X < 3
%           then \@tempcnta :=  24*X - 6
%           else if X = 3
%                  then \@tempcnta := 49
%                  else \@tempcnta := 58  fi
%         fi
%     2 : if X < 3
%           then \@tempcnta :=  24*X - 3
%           else \@tempcnta := 51     % X must = 3
%         fi
%     3 : \@tempcnta := 16*X - 2
%     4 : \@tempcnta := 16*X + 7
%   endcase
%   if Y < 0
%     then \@tempcnta := \@tempcnta + 64
%   fi
%   \char\@tempcnta
%  END
 
\newif\if@negarg
 
\def\line(#1,#2)#3{\@xarg #1\relax \@yarg #2\relax
\@linelen #3\unitlength
\ifnum\@xarg =\z@ \@vline
  \else \ifnum\@yarg =\z@ \@hline \else \@sline\fi
\fi}
 
\def\@sline{\ifnum\@xarg<\z@ \@negargtrue \@xarg -\@xarg \@yyarg -\@yarg
  \else \@negargfalse \@yyarg \@yarg \fi
\ifnum \@yyarg >\z@ \@tempcnta\@yyarg \else \@tempcnta -\@yyarg \fi
\ifnum\@tempcnta>6 \@badlinearg\@tempcnta\z@ \fi
\ifnum\@xarg>6 \@badlinearg\@xarg \@ne \fi
\setbox\@linechar\hbox{\@linefnt\@getlinechar(\@xarg,\@yyarg)}%
\ifnum \@yarg >\z@ \let\@upordown\raise \@clnht\z@
   \else\let\@upordown\lower \@clnht \ht\@linechar\fi
\@clnwd \wd\@linechar
\if@negarg \hskip -\wd\@linechar \def\@tempa{\hskip -2\wd\@linechar}\else
     \let\@tempa\relax \fi
\@whiledim \@clnwd <\@linelen \do
  {\@upordown\@clnht\copy\@linechar
   \@tempa
   \advance\@clnht \ht\@linechar
   \advance\@clnwd \wd\@linechar}%
\advance\@clnht -\ht\@linechar
\advance\@clnwd -\wd\@linechar
\@tempdima\@linelen\advance\@tempdima -\@clnwd
\@tempdimb\@tempdima\advance\@tempdimb -\wd\@linechar
\if@negarg \hskip -\@tempdimb \else \hskip \@tempdimb \fi
\multiply\@tempdima \@m
\@tempcnta \@tempdima \@tempdima \wd\@linechar \divide\@tempcnta \@tempdima
\@tempdima \ht\@linechar \multiply\@tempdima \@tempcnta
\divide\@tempdima \@m
\advance\@clnht \@tempdima
\ifdim \@linelen <\wd\@linechar
   \hskip \wd\@linechar
  \else\@upordown\@clnht\copy\@linechar\fi}
 
\def\@hline{\ifnum \@xarg <\z@ \hskip -\@linelen \fi
\vrule \@height \@halfwidth \@depth \@halfwidth \@width \@linelen
\ifnum \@xarg <\z@ \hskip -\@linelen \fi}
 
 \def\@getlinechar(#1,#2){\@tempcnta#1\relax\multiply\@tempcnta 8%
\advance\@tempcnta -9\ifnum #2>\z@ \advance\@tempcnta #2\relax\else
\advance\@tempcnta -#2\relax\advance\@tempcnta 64 \fi
\char\@tempcnta}
 
\def\vector(#1,#2)#3{\@xarg #1\relax \@yarg #2\relax
\@tempcnta \ifnum\@xarg<\z@ -\@xarg\else\@xarg\fi
\ifnum\@tempcnta<5\relax
\@linelen #3\unitlength
\ifnum\@xarg =\z@ \@vvector
  \else \ifnum\@yarg =\z@ \@hvector \else \@svector\fi
\fi
\else\@badlinearg\fi}
 
\def\@hvector{\@hline\hbox to\z@{\@linefnt
\ifnum \@xarg <\z@ \@getlarrow(1,0)\hss\else
    \hss\@getrarrow(1,0)\fi}}
 
\def\@vvector{\ifnum \@yarg <\z@ \@downvector \else \@upvector \fi}
 
\def\@svector{\@sline
\@tempcnta\@yarg \ifnum\@tempcnta <\z@ \@tempcnta -\@tempcnta\fi
\ifnum\@tempcnta <5%
  \hskip -\wd\@linechar
  \@upordown\@clnht \hbox{\@linefnt  \if@negarg
  \@getlarrow(\@xarg,\@yyarg)\else \@getrarrow(\@xarg,\@yyarg)\fi}%
\else\@badlinearg\fi}
 
\def\@getlarrow(#1,#2){\ifnum #2=\z@ \@tempcnta'33 \else
\@tempcnta #1\relax\multiply\@tempcnta \sixt@@n \advance\@tempcnta
-9 \@tempcntb #2\relax\multiply\@tempcntb \tw@
\ifnum \@tempcntb >\z@ \advance\@tempcnta \@tempcntb
\else\advance\@tempcnta -\@tempcntb\advance\@tempcnta 64
\fi\fi\char\@tempcnta}
 
\def\@getrarrow(#1,#2){\@tempcntb #2\relax
\ifnum\@tempcntb <\z@ \@tempcntb -\@tempcntb\relax\fi
\ifcase \@tempcntb\relax \@tempcnta'55 \or
\ifnum #1<\thr@@ \@tempcnta #1\relax\multiply\@tempcnta
24\advance\@tempcnta -6 \else \ifnum #1=\thr@@ \@tempcnta 49
\else\@tempcnta 58 \fi\fi\or
\ifnum #1<\thr@@ \@tempcnta=#1\relax\multiply\@tempcnta
24\advance\@tempcnta -\thr@@ \else \@tempcnta 51 \fi\or
\@tempcnta #1\relax\multiply\@tempcnta
\sixt@@n \advance\@tempcnta -\tw@ \else
\@tempcnta #1\relax\multiply\@tempcnta
\sixt@@n \advance\@tempcnta 7 \fi\ifnum #2<\z@ \advance\@tempcnta 64 \fi
\char\@tempcnta}
 
 
 
\def\@vline{\ifnum \@yarg <\z@ \@downline \else \@upline\fi}
 
\def\@upline{\hbox to \z@{\hskip -\@halfwidth \vrule \@width \@wholewidth
   \@height \@linelen \@depth \z@\hss}}
 
\def\@downline{\hbox to \z@{\hskip -\@halfwidth \vrule \@width \@wholewidth
   \@height \z@ \@depth \@linelen \hss}}
 
\def\@upvector{\@upline\setbox\@tempboxa\hbox{\@linefnt\char'66}\raise
     \@linelen \hbox to\z@{\lower \ht\@tempboxa\box\@tempboxa\hss}}
 
\def\@downvector{\@downline\lower \@linelen
      \hbox to \z@{\@linefnt\char'77\hss}}
 
% \dashbox{D}(X,Y) ==
%  BEGIN
%  leave vertical mode
%  \hbox to 0pt {
%       \baselineskip := 0pt
%       \lineskip     := 0pt
%  %% HORIZONTAL DASHES
%       \@dashdim := X * \unitlength
%       \@dashcnt := \@dashdim + 200 % to prevent roundoff error
%       \@dashdim := D * \unitlength
%       \@dashcnt := \@dashcnt / \@dashdim
%       if \@dashcnt is odd
%         then \@dashdim := 0pt
%              \@dashcnt := (\@dashcnt + 1) / 2
%         else \@dashdim := \@dashdim / 2
%              \@dashcnt := \@dashcnt / 2 - 1
%              \box\@dashbox   := \hbox{\vrule height \@halfwidth
%                                    depth \@halfwidth width \@dashdim}
%              \put(0,0){\copy\@dashbox}
%              \put(0,Y){\copy\@dashbox}
%              \put(X,0){\hskip -\@dashdim\copy\@dashbox}
%              \put(X,Y){\hskip -\@dashdim\box\@dashbox}
%              \@dashdim := 3 * \@dashdim
%       fi
%       \box\@dashbox := \hbox{\vrule height \@halfwidth
%                                 depth \@halfwidth width D * \unitlength
%                              \hskip D * \unitlength}
%       \@tempcnta := 0
%       \put(0,0){\hskip \@dashdim
%                while \@tempcnta < \@dascnt
%                  do \copy\@dashbox
%                     \@tempcnta := \@tempcnta + 1
%                  od
%               }
%       \@tempcnta := 0
%       put(0,Y){\hskip \@dashdim
%                while \@tempcnta < \@dascnt
%                  do \copy\@dashbox
%                     \@tempcnta := \@tempcnta + 1
%                  od
%               }
%
% %% vertical dashes
%       \@dashdim := Y * \unitlength
%       \@dashcnt := \@dashdim + 200 % to prevent roundoff error
%       \@dashdim := D * \unitlength
%       \@dashcnt := \@dashcnt / \@dashdim
%       if \@dashcnt is odd
%         then \@dashdim := 0pt
%              \@dashcnt := (\@dashcnt + 1) / 2
%         else \@dashdim := \@dashdim / 2
%              \@dashcnt := \@dashcnt / 2 - 1
%              \box\@dashbox   := \hbox{\hskip -\@halfwidth
%                                       \vrule width \@wholewidth
%                                                height \@dashdim  }
%              \put(0,0){\copy\@dashbox}
%              \put(X,0){\copy\@dashbox}
%              \put(0,Y){\lower\@dashdim\copy\@dashbox}
%              \put(X,Y){\lower\@dashdim\copy\@dashbox}
%              \@dashdim := 3 * \@dashdim
%       fi
%       \box\@dashbox := \hbox{\vrule width \@wholewidth
%                                 height D * \unitlength       }
%       \@tempcnta := 0
%       put(0,0){\hskip -\halfwidth
%                \vbox{while \@tempcnta < \@dashcnt
%                       do \vskip D*\unitlength
%                          \copy\@dashbox
%                          \@tempcnta := \@tempcnta + 1
%                       od
%                      \vskip \@dashdim
%                     } }
%       \@tempcnta := 0
%       put(X,0){\hskip -\halfwidth
%                \vbox{while \@tempcnta < \@dashcnt
%                       do \vskip D*\unitlength
%                          \copy\@dashbox
%                          \@tempcnta := \@tempcnta + 1
%                       od
%                      \vskip \@dashdim
%                     }
%               }
%    }     % END DASHES
%
%  \@imakepicbox(X,Y)
% END
 
\def\dashbox#1(#2,#3){\leavevmode\hbox to\z@{\baselineskip \z@
\lineskip \z@
\@dashdim #2\unitlength
\@dashcnt \@dashdim \advance\@dashcnt 200
\@dashdim #1\unitlength\divide\@dashcnt \@dashdim
\ifodd\@dashcnt\@dashdim \z@
\advance\@dashcnt \@ne \divide\@dashcnt \tw@
\else \divide\@dashdim \tw@ \divide\@dashcnt \tw@
\advance\@dashcnt \m@ne
\setbox\@dashbox \hbox{\vrule \@height \@halfwidth \@depth \@halfwidth
\@width \@dashdim}\put(0,0){\copy\@dashbox}%
\put(0,#3){\copy\@dashbox}%
\put(#2,0){\hskip-\@dashdim\copy\@dashbox}%
\put(#2,#3){\hskip-\@dashdim\box\@dashbox}%
\multiply\@dashdim \thr@@
\fi
\setbox\@dashbox \hbox{\vrule \@height \@halfwidth \@depth \@halfwidth
\@width #1\unitlength\hskip #1\unitlength}\@tempcnta\z@
\put(0,0){\hskip\@dashdim \@whilenum \@tempcnta <\@dashcnt
\do{\copy\@dashbox\advance\@tempcnta \@ne }}\@tempcnta\z@
\put(0,#3){\hskip\@dashdim \@whilenum \@tempcnta <\@dashcnt
\do{\copy\@dashbox\advance\@tempcnta \@ne }}%
\@dashdim #3\unitlength
\@dashcnt \@dashdim \advance\@dashcnt 200
\@dashdim #1\unitlength\divide\@dashcnt \@dashdim
\ifodd\@dashcnt \@dashdim \z@
\advance\@dashcnt \@ne \divide\@dashcnt \tw@
\else
\divide\@dashdim \tw@ \divide\@dashcnt \tw@
\advance\@dashcnt \m@ne
\setbox\@dashbox\hbox{\hskip -\@halfwidth
\vrule \@width \@wholewidth
\@height \@dashdim}\put(0,0){\copy\@dashbox}%
\put(#2,0){\copy\@dashbox}%
\put(0,#3){\lower\@dashdim\copy\@dashbox}%
\put(#2,#3){\lower\@dashdim\copy\@dashbox}%
\multiply\@dashdim \thr@@
\fi
\setbox\@dashbox\hbox{\vrule \@width \@wholewidth
\@height #1\unitlength}\@tempcnta\z@
\put(0,0){\hskip -\@halfwidth \vbox{\@whilenum \@tempcnta <\@dashcnt
\do{\vskip #1\unitlength\copy\@dashbox\advance\@tempcnta \@ne }%
\vskip\@dashdim}}\@tempcnta\z@
\put(#2,0){\hskip -\@halfwidth \vbox{\@whilenum \@tempcnta<\@dashcnt
\do{\vskip #1\unitlength\copy\@dashbox\advance\@tempcnta \@ne }%
\vskip\@dashdim}}}\@makepicbox(#2,#3)}
 
% CIRCLES AND OVALS
%
%  USER COMMANDS:
%
%  \circle{D} : Produces the circle with the diameter as close as
%               possible to D * \unitlength.  \put(X,Y){\circle{D}}
%               puts the circle with its center at (X,Y).
%
%  \oval(X,Y) : Makes an oval as round as possible that fits in the
%               rectangle of width X * \unitlength and height
%               Y * \unitlength. The reference point is the center.
%
% \oval(X,Y)[POS] : Save as \oval(X,Y) except it draws only the
%                   half or quadrant of the oval indicated by POS.
%                   E.G., \oval(X,Y)[t] draws just the top half
%                   and \oval(X,Y)[br] draws just the bottom right
%                   quadrant.  In all cases, the reference point is
%                   the same as the unqualified \oval(X,Y) command.
%
% \@ovvert {DELTA1} {DELTA2} : Makes a vbox containing either the left side
%        or the right side of the oval being constructed.  The baseline
%        will coincide with the outside bottom edge of the oval; the left
%        side of the box will coincide with the left edge of the vertical
%        rule.  The width of the box will be \@tempdima.
%        DELTA1 and DELTA2 are added to the character number in \@tempcnta
%        to get the characters for the top and bottom quarter circle pieces.
%
% \@ovhorz : Makes an hbox containing the straight rule for either the
%         top or the bottom of the oval being constructed.  The baseline
%         will coincide with bottom edge of the rule; the left side of
%         the box will coincide with the left side of the oval.
%         The width of the box will be \@ovxx.
%
% \@getcirc {DIAM} : Sets \@tempcnta to the character number
%                   of the top-right quarter circle with the largest
%                   diameter less than or equal to DIAM.
%                   Sets \@tempboxa to an hbox containing that character.
%                   Sets \@tempdima to \wd \@tempboxa, which is the distance
%                   from the circle's left outside edge to its right
%                   inside edge.
%                   (These characters are like those described in the
%                   TeXbook, pp. 389-90.)
%
% \@getcirc {DIAM} ==
%   BEGIN
%     \@tempcnta       := integer coercion of (DIAM + 2pt)     %% + 2pt added
%     \@tempcnta       := \@tempcnta / integer coercion of 4pt %%    1 Nov 88
%     if \@tempcnta > 10
%       then \@tempcnta := 10 fi
%     if \@tempcnta > 0
%       then \@tempcnta := \@tempcnta-1
%       else LaTeX Warning: Oval too small.
%     fi
%     \@tempcnta       := 4 * \@tempcnta
%     \@tempboxa       := \hbox{\@circlefnt \char \@tempcnta}
%     \@tempdima       := \wd \@tempboxa
%   END
%
% \@put{X}{Y}{OBJ} ==
%   BEGIN
%     \raise Y \hbox to 0pt{\hskip X OBJ \hss}
%   END
%
% \@oval(X,Y)[POS] ==
%   BEGIN
%     \begingroup
%       \boxmaxdepth := \maxdimen
%       @ovt := @ovb := @ovl := @ovr := true
%       for all E in POS
%         do  @ovE := false od
%       \@ovxx      := X * \unitlength
%       \@ovyy      := Y * \unitlength
%       \@tempdimb := min(\@ovxx,\@ovyy)
%       \@getcirc{\@tempdimb-2pt}   %% "-2pt" added 7 Dec 89
%       \@ovro     := \ht \@tempboxa
%       \@ovri     := \dp \@tempboxa
%       \@ovdx     := \@ovxx - \@tempdima
%       \@ovdx     := \@ovdx/2
%       \@ovdy     := \@ovyy - \@tempdima
%       \@ovdy     := \@ovyy/2
%       \@circlefnt
%       \@tempboxa :=
%           \hbox{
%                 if @ovr
%                   then \@ovvert{3}{2} \kern -\@tempdima
%                 fi
%                 if @ovl
%                   then \kern \@ovxx \@ovvert{0}{1} \kern -\@tempdima
%                        \kern -\@ovxx
%                 fi
%                 if @ovt
%                   then \@ovhorz \kern -\@ovxx
%                 fi
%                 if @ovb
%                   then \raise \@ovyy \@ovhorz
%                 fi
%                }
%       \@ovdx    := \@ovdx + \@ovro
%       \@ovdy    := \@ovdy + \@ovro
%      \ht\@tempboxa := \dp\@tempboxa := 0
%       \@put{-\@ovdx}{-\@ovdy}{\box\@tempboxa}
%    \endgroup
%   END
%
% \@ovvert {DELTA1} {DELTA2} ==
%   BEGIN
%      \vbox to \@ovyy {
%                      if @ovb
%                        then \@tempcntb := \@tempcnta + DELTA1
%                             \kern -\@ovro
%                             \hbox { \char \@tempcntb }
%                             \nointerlineskip
%                        else \kern \@ovri \kern \@ovdy
%                      fi
%                      \leaders \vrule width \@wholewidth \vfil
%                      \nointerlineskip
%                      if @ovt
%                        then \@tempcntb := \@tempcnta + DELTA2
%                             \hbox { \char \@tempcntb }
%                        else \kern \@ovdy \kern \@ovro
%                      fi
%                     }
%   END
%
% \@ovhorz ==
%   BEGIN
%    \hbox to \@ovxx{
%                   \kern \@ovro
%                   if @ovr
%                     then
%                     else \kern \@ovdx
%                   fi
%                   \leaders \hrule height \@wholewidth \hfil
%                   if @ovl
%                     then
%                     else \kern \@ovdx
%                   fi
%                   \kern \@ovri
%                  }
%   END
%
% \circle{DIAM} ==
%   BEGIN
%    \begingroup
%    \boxmaxdepth := maxdimen
%    \@tempdimb := DIAM *\unitlength
%    if \@tempdimb > 15.5pt
%      then \@getcirc{\@tempdimb}
%           \@ovro := \ht \@tempboxa
%           \@tempboxa := \hbox{
%                   \@circlefnt
%                   \@tempcnta := \@tempcnta + 2
%                   \char \@tempcnta
%                   \@tempcnta := \@tempcnta - 1
%                   \char \@tempcnta
%                   \kern -2\@tempdima
%                   \@tempcnta := \@tempcnta + 2
%                   \raise \@tempdima \hbox { \char \@tempcnta }
%                   \raise \@tempdima \box\@tempboxa
%                  }
%           \ht\@tempboxa := \dp\@tempboxa := 0
%           \@put{-\@ovro}{-\@ovro}{\@tempboxa}
%      else
%           \@circ{\@tempdimb}{96}
%    fi
%   \endgroup
%   END
%
% \circle*{DIAM}  ==  \@dot{DIAM} == \@circ{DIAM*\unitlength}{112}
%
% \@circ{DIAM}{CHAR} ==
%  BEGIN
%   \@tempcnta := integer coercion of (DIAM + .5pt)/1pt.
%   if \@tempcnta > 15 then \@tempcnta := 15 fi
%   if \@tempcnta > 1  then \@tempcnta := \@tempcnta - 1 fi
%   \@tempcnta := \@tempcnta + CHAR
%   \@circlefnt
%   \char \@tempcnta
%  END
%
 
\newif\if@ovt
\newif\if@ovb
\newif\if@ovl
\newif\if@ovr
\newdimen\@ovxx
\newdimen\@ovyy
\newdimen\@ovdx
\newdimen\@ovdy
\newdimen\@ovro
\newdimen\@ovri
 
%% \advance\@tempdima 2pt\relax added 1 Nov 88 to fix bug in which
%% size of drawn circle not monotonic function of argument of \circle,
%% caused by different rounding for dimensions of large and small circles.
%
\def\@getcirc#1{\@tempdima #1\relax \advance\@tempdima 2\p@
  \@tempcnta\@tempdima
  \@tempdima 4\p@ \divide\@tempcnta\@tempdima
  \ifnum \@tempcnta >10\relax \@tempcnta 10\relax\fi
  \ifnum \@tempcnta >\z@ \advance\@tempcnta\m@ne
    \else \@warning{Oval too small}\fi
  \multiply\@tempcnta 4\relax
  \setbox \@tempboxa \hbox{\@circlefnt
  \char \@tempcnta}\@tempdima \wd \@tempboxa}
 
\def\@put#1#2#3{\raise #2\hbox to\z@{\hskip #1#3\hss}}
 
\def\oval(#1,#2){\@ifnextchar[{\@oval(#1,#2)}{\@oval(#1,#2)[]}}
 
\def\@oval(#1,#2)[#3]{\begingroup\boxmaxdepth \maxdimen
  \@ovttrue \@ovbtrue \@ovltrue \@ovrtrue
  \@tfor\@tempa :=#3\do{\csname @ov\@tempa false\endcsname}\@ovxx
  #1\unitlength \@ovyy #2\unitlength
  \@tempdimb \ifdim \@ovyy >\@ovxx \@ovxx\else \@ovyy \fi
  \advance \@tempdimb -2\p@
  \@getcirc \@tempdimb
  \@ovro \ht\@tempboxa \@ovri \dp\@tempboxa
  \@ovdx\@ovxx \advance\@ovdx -\@tempdima \divide\@ovdx \tw@
  \@ovdy\@ovyy \advance\@ovdy -\@tempdima \divide\@ovdy \tw@
  \@circlefnt \setbox\@tempboxa
  \hbox{\if@ovr \@ovvert32\kern -\@tempdima \fi
  \if@ovl \kern \@ovxx \@ovvert01\kern -\@tempdima \kern -\@ovxx \fi
  \if@ovt \@ovhorz \kern -\@ovxx \fi
  \if@ovb \raise \@ovyy \@ovhorz \fi}\advance\@ovdx\@ovro
  \advance\@ovdy\@ovro \ht\@tempboxa\z@ \dp\@tempboxa\z@
  \@put{-\@ovdx}{-\@ovdy}{\box\@tempboxa}%
  \endgroup}
 
\def\@ovvert#1#2{\vbox to\@ovyy{%
    \if@ovb \@tempcntb \@tempcnta \advance \@tempcntb #1\relax
      \kern -\@ovro \hbox{\char \@tempcntb}\nointerlineskip
    \else \kern \@ovri \kern \@ovdy \fi
    \leaders\vrule \@width \@wholewidth\vfil \nointerlineskip
    \if@ovt \@tempcntb \@tempcnta \advance \@tempcntb #2\relax
      \hbox{\char \@tempcntb}%
    \else \kern \@ovdy \kern \@ovro \fi}}
 
\def\@ovhorz{\hbox to \@ovxx{\kern \@ovro
    \if@ovr \else \kern \@ovdx \fi
    \leaders \hrule \@height \@wholewidth \hfil
    \if@ovl \else \kern \@ovdx \fi
    \kern \@ovri}}
 
\def\circle{\@ifstar{\@dot}{\@circle}}
\def\@circle#1{\begingroup \boxmaxdepth \maxdimen \@tempdimb #1\unitlength
   \ifdim \@tempdimb >15.5\p@ \@getcirc\@tempdimb
      \@ovro\ht\@tempboxa
     \setbox\@tempboxa\hbox{\@circlefnt
      \advance\@tempcnta\tw@ \char \@tempcnta
      \advance\@tempcnta\m@ne \char \@tempcnta \kern -2\@tempdima
      \advance\@tempcnta\tw@
      \raise \@tempdima \hbox{\char\@tempcnta}\raise \@tempdima
        \box\@tempboxa}\ht\@tempboxa\z@ \dp\@tempboxa\z@
      \@put{-\@ovro}{-\@ovro}{\box\@tempboxa}%
   \else  \@circ\@tempdimb{96}\fi\endgroup}
 
\def\@dot#1{\@tempdimb #1\unitlength \@circ\@tempdimb{112}}
 
\def\@circ#1#2{\@tempdima #1\relax \advance\@tempdima .5\p@
   \@tempcnta\@tempdima \@tempdima \p@
   \divide\@tempcnta\@tempdima
   \ifnum\@tempcnta >15\relax \@tempcnta 15\relax \fi
   \ifnum \@tempcnta >\z@ \advance\@tempcnta\m@ne\fi
   \advance\@tempcnta #2\relax
   \@circlefnt \char\@tempcnta}
 
 
%INITIALIZATION
\thinlines
 
\newcount\@xarg
\newcount\@yarg
\newcount\@yyarg
\newcount\@multicnt
\newdimen\@xdim
\newdimen\@ydim
\newbox\@linechar
\newdimen\@linelen
\newdimen\@clnwd
\newdimen\@clnht
\newdimen\@dashdim
\newbox\@dashbox
\newcount\@dashcnt
 
 
 
 \message{theorem,}
%       ****************************************
%       *         THEOREM ENVIRONMENTS         *
%       ****************************************
%
%  The user creates his own theorem-like environments with the command
%      \newtheorem{NAME}{TEXT}[COUNTER]  or
%      \newtheorem{NAME}[OLDNAME]{TEXT}
%  This defines the environment NAME to be just as one would expect a
%  theorem environment to be, except that it prints ``TEXT'' instead of
%  ``Theorem''.
%
%  If OLDNAME is given, then environments NAME and OLDNAME use the same
%  counter, so using a NAME environment advances the number of the next
%  NAME environment, and vice-versa.
%
%  If COUNTER is given, then environment NAME is numbered within COUNTER.
%  E.g., if COUNTER = subsection, then the first NAME in subsection 7.2
%  is numbered TEXT 7.2.1.
%
%  The way NAME environments are numbered can be changed by redefining
%  \theNAME.
%
%  DOCUMENT STYLE PARAMETERS
%
%  \@thmcounter{COUNTER} : A command such that
%               \edef\theCOUNTER{\@thmcounter{COUNTER}}
%         defines \theCOUNTER to produce a number for a theorem environment.
%         The default is:
%            BEGIN \noexpand\arabic{COUNTER} END
%
%  \@thmcountersep : A separator placed between a theorem number and
%         the number of the counter within which it is numbered.
%         E.g., to make the third theorem of section 7.2 be numbered
%         7.2-3, \@thmcountersep should be \def'ed to '-'.  Its
%         default is '.'.
%
%  \@begintheorem{NAME}{NUMBER} : A command that begins a theorem
%         environment for a 'theorem' named 'NAME NUMBER' --
%         e.g., \@begintheorem{Lemma}{3.7} starts Lemma 3.7.
%
%  \@opargbegintheorem{NAME}{NUMBER}{OPARG} : A command that begins a theorem
%         environment for a 'theorem' named 'NAME NUMBER' with optional
%         argument OPARG -- e.g., \@begintheorem{Lemma}{3.7}{Jones}
%         starts `Lemma 3.7 (Jones):'.
%
%  \@endtheorem : A command that ends a theorem environment.
%
% \newtheorem{NAME}{TEXT}[COUNTER] ==
%   BEGIN
%     if \NAME is definable
%       then \@definecounter{NAME}
%            if COUNTER present
%              then \@addtoreset{NAME}{COUNTER} fi
%                   \theNAME ==  BEGIN \theCOUNTER \@thmcountersep
%                                       eval\@thmcounter{NAME}      END
%              else \theNAME ==  BEGIN eval\@thmcounter{NAME} END
%            \NAME == \@thm{NAME}{TEXT}
%            \endNAME == \@endtheorem
%       else  error
%     fi
%   END
%
% \newtheorem{NAME}[OLDNAME]{TEXT}==
%   BEGIN
%     if \NAME is definable
%       then \theNAME == \theOLDNAME
%            \NAME == \@thm{OLDNAME}{TEXT}
%            \endNAME == \@endtheorem
%       else  error
%     fi
%   END
%
% \@thm{NAME}{TEXT} ==
%   BEGIN
%    \refstepcounter{NAME}
%    if next char = [
%       then \@ythm{NAME}{TEXT}
%       else \@xthm{NAME}{TEXT}
%    fi
%   END
%
% \@xthm{NAME}{TEXT} ==
%   BEGIN
%    \@begintheorem{TEXT}{\theNAME}
%    \ignorespaces
%   END
%
% \@ythm{NAME}{TEXT}[OPARG] ==
%   BEGIN
%    \@opargbegintheorem{TEXT}{\theNAME}{OPARG}
%    \ignorespaces
%   END
%
\def\newtheorem#1{\@ifnextchar[{\@othm{#1}}{\@nthm{#1}}}
 
\def\@nthm#1#2{%
\@ifnextchar[{\@xnthm{#1}{#2}}{\@ynthm{#1}{#2}}}
 
\def\@xnthm#1#2[#3]{\expandafter\@ifdefinable\csname #1\endcsname
{\@definecounter{#1}\@addtoreset{#1}{#3}%
\expandafter\xdef\csname the#1\endcsname{\expandafter\noexpand
  \csname the#3\endcsname \@thmcountersep \@thmcounter{#1}}%
\global\@namedef{#1}{\@thm{#1}{#2}}\global\@namedef{end#1}{\@endtheorem}}}
 
\def\@ynthm#1#2{\expandafter\@ifdefinable\csname #1\endcsname
{\@definecounter{#1}%
\expandafter\xdef\csname the#1\endcsname{\@thmcounter{#1}}%
\global\@namedef{#1}{\@thm{#1}{#2}}\global\@namedef{end#1}{\@endtheorem}}}

%% RmS 92/01/10: check for existence of theorem environment #2
\def\@othm#1[#2]#3{%
  \@ifundefined{c@#2}{\@latexerr{No theorem environment `#2' defined}\@eha}%
  {\expandafter\@ifdefinable\csname #1\endcsname
  {\global\@namedef{the#1}{\@nameuse{the#2}}%
\global\@namedef{#1}{\@thm{#2}{#3}}%
\global\@namedef{end#1}{\@endtheorem}}}}
 
\def\@thm#1#2{\refstepcounter
    {#1}\@ifnextchar[{\@ythm{#1}{#2}}{\@xthm{#1}{#2}}}
 
\def\@xthm#1#2{\@begintheorem{#2}{\csname the#1\endcsname}\ignorespaces}
\def\@ythm#1#2[#3]{\@opargbegintheorem{#2}{\csname
       the#1\endcsname}{#3}\ignorespaces}
 
%DEFAULT VALUES
\def\@thmcounter#1{\noexpand\arabic{#1}}
\def\@thmcountersep{.}
%deleted September 2, 1986 MDK
%\def\@makethmnumber#1#2{\bf #1 #2:}

%% RmS 91/08/14 Moved \it after \item to make it work with NFSS
\def\@begintheorem#1#2{\trivlist \item[\hskip \labelsep{\bf #1\ #2}]\it}
\def\@opargbegintheorem#1#2#3{\trivlist
      \item[\hskip \labelsep{\bf #1\ #2\ (#3)}]\it}
\def\@endtheorem{\endtrivlist}
 
 
 \message{lengths,}
%     ****************************************
%     *              LENGTHS                 *
%     ****************************************
%
% USER COMMANDS:
%
%   \newlength{\NAME}          == \newskip\NAME
%   \setlength{\NAME}{VALUE}   == \NAME :=L VALUE
%   \addtolength{\NAME}{VALUE} == \NAME :=L \NAME + VALUE
%   \settowidth{\NAME}{TEXT}   == \NAME :=L width of \hbox{TEXT}
%
\def\newlength#1{\@ifdefinable#1{\newskip#1}}
\def\setlength#1#2{#1#2\relax}
\def\addtolength#1#2{\advance#1 #2\relax}
\def\settowidth#1#2{\setbox\@tempboxa\hbox{#2}#1\wd\@tempboxa\relax}
   %% \relax added 24 Mar 86
 
 \message{title,}
%     *****************************************
%     *              THE TITLE                *
%     *****************************************
%
% The user defines the title, author, date by the declarations \title{NAME},
% \author{NAME} and \date{DATE}.  Inside these, he can use the \thanks
% command to make footnoted acknowledgements, notice of address, etc.
% The \maketitle command produces the actual title.  Note: multiple authors
% are separated with the \and command.
 
\def\title#1{\gdef\@title{#1}}
 
\def\author#1{\gdef\@author{#1}}
 
\def\date#1{\gdef\@date{#1}}
\gdef\@date{\today}          %Default is today's date
 
\def\thanks#1{\footnotemark\begingroup
\def\protect{\noexpand\protect\noexpand}\xdef\@thanks{\@thanks
  \protect\footnotetext[\the\c@footnote]{#1}}\endgroup}
 
\def\@thanks{}
 
\def\and{%%                             % \begin{tabular}
\end{tabular}\hskip 1em plus.17fil\begin{tabular}[t]{c}%% \end{tabular}
}
 
 
 
 \message{sectioning,}
%     *****************************************
%     *              SECTIONING               *
%     *****************************************
%
%
% \@startsection {NAME}{LEVEL}{INDENT}{BEFORESKIP}{AFTERSKIP}{STYLE}
%            optional * [ALTHEADING]{HEADING}
%    Generic command to start a section.
%    NAME       : e.g., 'subsection'
%    LEVEL      : a number, denoting depth of section -- e.g., chapter=1,
%                 section = 2, etc.
%    INDENT     : Indentation of heading from left margin
%    BEFORESKIP : Absolute value = skip to leave above the heading.
%                 If negative, then paragraph indent of text following
%                 heading is suppressed.
%    AFTERSKIP  : if positive, then skip to leave below heading, else
%                 negative of skip to leave to right of run-in heading.
%    STYLE      : commands to set style
%  If '*' missing, then increments the counter.  If it is present, then
%  there should be no [ALTHEADING] argument.
%  Uses the counter 'secnumdepth' whose value is the highest section
%  level that is to be numbered.
%
%  WARNING: The \@startsection command should be at the same or higher
%  grouping level as the text that follows it.  For example, you
%  should NOT do something like
%      \def\foo{ \begingroup ...
%                   \paragraph{...}
%                 \endgroup}
%
% \@startsection {NAME}{LEVEL}{INDENT}{BEFORESKIP}{AFTERSKIP}{STYLE} ==
%    BEGIN
%     IF  @noskipsec = T  THEN  \leavevmode  FI  % true if previous section
%                                                % had no body.
%     \par
%     \@tempskipa  := BEFORESKIP
%     @afterindent := T
%     IF \@tempskipa < 0  THEN  \@tempskipa  := -\@tempskipa
%                               @afterindent := F
%     FI
%     IF @nobreak = true
%       THEN \everypar == null
%       ELSE \addpenalty{\@secpenalty}
%            \addvspace{\@tempskipa}
%     FI
%     IF * next
%       THEN \@ssect{INDENT}{BEFORESKIP}{AFTERSKIP}{STYLE}
%       ELSE \@dblarg{\@sect
%                       {NAME}{LEVEL}{INDENT}{BEFORESKIP}{AFTERSKIP}{STYLE}}
%     FI
% END
%
% \@sect{NAME}{LEVEL}{INDENT}{BEFORESKIP}{AFTERSKIP}{STYLE}[ARG1]{ARG2} ==
%   BEGIN
%    IF LEVEL > \c@secnumdepth
%      THEN \@svsec :=L null
%      ELSE \refstepcounter{NAME}
%           \@svsec :=L BEGIN \theNAME END
%    FI
%    IF AFTERSKIP > 0
%      THEN \begingroup
%              STYLE
%              \@hangfrom{\hskip INDENT\@svsec}
%              {\interlinepenalty 10000 ARG2\par}
%           \endgroup
%           \NAMEmark{ARG1}
%           \addcontentsline{toc}{NAME}
%              { IF  LEVEL > \c@secnumdepth
%                  ELSE \protect\numberline{\theNAME}  FI
%                ARG1 }
%      ELSE \@svsechd == BEGIN  STYLE
%                               \hskip INDENT\@svsec
%                               ARG2
%                               \NAMEmark{ARG1}
%                               \addcontentsline{toc}{NAME}
%                                  { IF  LEVEL > \c@secnumdepth
%                                      ELSE \protect\numberline{\theNAME}  FI
%                                    ARG1 }
%                        END
%    FI
%    \@xsect{AFTERSKIP}
% END
%
% \@xsect{AFTERSKIP} ==
%  BEGIN
%    IF AFTERSKIP > 0
%      THEN \par \nobreak
%           \vskip AFTERSKIP
%           \@afterheading
%      ELSE @nobreak :=G F
%           @noskipsec :=G T
%           \everypar{ IF @noskipsec = T
%                        THEN @noskipsec :=G F
%                             \clubpenalty :=G 10000
%                             \hskip -\parindent
%                             \begingroup
%                               \@svsechd
%                             \endgroup
%                             \unskip
%                             \hskip -AFTERSKIP \relax %% relax added 14 Jan 91
%                        ELSE \clubpenalty :=G \@clubpenalty
%                             \everypar := NULL
%                      FI
%                    }
%    FI
%
%   END
%
% \@ssect{INDENT}{BEFORESKIP}{AFTERSKIP}{STYLE}{ARG} ==
%   BEGIN
%    IF AFTERSKIP > 0
%      THEN \begingroup
%             STYLE
%             \@hangfrom{\hskip INDENT}{\interlinepenalty 10000 ARG\par}
%           \endgroup
%      ELSE \@svsechd == BEGIN STYLE
%                              \hskip INDENT
%                              ARG
%                        END
%    FI
%    \@xsect{AFTERSKIP}
%   END
%
% \@afterheading ==
%  BEGIN
%    @nobreak :=G true
%    \everypar := BEGIN  IF @nobreak = T
%                          THEN @nobreak  :=G false
%                               \clubpenalty :=G 10000
%                               IF @afterindent = F
%                                 THEN remove \lastbox
%                               FI
%                          ELSE \clubpenalty :=G \@clubpenalty
%                               \everypar := NULL
%                       FI
%                 END
%  END
%
% \@secpenalty : The penalty (usually negative) put before a section
%                heading unless it immediately follows another one.
%
\newcount\@secpenalty
\@secpenalty = -300
 
 
\newif\if@noskipsec \@noskipsectrue

 
\def\@startsection#1#2#3#4#5#6{\if@noskipsec \leavevmode \fi
   \par \@tempskipa #4\relax
   \@afterindenttrue
   \ifdim \@tempskipa <\z@ \@tempskipa -\@tempskipa \@afterindentfalse\fi
   \if@nobreak \everypar{}\else
     \addpenalty{\@secpenalty}\addvspace{\@tempskipa}\fi \@ifstar
     {\@ssect{#3}{#4}{#5}{#6}}{\@dblarg{\@sect{#1}{#2}{#3}{#4}{#5}{#6}}}}
 
\def\@sect#1#2#3#4#5#6[#7]#8{\ifnum #2>\c@secnumdepth
     \let\@svsec\@empty\else
     \refstepcounter{#1}\edef\@svsec{\csname the#1\endcsname\hskip 1em}\fi
     \@tempskipa #5\relax
      \ifdim \@tempskipa>\z@
        \begingroup #6\relax
          \@hangfrom{\hskip #3\relax\@svsec}{\interlinepenalty \@M #8\par}%
        \endgroup
       \csname #1mark\endcsname{#7}\addcontentsline
         {toc}{#1}{\ifnum #2>\c@secnumdepth \else
                      \protect\numberline{\csname the#1\endcsname}\fi
                    #7}\else
        \def\@svsechd{#6\hskip #3\relax  %% \relax added 2 May 90
                   \@svsec #8\csname #1mark\endcsname
                      {#7}\addcontentsline
                           {toc}{#1}{\ifnum #2>\c@secnumdepth \else
                             \protect\numberline{\csname the#1\endcsname}\fi
                       #7}}\fi
     \@xsect{#5}}
 
\def\@xsect#1{\@tempskipa #1\relax
      \ifdim \@tempskipa>\z@
       \par \nobreak
       \vskip \@tempskipa
       \@afterheading
    \else \global\@nobreakfalse \global\@noskipsectrue
       \everypar{\if@noskipsec \global\@noskipsecfalse
                   \clubpenalty\@M \hskip -\parindent
                   \begingroup \@svsechd \endgroup \unskip
                   \hskip -#1\relax  % relax added 14 Jan 91
                  \else \clubpenalty \@clubpenalty
                    \everypar{}\fi}\fi\ignorespaces}
 
\def\@ssect#1#2#3#4#5{\@tempskipa #3\relax
   \ifdim \@tempskipa>\z@
     \begingroup #4\@hangfrom{\hskip #1}{\interlinepenalty \@M #5\par}\endgroup
   \else \def\@svsechd{#4\hskip #1\relax #5}\fi
    \@xsect{#3}}
 
\newif\if@afterindent \@afterindenttrue
 
\def\@afterheading{\global\@nobreaktrue
      \everypar{\if@nobreak
                   \global\@nobreakfalse
                   \clubpenalty \@M
                   \if@afterindent \else {\setbox\z@\lastbox}\fi
                 \else \clubpenalty \@clubpenalty
                    \everypar{}\fi}}
 
 
% \@hangfrom{TEXT} : Puts TEXT in a box, and makes a hanging indentation
%    of the following material up to the first \par.  Should be used
%    in vertical mode.
%
\def\@hangfrom#1{\setbox\@tempboxa\hbox{#1}%
      \hangindent \wd\@tempboxa\noindent\box\@tempboxa}
 
\newcount\c@secnumdepth
\newcount\c@tocdepth
 
% \secdef{UNSTARCMDS}{STARCMDS} :
%    When defining a \chapter or \section command without using
%    \@startsection, you can use \secdef as follows:
%       \def\chapter { ... \secdef \CMDA \CMDB }
%       \def\CMDA    [#1]#2{ ... }  % Command to define \chapter[...]{...}
%       \def\CMDB    #1{ ... }      % Command to define \chapter*{...}
 
\def\secdef#1#2{\@ifstar{#2}{\@dblarg{#1}}}
 
% Initializations
%
\def\sectionmark#1{}
\def\subsectionmark#1{}
\def\subsubsectionmark#1{}
\def\paragraphmark#1{}
\def\subparagraphmark#1{}
 
 \message{contents,}
%     *****************************************
%     *        TABLE OF CONTENTS, ETC.        *
%     *****************************************
%
% CONVENTIONS:
%   \tf@foo = file number for output for table foo.  The file is
%             opened only if @filesw = true.
%
%   \contentsline{TYPE}{ENTRY}{PAGE}
%       Macro to produce a TYPE entry in a table of contents, etc.
%       It will appear in the .TOC or other file.  For example,
%       The entry for subsection 1.4.3 in the table of contents might
%       be produced by:
%       \contentsline{subsection}{\makebox{30pt}[r]{1.4.3} Gnats and Gnus}{22}
%       The \protect command causes command sequences to be written
%       without expanding them.
%
%   \l@TYPE{ENTRY}{PAGE}
%       Macro defined by document style for making an entry of
%       type TYPE in a table of contents, etc.  E.g., the document
%       style should define \l@chapter, \l@section, etc.
%
%   \addcontentsline{TABLE}{TYPE}{ENTRY}
%      User command for adding his own entry to a table of contents, etc.
%      It adds the entry
%         \contentsline{TYPE}{ENTRY}{page}
%      to the .TABLE file.
%
%   \addtocontents{TABLE}{TEXT} : Adds TEXT to the .TABLE file, with no
%      page number.
%
%  Note: When used in the ENTRY or TEXT of one of the above commands,
%  \protect causes the following control sequence to be written
%  on the file without being expanded.  The sequence will be expanded
%  when the table of contents entry is processed.
%
%  SURPRISE: \index, \glossary,  and \label are no-ops inside an
%  \addcontentsline or \addtocontents command argument.  This could cause a
%  problem if the user puts an \index or \label into one of the commands he
%  writes, or into the optional 'short version' argument of a \section or
%  \caption command.
%
% \addcontentsline{TABLE}{TYPE}{ENTRY}  ==
%   BEGIN
%     if @filesw = true
%       then  \begingroup
%               \index == \label == \glossary == \@gobble
%               \protect{ARG} == \string\string\string ARG \string\space\space
%               \@temptokena := \thepage
%               \@tempa  == write \string\contentsline
%                              {TYPE}{ENTRY}{\the\@temptokena}
%               \@tempa
%               IF vmode and @nobreak = true  THEN  \nobreak FI
%              \endgroup
%     fi
%   END
%
% \@starttoc{EXT} : Used to define \tableofcontents, \listoffigures, etc.--
%      e.g., \@starttoc{lof} is used in \listoffigures.  This command reads
%      the .EXT file and sets up to write the new .EXT file.
%
% \@starttoc{EXT} ==
%   BEGIN
%     \begingroup
%        \makeatletter
%        read file \jobname.EXT
%        IF @filesw = true
%          THEN  open \jobname.EXT as file \tf@EXT
%        FI
%        @nobreak :=G FALSE  %% added 24 May 89
%     \endgroup
%   END

%% RmS 92/01/14: added \immediate to \openout as all \write commands
%%               are also executed \immediate
\def\@starttoc#1{\begingroup
  \makeatletter
  \@input{\jobname.#1}\if@filesw \expandafter\newwrite\csname tf@#1\endcsname
             \immediate\openout \csname tf@#1\endcsname \jobname.#1\relax
  \fi \global\@nobreakfalse \endgroup}
 
 
\let\protect=\relax
 
\def\addcontentsline#1#2#3{\if@filesw \begingroup
\let\label\@gobble \let\index\@gobble \let\glossary\@gobble
\def\protect##1{\string\string\string##1\string\space
   \space}\@temptokena{\thepage}%
\edef\@tempa{\write \@auxout{\string\@writefile{#1}{\protect
     \contentsline{#2}{#3}{\the\@temptokena}}}}\@tempa
   \if@nobreak \ifvmode\nobreak\fi\fi\endgroup\fi}
 
\long\def\addtocontents#1#2{\if@filesw \begingroup
\let\label\@gobble \let\index\@gobble \let\glossary\@gobble
\def\protect##1{\string\string\string##1\string\space\space}%
\edef\@tempa{\write \@auxout {\string\@writefile{#1}{#2}}}\@tempa
  \if@nobreak \ifvmode\nobreak\fi\fi\endgroup\fi}
 
\def\contentsline#1{\csname l@#1\endcsname}
 
% \@dottedtocline{LEVEL}{INDENT}{NUMWIDTH}{TITLE}{PAGE} :
%   Macro to produce a table of contents line with the following
%   parameters:
%     LEVEL    : If LEVEL > \c@tocdepth, then no line produced.
%     INDENT   : Total indentation from the left margin.
%     NUMWIDTH : Width of box for number if the TITLE has a
%                \numberline command.
%                As of 25 Jan 88, this is also the amount of extra indentation
%                added to second and later lines of a multiple line entry.
%     TITLE    : Contents of entry.
%     PAGE     : Page number.
%
%  Uses the following parameters, which must be set by the document style.
%  They should be defined with \def's.
%    \@pnumwidth : Width of box in which page number is set.
%    \@tocrmarg  : Right margin indentation for all but last line of
%                  multiple-line entries.
%    \@dotsep    : Separation between dots, in mu units.  Should be \def'd to
%                  a number like 2 or 1.7
%

%% RmS 91/09/29: added \reset@font for page number 
\def\@dottedtocline#1#2#3#4#5{\ifnum #1>\c@tocdepth \else
  \vskip \z@ plus.2\p@
  {\leftskip #2\relax \rightskip \@tocrmarg \parfillskip -\rightskip
    \parindent #2\relax\@afterindenttrue
   \interlinepenalty\@M
   \leavevmode
   \@tempdima #3\relax \advance\leftskip \@tempdima \hbox{}\hskip -\leftskip
    #4\nobreak\leaders\hbox{$\m@th \mkern \@dotsep mu.\mkern \@dotsep
       mu$}\hfill \nobreak
           \hbox to\@pnumwidth{\hfil\reset@font\rm #5}\par}\fi}
 
 
%%% Note: \nobreak's added 7 Jan 86 to prevent bad line break that
%%% left the page number dangling by itself at left edge of a new line.
%%%
%%% Changed 25 Jan 88 to use \leftskip instead of \hangindent so
%%% leaders of multiple-line contents entries would line up properly.
 
% \numberline{NUMBER} : For use in a \contentsline command.
%   It puts NUMBER flushleft in a box of width \@tempdima
%   (Before 25 Jan 88 change, it also added \@tempdima to the hanging
%   indentation.)
 
\def\numberline#1{\hbox to\@tempdima{#1\hfil}}
 
 
 \message{index,}
%       ****************************************************
%       *            INDEX COMMANDS AND GLOSSARY           *
%       ****************************************************
%
% \makeindex ==
%   BEGIN
%    if \@filesw = T
%      then  open file \jobname.IDX as \@indexfile
%             \index ==  BEGIN \@bsphack
%                              \begingroup
%                                 \protect{X} == \string X\space
%                                       %% added 3 Feb 87 for \index commands
%                                       %% in \footnotes
%                                  re-\catcode special characters to 'other'
%                                  \@wrindex
%    fi
%   END
%
%  \@wrindex{ITEM} ==
%    BEGIN
%        write of {\indexentry{ITEM}{page number}}
%      \endgroup
%      \@esphack
%    END
 
%  INITIALIZATION:
%
%  \index == BEGIN \@bsphack
%                  \begingroup
%                     re-\catcode special characters (in case '%' there)
%                     \@index
%            END
%
%  \@index{ITEM} == BEGIN \endgroup \@esphack END
%
% Changes made 14 Apr 89 to write \glossaryentry's instead of
% \indexentry's on the .glo file.
 
\def\makeindex{\if@filesw \newwrite\@indexfile
  \immediate\openout\@indexfile=\jobname.idx
  \def\index{\@bsphack\begingroup
             \def\protect####1{\string####1\space}\@sanitize
             \@wrindex}\typeout
  {Writing index file \jobname.idx }\fi}
 
\def\@wrindex#1{\let\thepage\relax
   \edef\@tempa{\write\@indexfile{\string
      \indexentry{#1}{\thepage}}}\expandafter\endgroup\@tempa
   \if@nobreak \ifvmode\nobreak\fi\fi\@esphack}
 
\def\index{\@bsphack\begingroup \@sanitize\@index}
 
\def\@index#1{\endgroup\@esphack}
 
\def\makeglossary{\if@filesw \newwrite\@glossaryfile
  \immediate\openout\@glossaryfile=\jobname.glo
  \def\glossary{\@bsphack\begingroup\@sanitize\@wrglossary}\typeout
  {Writing glossary file \jobname.glo }\fi}
 
\def\@wrglossary#1{\let\thepage\relax
   \edef\@tempa{\write\@glossaryfile{\string
      \glossaryentry{#1}{\thepage}}}\expandafter\endgroup\@tempa
   \if@nobreak \ifvmode\nobreak\fi\fi\@esphack}
 
\def\glossary{\@bsphack\begingroup\@sanitize\@index}
 
 \message{bibliography,}
%      ****************************************
%      *            BIBLIOGRAPHY              *
%      ****************************************
%
%  A bibliography is created by the bibliography environment, which
%  generates a title such as ``References'', and a list of entries.
%  The BIBTeX program will create a file containing such an environment,
%  which will be read in by the \bibliography command.  With
%  BIBTeX, the following commands will be used.
%
%  \bibliography{FILE1,FILE2, ... ,FILEn} : specifies
%     the bibdata files.  Writes a \bibdata entry on the .aux file
%     and tries to read in mainfile.BBL.
%
%  \bibliographystyle{STYLE} : Writes a \bibstyle entry on the .aux file.
%
%  The thebibliography environment is a list environment.  To save the
%  use of an extra counter, it should use  enumiv  as the item counter.
%  Instead of using \item, items in the bibliography are produced by the
%  following commands:
%    \bibitem{NAME}             : Produces a numbered entry cited as NAME.
%    \bibitem[LABEL]{NAME}      : Produces an entry labeled by LABEL and
%                                 cited by NAME.
%  The former is used for bibliographies with citations like [1], [2], etc.;
%  the latter is used for citations like [Knuth82].
%
%  The document style must define the thebibliography environment.  This
%  environment has a single argument, which is the widest bibliography
%  label-- e.g., if the [Knuth67] is the widest entry, then thist argument
%  will be Knuth67.  The \thebibliography command must begin a list
%  environment, which the \endthebibliography command ends.
%
%  Entries are cited by the command \cite{NAME}.
%
%  PARAMETERS
%
%   \@cite     : A macro such that \@cite{LABEL1,LABEL2}{NOTE}
%                produces the output for a \cite[NOTE]{FOO1,FOO2} command,
%                where entry FOOi is defined by \bibitem[LABELi]{FOOi}.
%                The switch @tempswa is true if the optional NOTE argument
%                is present.
%                The default definition is :
%                  \@cite{LABELS}{NOTE} ==
%                     BEGIN [LABELS
%                           IF @tempswa = T THEN , NOTE FI
%                           ]
%                     END
%
%   \@biblabel : A macro to produce the label in the bibliography
%                entry.  For \bibitem[LABEL]{NAME}, the label is
%                generated by \@biblabel{LABEL}.  It has the default
%                definition \@biblabel{LABEL} -> [LABEL].
%  CONVENTION
%
%  \b@FOO : The name or number of the reference created by \cite{FOO}
%           E.g., if \cite{FOO} -> [17] , then \b@FOO -> 17.
%
%
 
\def\bibitem{\@ifnextchar[{\@lbibitem}{\@bibitem}}
 
%% RmS 92/02/26: Added \hfill to restore left-alignment of
%%               bibliography labels in alpha style
\def\@lbibitem[#1]#2{\item[\@biblabel{#1}\hfill]\if@filesw
      {\def\protect##1{\string ##1\space}\immediate
       \write\@auxout{\string\bibcite{#2}{#1}}}\fi\ignorespaces}
%% Placement of `}' in def of \@lbibitem corrected 29 Apr 87
%% (Error found by Arthur Ogawa.)

%% RmS 91/11/13: Changed counter enumi to enumiv,
%%               as it says in the comment above
%% RmS 92/01/10: Changed \c@enumiv to \value{\@listctr}.
\def\@bibitem#1{\item\if@filesw \immediate\write\@auxout
       {\string\bibcite{#1}{\the\value{\@listctr}}}\fi\ignorespaces}
 
\def\bibcite#1#2{\global\@namedef{b@#1}{#2}}
 
\let\citation\@gobble
 
\def\cite{\@ifnextchar [{\@tempswatrue\@citex}{\@tempswafalse\@citex[]}}
 
% \penalty\@m added to definition of \@citex to allow a line
% break after the `,' in citations like [Jones80,Smith77]
% (Added 23 Oct 86)
%
% space added after the `,' (21 Nov 87)
%
%% RmS 91/10/25: added \reset@font, suggested by Bernd Raichle.
%% RmS 91/11/06: added code to remove a leading blank
\def\@citex[#1]#2{\if@filesw\immediate\write\@auxout{\string\citation{#2}}\fi
  \let\@citea\@empty
  \@cite{\@for\@citeb:=#2\do
    {\@citea\def\@citea{,\penalty\@m\ }%
     \def\@tempa##1##2\@nil{\edef\@citeb{\if##1\space##2\else##1##2\fi}}%
     \expandafter\@tempa\@citeb\@nil
     \@ifundefined{b@\@citeb}{{\reset@font\bf ?}\@warning
       {Citation `\@citeb' on page \thepage \space undefined}}%
     \hbox{\csname b@\@citeb\endcsname}}}{#1}}
 
\let\bibdata=\@gobble
\let\bibstyle=\@gobble
 
\def\bibliography#1{\if@filesw\immediate\write\@auxout{\string\bibdata{#1}}\fi
  \@input{\jobname.bbl}}
 
\def\bibliographystyle#1{\if@filesw\immediate\write\@auxout
    {\string\bibstyle{#1}}\fi}
 
% \nocite{CITATIONS} : puts information on .AUX file to cause
%   BibTeX to include the CITATIONS list in the bibliography,
%   but puts nothing in the text.  (Added 14 Jun 85)
 
\def\nocite#1{\@bsphack
  \if@filesw\immediate\write\@auxout{\string\citation{#1}}\fi
  \@esphack}
 
 
 
%DEFAULT DEFINITIONS
 
\def\@cite#1#2{[{#1\if@tempswa , #2\fi}]}
%% RmS 92/01/14: removed \hfill in definition of \@biblabel
\def\@biblabel#1{[#1]}
 
 \message{floats,}
%     ****************************************
%     *               FLOATS                 *
%     ****************************************
%
%  The different types of floats are identified by a TYPE name, which is
%  the name of the counter for that kind of float.  For example, figures
%  are of type 'figure' and tables are of type 'table'.  Each TYPE has
%  associated a positive TYPE NUMBER, which is a power of two.  E.g.,
%  figures might be have type number 1, tables type number 2, programs
%  type number 4, etc.
%
%  The locations where a float can go are specified by a PLACEMENT
%  SPECIFIER, which is a list of the possible locations, each denoted
%  by a letter as follows:
%     h : here   - at the current location in the text.
%     t : top    - at the top of a text page.
%     b : bottom - at the bottom of a text page.
%     p : page   - on a separate float page.
%  For example, 'pht' specifies that the float can appear in any of three
%  locations: page, here or top.
%
% Where floats may appear on a page, and how many may appear there
% are specified by the following float placement parameters.  The
% numbers are named like counters so the user can set them with
% the ordinary counter-setting commands.
%
%  \c@topnumber            : Number of floats allowed at the top of a column.
%  \topfraction            : Fraction of column that can be devoted to floats.
%  \c@dbltopnumber, \dbltopfraction : Same as above, but for double-column
%                          floats.
%  \c@bottomnumber, \bottomfraction : Same as above for bottom of page.
%  \c@totalnumber          : Number of floats allowed in a single column,
%                          including in-text floats.
%  \textfraction         : Minimum fraction of column that must contain text.
%  \floatpagefraction    : Minimum fraction of page that must be taken
%                          up by float page.
%  \dblfloatpagefraction : Same as above, for double-column floats.
%
% The document style must define the following.
%
%    \fps@TYPE   : The default placement specifier for floats of type TYPE.
%
%    \ftype@TYPE : The type number for floats of type TYPE.
%
%    \ext@TYPE   : The file extension indicating the file on which the
%                  contents list for float type TYPE is stored.  For example,
%                  \ext@figure = 'lof'.
%
%    \fnum@TYPE  : A macro to generate the figure number for a caption.
%                  For example, \fnum@TYPE == Figure \thefigure.
%
%    \@makecaption{NUM}{TEXT} : A macro to make a caption, with NUM the value
%                  produced by \fnum@... and TEXT the text of the caption.
%                  It can assume it's in a \parbox of the appropriate width.
%
% \@float{TYPE}[PLACEMENT] : This macro begins a float environment for a
%     single-column float of type TYPE with PLACEMENT as the placement
%     specifier.  The default value of PLACEMENT is defined by \fps@TYPE.
%     The environment is ended by \end@float.
%     E.g., \figure == \@float{figure}, \endfigure == \end@float.
%
%  \caption ==
%    BEGIN
%     \refstepcounter{\@captype}
%     \@dblarg{\@caption{\@captype}}
%    END
%
%% In following definition, \par moved from after \addcontentsline to
%% before \addcontentsline because the \write could cause
%% an extra blank line to be added to the paragraph above the
%% caption.  (Change made 12 Jun 87)
%
%  \@caption{TYPE}[STEXT]{TEXT} ==
%   BEGIN
%     \par
%     \addcontentsline{\ext@TYPE}{TYPE}{\numberline{\theTYPE}{STEXT}}
%     \begingroup
%       \@parboxrestore
%       \normalsize
%       \@makecaption{\fnum@TYPE}{TEXT}
%       \par
%     \endgroup
%   END
%
%  \@float{TYPE}[PLACEMENT] ==
%   BEGIN
%     if hmode then \@bsphack
%                   \@floatpenalty := -10002
%              else \@floatpenalty := -10003
%     fi
%     \@captype ==L TYPE
%     if inner
%       then LaTeX Error: 'Not in outer paragraph mode.'
%            \@floatpenalty := 0
%       else if \@freelist nonempty
%              then \@currbox  :=L head of \@freelist
%                   \@freelist :=G tail of \@freelist
%                   \count\@currbox :=G 32*\ftype@TYPE + 16 +
%                                          bits determined by PLACEMENT
%              else \@floatpenalty := 0
%                   LaTeX Error: 'Too many unprocessed floats'
%            fi
%     fi
%     \@currbox :=G \vbox{  %% 15 Dec 87 -- removed \boxmaxdepth :=L 0pt
%                           %% that made box zero depth because it screwed
%                           %% things up.  Instead, added \vskip 0pt at end
%                     \hsize = \columnwidth
%                     \@parboxrestore
%   END
%
%  \end@float ==
%    BEGIN
%       \vskip 0pt %% makes 0 depth box -- added 15 Dec 87
%       }
%      if \@floatpenalty < 0
%        then add \@currbox to \@currlist
%             if \ht\@currbox > \textheight
%               then \ht\@currbox :=G \textheight fi
%             if \@floatpenalty < -10002
%               then \penalty -10004
%                    \vbox{}
%                    \penalty \@floatpenalty
%               else \vadjust{\penalty -10004
%                             \vbox{}
%                             \penalty \@floatpenalty}
%                    \@Esphack
%      fi    fi
%    END
%
%  \@dblfloat{TYPE}[PLACEMENT] : Macro to begin a float environment for a
%     double-column float of type TYPE with PLACEMENT as the placement
%     specifier.  The default value of PLACEMENT is 'tp'
%     The environment is ended by \end@dblfloat.
%     E.g., \figure* == \@dblfloat{figure}, \endfigure* == \end@dblfloat.
%
%  \@dblfloat{TYPE}[PLACEMENT] ==
%     Identical to \@float{TYPE}[PLACEMENT] except \hsize and \linewidth
%     are set to \textwidth.
%
%  \end@dblfloat ==
%     BEGIN    %%% { BRACE MATCHING
%       \vskip 0pt %% makes 0 depth box -- added 15 Dec 87
%       }
%       if \@floatpenalty < 0
%         then \@dbldeferlist :=G \@dbldeferlist * \@currbox
%       fi
%       if \@floatpenalty = -10002 then \@Esphack  fi
%     END
%
\newcount\@floatpenalty
 
\def\caption{\refstepcounter\@captype \@dblarg{\@caption\@captype}}
 
 
\long\def\@caption#1[#2]#3{\par\addcontentsline{\csname
  ext@#1\endcsname}{#1}{\protect\numberline{\csname
  the#1\endcsname}{\ignorespaces #2}}\begingroup
    \@parboxrestore
    \normalsize
    \@makecaption{\csname fnum@#1\endcsname}{\ignorespaces #3}\par
  \endgroup}
 
\def\@float#1{\@ifnextchar[{\@xfloat{#1}}{\edef\@tempa{\noexpand\@xfloat
    {#1}[\csname fps@#1\endcsname]}\@tempa}}
 
\def\@xfloat#1[#2]{\ifhmode \@bsphack\@floatpenalty -\@Mii\else
   \@floatpenalty-\@Miii\fi\def\@captype{#1}\ifinner
      \@parmoderr\@floatpenalty\z@
    \else\@next\@currbox\@freelist{\@tempcnta\csname ftype@#1\endcsname
       \multiply\@tempcnta\@xxxii\advance\@tempcnta\sixt@@n
       \@tfor \@tempa :=#2\do
                        {\if\@tempa h\advance\@tempcnta \@ne\fi
                         \if\@tempa t\advance\@tempcnta \tw@\fi
                         \if\@tempa b\advance\@tempcnta 4\relax\fi
                         \if\@tempa p\advance\@tempcnta 8\relax\fi
         }\global\count\@currbox\@tempcnta}\@fltovf\fi
    \global\setbox\@currbox\vbox\bgroup
%    \boxmaxdepth\z@     % commented out 15 Dec 87
    \hsize\columnwidth \@parboxrestore}
 
\def\end@float{\par\vskip\z@\egroup %% \par\vskip\z@ added 15 Dec 87
   \ifnum\@floatpenalty <\z@
     \@cons\@currlist\@currbox
     \ifdim \ht\@currbox >\textheight
%% RmS 91/11/06 added warning message
% perhaps we should use an error message
        \@warning{Float larger than \string\textheight}%
        \ht\@currbox\textheight \fi
     \ifnum\@floatpenalty <-\@Mii
        \penalty -\@Miv
        \@tempdima\prevdepth    %% saving and restoring \prevdepth added
        \vbox{}%                %% 26 May 87 to prevent extra vertical
        \prevdepth \@tempdima   %% space when used in vertical mode
        \penalty\@floatpenalty
%% RmS 92/03/18 changed \@esphack to \@Esphack
      \else \vadjust{\penalty -\@Miv \vbox{}\penalty\@floatpenalty}\@Esphack
     \fi\fi}
 
 
\def\@dblfloat{\if@twocolumn\let\@tempa\@dbflt\else\let\@tempa\@float\fi
  \@tempa}
 
\def\@dbflt#1{\@ifnextchar[{\@xdblfloat{#1}}{\@xdblfloat{#1}[tp]}}
 
\def\@xdblfloat#1[#2]{\@xfloat{#1}[#2]\hsize\textwidth\linewidth\textwidth}
 
\def\end@dblfloat{\if@twocolumn
     \par\vskip\z@\egroup %% \par\vskip\z@ added 15 Dec 87\egroup
     \ifnum\@floatpenalty <\z@
% make sure that we never exceed \textheight, otherwise float
% will never get typeset =91/03/15 FMi=
       \ifdim\ht\@currbox >\textheight
% perhaps we should use an error message
         \@warning{Float larger than \string\textheight}%
         \ht\@currbox\textheight \fi
        \@cons\@dbldeferlist\@currbox\fi
%% RmS 92/03/18 changed \@esphack to \@Esphack
      \ifnum \@floatpenalty =-\@Mii \@Esphack\fi\else\end@float\fi}
 
\newcount\c@topnumber
\newcount\c@dbltopnumber
\newcount\c@bottomnumber
\newcount\c@totalnumber
 
\def\@floatplacement{\global\@topnum\c@topnumber
   \global\@toproom \topfraction\@colht
   \global\@botnum  \c@bottomnumber
   \global\@botroom \bottomfraction\@colht
   \global\@colnum  \c@totalnumber
   \@fpmin   \floatpagefraction\@colht}
 
\def\@dblfloatplacement{\global\@dbltopnum\c@dbltopnumber
   \global\@dbltoproom \dbltopfraction\@colht
   \@fpmin \dblfloatpagefraction\textheight
   \@fptop \@dblfptop
   \@fpsep \@dblfpsep
   \@fpbot \@dblfpbot}
 
%   MARGINAL NOTES:
%
%   Marginal notes use the same mechanism as floats to communicate
%   with the \output routine.  Marginal notes are distinguished from
%   floats by having a negative placement specification.  The command
%   \marginpar [LTEXT]{RTEXT} generates a marginal note in a parbox,
%   using LTEXT if it's on the left and RTEXT if it's on the right.
%   (Default is RTEXT = LTEXT.)  It uses the following parameters.
%
%   \marginparwidth : Width of marginal notes.
%   \marginparsep   : Distance between marginal note and text.
%        the page layout to determine how to move the marginal
%        note into the margin.   E.g., \@leftmarginskip ==
%        \hskip -\marginparwidth \hskip -\marginparsep .
%   \marginparpush  :  Minimum vertical separation between \marginpar's
%
%  Marginal notes are normally put on the outside of the page
%  if @mparswitch = true, and on the right if @mparswitch = false.
%  The command \reversemarginpar reverses the side where they
%  are put.  \normalmarginpar undoes \reversemarginpar.
%  These commands have no effect for two-column output.
%
%  SURPRISE: if two marginal notes appear on the same line of
%  text, then the second one could appear on the next page, in
%  a funny position.
%
%
%  \marginpar [LTEXT]{RTEXT} ==
%   BEGIN
%     if hmode then \@bsphack
%                   \@floatpenalty := -10002
%              else \@floatpenalty := -10003
%     fi
%     if inner
%       then LaTeX Error: 'Not in outer paragraph mode.'
%            \@floatpenalty := 0
%       else if \@freelist has two elements:
%              then get \@marbox, \@currbox  from \@freelist
%                   \count\@marbox :=G -1
%              else \@floatpenalty := 0
%                   LaTeX Error: 'Too many unprocessed floats'
%                   \@currbox, \@marbox := \@tempboxa    %%use \def
%            fi
%     fi
%     if optional argument
%       then %% \@xmpar ==
%            \@savemarbox\@marbox{LTEXT}
%            \@savemarbox\@currbox{RTEXT}
%       else %% \@ympar ==
%            \@savemarbox\@marbox{RTEXT}
%            \box\@currbox :=G \box\@marbox
%    fi
%    %% \@xympar ==
%    if \@floatpenalty < 0 then add \@marbox to \@currlist fi
%    \setbox\@tempboxa =L     %% added 3 Jan 88 to correct bug introduced
%       { \end@float %%%% BRACE MATCHING}        %% by 15 Dec 87 change
%   END
%
% \@savemarbox\BOX{TEXT} ==
%   BEGIN
%     \BOX  :=G \vtop{ \hsize = \marginparwidth
%                      \@parboxrestore
%                      TEXT
%                    }
%   END
%
% \reversemarginpar == BEGIN \@mparbottom   :=G 0
%                            @reversemargin :=G true
%                      END
%
% \normalmarginpar  == BEGIN \@mparbottom   :=G 0
%                            @reversemargin :=G false
%                      END
%
 
\def\marginpar{\ifhmode \@bsphack\@floatpenalty -\@Mii\else
   \@floatpenalty-\@Miii\fi\ifinner
      \@parmoderr\@floatpenalty\z@
    \else\@next\@currbox\@freelist{}{}\@next\@marbox\@freelist{\global
      \count\@marbox\m@ne}{\@floatpenalty\z@ \@fltovf\def\@currbox{\@tempboxa
           }\def\@marbox{\@tempboxa}}\fi
     \@ifnextchar [{\@xmpar}{\@ympar}}
 
\long\def\@xmpar[#1]#2{\@savemarbox\@marbox{#1}\@savemarbox\@currbox
   {#2}\@xympar}
 
\long\def\@ympar#1{\@savemarbox\@marbox{#1}\global\setbox\@currbox
     \copy\@marbox\@xympar}
 
\long\def\@savemarbox#1#2{\global\setbox#1\vtop{\hsize\marginparwidth
   \@parboxrestore #2}}
 
\def\@xympar{\ifnum\@floatpenalty <\z@\@cons\@currlist\@marbox\fi
     \setbox\@tempboxa\vbox   %% added 3 Jan 88
%% RmS 92/03/18 added \global\@ignorefalse
     \bgroup\end@float\global\@ignorefalse\@esphack}
 
\def\reversemarginpar{\global\@mparbottom\z@ \@reversemargintrue}
\def\normalmarginpar{\global\@mparbottom\z@ \@reversemarginfalse}
 
 
 \message{footnotes,}
%       ****************************************
%       *             FOOTNOTES                *
%       ****************************************
%
%   \footnote{NOTE}       : User command to insert a footnote.
%
%   \footnote[NUM]{NOTE}  : User command to insert a footnote numbered
%                           NUM, where NUM is a number -- 1, 2,
%                           etc.  For example, if footnotes are numbered
%                           *, **, etc. within pages, then \footnote[2]{...}
%                           produces footnote '**'.  This command does not
%                           step the footnote counter.
%
%   \footnotemark[NUM]    : Command to produce just the footnote mark in
%                           the text, but no footnote.  With no argument,
%                           it steps the footnote counter before generating
%                           the mark.
%
%   \footnotetext[NUM]{TEXT} : Command to produce the footnote but no
%                              mark.  \footnote is equivalent to
%                              \footnotemark \footnotetext .
%
%   As in PLAIN, footnotes use \insert\footins, and the following parameters:
%
%   \footnotesize   : Size-changing command for footnotes.
%
%   \footnotesep    : The height of a strut placed at the beginning of
%                     every footnote.
%   \skip\footins   : Space between main text and footnotes.  The rule
%                     separating footnotes from text occurs in this space.
%                     This space lies above the strut of height \footnotesep
%                     which is at the beginning of the first footnote.
%   \footnoterule   : Macro to draw the rule separating footnotes from text.
%                     It is executed right after a \vspace of \skip\footins.
%                     It should take zero vertical space--i.e., it should to
%                     a negative skip to compensate for any positive space
%                     it occupies.  (See PLAIN.TEX.)
%
%   \interfootnotelinepenalty : Interline penalty for footnotes.
%
%   \thefootnote : In usual LaTeX style, produces the footnote number.
%                  If footnotes are to be numbered within pages, then the
%                  document style file must include an \@addtoreset command
%                  to cause the footnote counter to be reset when the page
%                  counter is stepped.  This is not a good idea, though,
%                  because the counter will not always be reset in time
%                  to ensure that the first footnote on a page is footnote
%                  number one.
%
%   \@thefnmark : Holds the current footnote's mark--e.g., \dag or '1' or 'a'.
%
%   \@mpfnnumber     : A macro that generates the numbers for \footnote
%                      and \footnotemark commands. It == \thefootnote
%                      outside a minipage environment, but can be changed
%                      inside to generate numbers for \footnote's.
%
%   \@makefnmark : A macro to generate the footnote marker from \@thefnmark
%                  The default definition is \hbox{$^\@thefnmark$}.
%
%   \@makefntext{NOTE} :
%        Must produce the actual footnote, using \@thefnmark as the mark
%        of the footnote and NOTE as the text.  It is called when effectively
%        inside a \parbox, with \hsize = \columnwidth.  For example, it might
%        be as simple as
%               $^{\@thefnmark}$ NOTE
%
% In a minipage environment, \footnote and \footnotetext are redefined
% so that
%    (a) they use the counter mpfootnote
%    (b) the footnotes they produce go at the bottom of the minipage.
% The switch is accomplished by letting \@mpfn == footnote or mpfootnote
% and \thempfn == \thefootnote or \thempfootnote, and by redefining
% \@footnotetext to be \@mpfootnotetext in the minipage.
%
% \footnote{NOTE}  ==
%  BEGIN
%    \stepcounter{\@mpfn}
%    \@thefnmark :=G eval (\thempfn)
%    \@footnotemark
%    \@footnotetext{NOTE}
%  END
%
% \footnote[NUM]{NOTE} ==
%  BEGIN
%    begingroup
%       counter \@mpfn :=L NUM
%       \@thefnmark :=G eval (\thempfn)
%    endgroup
%    \@footnotemark
%    \@footnotetext{NOTE}
%  END
%
% \@footnotetext{NOTE} ==
%  BEGIN
%    \insert into \footins
%       {\footnotesize
%        \interlinepenalty :=L \interfootnotelinepenalty
%        \splittopskip     :=L \footnotesep
%        \splitmaxdepth    :=L \dp\strutbox
%        \floatingpenalty  :=L 20000
%        \hsize :=L \columnwidth
%        \@parboxrestore
%        set \@currentlabel to make \label command work right
%        \@makefntext{\rule{0pt}{\footnotesep} NOTE}
%       }
%  END
%
% \footnotemark      ==
%  BEGIN \stepcounter{footnote}
%        \@thefnmark :=G eval(\thefootnote)
%        \@footnotemark
%  END
%
% \footnotemark[NUM] ==
%   BEGIN
%       begingroup
%         footnote counter :=L NUM
%        \@thefnmark :=G eval(\thefootnote)
%       endgroup
%       \@footnotemark
%   END
%
% \@footnotemark ==
%   BEGIN
%    \leavevmode
%    IF hmode THEN \@x@sf := \the\spacefactor FI
%    \@makefnmark          % put number in main text
%    IF hmode THEN \spacefactor := \@x@sf FI
%   END
%
% \footnotetext      ==
%    BEGIN \@thefnmark :=G eval (\thempfn)
%          \@footnotetext
%    END
%
% \footnotetext[NUM] ==
%    BEGIN begingroup  counter \@mpfn :=L NUM
%                      \@thefnmark :=G eval (\thempfn)
%          endgroup
%          \@footnotetext
%    END
%
 
\@definecounter{footnote}
\def\thefootnote{\arabic{footnote}}
 
\@definecounter{mpfootnote}
\def\thempfootnote{\alph{mpfootnote}}
 
% Default definition
\def\@makefnmark{\hbox{$^{\@thefnmark}\m@th$}}
 
\newdimen\footnotesep

%% RmS 91/11/01: Added \let\protect\noexpand in \footnote, \footnotemark,
%%               and \footnotetext, since \xdef is used.
%% RmS 91/11/22: Added \let\protect\noexpand in \@xfootnote, \@xfootnotemark,
%%               and \@xfootnotetext.

\def\footnote{\@ifnextchar[{\@xfootnote}{\stepcounter{\@mpfn}%
     \begingroup\let\protect\noexpand
       \xdef\@thefnmark{\thempfn}\endgroup
     \@footnotemark\@footnotetext}}
 
\def\@xfootnote[#1]{\begingroup \csname c@\@mpfn\endcsname #1\relax
   \let\protect\noexpand
   \xdef\@thefnmark{\thempfn}\endgroup
   \@footnotemark\@footnotetext}

%% RmS 91/09/29: added \reset@font
\long\def\@footnotetext#1{\insert\footins{\reset@font\footnotesize
    \interlinepenalty\interfootnotelinepenalty
    \splittopskip\footnotesep
    \splitmaxdepth \dp\strutbox \floatingpenalty \@MM
    \hsize\columnwidth \@parboxrestore
   \edef\@currentlabel{\csname p@footnote\endcsname\@thefnmark}\@makefntext
    {\rule{\z@}{\footnotesep}\ignorespaces
      #1\strut}}}
 
\def\footnotemark{\@ifnextchar[{\@xfootnotemark}{\stepcounter{footnote}%
     \begingroup\let\protect\noexpand
       \xdef\@thefnmark{\thefootnote}\endgroup
     \@footnotemark}}
 
\def\@xfootnotemark[#1]{\begingroup \c@footnote #1\relax
   \let\protect\noexpand
   \xdef\@thefnmark{\thefootnote}\endgroup \@footnotemark}
 
\def\@footnotemark{\leavevmode\ifhmode
  \edef\@x@sf{\the\spacefactor}\fi \@makefnmark
   \ifhmode\spacefactor\@x@sf\fi\relax}
 
\def\footnotetext{\@ifnextchar [{\@xfootnotenext}%
   {\begingroup\let\protect\noexpand
      \xdef\@thefnmark{\thempfn}\endgroup
    \@footnotetext}}
 
\def\@xfootnotenext[#1]{\begingroup \csname c@\@mpfn\endcsname #1\relax
   \let\protect\noexpand
   \xdef\@thefnmark{\thempfn}\endgroup \@footnotetext}
 
\def\@mpfn{footnote}
\def\thempfn{\thefootnote}
 
 \message{initial,}
%          ****************************************
%          *    INITIAL DECLARATION COMMANDS      *
%          ****************************************
%
%                    DOCUMENT STYLE
%                    --------------
%
% The user starts his file with the command
%       \documentstyle [OPTION1, ... ,OPTIONk]{STYLE}
% which saves the OPTION's and \input's the file STYLE.STY.  When the
% STYLE.STY file issues the command \@options, the following happens
% for each i :
%     IF \ds@OPTIONi is defined
%       THEN execute \ds@OPTIONi
%       ELSE save OPTIONi on a list of unprocessed options.
%     FI
% After STYLE.STY has been executed, the file OPTIONi.STY is read for
% each OPTIONi on the list of unprocessed options.
%
% \documentstyle ==
%   BEGIN
%      IF next char = [
%        THEN  \@documentstyle
%        ELSE  \@documentstyle[]
%      FI
%   END
%
% \@documentstyle[OPTIONS]{STYLE} ==
%   BEGIN
%     \makeatletter
%     \@optionlist  :=  OPTIONS
%     \@optionfiles :=G null
%     \input STYLE.STY
%     \@elt == \input
%     \@optionfiles
%     \@elt == \relax
%     \makeatother
%   END
%
%  \@options ==
%    BEGIN
%      \@elt := \relax
%      FOR \@tempa := \@optionlist
%         DO  IF \ds@[eval(\@tempa)] defined
%               THEN  \ds@[eval(\@tempa)]
%               ELSE  \@optionfiles :=G \@optionfiles *
%                                           \@elt eval(\@tempa) \relax
%         OD  FI
%    END
%
%              PAGE STYLE COMMANDS
%              -------------------
%  \pagestyle{STYLE}     : sets the page style of the current and succeeding
%                          pages to STYLE
%
%  \thispagestyle{STYLE} : sets the page style of the current page only
%                          to STYLE
%
%  To define a page style STYLE, you must define \ps@STYLE to set the page
%  style parameters.
%
%  HOW A PAGE STYLE MAKES RUNNING HEADS AND FEET:
%
% The \ps@... command defines the macros \@oddhead, \@oddfoot,
% \@evenhead, and \@evenfoot to define the running heads and feet.
% (See output routine.)  To make headings determined by the sectioning
% commands, the page style defines the commands \chaptermark,
% \sectionmark, etc., where \chaptermark{TEXT} is called by \chapter to
% set a mark.  The \...mark commands and the \...head macros are defined
% with the help of the following macros.  (All the \...mark commands
% should be initialized to no-ops.)
%
% MARKING CONVENTIONS:
% LaTeX extends TeX's \mark facility by producing two kinds of marks
% a 'left' and a 'right' mark, using the following commands:
%     \markboth{LEFT}{RIGHT} : Adds both marks.
%     \markright{RIGHT}      : Adds a 'right' mark.
%     \leftmark  : Used in the output routine, gets the current 'left'  mark.
%                  Works like TeX's \botmark.
%     \rightmark : Used in the output routine, gets the current 'right' mark.
%                  Works like TeX's \firstmark.
% The marking commands work reasonably well for right marks 'numbered
% within' left marks--e.g., the left mark is changed by a \chapter command and
% the right mark is changed by a \section command.  However, it does
% produce somewhat anomalous results if 2 \markboth's occur on the same page.
%
% Commands like \tableofcontents that should set the marks in some page styles
% use a \@mkboth command, which is \let by the pagestyle command (\ps@...)
% to \markboth for setting the heading or to \@gobbletwo to do nothing.
 
\def\documentstyle{\@ifnextchar[{\@documentstyle}{\@documentstyle[]}}
 
\def\@documentstyle[#1]#2{\makeatletter
   \def\@optionlist{#1}\gdef\@optionfiles{}\input #2.sty\relax
   \let\@elt\input \@optionfiles \let\@elt\relax \makeatother}
 
\def\@options{\let\@elt\relax
    \@for\@tempa:=\@optionlist\do
        {\@ifundefined{ds@\@tempa}{\xdef\@optionfiles{\@optionfiles
             \@elt \@tempa.sty\relax}}{\csname ds@\@tempa\endcsname}}}
 
\def\pagestyle#1{\@nameuse{ps@#1}}
\def\thispagestyle#1{\global\@specialpagetrue\gdef\@specialstyle{#1}}
 
% \head : An obsolete command that was used in the `myheadings'
%         page style.   (Removed 14 Jun 85)
% \def\head{\@ifnextchar[{\@xhead}{\@yhead}}
% \def\@xhead[#1]#2{\if #1l \def\@lhead{#2}\else \def\@rhead{#2}\fi}
% \def\@yhead#1{\def\@lhead{#1}\def\@rhead{#1}}
 
% Initialization
%
%\def\@lhead{} %% RmS 91/09/29: removed since no longer used
%\def\@rhead{} %%  ditto
 
 
% Default Initializations
%
\def\ps@empty{\let\@mkboth\@gobbletwo\let\@oddhead\@empty\let\@oddfoot\@empty
\let\@evenhead\@empty\let\@evenfoot\@empty}
 
\def\ps@plain{\let\@mkboth\@gobbletwo
     \let\@oddhead\@empty\def\@oddfoot{\reset@font\rm\hfil\thepage
     \hfil}\let\@evenhead\@empty\let\@evenfoot\@oddfoot}
 
\def\@leftmark#1#2{#1}
\def\@rightmark#1#2{#2}
 
%% test for @nobreak added 15 Apr 86 in \markboth and \markright
%% letting \label and \index to \relax added 22 Feb 86 so these
%%   commands can appear in sectioning command arguments
%% RmS 91/06/21 Same for \glossary
%%
 
\def\markboth#1#2{\gdef\@themark{{#1}{#2}}{\let\protect\noexpand
     \let\label\relax \let\index\relax \let\glossary\relax
     \mark{\@themark}}\if@nobreak\ifvmode\nobreak\fi\fi}
\def\markright#1{{\let\protect\noexpand
     \let\label\relax \let\index\relax \let\glossary\relax
     \expandafter\@markright\@themark
     {#1}\mark{\@themark}}\if@nobreak\ifvmode\nobreak\fi\fi}
 
\def\@markright#1#2#3{\gdef\@themark{{#1}{#3}}}
\def\leftmark{\expandafter\@leftmark\botmark}
\def\rightmark{\expandafter\@rightmark\firstmark}
 
% Initialization
%
\def\@themark{{}{}}
 
 
%  OTHER
%  -----
%
%   \raggedbottom : Typesets pages with no vertical stretch, so they have
%                   their natural height instead of all being exactly the
%                   same height.  (Uses a space of .0001fil to avoid
%                   interfering with the 1fil space of \newpage.)
%
%   \flushbottom  : Inverse of \raggedbottom - makes all pages the same
%                   height.
%
%   \sloppy : Resets TeX's parameters so it accepts worse line and page
%             breaks, and slightly more overfull boxes.
%
%   \fussy  : Resets TeX's parameters to their normal finnicky values.
%
 
\def\raggedbottom{\def\@textbottom{\vskip \z@ plus.0001fil}\let\@texttop\relax}
\def\flushbottom{\let\@textbottom\relax \let\@texttop\relax}
 
% Default definitions
%  \sloppy will never (well, hardly ever) produce overfull boxes, but may
%  produce underfull ones.  (14 June 85)
%  A sloppypar environment is equivalent to {\par \sloppy ... \par}.
\def\sloppy{\tolerance \@M \hfuzz .5\p@ \vfuzz .5\p@}
\def\sloppypar{\par\sloppy}
\def\endsloppypar{\par}
\def\fussy{\tolerance 200 \hfuzz .1\p@ \vfuzz .1\p@}
 
 
 
% LaTeX default is no overfull box rule.  Changed by document
% style option
 
\overfullrule 0pt
 
 \message{output,}
%     ****************************************
%     *               OUTPUT                 *
%     ****************************************
%
%
%  PAGE LAYOUT PARAMETERS
%
%   \topmargin      : Extra space added to top of page.
%   @twoside        : boolean.  T if two-sided printing
%   \oddsidemargin  : IF @twoside = T
%                         THEN extra space added to left of odd-numbered
%                              pages.
%                         ELSE extra space added to left of all pages.
%   \evensidemargin : IF @twoside = T
%                         THEN extra space added to left of even-numbered
%                              pages.
%   \headheight     : height of head
%   \headsep        : separation between head and text
%   \footskip       : distance separation between baseline of last
%                     line of text and baseline of foot.
%                     Note difference between \footSKIP and \headSEP.
%   \textheight     : height of text on page, excluding head and foot
%   \textwidth      : width of printing on page
%   \columnsep      : IF @twocolumn = T
%                       THEN width of space between columns
%   \columnseprule  : IF @twocolumn = T
%                       THEN width of rule between columns (0 if none).
%   \columnwidth    : IF @twocolumn = T
%                       THEN (\textwidth - \columnsep)/2
%                       ELSE \textwidth
%                     It is set by the \@maketwocolumn and \@makeonecolumn
%                     commands.
%   \@textbottom    : Command executed at bottom of vbox holding text of page
%                     (including figures).  The \raggedbottom command
%                     almost \let's this to \vfil (actually sets it to
%                     \vskip \z@ plus.0001fil). %expanded 18 Jun 86
%
%   \@texttop       : Command executed at top of vbox holding text of page
%                     (including figures).  Used by letter style; can also
%                     be used to produce centered pages.  Is \let to \relax
%                     by \raggedbottom and \flushbottom.
%
%   Page layout must also initialize \@colht and \@colroom to \textheight.
%
%  PAGE STYLE PARAMETERS:
%
%   \floatsep       : Space left between floats.
%   \textfloatsep   : Space between last top float or first bottom float
%                     and the text.
%   \topfigrule     : Command to place rule (or whatever) between floats
%                     at top of page and text.  Executed in inner vertical
%                     mode right before the \textfloatsep skip separating
%                     the floats from the text.  Must occupy zero vertical
%                     space.  (See \footnoterule.)
%   \botfigrule     : Same as \topfigrule, but put after the \textfloatsep
%                     skip separating text from the floats at bottom of page.
%   \intextsep      : Space left on top and bottom of an in-text float.
%   \@maxsep        : The maximum of \floatsep, \textfloatsep and \intextsep
%   \dblfloatsep    : Space between double-column floats.
%   \dbltextfloatsep : Space between top or bottom double-column floats
%                      and text.
%   \dblfigrule     : Similar to \topfigrule, but for double-column floats.
%   \@dblmaxsep     : The maximum of \dblfloatsep and \dbltexfloatsep
%   \@fptop         : Glue to go at top of float column -- must be 0pt +
%                     stretch
%   \@fpsep         : Glue to go between floats in a float column.
%   \@fpbot         : Glue to go at bottom of float column -- must be 0pt +
%                     stretch
%   \@dblfptop, \@dblfpsep, \@dblfpbot
%                   : Analogous for double-column float page in two-column
%                     format.
%
%  FOOTNOTES: As in PLAIN, footnotes use \insert\footins.
%
%  PAGE LAYOUT SWITCHES AND MACROS
%
%   @twocolumn      : Boolean.  T if two columns per page.
%
%  PAGE STYLE MACROS AND SWITCHES
%
%   \@oddhead        : IF @twoside = T
%                           THEN macro to generate head of odd-numbered pages.
%                           ELSE macro to generate head of all pages.
%   \@evenhead       : IF @twoside = T
%                           THEN macro to generate head of even-numbered pages.
%   \@oddfoot        : IF @twoside = T
%                           THEN macro to generate foot of odd-numbered pages.
%                           ELSE macro to generate foot of all pages.
%   \@evenfoot       : IF @twoside = T
%                           THEN macro to generate foot of even-numbered pages.
%   @specialpage    : boolean.  T if current page is to have a special format.
%   \@specialstyle  : If its value is  foo then
%                      IF @specialpage = T
%                        THEN the command \ps@foo is executed to temporarily
%                             reset the page style parameters before composing
%                             the current page.  This command should execute
%                             only \def's and \edef's, making only local
%                             definitions.
%
%
%  FLOAT PLACEMENT PARAMETERS
%
% The following parameters are set by the macro \@floatplacement.
% When \@floatplacement is called,
% \@colht is the height of the page or column being built.  I.e.:
%         * For single-column page it equals \textheight.
%         * For double-column page it equals \textheight - height
%           of double-column floats on page.
% Note that some are set globally and some locally:
%    \@topnum  :=G Maximum number of floats allowed on the top of a column.
%    \@toproom :=G Maximum amount of top of column devoted to floats--
%                  excluding \textfloatsep separation below the floats and
%                  \floatsep separation between them.  For two-column
%                  output, should be computed as a function of \@colht.
%    \@botnum, \@botroom
%                : Analogous to above.
%    \@colnum  :=G Maximum number of floats allowed in a column, including
%                  in-text floats.
%    \@textmin :=L Minimum amount of text (excluding footnotes) that must
%                  appear on a text page.  %% 27 Sep 85 : made local to
%                                          %% \@addtocurcol and \@addtonextcol
%    \@fpmin   :=L Minimum height of floats in a float column.
%
% The macro \@dblfloatplacement sets the following parameters.
%    \@dbltopnum  :=G Maximum number of double-column floats allowed at the
%                     top of a two-column page.
%    \@dbltoproom :=G Maximum height of double-column floats allowed at
%                     top of two-column page.
%    \@fpmin      :=L Minimum height of floats in a float column.
% It should also perform the following local assignments where necessary
% -- i.e., where the new value differs from the old one:
%      \@fptop       :=L \@dblfptop
%      \@fpsep       :=L \@dblfpsep
%      \@fpbot       :=L \@dblfpbot
%
%  OUTPUT ROUTINE VARIABLES
%
%  \@colht : The total height of the current column.  In single column
%            style, it equals \textheight.  In two-column style, it is
%            \textheight minus the height of the double-column floats
%            on the current page.  MUST BE INITIALIZED TO \textheight.
%
%  \@colroom : The height available in the current column for text and
%              footnotes.  It equals \@colht minus the height of all
%              floats committed to the top and bottom of the current
%              column.
%
%  \footins : Footnote insertion number.
%
%  \@maxdepth : Saved value of TeX's \maxdepth.  Must be set
%               when any routine sets \maxdepth.
%
%            CALLING THE OUTPUT ROUTINE
%            --------------------------
%
% The output routine is called either by TeX's normal page-breaking
% mechanism, or by a macro putting a penalty < or = -10000 in the output
% list.  In the latter case, the penalty indicates why the output
% routine was called, using the following code.
%
%   penalty   reason
%   -------   ------
%   -10000    \pagebreak
%             \newpage
%   -10001    \clearpage (called with \penalty -10000 \vbox{} \penalty -10001
%   -10002    float insertion, called from horizontal mode
%   -10003    float insertion, called from vertical mode.
%   -10004    float insertion.
%
% Note: A float or marginpar puts the following sequence in the output
%       list:  (i) a penalty of -10004,
%             (ii) a null \vbox
%            (iii) a penalty of -10002 or -10003.
%       This solves two special problems:
%         1. If the float comes right after a \newpage or \clearpage,
%            then the first penalty is ignored, but the second one
%            invokes the output routine.
%         2. If there is a split footnote on the page, the second 'page'
%            puts out the rest of the footnote.
%
%             THE OUTPUT ROUTINE
%             ------------------
%
% FUNCTIONS USED IN THE OUTPUT ROUTINE:
%
% \@outputpage : Produces an output page with the contents of box
%              \@outputbox as the text part.  Also sets
%              \@colht :=G \textheight.   The page style is determined
%              as follows.
%                IF  @thispagestyle = true
%                  THEN  use \thispagestyle style
%                  ELSE  use ordinary page style.
%
% \@tryfcolumn\FLIST : Tries to form a float column composed of floats from
%              \FLIST with with the following parameters:
%                \@colht : height of box
%                \@fpmin : minimum height of floats in the box
%                \@fpsep : interfloat space
%                \@fptop : glue at top of box
%                \@fpbot : glue at bottom of box.
%              If it succeeds, then it does the following:
%                * \@outputbox :=L the composed float box.
%                * @fcolmade   :=L true
%                * \FLIST      :=G \FLIST - floats put in box
%                * \@freelist  :=G \@freelist + floats put in box
%              If it fails, then:
%                * @fcolmade :=L false
%           NOTE: BIT MUST BE A SINGLE TOKEN!
%
% \@makefcolumn \FLIST : Same as \@tryfcolumn except that it
%             fails to make a float column only if \FLIST is empty.
%             Otherwise, it makes a float column containing at least
%             the first box in \FLIST, disregarding \@fpmin.
%
% \@startcolumn :
%       Calls \@tryfcolumn\@deferlist{8}.  If \@tryfcolumn returns with
%       @fcolmade = false, then:
%                * Globally sets \@toplist and \@botlist to floats
%                  from \@deferlist to go at top and bottom of column,
%                  deleting them from \@deferlist.  It does
%                  this using \@colht as the total height, the page
%                  style parameters \@floatsep and \@textfloatsep, and
%                  the float placement parameters \@topnum, \@toproom,
%                  \@botnum, \@botroom, \@colnum and \textfraction.
%                * Globally sets \@colroom to \@colht minus the height
%                  of the added floats.
%
% \@startdblcolumn :
%      Calls \@tryfcolumn\@dbldeferlist{8}.  If \@tryfcolumn returns
%      with @fcolmade = false, then:
%               * Globally sets \@dbltoplist to floats from \@dbldeferlist
%                 to go at top and bottom of column, deleting them from
%                 \@dbldeferlist.  It does this using \textheight as the
%                 total height, and the parameters \@dblfloatsep, etc.
%               * Globally sets \@colht to \textheight minus the height
%                 of the added floats.
%
% \@combinefloats : Combines the text from box
%              \@outputbox with the floats from \@toplist and \@botlist,
%              putting the new box in \@outputbox.  It uses \floatsep and
%              \textfloatsep for the appropriate separations.  It puts the
%              elements of \TOPLIST and \BOTLIST onto \@freelist, and makes
%              those lists null.
%
% \@makecol : Makes the contents of \box255 plus the accumulated
%              footnotes, plus the floats in \@toplist and \@botlist,
%              into a single column of height \@colht, which it puts
%              into box \@outputbox.  It puts boxes in \@midlist back
%              onto \@freelist and restores \maxdepth.
%
% \@opcol : Outputs a column whose text is in box \@outputbox
%              If @twocolumn = false, then it calls \@outputpage,
%              sets \@colht :=G \textheight, and calls \@floatplacement.
%
%              If @twocolumn = true, then:
%                  If @firstcolumn = true, then it puts box \@outputbox
%                  into \@leftcolumn and sets @firstcolumn :=G false.
%
%                  If @firstcolumn = false, then it puts out the current
%                  two-column page, any possible two-column float pages,
%                  and determines \@dbltoplist for the next page.
%
% \@opcol ==
%  BEGIN
%   \@mparbottom :=G 0pt
%   if @twocolumn = true
%     then %% \@outputdblcol ==
%          if @firstcolumn = true
%            then @firstcolumn :=G false
%                 \@leftcolumn :=G \@outputbox
%            else @firstcolumn :=G true
%                 \@outputbox  := \vbox{
%                                   \hbox to \textwidth{
%                                     \hbox to\columnwidth{\box\@leftcolumn
%                                                          \hss}
%                                     \hfil \vrule width \columnseprule \hfil
%                                     \hbox to\columnwidth{\box\@outputbox}
%                                                           \hss}             }
%                 \@combinedblfloats
%                 \@outputpage
%                 \begingroup
%                    \@dblfloatplacement
%                    \@startdblcolumn
%                    while @fcolmade = true
%                      do  \@outputpage
%                          \@startdblcolumn  od
%                 \endgroup
%          fi
%     else
%       \@outputpage
%       \@colht :=G \textheight
%   fi
%  END
%
%  \@makecol ==
%    BEGIN
%     ifvoid \insert\footins
%        then  \@outputbox := \box255
%        else  \@outputbox := \vbox {\boxmaxdepth :=L \maxdepth
%                                              %added 21 Jan 87
%                                    \unvbox255
%                                    \vskip \skip\footins
%                                    \footnoterule
%                                    \unvbox\footins
%                                   }
%    fi
%    \@freelist :=G \@freelist * \@midlist
%    \@midlist  :=G empty
%    \@combinefloats
%    \@outputbox := \vbox to \@colht{\boxmaxdepth := \maxdepth
%                                     \@texttop
%                                     temp :=L \dp\@outputbox
%                                     \unvbox\@outputbox
%                                     \vskip -temp
%                                     \@textbottom}
%    \maxdepth :=G \@maxdepth
%    END
%
% \@outputpage ==
%   BEGIN
%     \begingroup          %%% added 11 Jun 85 to keep special page
%                          %%%  declarations local to this output page
%     \catcode`\  := 10       %%make sure space is really a space
%     \- := \@dischyph     %%% Added 4 Aug 88 in event output routine
%     \' := \@acci         %%% called inside a tabbing environment.
%     \` := \@accii
%     \= := \@acciii
%     if @specialpage = T
%       then @specialpage :=G F
%            execute \ps@[eval(\@specialstyle)]  fi
%     if \@twoside = T
%       then if \count0 odd
%                    \@thehead       ==L \@oddhead
%                    \@thefoot       ==L \@oddfoot
%                    \@themargin     ==L \oddsidemargin
%               else \@thehead       ==L \@evenhead
%                    \@thefoot       ==L \@evenfoot
%                    \@themargin     ==L \evensidemargin  fi fi
%     \shipout\vbox
%       {\normalsize          % set fonts size for head and foot
%        \baselineskip :=L \lineskip :=L 0pt
%        \par :=L \@@par       %% added 15 Sep 87 for robustness
%        \vskip \topmargin
%        \moveright\@themargin\vbox
%              { \box\@tempboxa := \vbox to \headheight{\vfil
%                                      \hbox to \textwidth
%                                         {\index == \label == 
%                                           \glossary == \@gobble
%                                              %% Added 22 Feb 87 as bug fix
%                                              %% RmS 91/06/21 \glossary added
%                                          \@thehead}}
%                \dp\@tempboxa := 0pt   % Don't skip space for descenders in
%                \box\@tempboxa         % running head.
%                \vskip \headsep
%                \box\@outputbox
%                \baselineskip\footskip
%                \hbox to \textwidth{\index == \label == \glossary == \@gobble
%                                        %%% added 22 Feb 87 as bug fix
%                                        %%% RmS 91/06/21 \glossary added
%                                    \@thefoot}
%               }
%       }
%     \@colht :=G \textheight
%     \endgroup                %% added 11 Jun 85
%     \stepcounter{page}
%     \firstmark ==L \botmark   %% So marks work properly on float
%                               %% pages. (14 Jun 85)
%   END
%
% \@startcolumn ==
%  BEGIN
%    \@colroom :=G \@colht
%    if \@deferlist is empty
%      then @fcolmade := false
%      else \@tryfcolumn\@deferlist       %% else clause == \@xstartcol
%           if @fcolmade = false
%             then \begingroup
%                    \@tempb     :=L \@deferlist
%                    \@deferlist :=G empty
%                    \@elt \BOX  ==  BEGIN \@currbox == \BOX      % use \gdef
%                                          \@addtonextcol
%                                    END  == \@scolelt
%                    \@tempb
%                  \endgroup
%    fi    fi
%  END
%
% \@startdblcolumn ==
%  BEGIN
%    \@colht :=G \textheight
%      \@tryfcolumn\@dbldeferlist      %% else clause == \@xstartcol
%      if @fcolmade = false
%        then \begingroup
%               \@tempb        :=L \@dbldeferlist
%               \@dbldeferlist :=G empty
%               \@elt \BOX     ==  BEGIN \@currbox == \BOX      % use \gdef
%                                        \@addtodblcol
%                                  END  == \@sdblcolelt
%               \@tempb
%             \endgroup
%    fi    fi
%  END
%
% \output ==
%  BEGIN
%   case of \outputpenalty
%      > -10001  -> \@makecol
%                   \@opcol
%                   \@floatplacement
%                   \@startcolumn
%                   while  @fcolmade = true
%                      do \@opcol
%                         \@startcolumn
%                      od
%
%  %%% \@specialoutput ==
%
%      -10001   ->  %% \@doclearpage ==
%                      if there are no footnote insertions
%                        then unbox the \writes at the head of \box255
%                               and throw away the rest
%                             \@deferlist :=G \@toplist * \@botlist
%                                                * \@deferlist
%                             \@toplist :=G \@botlist :=G empty
%                             \@colroom :=G \@colht
%                             if \@currlist not empty
%                               then  LaTeX error: float(s) lost
%                                     \@currlist :=G empty
%                             fi
%                             \@makefcolumn\@deferlist
%                             while  @fcolmade = true
%                                  do \@opcol
%                                     \@makefcolumn\@deferlist
%                                  od
%                             if @twocolumn
%                               then
%                                 if @firstcolumn = true
%                                   then \@dbldeferlist :=G \@dbltoplist *
%                                                           \@dbldeferlist
%                                        \@dbltoplist   :=G empty
%                                        \@colht :=G \textheight
%                                        \begingroup
%                                           \@dblfloatplacement
%                                           \@makefcolumn\@dbldeferlist
%                                           while  @fcolmade = true
%                                             do \@outputpage
%                                                \@makefcolumn\@dbldeferlist
%                                             od
%                                        \endgroup
%                                   else  \vbox{} \clearpage
%                             fi fi
%                        else \box255 := \vbox{\box255\vfil}
%                             \@makecol
%                             \@opcol
%                             \clearpage
%                      fi
%    < -10001     ->
%              if \outputpenalty < -10003
%                then if \outputpenalty <-20000  %% true only at end
%                        then \deadcycles := 0
%                     fi
%                     box \@holdpg :=G box255
%                else throw away box 255
%                     \@pagedp :=L natural depth of box \@holdpg
%                     \@pageht :=L natural ht of box \@holdpg
%                     \unvbox box \@holdpg  %% put text back
%                     if \@currlist nonempty
%                        then \@currbox  :=L head of \@currlist
%                             \@currlist :=G tail of \@currlist
%                             if \count\@currbox > 0
%                           %% Changed 28 Feb 88 so \@pageht and \@pagedp
%                           %% aren't changed for a marginal note
%                               then  %% this is a float
%                                 if there are footnote insertions
%                                   then advance \@pageht and \@pagedp and
%                                   reinsert footnotes
%                                 fi
%                                 \@addtocurcol
%                               else  %% this is a marginal note
%                                 if there are footnote insertions
%                                   reinsert footnotes
%                                 fi
%                                 \@addmarginpar
%                             fi
%                        else  THIS SHOULDN'T HAPPEN
%                     fi
%                     if \outputpenalty < 0           %% TO PERMIT PAGE BREAK
%                       then \penalty\interlinepenalty fi %% IF \@addtocurcol
%                                               %%  DIDN'T INSERT A PENALTY
%              fi
%   end case
%  \vsize :=G if \outputpenalty > -10004 then \@colroom  %%normal case
%                                        else \maxdimen  %%processing float
%             fi
%  END
%
% \@combinefloats ==
%  BEGIN
%    if \@toplist nonempty
%      then %%\@cfla ==
%           \@elt\BOX == BEGIN  \@tempbox := \vbox{\unvbox\@tempbox
%                                                  \box\BOX
%                                                  \vskip \floatsep}
%                         END  ==  \@comflelt
%            \@tempbox := null
%            \@toplist
%            \@outputbox := \vbox{\boxmaxdepth :=L \maxdepth
%                                              %added 21 Jan 87
%                                 \unvbox\@tempbox
%                                 \vskip - \floatsep
%                                 \topfigrule
%                                 \vskip \textfloatsep
%                                 \unvbox\@outputbox                }
%            \@elt == \relax
%            \@freelist :=G \@freelist * \@toplist
%            \@toplist  :=G null
%    fi
%    if \@botlist nonempty
%      then  %%\@cflb ==
%            \@elt == \@comflelt
%            \@tempbox   := null
%            \@botlist
%            \@outputbox := \vbox{ \unvbox\@outputbox
%                                  \vskip \textfloatsep
%                                  \botfigrule
%                                  \unvbox\@tempbox
%                                  \vskip  - \floatsep   }
%            \@elt == \relax
%            \@freelist :=G \@freelist * \@botlist
%            \@botlist  :=G null
%    fi
%  END
%
% \@combinedblfloats ==
%  BEGIN
%    if \@dbltoplist nonempty
%      then \@elt == \@comdblflelt
%            \@tempbox := null
%            \@dbltoplist
%            \@outputbox := \vbox to \textheight
%                              {\boxmaxdepth :=L \maxdepth
%                               \unvbox\@tempbox
%                               \vskip - \dblfloatsep
%                               \dblfigrule
%                               \vskip \dbltextfloatsep
%                               \box\@outputbox                         }
%            \@elt == \relax
%            \@freelist :=G \@freelist * \@dbltoplist
%            \@dbltoplist  :=G null
%    fi
%  END
%
%
%            USER COMMANDS THAT CALL OR AFFECT THE OUTPUT ROUTINE
%            ----------------------------------------------------
%
% \newpage == BEGIN \par\vfil\penalty -10000 END
%
% \clearpage == BEGIN \newpage
%                     \write -1{}    % Part of hack to make sure no
%                     \vbox{}        % \write's get lost.
%                     \penalty -10001
%               END
%
% \cleardoublepage == BEGIN \clearpage
%                           if @twoside = true and c@page is even
%                             then \hbox{} \newpage fi
%                     END
%
% \twocolumn ==
%  BEGIN
%    \clearpage
%    \columnwidth :=G .5(\textwidth - \columnsep)
%    \hsize       :=G \columnwidth
%    @twocolumn   :=G true
%    @firstcolumn :=G true
%    \@dblfloatplacement
%  END
%
% \onecolumn ==
%  BEGIN
%    \clearpage
%    \columnwidth :=G \textwidth
%    \hsize       :=G \columnwidth
%    @twocolumn   :=G false
%    \@floatplacement
%  END
%
%
% \topnewpage{BOX} : starts a new page and puts BOX in a parbox of width
%     \textwidth across the top.  Useful for full-width titles for
%     double-column pages.
%     SURPRISE: The stretch from \@dbltextfloatsep will be inserted
%     between the BOX and the top of the two columns.
%
% \topnewpage{BOX} ==
%  BEGIN
%   \clearpage
%   Take \@currbox from \@freelist
%   \box\@currbox :=G \parbox{BOX \par
%                             \vskip - \@dbltextfloatsep}
%   \count\@currbox :=G 2
%   \@dbltopnum     :=G 1
%   \@dbltoproom    :=G maxdimension
%   \@addtodblcol
%   \vsize          :=G \@colht
%   \@colroom       :=G \@colht
%  END
 
 
%            FLOAT-HANDLING MECHANISMS
%            -------------------------
%
% The float environment obtains an insertion number B from the
% \@freelist (see below for a description of list manipulation), puts
% the float into box B and sets \count B to a FLOAT SPECIFIER.  For
% a normal (not double-column) float, it then causes a page break
% in one of the following two ways:
%   - In outer hmode: \vadjust{\penalty -10002}
%   - In vmode :      \penalty -10003.
% For a double-column float, it puts B onto the \@dbldeferlist.
% The float specifier has two components:
%    * A PLACEMENT SPECIFICATION, describing where the float may
%      be placed.
%    * A TYPE, which is a power of two--e.g., figures might be
%      type 1 floats, tables type 2 floats, programs type 4 floats, etc.
% The float specifier is encoded as follows, where bit 0 is the least
% significant bit.
%
%  Bit    Meaning
%  ---    -------
%   0     1 iff the float may go where it appears in the text.
%   1     1 iff the float may go on the top of a page.
%   2     1 iff the float may go on the bottom of a page.
%   3     1 iff the float may go on a float page.
%   4     always 1
%   5     1 iff a type 1 float
%   6     1 iff a type 2 float
%   etc.
%
%  A negative float specifier is used to indicate a marginal note.
%
%     MACROS AND DATA STRUCTURES FOR PROCESSING FLOATS
%     ------------------------------------------------
%
%  A FLOAT LIST consisting of the floats in boxes \boxa ... \boxN has the form:
%         \@elt \boxa ... \@elt \boxN
%  where  \boxI is defined by
%         \newinsert\boxI
%  Normally, \@elt is \let to \relax.  A test can be performed on the entire
%  float list by locally \def'ing \@elt appropriately and executing
%  the list.  This is a lot more efficient than looping through the list.
%
%  The following macros are used for manipulating float lists.
%
%  \@next \CS \LIST {NONEMPTY}{EMPTY} ==  %% NOTE: ASSUME \@elt = \relax
%    BEGIN  assume that \LIST == \@elt \B1 ... \@elt \Bn
%           if n = 0
%             then  EMPTY
%             else  \CS    :=L \B1
%                   \LIST  :=G \@elt \B2 ... \@elt \Bn
%                   NONEMPTY
%           fi
%    END
%
%
%  \@bitor\NUM\LIST : Globally sets switch @test to the disjunction for all I
%        of bit  log2 \NUM of the float specifiers of all the floats in
%        \LIST.  I.e., @test is set to true iff there is at least one
%        float in \LIST having bit  log2 \NUM  of its float specifier
%        equal to 1.
%
%  Note: log2 [(\count I)/32] is the bit number corresponding to the
%  type of float I.  To see if there is any float in \LIST having
%  the same type as float I, you run \@bitor with \NUM = [(\count I)/32] * 32.
%
% \@bitor\NUM\LIST ==
%   BEGIN
%      @test :=G false
%      { \@elt \CTR ==  if \count\CTR / \NUM is odd
%                        then  @test := true       fi
%        \LIST
%      }
%   END
%
%
% \@cons\LIST\NUM : Globally sets \LIST := \LIST * \@elt \NUM
%
% \@cons\LIST\NUM ==
%   BEGIN {  \@elt == \relax
%            \LIST :=G \LIST \@elt \NUM
%         }
%
%  BOX LISTS FOR FLOAT-PLACEMENT ALGORITHMS
%
%    \@freelist     : List of empty boxes for placing new floats.
%    \@toplist      : List of floats to go at top of current column.
%    \@midlist      : List of floats in middle of current column.
%    \@botlist      : List of floats to go at bottom of current column.
%    \@deferlist    : List of floats to go after current column.
%    \@dbltoplist   : List of double-col. floats to go at top of current page.
%    \@dbldeferlist : List of double-column floats to go on subsequent pages.
%
%  FLOAT-PLACEMENT ALGORITHMS
%
% \@tryfcolumn \FLIST ==
%  BEGIN
%   @fcolmade    :=G false
%   \@trylist    :=G \FLIST
%   \@failedlist :=G empty
%   \begingroup
%    \@elt == \@xtryfc
%    \@trylist
%   \endgroup
%   if @fcolmade = true
%     then  \@vtryfc \FLIST
%   fi
%  END
%
%  \@vtryfc ==
%    BEGIN
%      \@outputbox :=G \vbox{}
%      \@elt\BOX == BEGIN
%                     \@outputbox :=L \vbox{ \unvbox \@outputbox
%                                            \vskip \@fpsep
%                                            \box\BOX            }
%                   END  == \@wtryfc
%      \@flsucceed
%      \@outputbox :=G \vbox to \@colht{ \vskip \@fptop
%                                        \vskip -\@fpsep
%                                        \unvbox \@outputbox
%                                        \vskip \@fpbot      }
%      \@elt      ==  \relax
%      \@freelist :=G \@freelist * \@flsucceed
%      \FLIST     :=G \@failedlist * \@flfail
%   END
%
% \@xtryfc \BOX ==
%  BEGIN
%    remove first element from \@trylist
%    \@currtype := (\count\BOX / 32) * 32
%    \@bitor \@currtype \@failedlist    % @test := true if type on list
%    \@testfp \BOX                      % @test := true if no p-option
%    if ht of \BOX > \@colht
%      then @test :=G true
%    fi
%    if @test = true
%      then add \BOX to \@failedlist
%      else \@ytryfc \BOX
%    fi
%  END
%
% \@ytryfc ==
%  BEGIN
%   \begingroup
%     \@flsucceed :=G \@elt\BOX
%     \@flfail    :=G empty
%     \@tempdima  := \ht\BOX
%     \@elt == \@ztryfc
%     \@trylist
%     if \@tempdima > \@fpmin
%       then @fcolmade :=G true
%       else add \BOX to \@failedlist
%     fi
%   \endgroup
%   if @fcolmade = true  then  \@elt == \@gobble fi
%  END
%
% \@ztryfc \BOX ==
%  BEGIN
%   \@tempcnta := (\count\BOX / 32) * 32
%   \@bitor \@tempcnta {\@failedlist \@flfail}  % @test := true if on a list
%   \@testfp \BOX                               % @test := true if not p-option
%   \@tempdimb := \@tempdima + ht of \BOX + \@fpsep
%   if \@tempdimb > \@colht
%     then @test :=G true
%   fi
%   if @test = true
%     then add \BOX to \@flfail
%     else add \BOX to \@flsucceed
%          \@tempdima := \@tempdimb
%   fi
%  END
%
% \@testfp \BOX == BEGIN if bit 3 of \count\BOX = 0
%                          then @test :=G true      fi
%                  END
%
% \@makefcolumn \FLIST ==
%  BEGIN
%   \begingroup
%     \@fpmin =:L 0
%     \@testfp == \@gobble
%     \@tryfcolumn \FLIST
%   \endgroup
%  END
%
%  \@addtobot : Tries to put insert \@currbox on \@botlist.  Called only when:
%                  * \ht BOX + \@maxsep < \@colroom
%                  * type of \@currbox not on \@deferlist
%                  * \@colnum > 0
%                  * @insert = false
%               If it succeeds, then:
%                  * sets @insert true
%                  * decrements \@botroom by \ht BOX
%                  * decrements \@botnum and \@colnum by 1
%                  * decrements \@colroom by \ht BOX + either \floatsep
%                    or \textfloatsep, as appropriate.
%                  * sets \maxdepth to 0pt
%
%  \@addtotoporbot : Tries to put insert \@currbox on \@toplist or \@botlist.
%                    Called only under same conditions as \@addtobot.
%                    If it succeeds, then:
%                       * sets @insert true
%                       * decrements either \@toproom or \@botroom by \ht BOX
%                       * decrements \@colnum and either \@topnum or
%                         \@botnum by 1
%                       * decrements \@colroom by \ht BOX + either \floatsep
%                         or \textfloatsep, as appropriate.
%
% \@addtocurcol : Tries to add \@currbox to current column, setting @insert
%                 true if it succeeds, false otherwise.  It will add
%                 \@currbox to top only if bit 0 of \count \@currbox is 0, and
%                 to the bottom only if bit 0 = 0 or an earlier float of
%                 the same type is put on the bottom.
%                 If the float is put in the text, then
%                 \penalty\interlinepenalty is put
%                 right after the float, before the following \vskip, and
%                 \outputpenalty :=L 0.
%
% \@addtonextcol : Tries to add \@currbox to the next column, setting @insert
%                 true if it succeeds, false otherwise.
%
% \@addtodblcol : Tries to add \@currbox to the next double-column page,
%                 adding it to \@dbltoplist if it succeeds and \@dbldeferlist
%                 if it fails.
%
%  \@addtobot ==
%    BEGIN
%      if bit 2 of \count \@currbox = 1
%        then  if \@botnum > 0
%                 then if \@botroom > \ht \@currbox
%                        then \@botnum   :=G \botnum - 1
%                             \@colnum   :=G \@colnum - 1
%                             \@tempdima  :=L - \ht\@currbox -
%                                            if \@botlist empty
%                                                then \textfloatsep
%                                                else \floatsep
%                                            fi
%                             \@botroom  :=G \@botroom + \@tempdima
%                             \@colroom  :=G \@colroom + \@tempdima
%                              add \@currbox to \@botlist
%                              \maxdepth :=G 0pt
%                              @insert :=L true
%      fi      fi       fi
%    END
%
%  \@addtotoporbot ==
%    BEGIN
%      if bit 1 of \count \@currbox = 1
%        then if \@topnum > 0
%               then if \@toproom > \ht \@currbox
%                       then if \@currtype not on \@midlist or \@botlist
%                               then \@topnum    :=G \topnum - 1
%                                    \@colnum    :=G \@colnum - 1
%                                    \@tempdima  :=L - \ht\@currbox -
%                                                   if \@toplist empty
%                                                       then \textfloatsep
%                                                       else \floatsep
%                                                   fi
%                                    \@toproom   :=G \@toproom + \@tempdima
%                                    \@colroom   :=G \@colroom + \@tempdima
%                                    add \@currbox to \@toplist
%                                    @insert :=L true
%      fi     fi      fi      fi
%      if @insert = false then \@addtobot fi
%    END
%
% \@addtocurcol ==
%  BEGIN
%    @insert :=L false
%    \@textmin := \textfraction\@colht         %% added 27 Sep 85
%    if \@colroom > \ht \@currbox + max(\@pageht+\@pagedp, \@textmin)
%                           + \@maxsep
%       then if \@colnum > 0
%               then \@currtype := type of \@currbox
%                    if \@currtype not on \@deferlist
%                      then if \@currtype on \@botlist
%                             then \@addtobot
%                             else if bit0 of \count \@currbox = 1
%                                    then  decrement \@colnum
%                                          put \@currbox on \@midlist
%                                          add \@currbox + space +
%                                            \penalty \interlinepenalty to text
%                                          \outputpenalty :=L 0
%                                          @insert := true
%                                    else  \@addtotoporbot
%    fi      fi      fi     fi     fi
%    if @insert = false
%      then add \@currbox to \@deferlist
%    fi
%  END
%
% \@addtonextcol ==
%  BEGIN
%    @insert :=L false
%    \@textmin := \textfraction\@colht         %% added 27 Sep 85
%    if \@colroom > \ht \@currbox + \@textmin + \@maxsep
%       then if \@colnum > 0
%               \@currtype := type of \@currbox
%               then if \@currtype not on \@deferlist
%                      then \@addtotoporbot
%    fi      fi      fi
%    if @insert = false
%      then add \@currbox to \@deferlist
%    fi
%  END
%
%  \@addtodblcol ==
%    BEGIN
%      @insert :=L false
%      if bit 1 of \count \@currbox = 1
%        then  if \@dbltopnum > 0
%                 then if \@dbltoproom > \ht \@currbox
%                        then if type of \@currbox not on \@dbldeferlist
%                               then \@dbltopnum   :=G \@dbltopnum - 1
%                                    \@tempdima := -\ht\@currbox -
%                                                  if \@dbltoplist empty
%                                                     then \dbltextfloatsep
%                                                     else \dblfloatsep
%                                                  fi
%                                    \@dbltoproom  :=G \@dbltoproom+\@tempdima
%                                    \@colht       :=G \@colht+\@tempdima
%                                    add \@currbox to \@dbltoplist
%                                     @insert :=L true
%      fi      fi       fi    fi
%      if @insert = false then add \@currbox to \@dbldeferlist
%    END
%
%  \@addmarginpar ==
%   BEGIN
%     if \@currlist nonempty
%       then remove \@marbox from \@currlist  %% NOTE: \@currbox = left box
%            add \@marbox and \@currbox to \@freelist
%       else LaTeX error: ?  %% shouldn't happen
%     fi
%     \@tempcnta := 1     %% 1 = right, -1 = left
%     if @twocolumn = true
%       then if @firstcolumn = true
%              then \@tempcnta := -1
%            fi
%       else if @mparswitch = true
%              then if count0 odd
%                     else \@tempcnta := -1
%                   fi
%            fi
%            if @reversemargin = true
%               then \@tempcnta := -\@tempcnta
%            fi
%     fi
%     if \@tempcnta < 0 then \box\@marbox :=G \box\@currbox fi
%     \@tempdima   :=L maximum(\@mparbottom - \@pageht + ht of \@marbox, 0)
%     if \@tempdima > 0 then LaTeX warning: 'marginpar moved' fi
%     \@mparbottom :=G \@pageht + \@tempdima + depth of \@marbox
%                          + \marginparpush
%     \@tempdima   :=L \@tempdima - ht of \@marbox
%     height of \@marbox :=G depth of \@marbox :=G 0
%     \vskip -\@pagedp
%     \vskip \@tempdima
%     \nointerlineskip
%     \hbox{ if @tempcnta > 0 then \hskip \columnwidth
%                                 \hskip \marginparsep
%                            else \hskip -\marginparsep
%                                 \hskip -\marginparwidth
%            fi
%            \box\@marbox
%            \hss
%          }
%     \vskip -\@tempdima
%     \nointerlineskip
%     \hbox{\vrule height 0 width 0 depth \@pagedp}
%   END
 
 
\maxdeadcycles = 100 % floats and \marginpar's add a lot of dead cycles
 
\let\@elt\relax
 
\def\@next#1#2#3#4{\ifx#2\@empty #4\else
   \expandafter\@xnext #2\@@#1#2#3\fi}
 
\def\@xnext \@elt #1#2\@@#3#4{\def#3{#1}\gdef#4{#2}}
 
\newif\if@test
 
\def\@bitor#1#2{\global\@testfalse {\let\@elt\@xbitor
   \@tempcnta #1\relax #2}}

%% RmS 91/11/22: Added test for \count#1 being 0.
%%               Suggested by Chris Rowley.
\def\@xbitor #1{\@tempcntb \count#1
   \ifnum \@tempcnta =\z@
   \else
     \divide\@tempcntb\@tempcnta
     \ifodd\@tempcntb \global\@testtrue\fi
   \fi}
 
% DEFINITION OF FLOAT BOXES:
\newinsert\bx@A
\newinsert\bx@B
\newinsert\bx@C
\newinsert\bx@D
\newinsert\bx@E
\newinsert\bx@F
\newinsert\bx@G
\newinsert\bx@H
\newinsert\bx@I
\newinsert\bx@J
\newinsert\bx@K
\newinsert\bx@L
\newinsert\bx@M
\newinsert\bx@N
\newinsert\bx@O
\newinsert\bx@P
\newinsert\bx@Q
\newinsert\bx@R
 
 
 
\gdef\@freelist{\@elt\bx@A\@elt\bx@B\@elt\bx@C\@elt\bx@D\@elt\bx@E
               \@elt\bx@F\@elt\bx@G\@elt\bx@H\@elt\bx@I\@elt\bx@J
                \@elt\bx@K\@elt\bx@L\@elt\bx@M\@elt\bx@N
                \@elt\bx@O\@elt\bx@P\@elt\bx@Q\@elt\bx@R}
 
\gdef\@toplist{}
\gdef\@botlist{}
\gdef\@midlist{}
\gdef\@currlist{}
\gdef\@deferlist{}
\gdef\@dbltoplist{}
\gdef\@dbldeferlist{}
 
% PAGE LAYOUT PARAMETERS
\newdimen\topmargin
\newdimen\oddsidemargin
\newdimen\evensidemargin
\let\@themargin=\oddsidemargin
\newdimen\headheight
\newdimen\headsep
\newdimen\footskip
\newdimen\footheight % even though it never gets used.
\newdimen\textheight
\newdimen\textwidth
\newdimen\columnwidth
\newdimen\columnsep
\newdimen\columnseprule
\newdimen\@maxdepth    \@maxdepth = \maxdepth
\newdimen\marginparwidth
\newdimen\marginparsep
\newdimen\marginparpush
 
% PAGE STYLE PARAMETERS
\newskip\floatsep
\newskip\textfloatsep
\newskip\intextsep
\newdimen\@maxsep
\newskip\dblfloatsep
\newskip\dbltextfloatsep
\newdimen\@dblmaxsep
\newskip\@fptop
\newskip\@fpsep
\newskip\@fpbot
\newskip\@dblfptop
\newskip\@dblfpsep
\newskip\@dblfpbot
\let\topfigrule=\relax
\let\botfigrule=\relax
\let\dblfigrule=\relax
 
% INTERNAL REGISTERS
 
\newcount\@topnum
\newdimen\@toproom
\newcount\@dbltopnum
\newdimen\@dbltoproom
\newcount\@botnum
\newdimen\@botroom
\newcount\@colnum
\newdimen\@textmin
\newdimen\@fpmin
\newdimen\@colht
\newdimen\@colroom
\newdimen\@pageht
\newdimen\@pagedp
\newdimen\@mparbottom \@mparbottom\z@
\newcount\@currtype
\newbox\@outputbox
\newbox\@leftcolumn
\newbox\@holdpg
 
\newif\if@insert
\newif\if@fcolmade
\newif\if@specialpage \@specialpagefalse
\newif\if@twoside     \@twosidefalse
\newif\if@firstcolumn \@firstcolumntrue
\newif\if@twocolumn   \@twocolumnfalse
\newif\if@reversemargin \@reversemarginfalse
\newif\if@mparswitch  \@mparswitchfalse
 
\def\@thehead{\@oddhead} % initialization
\def\@thefoot{\@oddfoot}
 
\def\newpage{\par\vfil\penalty -\@M}
 
\def\clearpage{\newpage \write\m@ne{}\vbox{}\penalty -\@Mi}
 
\def\cleardoublepage{\clearpage\if@twoside \ifodd\c@page\else
    \hbox{}\newpage\if@twocolumn\hbox{}\newpage\fi\fi\fi}
 
\def\twocolumn{\clearpage \global\columnwidth\textwidth
   \global\advance\columnwidth -\columnsep \global\divide\columnwidth\tw@
   \global\hsize\columnwidth \global\linewidth\columnwidth
   \global\@twocolumntrue \global\@firstcolumntrue
   \@dblfloatplacement\@ifnextchar[{\@topnewpage}{}}
 
\def\onecolumn{\clearpage\global\columnwidth\textwidth
     \global\hsize\columnwidth \global\linewidth\columnwidth
     \global\@twocolumnfalse \@floatplacement}
 
\long\def\@topnewpage[#1]{\@next\@currbox\@freelist{}{}%
    \global\setbox\@currbox\vbox{\hsize\textwidth \@parboxrestore
     #1\par\vskip -\dbltextfloatsep}\global\count\@currbox\tw@
    \global\@dbltopnum\@ne \global\@dbltoproom\maxdimen\@addtodblcol
    \global\vsize\@colht \global\@colroom\@colht}

%% RmS 91/09/29: added reset of \par to the output routine.
%%               This avoids problems when the output routine is
%%               called within a list where \par may be a no-op.

\output{\let\par\@@par
  \ifnum\outputpenalty <-\@M\@specialoutput\else
  \@makecol\@opcol\@floatplacement\@startcolumn
  \@whilesw\if@fcolmade \fi{\@opcol\@startcolumn}\fi
  \global\vsize\ifnum\outputpenalty >-\@Miv \@colroom
                  \else \maxdimen\fi}
 
% CHANGES TO \@specialoutput:
% * \penalty\z@ changed to \penalty\interlinepenalty so \samepage
%   works properly with figure and table environments.
%   (Changed 23 Oct 86)
%
% * Definition of \@specialoutput changed 26 Feb 88 so \@pageht and \@pagedp
%   aren't changed for a marginal note.  (Change suggested by
%   Chris Rowley.)
%
\def\@specialoutput{\ifnum\outputpenalty >-\@Mii
    \@doclearpage
  \else
    \ifnum \outputpenalty <-\@Miii
       \ifnum\outputpenalty<-\@MM \deadcycles\z@\fi
       \global\setbox\@holdpg\vbox{\unvbox\@cclv}%
    \else \setbox\@tempboxa\box\@cclv
        \@pagedp\dp\@holdpg \@pageht\ht\@holdpg
        \unvbox\@holdpg
        \@next\@currbox\@currlist{\ifnum\count\@currbox >\z@
           \ifvoid\footins\else\advance\@pageht\ht\footins
             \advance\@pageht\skip\footins \advance\@pagedp\dp\footins
             \insert\footins{\unvbox\footins}\fi
            \@addtocurcol\else
           \ifvoid\footins\else\insert\footins{\unvbox\footins}\fi
            \@addmarginpar\fi}\@latexbug
    \ifnum \outputpenalty <\z@ \penalty\interlinepenalty\fi
  \fi\fi}
 
 
\def\@doclearpage{\ifvoid\footins
        \setbox\@tempboxa\vsplit\@cclv to\z@ \unvbox\@tempboxa
        \setbox\@tempboxa\box\@cclv
        \xdef\@deferlist{\@toplist\@botlist
            \@deferlist}\gdef\@toplist{}\gdef\@botlist{}\global\@colroom\@colht
             \ifx\@currlist
              \@empty\else\@latexerr{Float(s)
                 lost}\@ehb\gdef\@currlist{}\fi
       \@makefcolumn\@deferlist
        \@whilesw\if@fcolmade \fi{\@opcol
                                  \@makefcolumn\@deferlist}\if@twocolumn
           \if@firstcolumn
             \xdef\@dbldeferlist{\@dbltoplist
                 \@dbldeferlist}\gdef\@dbltoplist{}\global\@colht\textheight
              \begingroup \@dblfloatplacement \@makefcolumn\@dbldeferlist
               \@whilesw\if@fcolmade \fi{\@outputpage
                                         \@makefcolumn\@dbldeferlist}\endgroup
           \else \vbox{}\clearpage
        \fi\fi
     \else\setbox\@cclv\vbox{\box\@cclv\vfil}\@makecol\@opcol
      \clearpage
     \fi}
 
\def\@opcol{\global\@mparbottom\z@\if@twocolumn\@outputdblcol\else
    \@outputpage \global\@colht\textheight \fi}
 
\def\@outputdblcol{\if@firstcolumn \global\@firstcolumnfalse
    \global\setbox\@leftcolumn\box\@outputbox
  \else \global\@firstcolumntrue
    \setbox\@outputbox\vbox{\hbox to\textwidth{\hbox to\columnwidth
      {\box\@leftcolumn \hss}\hfil \vrule width\columnseprule\hfil
       \hbox to\columnwidth{\box\@outputbox \hss}}}\@combinedblfloats
       \@outputpage \begingroup \@dblfloatplacement \@startdblcolumn
       \@whilesw\if@fcolmade \fi{\@outputpage\@startdblcolumn}\endgroup
    \fi}
 
% Extra \@texttop somehow found its way into \@makecol.  Deleted
% 1 Dec 86.  (Found by Mike Harrison)
%% RmS 91/10/22: Replaced \dimen128 by \dimen@.
\def\@makecol{\ifvoid\footins \setbox\@outputbox\box\@cclv
   \else\setbox\@outputbox
     \vbox{\boxmaxdepth \maxdepth
     \unvbox\@cclv\vskip\skip\footins\footnoterule\unvbox\footins}\fi
     \xdef\@freelist{\@freelist\@midlist}\gdef\@midlist{}\@combinefloats
     \setbox\@outputbox\vbox to\@colht{\boxmaxdepth\maxdepth
        \@texttop\dimen@\dp\@outputbox\unvbox\@outputbox
        \vskip-\dimen@\@textbottom}%
     \global\maxdepth\@maxdepth}
 
\let\@texttop=\relax
\let\@textbottom=\relax
 
\def\@outputpage{\begingroup\catcode`\ =10
     \let\-\@dischyph \let\'\@acci \let\`\@accii \let\=\@acciii
    \if@specialpage
     \global\@specialpagefalse\@nameuse{ps@\@specialstyle}\fi
     \if@twoside
       \ifodd\count\z@ \let\@thehead\@oddhead \let\@thefoot\@oddfoot
            \let\@themargin\oddsidemargin
          \else \let\@thehead\@evenhead
          \let\@thefoot\@evenfoot \let\@themargin\evensidemargin
     \fi\fi
     \shipout
     \vbox{\reset@font %% RmS 91/08/15
           \normalsize \baselineskip\z@ \lineskip\z@
           \let\par\@@par %% 15 Sep 87
           \vskip \topmargin \moveright\@themargin
           \vbox{\setbox\@tempboxa
                   \vbox to\headheight{\vfil \hbox to\textwidth
                                       {\let\label\@gobble \let\index\@gobble
                                        \let\glossary\@gobble %% 21 Jun 91
                                         \@thehead}}% %% 22 Feb 87
                 \dp\@tempboxa\z@
                 \box\@tempboxa
                 \vskip \headsep
                 \box\@outputbox
                 \baselineskip\footskip
                 \hbox to\textwidth{\let\label\@gobble
                           \let\index\@gobble  %% 22 Feb 87
                           \let\glossary\@gobble %% 21 Jun 91
                           \@thefoot}}}\global\@colht\textheight
           \endgroup\stepcounter{page}\let\firstmark\botmark}
 
 
\def\@combinefloats{\boxmaxdepth\maxdepth \ifx\@toplist\@empty\else\@cfla\fi
    \ifx\@botlist\@empty\else\@cflb\fi}
 
\def\@cfla{\let\@elt\@comflelt \setbox\@tempboxa\vbox{}\@toplist
    \setbox\@outputbox\vbox{\unvbox\@tempboxa\vskip-\floatsep
    \topfigrule\vskip\textfloatsep \unvbox\@outputbox}\let\@elt\relax
    \xdef\@freelist{\@freelist\@toplist}\gdef\@toplist{}}
 
\def\@cflb{\let\@elt\@comflelt \setbox\@tempboxa\vbox{}\@botlist
    \setbox\@outputbox\vbox{\unvbox\@outputbox \vskip\textfloatsep
    \botfigrule\unvbox\@tempboxa \vskip-\floatsep}\let\@elt\relax
    \xdef\@freelist{\@freelist\@botlist}\gdef\@botlist{}}
 
\def\@comflelt#1{\setbox\@tempboxa
      \vbox{\unvbox\@tempboxa\box #1\vskip\floatsep}}
 
\def\@combinedblfloats{\ifx\@dbltoplist\@empty\else
    \let\@elt\@comdblflelt \setbox\@tempboxa\vbox{}\@dbltoplist
    \setbox\@outputbox\vbox to\textheight
      {\boxmaxdepth\maxdepth
       \unvbox\@tempboxa\vskip-\dblfloatsep
       \dblfigrule\vskip\dbltextfloatsep \box\@outputbox}\let\@elt\relax
    \xdef\@freelist{\@freelist\@dbltoplist}\gdef\@dbltoplist{}\fi}
 
 
\def\@comdblflelt#1{\setbox\@tempboxa
      \vbox{\unvbox\@tempboxa\box #1\vskip\dblfloatsep}}
 
 
\def\@startcolumn{\global\@colroom\@colht
    \ifx\@deferlist\@empty\global\@fcolmadefalse\else\@xstartcol\fi}
 
\def\@xstartcol{\@tryfcolumn\@deferlist \if@fcolmade\else
   \begingroup\edef\@tempb{\@deferlist}\gdef\@deferlist{}\let\@elt\@scolelt
   \@tempb\endgroup\fi}
 
\def\@scolelt#1{\def\@currbox{#1}\@addtonextcol}
 
\def\@startdblcolumn{\global\@colht\textheight
   \@tryfcolumn\@dbldeferlist \if@fcolmade\else
     \begingroup
       \edef\@tempb{\@dbldeferlist}\gdef\@dbldeferlist{}\let\@elt\@sdblcolelt
       \@tempb\endgroup\fi}
 
\def\@sdblcolelt#1{\def\@currbox{#1}\@addtodblcol}
 
\def\@tryfcolumn #1{\global\@fcolmadefalse \xdef\@trylist{#1}\xdef\@failedlist
   {}\begingroup \let\@elt\@xtryfc \@trylist \endgroup
    \if@fcolmade \@vtryfc #1\fi}
 
\def\@vtryfc #1{\global\setbox\@outputbox\vbox{}\let\@elt\@wtryfc
       \@flsucceed \global\setbox\@outputbox\vbox to\@colht{\vskip \@fptop
       \vskip -\@fpsep \unvbox \@outputbox \vskip \@fpbot}\let\@elt\relax
       \xdef #1{\@failedlist\@flfail}\xdef\@freelist{\@freelist\@flsucceed}}
 
\def\@wtryfc #1{\global\setbox\@outputbox\vbox{\unvbox\@outputbox
    \vskip\@fpsep\box #1}}
 
 
\def\@xtryfc #1{\@next\@tempa\@trylist{}{}\@currtype
  \count #1\divide\@currtype\@xxxii \multiply\@currtype\@xxxii
  \@bitor \@currtype \@failedlist \@testfp #1\ifdim
    \ht #1>\@colht \global\@testtrue\fi
    \if@test \@cons\@failedlist #1\else \@ytryfc #1\fi}
 
\def\@ytryfc #1{\begingroup \gdef\@flsucceed{\@elt #1}\gdef\@flfail
  {}\@tempdima\ht #1\let\@elt\@ztryfc \@trylist \ifdim \@tempdima >\@fpmin
     \global\@fcolmadetrue \else \@cons\@failedlist #1\fi
  \endgroup \if@fcolmade \let\@elt\@gobble \fi}
 
\def\@ztryfc #1{\@tempcnta\count #1\divide\@tempcnta\@xxxii
    \multiply\@tempcnta\@xxxii \@bitor \@tempcnta {\@failedlist
    \@flfail}\@testfp #1\@tempdimb\@tempdima \advance\@tempdimb\ht #1\advance
    \@tempdimb\@fpsep \ifdim \@tempdimb >\@colht \global\@testtrue\fi
    \if@test \@cons\@flfail #1\else \@cons\@flsucceed #1\@tempdima\@tempdimb
    \fi}
 
\def\@testfp #1{\@tempcnta\count #1\divide\@tempcnta 8\relax
   \ifodd\@tempcnta \else \global\@testtrue\fi}
 
\def\@makefcolumn #1{\begingroup  \@fpmin\z@ \let\@testfp\@gobble
   \@tryfcolumn #1\endgroup}
 
\def\@addtobot{\@tempcnta\count\@currbox\divide\@tempcnta4 \ifodd\@tempcnta
   \ifnum \@botnum >\z@ \ifdim \@botroom >\ht\@currbox
    \global\advance\@botnum\m@ne
    \global\advance\@colnum\m@ne
    \@tempdima -\ht\@currbox
    \advance\@tempdima -\ifx\@botlist\@empty \textfloatsep
       \else\floatsep\fi
    \global\advance\@botroom \@tempdima
    \global\advance\@colroom \@tempdima
    \@cons\@botlist\@currbox \global\maxdepth\z@
    \@inserttrue\fi\fi\fi}
 
\def\@addtotoporbot{\@tempcnta\count\@currbox \divide\@tempcnta\tw@
   \ifodd\@tempcnta \ifnum \@topnum >\z@ \ifdim\@toproom >\ht\@currbox
     \@bitor\@currtype{\@midlist\@botlist}\if@test\else
        \global\advance\@topnum\m@ne
        \global\advance\@colnum\m@ne
        \@tempdima-\ht\@currbox
        \advance\@tempdima
           -\ifx\@toplist\@empty \textfloatsep \else\floatsep\fi
        \global\advance\@toproom \@tempdima
        \global\advance\@colroom \@tempdima
        \@cons\@toplist\@currbox
        \@inserttrue
\fi\fi\fi\fi
\if@insert\else\@addtobot \fi}
 
\def\@addtonextcol{\@insertfalse \@textmin \textfraction\@colht
   \@tempdima\ht\@currbox
   \advance\@tempdima\@textmin\advance\@tempdima\@maxsep
   \ifdim\@colroom >\@tempdima
     \ifnum\@colnum >\z@
        \@currtype\count\@currbox \divide\@currtype\@xxxii
        \multiply\@currtype\@xxxii
        \@bitor\@currtype\@deferlist
        \if@test\else
          \@addtotoporbot
    \fi\fi\fi
    \if@insert\else \@cons\@deferlist\@currbox\fi}
 
\def\@addtodblcol{\@insertfalse
   \@tempcnta\count\@currbox \divide\@tempcnta\tw@
   \ifodd\@tempcnta
      \ifnum\@dbltopnum >\z@
         \ifdim\@dbltoproom >\ht\@currbox
           \@currtype\count\@currbox \divide\@currtype\@xxxii
               \multiply\@currtype\@xxxii
           \@bitor\@currtype\@dbldeferlist
           \if@test\else
              \global\advance\@dbltopnum\m@ne
              \@tempdima -\ht\@currbox
              \advance\@tempdima -\ifx\@dbltoplist\@empty
                 \dbltextfloatsep\else\dblfloatsep\fi
              \global\advance\@dbltoproom \@tempdima
              \global\advance\@colht \@tempdima
              \@cons\@dbltoplist\@currbox
              \@inserttrue
   \fi\fi\fi\fi
   \if@insert\else \@cons\@dbldeferlist\@currbox \fi}
 
% CHANGE TO \@addtocurcol:
% \penalty\z@ changed to \penalty\interlinepenalty so \samepage
% works properly with figure and table environments.
% (Changed 23 Oct 86)
%
\def\@addtocurcol{\@insertfalse \@textmin \textfraction\@colht
   \@tempdima\@pageht \advance\@tempdima\@pagedp
   \ifdim \@textmin >\@tempdima \@tempdima\@textmin \fi
       \advance\@tempdima\ht\@currbox \advance\@tempdima\@maxsep
   \ifdim\@colroom >\@tempdima
      \ifnum\@colnum >\z@
         \@currtype\count\@currbox \divide\@currtype\@xxxii
                \multiply\@currtype\@xxxii
         \@bitor\@currtype\@deferlist
         \if@test\else
            \@bitor\@currtype\@botlist
            \if@test \@addtobot \else
               \ifodd\count\@currbox
                 \global\advance\@colnum\m@ne
                 \@cons\@midlist\@currbox
                 \vskip\intextsep \box\@currbox
                 \penalty\interlinepenalty \vskip\intextsep
                 \ifnum\outputpenalty <-\@Mii \vskip -\parskip\fi
                 \outputpenalty\z@
                 \@inserttrue
               \else \@addtotoporbot
   \fi\fi\fi\fi\fi
   \if@insert\else\@cons\@deferlist\@currbox\fi}
 
\def\@addmarginpar{\@next\@marbox\@currlist{\@cons\@freelist\@marbox
    \@cons\@freelist\@currbox}\@latexbug\@tempcnta\@ne
    \if@twocolumn
        \if@firstcolumn \@tempcnta\m@ne \fi
    \else
      \if@mparswitch
         \ifodd\c@page \else\@tempcnta\m@ne \fi
      \fi
      \if@reversemargin \@tempcnta -\@tempcnta \fi
    \fi
    \ifnum\@tempcnta <\z@  \global\setbox\@marbox\box\@currbox \fi
    \@tempdima\@mparbottom \advance\@tempdima -\@pageht
       \advance\@tempdima\ht\@marbox \ifdim\@tempdima >\z@
       \@@warning{Marginpar on page \thepage\space moved}\else\@tempdima\z@ \fi
    \global\@mparbottom\@pageht \global\advance\@mparbottom\@tempdima
       \global\advance\@mparbottom\dp\@marbox
       \global\advance\@mparbottom\marginparpush
    \advance\@tempdima -\ht\@marbox
    \global\ht\@marbox\z@ \global\dp\@marbox\z@
    \vskip -\@pagedp \vskip\@tempdima\nointerlineskip
    \hbox to\columnwidth
      {\ifnum \@tempcnta >\z@
          \hskip\columnwidth \hskip\marginparsep
        \else \hskip -\marginparsep \hskip -\marginparwidth \fi
       \box\@marbox \hss}\nobreak   %% RmS 91/06/21 \nobreak added
    \vskip -\@tempdima
    \nointerlineskip
    \hbox{\vrule \@height\z@ \@width\z@ \@depth\@pagedp}}
 
 \message{debugging}
%     ****************************************
%     * DEBUGGING AND TEST INITIALIZATIONS  *
%     ****************************************
%
% DEBUGGING
\def\showoverfull{\tracingonline=1}
\tracingstats1   % SHOWS HOW MUCH STUFF TeX HAS USED
\def\showoutput{\tracingonline1\tracingoutput1
    \showboxbreadth99999\showboxdepth99999\errorstopmode}
\def\makeatletter{\catcode`\@=11\relax}
\def\makeatother{\catcode`\@=12\relax}
 
\newcount\@lowpenalty
\newcount\@medpenalty
\newcount\@highpenalty
 
% LIST
 
% ENUMERATION
 
% ITEMIZE
 
% ARRAY AND TABULAR
 
% THE PICTURE ENVIRONMENT
 
\unitlength = 1pt
\fboxsep = 3pt
\fboxrule = .4pt
 
%% FOOTNOTES
 
%\def\footnoterule{}  % INITIALIZED BY PLAIN
%\skip\footins{}      % INITIALIZED BY PLAIN
%\interfootnotelinepenalty % INITIALIZED BY PLAIN
 
\@maxdepth       = \maxdepth
 
% \vsize initialized because a \clearpage with \vsize < \topskip
%  causes trouble.
% \@colroom and \@colht also initialized because \vsize may be
%  set to them if a \clearpage is done before the \begin{document}
 
\vsize = 1000pt
\@colroom = \vsize
\@colht = \vsize

\endinput

\begin{document}
\begin{title}...\end{title}
...
%end LaTeX \opening template
%Copy proper
...
%Back matter
\bibliography
\closing
\end{verbatim}\endgroup
\noindent with in \verb|latex.cus|
\begingroup\small\begin{verbatim}
\def\head#1{\chapter{#1}}
\def\subhead#1{\section{#}}
\def\bibliography{\appendix
  \section*{Bibliography}
  \input{lit.dat}
  %\input{lit.tex}
   \frenchspacing
   \def\ls#1{\nul\\#1}%simple
  \input{lit.sel}
}
\def\closing{\end{document}}
\end{verbatim}\endgroup
\noindent
The above ideas came to mind when working on this paper.
They deserve development, because it has all to do with the
\begin{quote}
user$\leftrightarrow$environment interaction,
\end{quote}
which has always been important.

\paragraph*{Special texts}are computer programs.
First we like that these
reflect the structure and different quantities (constants, variables,
reserved words, comments etc.) of the program. Second we like that the
programs remain correct while formatting them (meaning: humans hands-off!).
These kind of texts come at two levels
\begin{itemize}
\item the small examples (less than a dozen of lines or so)
      which are part of  courseware, and
\item the documentation (and listings) of real-life programs.
\end{itemize}
Current practice is that for the first it does not really matter what you use.
For a survey see the compilation of Van Oostrum.
For the second Knuth developed \WEB, which stimulates a programmer to
design and {\em document\/} his program from the beginning, by rewarding him
with pretty-\TeX\ printing of it all via \TANGLE. Actually the hierarchical
way of working has been replaced by a relational approach, with the
documentation  related to the various items of a program.
For a survey see Knuth's literal programming article of 1984.

\subsection{Examples: math.}
The \TeX book has devoted at least 4 chapters to math mark-up:
typing math formula,
more about math,
fine points of math typing, and
displayed equations.

See also my Math into BLUes paper for % an anthology of pitfalls.
a survey and how to cope with situations which go wrong---not so much that
\TeX\ complains, but the results are different
from what we expected---by innocent mark-up.

\paragraph*{Displayed math}via (plain) \TeX.
A display is marked up by \$\$ at the beginning and the end.
Within a display the following is generally used
\begin{itemize}
\item just math mark-up
\item \verb|\displaylines|, for multi-liners
\item \verb|\(l)eqalign|, for aligned formulas\footnote{(l) denotes that
      the numbering appears at the left instead of the default right.}
\item \verb|\(l)eqalignno|, similar to the above, but numbering per line.
\end{itemize}
For numbering there is the primitive \verb|\eqno|.
\vskip1ex\noindent
>From a user point of view the following are representative structures
\begingroup%local
%From plain; in LaTeX context the defs below should be grouped
\newskip\centering \centering=0pt plus 1000pt minus 1000pt
\catcode`\@=11
\newdimen\netdpw
%The next is necessary because LaTeX redefined it!!!!!;
%\equalign with numbering went wrong!
%
\def\eqalign#1{%Changed TB361: dynamic number of alignment positions
%\, Had to be removed for two-column output, see TB189 (top)
\vcenter{\openup1\jot\m@th
    \ialign{\strut\hfil$\displaystyle{##}$&&  %Change is extra &
                       $\displaystyle{{}##}$\hfil\crcr
            #1\crcr}}\,}
\def\eqalignno#1{\displ@y \tabskip=\centering
   \halign to \displaywidth{\hfil$\displaystyle{##}$\tabskip=0pt
   &$\displaystyle{{}##}$\hfil\tabskip=\centering
   &\llap{$##$}\tabskip=0pt\crcr
   #1\crcr}}
\def\leqalignno#1{\displ@y \tabskip=\centering
   \halign to \displaywidth{\hfil$\displaystyle{##}$\tabskip=0pt
   &$\displaystyle{{}##}$\hfil\tabskip=\centering
   &\kern-\displaywidth\rlap{$##$}\tabskip=\displaywidth\crcr
   #1\crcr}}
\def\midinsert{\bgroup\smallskip}
\def\endinsert{\smallskip\egroup}
\catcode`\@=12

%Display math examples for Mededelingen van het Wiskundig Genootschap,
%13/4/92, cgl@rug.nl
\def\com#1{{\tt\char92#1}}
%
%Essential ways of formula numbering                     num.tex
\begin{itemize}
\item Labeled 1-line
$$\sin2x=2\sin x\, \cos x
     \eqno(\hbox{TB186})$$
\item Three lines, second flushed right
(relevant for 2-column printing)
$$\displaylines{F(z)=
a_0+{a_1\over z}+{a_2\over z^2}+\cdots
   +{a_{n-1}\over z^{n-1}}+R_n(z),\cr
           \hfill n=1,2,\dots\,,\cr
\hfill F(z)\sim\sum_{n=0}^\infty a_nz^{-n},
       \quad z\to\infty\qquad\qquad\hfill
       \llap{(TB ex19.16)}\cr}$$
\item Two lines aligned, with middle labeling
$$\eqalign{\cos2x&=2\cos^2x-1\cr
                 &=\cos^2x-\sin^2x\cr}
  \eqno(\hbox{TB193})$$
\item Two lignes aligned, with labeling per line
$$\eqalignno{
\cosh2x&=2\cosh^2x-1&(\hbox{TB192})\cr
       &=\cosh^2x+\sinh^2x\cr}$$
\end{itemize}
\noindent obtained via
\begingroup\small\begin{verbatim}
\begin{itemize}
\item Labeled 1-line
$$\sin2x=2\sin x\, \cos x
     \eqno(\hbox{TB186})$$
\item Three lines, second flushed right
(relevant for 2-column printing)
$$\displaylines{F(z)=
a_0+{a_1\over z}+{a_2\over z^2}+\cdots
   +{a_{n-1}\over z^{n-1}}+R_n(z),\cr
           \hfill n=1,2,\dots\,,\cr
\hfill F(z)\sim\sum_{n=0}^\infty a_nz^{-n},
       \quad z\to\infty\qquad\qquad\hfill
       \llap{(TB ex19.16)}\cr}$$
\item Two lines aligned,
      with middle labeling
$$\eqalign{\cos2x&=2\cos^2x-1\cr
                 &=\cos^2x-\sin^2x\cr}
  \eqno(\hbox{TB193})$$
\item Two lignes aligned,
      with labeling per line
$$\eqalignno{
\cosh2x&=2\cosh^2x-1&(\hbox{TB192})\cr
       &=\cosh^2x+\sinh^2x\cr}$$
\end{itemize}
\end{verbatim}\endgroup

\paragraph*{Matrices}via (plain) \TeX.
The examples show paradoxically that for practical use
we not only need \cs{matrix}, but
\begin{itemize}
\item \verb|\atop|, to stack elements on top of each other
\item \verb|\bordermatrix|, for bordered matrices, and this embedded within
      \verb|\displaylines|
\item \verb|\halign|, to handle partitioning, and
\item some macros tailored to our situations, like the icon set.
\end{itemize}
%Samples for TeXing matrices for Mededelingen Wiskundig Genootschap%cgl@rug.nl
\noindent Examples
\begin{itemize}
\item {Hypergeometric function}
$$M_n(z)={}_{n+1}F_n\biggl({k+a_0,
   \atop\phantom{kc_1}}
   {k+a_1,\dots,k+a_n
   \atop k+c_1,\dots,k+c_n};z\biggr)$$
via
\begingroup\small\begin{verbatim}
$$M_n(z)={}_{n+1}F_n\biggl({k+a_0,
   \atop\phantom{kc_1}}
   {k+a_1,\dots,k+a_n
   \atop k+c_1,\dots,k+c_n};z\biggr)$$
\end{verbatim}\endgroup

\item {Some matrix icons}, Wilkinson (1965)
%LaTeX use of linefonts for diag lines
\setlength{\unitlength}{1ex}
$$\icmat44\kern\unitlength\icllt44=
  \icllt44\icuh413\qquad AL=LH$$
$$\icmat63=\icmat63
\kern\unitlength\icurt63\qquad A=QR$$
%rectangular matrix\hfill$\icmat64$\\
%lower left triangular matrix\hfill $\icllt44$\\
%upper right triangular matrix \hfill$\icurt44$\\
%upper Hessenberg f form \hfill $\icuh413$
via
\begingroup\small\begin{verbatim}
$$\icmat44\kern\unitlength\icllt44=
  \icllt44\icuh413\qquad AL=LH$$
$$\icmat63=\icmat63
\kern\unitlength\icurt63\qquad A=QR$$
\end{verbatim}\endgroup

\noindent See for the matrix icon macros my paper on the issue.
%
\item {Matrix reductions},
Wilkinson(1965, p357) %---i.e.\ Math `hyphenation'
$$\displaylines{\indent
\bordermatrix{&      &\rm A &      \cr
              &\times&\times&\times\cr
              &\times&\times&\times\cr
              &\times&\times&\times\cr}
\bordermatrix{& &\rm N & \cr
              &1&      & \cr
              &0&1     & \cr
              &0&\times&1\cr}\hfill\cr
\hfill=
\bordermatrix{& &\rm N & \cr
              &1&      & \cr
              &0&1     & \cr
              &0&\times&1\cr}
\bordermatrix{&      &\rm H &      \cr
              &\times&\times&\times\cr
              &\times&\times&\times\cr
              &0     &\times&\times\cr}
}$$
via
\begingroup\small\begin{verbatim}
$$\displaylines{\indent
\bordermatrix{&      &\rm A &      \cr
              &\times&\times&\times\cr
              &\times&\times&\times\cr
              &\times&\times&\times\cr}
\bordermatrix{& &\rm N & \cr
              &1&      & \cr
              &0&1     & \cr
              &0&\times&1\cr}\hfill\cr
\hfill=
\bordermatrix{& &\rm N & \cr
              &1&      & \cr
              &0&1     & \cr
              &0&\times&1\cr}
\bordermatrix{&      &\rm H &      \cr
              &\times&\times&\times\cr
              &\times&\times&\times\cr
              &0     &\times&\times\cr}
}$$
\end{verbatim}\endgroup%
\item {Partitioning},
Wilkinson(1965, p291)
$$P_r=\left(\vcenter{
    \offinterlineskip\tabskip0pt
    \halign{
      \vrule height3ex depth1ex width 0pt
      \hfil$\enspace#\enspace$\hfil
      \vrule width.1pt\relax
      &\hfil$\enspace#\enspace$\hfil\cr
      I_{n-r}&0\cr
      \noalign{\hrule height.1pt\relax}
      0      &I-2v_rv_r^T\cr}
          }\right)           $$
via
\begingroup\small\begin{verbatim}
$$P_r=\left(\vcenter{
    \offinterlineskip\tabskip0pt
    \halign{
      \vrule height3ex depth1ex width 0pt
      \hfil$\enspace#\enspace$\hfil
      \vrule width.1pt\relax
      &\hfil$\enspace#\enspace$\hfil\cr
      I_{n-r}&0\cr
      \noalign{\hrule height.1pt\relax}
      0      &I-2v_rv_r^T\cr}
          }\right)           $$
\end{verbatim}\endgroup
\end{itemize}
\endgroup

\begingroup\noindent
Next some examples without the mark-up, just the
results, because they are real teasers.
\begin{itemize}
\item Braces and Matrices, Wilkinson(1965, p199)
$$\vcenter{
  \hbox{${\scriptstyle\phantom{n{-}}p}
    \left\{\vrule height4.5ex width0pt
         depth 0pt\right.$}\vglue3ex\relax
  \hbox{${\scriptstyle n{-}p}
    \left\{\vrule height3.0ex width0pt
         depth 0pt\right.\kern2pt$}
         \vglue.5ex\relax
      }
\bordermatrix{&\multispan4{\hfil
$\overbrace{\vrule height0pt width10.5ex
   depth0pt}^p$}\hfil
              &\multispan3{\enspace\hfil
$\overbrace{\vrule height0pt width7.5ex
   depth0pt}^{n-p}$}\hfil\cr
&\times&\times&\times&\times&\times&
 \times&\times\cr
&0     &\times&\times&\times&\times&
 \times&\times\cr
&0     &0     &\times&\times&\times&
 \times&\times\cr
&0     &0     &0     &\times&\times&
 \times&\times\cr
&0     &0     &0     &0     &\times&
 \times&\times\cr
&0     &0     &0     &0     &\times&
 \times&\times\cr
&0     &0     &0     &0     &\times&
 \times&\times\cr
}$$
\item
Matrices, braces, (dotted) partitioning
and icons; space efficient variant
%The simplest way is to make the 22-element
%separate, and measure the sizes.
%Subsequently one easily couples these
%sizes to the sizes of the braces.
%Hard things: automatic coupling,
%vertical dotted lines.
%
\def\vdts{\vbox{\baselineskip4pt
  \lineskiplimit0pt
  \vglue2pt\hbox{.}\hbox{.}\hbox{.}}}%
$$
\vcenter{\offinterlineskip%No interline
%                space in between parts
\halign{\hfil$#$&\hfil$#$\hfil\cr%2-column
%first row with braces, element 11 empty
{}&\hfil\enspace\mathop{\hbox to.9cm%
   {\downbracefill}}\limits^{\vbox{\hbox{
               $\scriptstyle p$}\kern2pt}}
        \enspace\hfil\mathop{\hbox to.6cm%
   {\downbracefill}}\limits^{\vbox{\hbox to
  0pt{\hss$\scriptstyle n-p$\hss}\kern2pt}}%
        \enspace\hfil\cr  % end first row
%Separation between first (border) row and
%rest
\noalign{\vglue1ex}
%first column with braces
\vcenter{\vfil
   \hbox{${\scriptstyle p}\left\{\vbox
   to.8cm{}\right.$}\vfil\vglue2ex\vfil
   \hbox{\llap{$\scriptstyle n{-}$}%
   ${\scriptstyle p}\left\{\vbox to.5cm{}
   \right.$}\vfil}
&%22-element is the matrix proper
\left(\vcenter{\offinterlineskip
\halign{\hfil$#$\hfil&\hfil$#$\hfil&
\hfil$#$\hfil&\hfil$#$\hfil
\tabskip=.5\tabskip&\vdts#&
\tabskip=2\tabskip
\hfil$#$\hfil&\hfil$#$\hfil&
\hfil$#$\hfil\cr%end template
\times&\times&\times&\times&&\times&
 \times&\times\cr
0     &\times&\times&\times&&\times&
 \times&\times\cr
0     &0     &\times&\times&&\times&
 \times&\times\cr
0     &0     &0     &\times&&\times&
 \times&\times\cr
\noalign{\vglue1ex}
\multispan8\dotfill\cr
0     &0     &0     &0     &&\times&
 \times&\times\cr
0     &0     &0     &0     &&\times&
 \times&\times\cr
0     &0     &0     &0     &&\times&
\times&\times\cr}%end halign (22)
}%end vcenter
\right)\cr %end 22-element
%Separation between last (border) row
%and rest
\noalign{\vglue1ex}
{}&\hfil\enspace\mathop{\hbox to.9cm{%
   \upbracefill}}\limits_{\vbox{\kern2pt
                      \icurt42}}
   \enspace\hfil
   \mathop{\hbox to.6cm{%
   \upbracefill}}\limits_{\vbox{\kern2pt
     \icmat4{1.5}}}\enspace\hfil%
\cr  % end last row
}%end halign
}%end vbox
$$
\item Other interesting two-dimensional
structures are commutative diagrams. Consult for those Spivak's
\LAMSTeX.\footnote{Within the graphics section I
         have supplied a simple example, however.}

\noindent Interestingly enough, simple commutative diagrams are done by
\verb|\matrix|, while I would expect some graphic commands.
\end{itemize}
\endgroup

\subsection{Examples: tables.}
For (full) rectangular tables \verb|\halign| or \verb|\valign| is generally
used, when they fit on the page.
Because of determining automatically the page breaks it might happen that
the page builder would like to split a table. Generally this is bad
typography, because we like to maintain the summary character of a table
all on one page.\footnote{When the latter is not important, for example for
   tables of values which goes on for pages, we can modify
   the row separator into a separator which allows line breaks.
   For tables which don't fit on a page
   there are special macros, like {\tt supertabular.sty}.}
A table smaller than the page should fit and in order to let that happen
we generally allow tables to float, that is they may be shifted around
a bit by the page builder.
For a survey on the issue see my Table Diversions paper, which also contains
a macro for handling bordered tables---the {\tt btable} macro (some 80 lines),
and used in this work (see later).

   Another important class of tables are the so-called trees. One can argue
whether they are tables or belong to graphics. Br\"uggeman-Klein has provided
a package called Tree\TeX. The user-interface looks good,
although I have not had any personal experience with it myself yet.

\paragraph*{Simple tables}via (plain) \TeX.
When I read Furuta a decade ago, I was impressed by the ease of mark-up
for tables via the tbl preprocessor of troff.\footnote{Because of
   that I was in favour of troff and its preprocessors.
   Happily a math professor stressed the importance of \TeX, and because
   UNIX was not widely available at the University, I entered \TeX-land.}
Below I'll show that a similar functionality---and some more, I also
   abstracted from the kinds of rules, and the positioning of the elements---%
is provided with respect to tables
by my btable macro for the class of bordered tables,
where the (possibly complicated) headers are treated separately
and independently from the (proper) table data, and the rowstub list.

\begingroup
%\input{btable.tex}
%btable.tex version 1, 15/7/92                            author: cgl@rug.nl
%Table Diversions is Published in EuroTeX92 proceedings, and MAPS92.2
%The article discusses typsetting tables via plain, surveys related work,
%introduces btable.tex, and provides a discipline for typesetting cell-blocks.
%Example of use                     (from the article)
%\def\data{2\cs7\cs6\rs
%          9\cs5\cs1\rs
%          4\cs3\cs8}
%\ruled\framed\btable\data
%Below is the btable macro listed
\newbox\tbl\let\ea=\expandafter
%Cell vertical size, row height and depth (separation implicit),
\newdimen\cvsize\newdimen\tsht\newdimen\tsdp\newdimen\tvsize\newdimen\thsize
%Parameter setting macros:   Rules
\def\hruled{\def\lineglue{\hrulefill}\def\colsep{}      \def\rowsep{\hrule}
   \let\rowstbsep=\colsep\let\headersep=\rowsep}
\def\vruled{\def\lineglue{\hfil}     \def\colsep{\vrule}\def\rowsep{}
   \let\rowstbsep=\colsep\let\headersep=\hrule}
\def\ruled {\def\lineglue{\hrulefill}\def\colsep{\vrule}\def\rowsep{\hrule}
   \let\rowstbsep=\colsep\let\headersep=\rowsep}
\def\nonruled{\def\lineglue{\hfil}   \def\colsep{}      \def\rowsep{}
   \def\rowstbsep{\vrule}\def\headersep{\hrule}}
\def\dotruled{\def\lineglue{\dotfill}\def\rowsep{\hbox to\thsize{\dotfill}}
\def\colsep{\lower1.5\tsdp\vbox to\cvsize{%
\leaders\hbox to0pt{\vrule height2pt depth2pt width0pt\hss.\hss}\vfil}}
\let\rowstbsep=\colsep\let\headersep=\rowsep}
%Parameter setting macros:   Controling positioning
\def\ctr{\def\lft{\hfil}\def\rgt{\hfil}}%Centered
\def\fll{\def\lft{}     \def\rgt{\hfil}}%Flushed left
\def\flr{\def\lft{\hfil}\def\rgt{}}     %Flushed right
%Parameter setting macros:   Framing
\def\framed{\let\frameit=\boxit}
\def\nonframed{\def\frameit##1{##1}}
\def\dotframed{\let\frameit=\dotboxit}
%
\def\btable#1{\vbox{\let\rsl=\rowstblst%Copy
\ifx\empty\template\ifx\empty\rowstblst
    \def\template{\colsepsurround\lft####\rgt&&\lft####\rgt\cr}
    \else\def\template{\colsepsurround####\hfil&&\lft####\rgt\cr}\fi
   \fi
\tsht=.775\cvsize\tsdp=.225\cvsize
\def\tstrut{\vrule height\tsht depth\tsdp width0pt}
%Logical mark up of column and row separators, via use of
\def\cs{&\colsepsurround\colsep\colsepsurround&}
\def\prs{&\colsepsurround\lineglue&}
\def\srp{&\lineglue\colsepsurround&}
\def\rs{\colsepsurround\tstrut\cr
        \ifx\empty\rowsep\else\noalign{\rowsep}\fi
        \ifx\empty\rowstblst\else\ea\nxtrs\fi}
\def\rss{&\colsepsurround\rowstbsep\colsepsurround&}
\def\hs{\colsepsurround\tstrut\cr
       \ifx\empty\headersep\else\noalign{\headersep}\fi
       \ifx\empty\rowstblst\else\ea\nxtrs\fi}
\preinsert
\setbox\tbl=\vbox{\tabskip=0pt\relax\offinterlineskip
\halign{\span\template\ifx\empty\first\ifx\empty\rowstblst\else
\ifx\empty\header\else\ea\rss\fi\fi\else\first\ea\rss\fi
\ifx\empty\header\ifx\empty\first\if\empty\rsl\else\ea\nxtrs\fi
                 \else\ea\hs\fi
\else\header\ea\hs\fi
#1\colsepsurround\tstrut\crcr}    }%end setbox
\postinsert
\ifx\caption\empty\else\hbox to\thsize{\strut\hfil\caption\hss}\captionsep\fi
\frameit{\copy\tbl}
\ifx\footer\empty\else\footersep\hbox{\vtop{\noindent\hsize=\thsize%
\footer}}\fi                             }}
%Defaults
\cvsize=4ex\tsht=.775\cvsize\tsdp=.225\cvsize\def\colsepsurround{\kern.5em}
\ctr\nonruled
\def\caption{}\def\first{}\def\header{}\def\rowstblst{}\def\footer{}\def\data{}
\def\captionsep{\medskip}    \def\headersep{\hrule}
\def\footersep{\smallskip}   \def\rowstbsep{\vrule}
\def\preinsert{}
\def\postinsert{\global\thsize=\wd\tbl
                \global\tvsize=\ht\tbl\global\advance\tvsize by\dp\tbl}
\ctr\nonruled\nonframed\def\template{}              %end Defaults
%Auxiliaries
\def\boxit#1{\vbox{\hrule\hbox{\vrule\vbox{#1}\vrule}\hrule}}
\def\dotboxit#1{\vbox{\offinterlineskip\hbox to\thsize{\dotfill}%
\hbox{\lower\tsdp\vbox to\tvsize{%
\leaders\hbox to0pt{\hss\vrule height2pt depth2pt width0pt.\hss}\vfil}%
\vbox{#1}\lower\tsdp\vbox to\tvsize{%
\leaders\hbox to0pt{\hss\vrule height2pt depth2pt width0pt.\hss}\vfil}}%
\hbox to\thsize{\dotfill}}}
%And to account for logical columns with \multispan
\def\spicspan{\span\omit}
\def\multispan#1{\omit\mscount=#1\multiply\mscount by2 \advance\mscount by-1
\loop\ifnum\mscount>1 \spicspan\advance\mscount by-1 \repeat}
%To process FIFO
\def\bfifo#1{\ifx\efifo#1\let\nxt=\relax\else\def\nxt{\process#1\bfifo}%
             \fi\nxt}%end \bfifo
\def\process#1{\hbox to0pt{\hss#1\hss}\kern.5ex}
%To handle row stub list
\def\nxtrs{\ifx\empty\rsl%\let\nxtel=\relax
\else\def\nxtel{\ea\nrs\rsl\srn}\ea\nxtel\fi}%next Row Stub
\def\nrs#1#2\srn{\gdef\rsl{#2}#1\rss}   %end btable.tex
%
\def\data{11\cs12\rs21\cs22}
\begin{itemize}
\item just framed data
   $$\vcenter{\framed\btable\data}$$
\item add header and rowstubs
  \def\header{\multispan2\hfill
                   Header\hfill}
  \def\rowstblst{{$1^{st}$ row}%
               {{$2^{nd}$ row}}}
  $$\vcenter{\btable\data}$$
\item add caption and footer,
      vary via dotted lines
  \def\caption{Caption}\def\footer{Footer}
  $$\vcenter{\dotruled\btable\data}$$
\item vary with ruled and framed
  $$\vcenter{\ruled\framed\btable\data}$$
\end{itemize}
via
\begingroup\small\begin{verbatim}
\def\data{11\cs12\rs21\cs22}
\begin{itemize}
\item just framed data
   $$\vcenter{\framed\btable\data}$$
\item add header and rowstubs
  \def\header{\multispan2\hfill
                   Header\hfill}
  \def\rowstblst{{$1^{st}$ row}%
               {{$2^{nd}$ row}}}
  $$\vcenter{\btable\data}$$
\item add caption and footer,
      vary via dotted lines
  \def\caption{Caption}\def\footer{Footer}
  $$\vcenter{\dotruled\btable\data}$$
\item vary with ruled and framed
  $$\vcenter{\ruled\framed\btable\data}$$
\end{itemize}
\end{verbatim}\endgroup

\paragraph*{Real-life.} AT\&T's example
from the tbl (troff) documentation, also supplied in \TeX book p.247
\begingroup
%\cite{lesk79}
\def\caption{AT\&T Common Stock}
\def\header{Year\cs Price\cs Dividend}
\catcode`?=\active \def?{\kern1.1ex}
\def\data{1971\cs41--54\cs\llap{\$}2.60\rs
             2\cs41--54\cs         2.70\rs
             3\cs46--55\cs         2.87\rs
             4\cs40--53\cs         3.24\rs
             5\cs45--52\cs         3.40\rs
             6\cs51--59\cs         ?.95\rlap*}
\def\footer{* (first quarter only)}
$$\vcenter{\vbox{\small
  \framed\ruled\btable\data}}$$
The above is obtained via \verb|\btable| as follows
\endgroup
%
\begingroup\small\begin{verbatim}
\def\caption{AT\&T Common Stock}
\def\header{Year\cs Price\cs Dividend}
\catcode`?=\active \def?{\kern1.1ex}
\def\data{1971\cs41--54\cs\llap{\$}2.60\rs
             2\cs41--54\cs         2.70\rs
             3\cs46--55\cs         2.87\rs
             4\cs40--53\cs         3.24\rs
             5\cs45--52\cs         3.40\rs
             6\cs51--59\cs         ?.95\rlap*}
\def\footer{* (first quarter only)}
$$\vcenter{\vbox{\small
  \framed\ruled\btable\data}}$$
\end{verbatim}
\endgroup
\endgroup
\subsection{Examples: graphics.}
The portable way is via \mmt, \LaTeX's picture environment,
or PiC\TeX.
For a survey see Clark's Portable Graphics in \TeX\ paper.
\TeX tures on the Macintosh by Blue Sky Research is famous for its
(non-portable) pic{\em tures\/} with \TeX. For inclusion of
photographs and in general halftones, see the work of Sowa.\footnote{On the
   Mac one can easily incorporate photos after they
   having been put onto CD in digitized form. Kodak provides the latter
   service.}
For drawing on the screen and get (La)\TeX\ code out see GNUplot
or \TeX CAD, for example.

Many disciplines make use of
special graphic diagrams. In this paper for example I won't
provide examples of
trees,
(math) graphs in general,
(advanced) commutative diagrams,
nor Feynmann diagrams,
to name but a few classes known to me.
\begin{itemize}
\item simple line diagrams via \mmt
\begingroup
\def\strut{\vrule height2.5ex depth1ex width0pt}
\def\fbox#1{\setbox0\hbox{\strut
 $\;$#1$\,$}\leavevmode\rlap{\copy0}%
 \makelightbox}
\def\element#1{\hbox to15ex{\hss#1\hss}}
\def\vconnector{\element{\strut\vrule}}
$$\hbox{\vbox{%
\element{\fbox{amsppt.sty}}
\vconnector
\element{\fbox{amstex.tex}}
\vconnector
\element{\fbox{\TeX}}}
\qquad\qquad\qquad
\vbox{%
\element{\fbox{amsart.sty}}
\vconnector
\element{\llap{\fbox{amstex.sty}---}%
 \fbox{\LaTeX}}
\vconnector
\element{\fbox{\TeX}}}
}$$
\endgroup
\noindent via
\begingroup\small\begin{verbatim}
$$\hbox{\vbox{%
\element{\fbox{amsppt.sty}}
\vconnector
\element{\fbox{amstex.tex}}
\vconnector
\element{\fbox{\TeX}}
}\qquad\qquad\qquad\vbox{%
\element{\fbox{amsart.sty}}
\vconnector
\element{\llap{\fbox{amstex.sty}---}
 \fbox{\LaTeX}}
\vconnector
\element{\fbox{\TeX}}
}}$$
\end{verbatim}\endgroup

\noindent with the auxiliaries
\begingroup\small\begin{verbatim}
\def\strut{\vrule height2.5ex depth1ex
 width0pt}
\def\fbox#1{\setbox0\hbox{\strut
 $\;$#1$\,$}\leavevmode\rlap{\copy0}%
 \makelightbox}
\def\element#1{\hbox to15ex{\hss#1\hss}}
\def\vconnector{\element{\strut\vrule}} .
\end{verbatim}\endgroup
\item flow chart borrowed from Furuta, via \LaTeX

\noindent
%\input{pic.pic}
\begingroup\small
\setlength{\unitlength}{4ex}
\begin{picture}(14,4)(0,-1)
\put(1, 1){\oval(2, 1)}
\put(1, 1){\makebox(0, 0){Start}}
\put(2, 1){\vector(1, 0){1.5}}
\put(3.5,.25){\framebox(2,1.5){\shortstack
         {\tiny Edit\\\tiny Document}}}
\put(5.5, 1){\vector(1, 0){1.5}}
\put(7,.25){\framebox(2,1.5){\shortstack
       {\tiny Format\\\tiny Document}}}
\put(9, 1){\vector(1, 0){1.5}}
\put(11.5, 1){\oval(2, 1)}
\put(11.5, 1){\makebox(0, 0){End}}
\bezier{75}(4.5,.25)(6.25,-1)(8,.25)
\put(4.5,.25){\vector(-2, 1){0}}
\bezier{150}(4.5,1.75)(8,4)(11.5,1.5)
\put(11.5,1.5){\vector(2,-1){0}}
\end{picture}
\endgroup

\par\noindent
via
\begingroup\small\begin{verbatim}
\setlength{\unitlength}{4ex}
\begin{picture}(14,4)(0,-1)
\put(1, 1){\oval(2, 1)}
\put(1, 1){\makebox(0, 0){Start}}
\put(2, 1){\vector(1, 0){1.5}}
\put(3.5,.25){\framebox(2,1.5){\shortstack
         {\tiny Edit\\\tiny Document}}}
\put(5.5, 1){\vector(1, 0){1.5}}
\put(7,.25){\framebox(2,1.5){\shortstack
       {\tiny Format\\\tiny Document}}}
\put(9, 1){\vector(1, 0){1.5}}
\put(11.5, 1){\oval(2, 1)}
\put(11.5, 1){\makebox(0, 0){End}}
\bezier{75}(4.5,.25)(6.25,-1)(8,.25)
\put(4.5,.25){\vector(-2, 1){0}}
\bezier{150}(4.5,1.75)(8,4)(11.5,1.5)
\put(11.5,1.5){\vector(2,-1){0}}
\end{picture}
\end{verbatim}\endgroup
\par\noindent
Although the specification is
not as easy as via the pic preprocessor,
it is not difficult when we
start from a template, like the one above.
Cumbersome is the treatment of the arrow heads, but these can be hidden.
\item a pie-chart via \LaTeX

%\input{lus.pic}
\setlength{\unitlength}{6ex}
\begin{picture}(6, 5)(-3, -2)
%1st quadrant
%\bezier{100}(2, 0)(2, .54)(1.79, .89)
     % 0  - `30' 2:1-lijn
\bezier{60}(1.79, .89)(1.46, 1.46)(1, 1.73)
     % `30' - 60
\bezier{60}(1, 1.73)(.54, 2)(0, 2)
     % 60 - 90
%2nd quadrant
\bezier{60}(0, 2)(-.54, 2)(-1, 1.73)
     % 90 - 120
\bezier{60}(-1,1.73)(-1.46,1.46)(-1.73,1)
     %120 - 150
\bezier{60}(-1.73, 1)(-2, .54)(-2, 0)
     %150 - 180
%3rd quadrant
\bezier{60}(-2, 0)(-2, -.54)(-1.73, -1)
     %180 - 210
\bezier{60}(-1.73,-1)(-1.46,-1.46)(-1,-1.73)
     %210 -240
\bezier{60}(-1, -1.73)(-.54, -2)(0, -2)
     %240 - 270
%4th quadrant
\bezier{60}(0, -2)(.54, -2)(1, -1.73)
     %270 - 300
\bezier{60}(1,-1.73)(1.46,-1.46)(1.73,-1)
     %300 - 330
\bezier{60}(1.73, -1)(2, -.54)(2, 0)
     %330 - 360
%division lines
\put(0, 0){\line(1, 0){2}}
\put(0, 0){\line(2, 1){1.79}}
     %1.79 = 2 cos arctg .5
%\put(0, 0){\line(-2, 1){1.79}}
\bezier{75}(0, 0)(-.81, .59)(-1.61, 1.18)
     %-.81 = cos 144; .59 = sin 144
%\put(0, 0){\line(-1, -2){.89}}
     % .89 = 2 sin arctg .5
\bezier{75}(0, 0)(-.59, -.81)(-1.18, -1.62)
     %-.59 = cos -126; -.81 = sin -126
%piece
\bezier{60}(2.5, 0.1)(2.5, .64)(2.29, .99)
     % shift .5, .1
\put(0.5, 0.1){\line(1, 0){2}}
\put(0.5, 0.1){\line(2, 1){1.79}}
%Candles:
\bezier{20}(0,1.31)(-.15,1.45)(0,1.61)
\bezier{20}(0,1.31)(.175,1.45)(0,1.61)
\put(-.1,.1){\line(0,1){1.2}}
\put(.1,.05){\line(0,1){.95}}
\put(.1,1){\line(-2,3){.2}}
%left
\bezier{20}(-.25,1.46)(-.40,1.6)(-.25,1.76)
\bezier{20}(-.25,1.46)(-.075,1.6)(-.25,1.76)
\put(-.35,.25){\line(0,1){1.2}}
\put(-.15,.2){\line(0,1){.95}}
\put(-.15,1.15){\line(-2,3){.2}}
%right
\bezier{20}(.25,1.46)(.40,1.6)(.25,1.76)
\bezier{20}(.25,1.46)(.075,1.6)(.25,1.76)
\put(.35,.25){\line(0,1){1.2}}
\put(.15,.2){\line(0,1){.95}}
\put(.15,1.15){\line(2,3){.2}}
%leftleft
\bezier{20}(-.5,1.61)(-.65,1.75)(-.5,1.91)
\bezier{20}(-.5,1.61)(-.325,1.75)(-.5,1.91)
\put(-.6,.4){\line(0,1){1.2}}
\put(-.4,.35){\line(0,1){.95}}
\put(-.4,1.3){\line(-2,3){.2}}
%rightright
\bezier{20}(.5,1.61)(.65,1.75)(.5,1.91)
\bezier{20}(.5,1.61)(.325,1.75)(.5,1.91)
\put(.6,.4){\line(0,1){1.2}}
\put(.4,.35){\line(0,1){.95}}
\put(.4,1.3){\line(2,3){.2}}
%texts
\put(-1, -.1){\makebox(0, 0){\strut Happy}}
\put(.5, -1){\makebox(0, 0){\strut Birthday}}
\put(1.9, .35){\makebox(0, 0){\strut NTG}}
\end{picture}

\par\noindent via
\begingroup\small
%\begingroup\small
\begin{verbatim}
\setlength{\unitlength}{6ex}
\begin{picture}(6, 5)(-3, -2)
%1st quadrant
%\bezier{100}(2, 0)(2, .54)(1.79, .89)
     % 0  - `30' 2:1-line
\bezier{60}(1.79, .89)(1.46, 1.46)(1, 1.73)
     % `30' - 60
\bezier{60}(1, 1.73)(.54, 2)(0, 2)
     % 60 - 90
%2nd quadrant
\bezier{60}(0, 2)(-.54, 2)(-1, 1.73)
     % 90 - 120
\bezier{60}(-1,1.73)(-1.46,1.46)(-1.73,1)
     %120 - 150
\bezier{60}(-1.73, 1)(-2, .54)(-2, 0)
     %150 - 180
%3rd quadrant
\bezier{60}(-2, 0)(-2, -.54)(-1.73, -1)
     %180 - 210
\bezier{60}(-1.73,-1)(-1.46,-1.46)(-1,-1.73)
     %210 -240
\bezier{60}(-1, -1.73)(-.54, -2)(0, -2)
     %240 - 270
%4th quadrant
\bezier{60}(0, -2)(.54, -2)(1, -1.73)
     %270 - 300
\bezier{60}(1,-1.73)(1.46,-1.46)(1.73,-1)
     %300 - 330
\bezier{60}(1.73, -1)(2, -.54)(2, 0)
     %330 - 360
%division lines
\put(0, 0){\line(1, 0){2}}
\put(0, 0){\line(2, 1){1.79}}
     %1.79 = 2 cos arctg .5
%\put(0, 0){\line(-2, 1){1.79}}
\bezier{75}(0, 0)(-.81, .59)(-1.61, 1.18)
     %-.81 = cos 144; .59 = sin 144
%\put(0, 0){\line(-1, -2){.89}}
     % .89 = 2 sin arctg .5
\bezier{75}(0, 0)(-.59, -.81)(-1.18, -1.62)
     %-.59 = cos -126; -.81 = sin -126
%piece
\bezier{60}(2.5, 0.1)(2.5, .64)(2.29, .99)
     % shift .5, .1
\put(0.5, 0.1){\line(1, 0){2}}
\put(0.5, 0.1){\line(2, 1){1.79}}
%Candles:
\bezier{20}(0,1.31)(-.15,1.45)(0,1.61)
\bezier{20}(0,1.31)(.175,1.45)(0,1.61)
\put(-.1,.1){\line(0,1){1.2}}
\put(.1,.05){\line(0,1){.95}}
\put(.1,1){\line(-2,3){.2}}
%left
\bezier{20}(-.25,1.46)(-.40,1.6)(-.25,1.76)
\bezier{20}(-.25,1.46)(-.075,1.6)(-.25,1.76)
\put(-.35,.25){\line(0,1){1.2}}
\put(-.15,.2){\line(0,1){.95}}
\put(-.15,1.15){\line(-2,3){.2}}
%right
\bezier{20}(.25,1.46)(.40,1.6)(.25,1.76)
\bezier{20}(.25,1.46)(.075,1.6)(.25,1.76)
\put(.35,.25){\line(0,1){1.2}}
\put(.15,.2){\line(0,1){.95}}
\put(.15,1.15){\line(2,3){.2}}
%leftleft
\bezier{20}(-.5,1.61)(-.65,1.75)(-.5,1.91)
\bezier{20}(-.5,1.61)(-.325,1.75)(-.5,1.91)
\put(-.6,.4){\line(0,1){1.2}}
\put(-.4,.35){\line(0,1){.95}}
\put(-.4,1.3){\line(-2,3){.2}}
%rightright
\bezier{20}(.5,1.61)(.65,1.75)(.5,1.91)
\bezier{20}(.5,1.61)(.325,1.75)(.5,1.91)
\put(.6,.4){\line(0,1){1.2}}
\put(.4,.35){\line(0,1){.95}}
\put(.4,1.3){\line(2,3){.2}}
%texts
\put(-1, -.1){\makebox(0, 0){\strut Happy}}
\put(.5, -1){\makebox(0, 0){\strut Birthday}}
\put(1.9, .35){\makebox(0, 0){\strut NTG}}
\end{picture}
\end{verbatim}\endgroup
\noindent
The above use of the bezier splines makes the
creation of scaling invariant circles
easier than the approach by Ramek in the proceedings of
\TeX eter '88.

\item commutative diagrams (\LAMSTeX, \ldots). As a simple example
the calculation flow of the autocorrelation
function,
$a_f$, inspired by the \TeX Book ex18.46, p.358.
${\cal F}$ denotes the Fourier transform and
${\cal F}\strut^{-}$ the inverse Fourier transform
\vskip1ex
%
\def\lllongrightarrow{\relbar\joinrel%
       \relbar\joinrel\relbar\joinrel%
       \relbar\joinrel\rightarrow}
\def\llongrightarrow{\relbar\joinrel%
        \relbar\joinrel\rightarrow}
\def\normalbaselines{%
           \baselineskip20pt
           \lineskip3pt
           \lineskiplimit3pt}
\def\mapright#1{\smash{\mathop{
   \llongrightarrow}\limits^{#1}}}
\def\lmapright#1{\smash{\mathop{
   \lllongrightarrow}\limits^{#1}}}
\def\mapdown#1{\Big\downarrow
      \rlap{$\vcenter{\hbox{$#1$}}$}}
\def\mapup#1{\Big\uparrow
      \rlap{$\vcenter{\hbox{$#1$}}$}}
$$%Diagram
\matrix{f&\lmapright\otimes&a_f\cr
    \mapdown{{\cal F}}&&\mapup{%
            {\cal F}\strut^{-}}\cr
    \hbox to 0pt{\hss${\cal F}(f)$\hss}
    &\mapright\times\hfil&
    \hbox to 0pt{\hss$\bigl({\cal F}(f)
                     \bigr)^2$\hss}\cr}
\qquad$$%a little to the left
via
\begingroup\small\begin{verbatim}
$$\matrix{f&\lmapright\otimes&a_f\cr
    \mapdown{{\cal F}}&&\mapup{%
            {\cal F}\strut^{-}}\cr
    \hbox to 0pt{\hss${\cal F}(f)$\hss}
    &\mapright\times\hfil&
    \hbox to 0pt{\hss$\bigl({\cal F}(f)
                     \bigr)^2$\hss}\cr}$$
\end{verbatim}\endgroup
\noindent with auxiliaries
\begingroup\small\begin{verbatim}
\def\lllongrightarrow{\relbar\joinrel%
       \relbar\joinrel\relbar\joinrel%
       \relbar\joinrel\rightarrow}
\def\llongrightarrow{\relbar\joinrel%
        \relbar\joinrel\rightarrow}
\def\normalbaselines{%
           \baselineskip20pt
           \lineskip3pt
           \lineskiplimit3pt}
\def\mapright#1{\smash{\mathop{
   \llongrightarrow}\limits^{#1}}}
\def\lmapright#1{\smash{\mathop{
   \lllongrightarrow}\limits^{#1}}}
\def\mapdown#1{\Big\downarrow
      \rlap{$\vcenter{\hbox{$#1$}}$}}
\def\mapup#1{\Big\uparrow
      \rlap{$\vcenter{\hbox{$#1$}}$}}
\end{verbatim}\endgroup

\item \MF{} coupled to \TeX. Leading in this area is the work of Hoenig, for
example see his `When \TeX\ and \MF{} work together.'
He has worked out the printing along curved lines,
and the typesetting of paragraphs which flow around
arbitrary shapes!
Very powerul, but not simple to use for the moment.
It looks like going back to the roots,
because Knuth's first version of the `\TeX book' contained it all:
`\TeX\ and \MF{}, New directions in typesetting.'

\item (encapsulated) \PS. Knuth left some niches for handling these
  kinds of things via the \cs{special} command. A very nice survey of
  the possibilities which can be obtained when incorporating \PS{} is
  given in Goossens' `\PS{} en \LaTeX,' which is also a chapter in
  the \LaTeX-companion. A survey of the various user approaches has been
  compiled by Anita Hoover.
\item Screen drawings. An example is GNUplot. Cameron in \TeX line
  characterized these kind of systems as
\begin{quote}
  `\ldots There are a couple of programs available which take all the
  calculation out: you draw your picture using the mouse, and it is
  automagically compiled into \LaTeX\ source. But for complicated
  figures, mathematical insight or computational power may be required.'
\end{quote}
  An example of figures that require math insight are Hoenig's `Fractal
  images with \TeX.'
  We can add to that the reuse aspect, which is hindered by the drawing
  approach, and the unreadable nature of  machine-generated code.
  But certainly these tools have their niche in the spectrum of tools for EP.
\end{itemize}

\section{Front matter}
Much attention is paid to front matter:
cover,
publication characteristics (source, ISBN or other classification),
title etc.,
abstract,
keywords,
table of contents and the like if not considered as an appendix,
foreword.
Basically the style or format can handle these easily.
Because of the eye-catching need of a cover a designer
is generally involved and the cover, especially
the graphics, typeset by different means.
The page with publication characteristics  can be left to
the copy editor.
For the others just obey the mark-up characteristics, as demanded
by the style file.
\section{Back matter}
As back matter we have the various appendices. Two kinds are
noteworthy: the list of references and the index. Both are
complicated because of the {\em cross-referencing on the fly.}
\paragraph*{Bibliography creation.}%
With a publication we have the problem of
handling a list of references,
to extract them from our literature database,
and to format them appropriately.
We also like to cross-reference them to the list of references,
such that it is adaptable to different journal traditions,
with respect to formatting of citations.
There are tools available to do that, for example
\LaTeX's \BibTeX\ with its {\tt thebibliography} environment, and
\AmSTeX's \cs{ref} and \cs{endref} structures.
I designed my own `little language
within \TeX' to handle that all in a one-pass job within \AllTeX.
To get the flavour, the bibliography at the end of this paper
has been prepared via
\begingroup\small\begin{verbatim}
\section*{References}
\input{lit.dat}%database file
  \def\tubissue#1(#2){{\sl TUGboat\/}
                        {\bf#1} (#2)}
  \def\ls#1{\ea\bibentry#1\endgraf}
\input{lit.sel}%file with names
\end{verbatim}\endgroup
See my BLUe's Bibliography paper
for more details, and my solution of the cross-referencing in a one-pass job.

\paragraph*{Index preparation.}
This is complicated because of the dynamic allocation of page numbers
and inclusion of these in the index. It is also an art to provide the
right entries. Generally (external) sorting needs to be done too,
next to the formatting. A complicated job.
\begin{quote}\TeX nically
   there is the tool Makeindex, to cooperate with (La)\TeX.
\end{quote}
Knuth provided a mark-up mechanism for extracting
the index entries and let the OTR add the page numbers.
These items are writen on a file
for further processing, like sorting, and adding comments and the like.
I consider that very powerful, but it is not completely automatic.
The user, or publisher, has to account for the finishing touch, for the
moment. For a survey of the intricacies which come along
when writing automatic index generators, see
the report of Chen and Harrison about developing Makeindex.

I have exercised index preparation \`a la Knuth in my Sorting in BLUe paper.
Although the approach of doing it all within \TeX\ looks promising,
it still needs  some polishing for BLU to become  useful.

\section{Guidelines for Choosing}
Given the above-mentioned variety of tools %and your personal circumstances
the following questions can be useful
\begin{quote}
What facilities does your publisher provide?\\
What is the document like?                           \\
What tools are already in use?                        \\
Whom is it aimed at?                               \\
How many authors are involved?                         \\
%(many authors many publications?),                    \\
Is (partial, e.g.\  bibliographical) reuse also envisaged?\\
Is future use, different from formatting, in sight?
\end{quote}
\noindent
First, contact your publisher and agree upon the tools to be used.\\
Next best, when you are on your own, consider
\begin{description}
\item[]No structure \hfill it does not matter \\%what will be used
       (For right-to-left etc.\hfill \TeXXeT)
\item[]Scientific papers \hfill \AllTeX%can best be used
\item[]Reuse \hfill \AllTeX, SGML? %can best be used
\item[]Various authors\hfill \AllTeX, SGML? % as a uniform language
%\TeX\ which is {\em de facto\/} in use for that purpose
\item[]Future (nonformatting) use\hfill \AllTeX, SGML?
\end{description}
A user sufficiently fluent in di-roff  would like me to
substitute x-roff for \TeX\ in the table above. Be my guest, I don't
have experience with x-roff.

\section*{Acknowledgements}
This paper has been processed via \LaTeX\ because I needed the
functionality of the picture environment. The standard formatting of
the peculiar \TeX-related names have been borrowed from \verb|tugboat.cmn|.
I used the \verb|ltugproc.sty|---style for
TUG proceedings---because of the nice way the front matter
is typeset.

I like to thank Christina Thiele for polishing my English
and pointing to the right use of fonts for established names.\footnote{Although
   this has its difficulties simply using \cs{MF} for example goes wrong
   when we vary size.}
Gerard van Nes blew his horn once again. Thank you!

\section*{Conclusions}
For high-quality computer-assisted typography \TeX{} etc.{} is
a flexible and excellent craftsman-like tool,
with the following characteristics
\begin{itemize}
\item \TeX\ is in the PD, available for all platforms
\item flavours of \TeX\ have been added
to facilitate its use, next to macro toolboxes
\item formats and style files have been added to facilitate the publication
process
\item \TeX\ can be used with many fonts, and takes its own font generation tool
\item drivers, WSYIWYG interfaces are commercially available
\item working environments are provided by user groups
\item lingua franca for scientific email communication
\item publishing houses accept compuscripts marked up by (La)\TeX
\item users maintain digests, discussion lists and file servers
\item some 10k organized users, with many books published via \AllTeX.
\end{itemize}
\noindent
Once mastered \TeX\ is a nice basic tool.
However, the way to error-less mark-up is hard and haunting, unless,
supported (by a publisher) with
generic styles,
user's guides, and
templates.
Using \LaTeX\ {\em as-is\/} and
supported by publishers is much simpler
than learning \TeX\ per se.

The \TeX\ arcana is complex.
(Professional) Education is paramount!
The twins \TeX-\MF{} will serve for a lifetime.
And above all let us keep it simple!
\section*{References}
\begin{thebibliography}{xxxxx}
\input{lit.dat}
%\input{lit.lat}
  \def\tubissue#1(#2){{\sl TUGboat\/} {\bf#1} (#2)}
  \def\ls#1{\ea\bibitem{}#1\endgraf}
\input{lit.sel}
\end{thebibliography}

\end{document}









