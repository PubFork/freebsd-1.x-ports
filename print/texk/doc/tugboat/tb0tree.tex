%		tb11eppstein.tex

%\Title Trees in \TeX
%\\David Eppstein
%\endx

%	This paper was published in TUGboat 6#1, March 1985.
%	David Eppstein's address (as of 15 June 1988) is
%		Computer Science Department
%		Columbia University
%		New York, NY 10027
%		Eppstein@cs.Columbia.edu

% First the tree macro definitions.

\input treedef

% Now the paper itself

{\obeyspaces\gdef {\ifvmode\indent\fi\space}}
{\catcode `|=0 \catcode`\\=12 |gdef|vbarg#1\endvb{#1|endgroup|medskip}}
\def\makeother#1{\catcode`#112\relax}
\def\vb{\medskip \begingroup \verb \obeylines \obeyspaces \tt \vbarg}
\def\verb{\let\do\makeother \dospecials}
\catcode`\*=\active
\def*{\begingroup\verb\tt\dostar}
\def\dostar#1*{#1\endgroup}

% \centerline{\bf Trees in \TeX}
% \centerline{David Eppstein; February 6, 1985}

\beginsection Introduction

There are many possible uses for trees in typeset text.  The following
taxonomy illustrates some of them.

\tree
  Tree
  Uses

  \subtree
    Computer
    Science

    \subtree
      Data
      Structures
      
      \leaf{Search Tree}
      \leaf{Priority Queue}
    \endsubtree

    \subtree
      Parsing

      \leaf{Parse Tree}
      \leaf{Symbol Table}
    \endsubtree

    \subtree
      Structured
      Programming
    \endsubtree
  \endsubtree

  \subtree
    Genealogy
    \leaf{Ancestors}
    \leaf{Descendants}
  \endsubtree

  \subtree
    Taxonomies
    \leaf{Tree Uses}
  \endsubtree
\endtree

Unfortunately \TeX\ provides no easy way to typeset such trees.
One possible method is given in exercise 22.14 of the \TeX book: using
\TeX's alignment primitives by hand.  This method becomes very clumsy as the
trees grow, however.  A more general technique is to write a set of tree
construction macros; that is the approach taken in this paper.  The
taxonomy above was typeset with the following input:

\vb
\tree
  Tree
  Uses

  \subtree
    Computer
    Science

    \subtree
      Data
      Structures

      \leaf{Search Tree}
      \leaf{Priority Queue}
    \endsubtree
    ...
  \endsubtree
  ...
\endtree
\endvb

It turns out that \TeX's alignment primitives are not very well suited to
automatic generation of trees.  The left edges of the trees at each
level can easily be made to line up, but it is difficult to center
lines of text for the root of a tree in vertical relation to its subtrees.
Instead, the macros described here construct trees from boxes and
glue, doing the alignment themselves.  This is not quite as simple as
it sounds---it would be incorrect to set subtrees independently of
each other, because then the edges would not line up.  For instance,
in the taxonomy above, the text ``Search Tree'' should line up
with ``Parse Tree''.  A one-pass algorithm would set the former
somewhat to the right of the latter.

To solve this problem, the macros described here set a tree using
three passes.  First, a data structure is built up from the tree
definition.  Second, that data structure is used to calculate the
width of each level of the tree, so that the subtrees can be aligned with each
other.  Finally, the data structure and the calculated list of widths
are used to set the system of boxes, glue, and rules that make up the tree.

\beginsection Pass 1: Internal storage of the tree structure

There are several possible ways to store the structure defined by the
tree macros.  Since we want to remember already-set text (the words at
the roots of each tree of subtree) we will use a nested structure of boxes.
Each subtree is stored in an *\hbox*, so that pieces of it can be pulled
off easily using *\lastbox* and *\unskip*.  To distinguish it from
another subtree, the text at the root of a subtree is stored in a *\vbox*.
To make this clearer, let us return to our original taxonomy.
We shall ignore for the moment
the details inside the text *\vbox*es, and the glue between boxes.
After the first pass, the tree as a whole would look like the
following set of boxes:

\vb
\hbox{\vbox{Tree Uses}
  \hbox{\vbox{Computer Science}
    \hbox{\vbox{Data Structures}
      \hbox{\vbox{Search Tree}}
      \hbox{\vbox{Priority Queue}}}
    ...}
  ...}
\endvb

Now we can begin defining the tree macros.  We start defining a tree
with the *\tree* macro; this merely sets up the assignment of the
boxed tree structure into a box called *\treebox*.  Starting a subtree
is similar, but there is no assignment; also, if it is the first
*\subtree* of its tree or subtree, we must stop making the *\vbox*
containing the root text.  A leaf is merely a subtree without any
sub-subtrees.

\vb
\newbox\treebox
\def\tree{\global\setbox\treebox=\boxtree}
\def\subtree{\ettext \boxtree}
\def\leaf#1{\subtree#1\endsubtree}
\endvb

Finishing a subtree merely involves first making sure the root text is
complete, and second
completing the box that was started in the expansion of *\subtree*.
Finishing a whole tree involves both of those steps, but then after the
box is completed the remaining two passes must be run.

\vb
\def\endsubtree{\ettext \egroup}
\def\endtree{\endsubtree \settreesizes \typesettree}
\endvb


Now all that remains to be defined of the first pass is the construction
of the *\vbox* containing the root text.  The difficulty here is
convincing \TeX\ to make the *\vbox* only as wide as the widest line
of text, rather than the width of the entire page.  One solution is to
put the text in an *\halign*, with *\crcr* implicit at the end of each line.
The *\iftreetext* test is used to tell whether we are still inside the
*\halign* and *\vbox*, so that *\ettext* can tell whether it should do
anything.  It is globally false, but within the *\vbox* it gets set to true.

\vb
\newif\iftreetext\treetextfalse         % Whether still aligning text
\def\boxtree{\hbox\bgroup               % Start outer box of tree or subtree
  \baselineskip 2.5ex                   % Narrow line spacing slightly
  \tabskip 0pt                          % No spurious glue in alignment
  \vbox\bgroup                          % Start inner text \vbox
  \treetexttrue                         % Remember for \ettext
  \let\par\crcr \obeylines              % New line breaks without explicit \cr
  \halign\bgroup##\hfil\cr}             % Start alignment with simple template
\def\ettext{\iftreetext                 % Are we still in inner text \vbox?
  \crcr\egroup \egroup \fi}             % Yes, end alignment and box
\endvb

\beginsection Pass 2: Calculation of widths at each level

Here we calculate a list of the dimensions of each level of the tree;
that is, the widths of the widest *\vbox* at each level.
To do this, we need to be able to maintain lists of things.  Since these
are dimensions rather than boxes of text it will be most convenient to
use macros like the ones given on page 378 of the \TeX book.  However,
it turns out that we need to set our lists both locally to some
grouping and also globally.  Therefore, we will use a stripped down
version of those list macros that can handle the *\global* flag.
To implement this feature, we have to lose some others; the chief losses
are that the contents of the lists will be macro-expanded by various of
the list manipulation macros, and that we can't use redefinitions of *\\*
to perform some operation on the whole list.

To initialize a control sequence to the empty list, we do *\let\csname\nil*.
Then to add an element to the start of the list we do *\cons{tokens}\csname*,
and to remove that element we do *\cdr\csname*.  Both *\cons* and *\cdr*
can be prefixed with *\global*.  The first element on the list can be
expanded into the token stream by doing *\car\csname*.  There is no
error checking, so giving *\car* or *\cdr* the empty list will cause
mysterious errors later on.  Because of the macro expansion performed
by *\cons* and *\cdr*, the token used to separate list elements
expands to itself, and unlike the \TeX book macros cannot be redefined
to do anything useful.

\vb
\def\cons#1#2{\edef#2{\xmark #1#2}}     % Add something to start of list.
\def\car#1{\expandafter\docar#1\docar}  % Take first element of list
\def\docar\xmark#1\xmark#2\docar{#1}    % ..by ignoring rest in expansion.
\def\cdr#1{\expandafter\docdr#1\docdr#1}% Similarly, drop first element.
\def\docdr\xmark#1\xmark#2\docdr#3{\def#3{\xmark #2}}
\def\xmark{\noexpand\xmark}             % List separator expands to self.
\def\nil{\xmark}                        % Empty list is just a separator.
\endvb

We calculate the level widths by calling *\setsizes* on the tree; it
will in turn call itself recursively for each of its subtrees.  The
tree being sized will be in *\box0*, which is used as scratch in this
macro, and the list of widths already found for this level and below
will be in *\treesizes* (initially *\nil*).  When the macro exits,
*\treesizes* will be updated with the widths found in the various
levels of the given tree or subtree.  A new *\dimen*, *\treewidth*, is
used within the macro to remember the previous maximum width at the
level of the tree's root.

\vb
\def\settreesizes{\setbox0=\copy\treebox \global\let\treesizes\nil \setsizes}
\newdimen\treewidth                     % Width of this part of the tree.
\def\setsizes{\setbox0=\hbox\bgroup     % Get a horiz list as a workspace.
  \unhbox0\unskip                       % Take tree, unpack it into horiz list.
  \inittreewidth                        % Get old width at this level.
  \sizesubtrees                         % Recurse through all subtrees.
  \sizelevel                            % Now set width from remaining \vbox.
  \egroup}                              % All done, finish our \hbox.
\endvb

The first thing *\setsizes* does is to find out what the previous
maximum at this level was, and store it in *\treewidth*.  If
*\treesizes* is *\nil*, we haven't seen anything this deep in the tree
before, so the previous size is zero.  Otherwise, it is *\car\treesizes*,
and we also do *\cdr\treesizes* to prepare for later recursive calls
to *\setsizes*.

\vb
\def\inittreewidth{\ifx\treesizes\nil   % If this is the first at this level
    \treewidth=0pt                      % ..then we have no previous max width.
  \else \treewidth=\car\treesizes       % Otherwise take old max level width
    \global\cdr\treesizes               % ..and advance level width storage
    \fi}                                % ..in preparation for next level.
\endvb

At this point, we have a horizontal list (the *\hbox* in *\setsizes*)
containing the *\vbox* for the text at the root of this subtree,
followed by the *\hbox*es for all of its sub-subtrees.  We loop pulling boxes
from the end of the list with *\lastbox* until we find the text *\vbox*,
calling *\setsizes* recursively for each *\hbox* we come across.

\vb
\def\sizesubtrees{\loop                 % For each box in horiz list (subtree)
  \setbox0=\lastbox \unskip             % ..pull it off list and flush glue.
  \ifhbox0 \setsizes                    % If hbox, it's a subtree - recurse
  \repeat}                              % ..and loop; end loop on tree text.
\endvb

Now all that remains to do in this call to *\setsizes* is to update
*\treewidth* if the text box, which can be found in *\box0*, is wider
than the previous maximum.  Then we add the (possibly updated) value
of *\treewidth* as a text string back onto the head of *\treesizes*.

\vb
\def\sizelevel{\ifdim\treewidth<\wd0    % If greater than previous maximum
   \treewidth=\wd0 \fi                  % Then set max to new high
 \global\cons{\the\treewidth}\treesizes}% In either case, put back on list
\endvb

\beginsection Pass 3: Typesetting the tree

We are now ready to begin actual construction of the tree.  This is
done by calling *\maketree*, which like *\setsizes* calls itself
recursively for all subtrees.  It adds an *\hbox* containing the given
subtree (which it finds in *\treebox*) to the current horizontal list;
thus the outer call to *\maketree* sends the whole tree into \TeX's
output stream.  

\vb
\newdimen\treeheight                    % Height of this part of the tree.
\newif\ifleaf                           % Tree has no subtrees (is a leaf).
\newif\ifbotsub                         % Bottom subtree of parent.
\newif\iftopsub                         % Top subtree of parent.
\def\typesettree{\medskip \maketree \medskip}   % Make whole tree with spacing.
\def\maketree{\hbox{\treewidth=\car\treesizes   % Get width at this level.
  \cdr\treesizes                        % Set up width list for recursion.
  \makesubtreebox\unskip                % Set \treebox to text, make subtrees.
  \ifleaf \makeleaf                     % No subtrees, add glue.
  \else \makeparent \fi}}               % Have subtrees, stick them at right.
\endvb

After *\maketree* sets *\treewidth* from *\treesizes*, it calls
*\makesubtreebox*.  This opens up the horizontal list describing this
level of the tree, and checks whether it has subtrees.  If not,
*\ifleaf* is set to true; otherwise it is set to false, and *\box0*
is set to contain a *\vbox* of them with their connecting rules,
except for the horizontal rule leading from the tree text to the
subtrees.  In any case *\treebox* is set to the *\vbox* containing the
tree text.

\vb

{\catcode`@=11                          % Be able to use \voidb@x.
\gdef\makesubtreebox{\unhbox\treebox    % Open up tree or subtree.
  \unskip\global\setbox\treebox\lastbox % Pick up very last box.
  \ifvbox\treebox                       % If we're already at the \vbox
    \global\leaftrue \let\next\relax    % ..then this is a leaf.
  \else \botsubtrue                     % Otherwise, we have subtrees.
    \setbox0\box\voidb@x                % Init stack of processed subs
    \botsubtrue \let\next\makesubtree   % ..and call \maketree on them.
  \fi \next}}                           % Finish up for whichever it was.
\endvb

If this tree or subtree itself has subtrees, we need to put them and
their connections in *\box0* for *\makesubtreebox*.  We come here with
the bottom subtree in *\treebox*, the remaining list of subtrees in
the current horizontal list, and the already processed subtrees
stacked in *\box0*.  The *\ifbotsub* test will be true for the first
call, that is, the bottom subtree.  Here we process the subtree in
*\treebox*.  If this is the top subtree, we return; otherwise we tail
recurse to process the remaining subtrees.  We use *\box1* as
another scratch variable; this is safe because the *\hbox* in *\maketree*
puts us inside a group, and also because we are not changing the output list.

\penalty-200		%%%%%%%%%%%%%%%%%%%%	added for TUGboat

\vb

\def\makesubtree{\setbox1\maketree      % Call \maketree on this subtree.
  \unskip\global\setbox\treebox\lastbox % Pick up box before it.
  \treeheight=\ht1                      % Get height of subtree we made
  \advance\treeheight 2ex               % Add some room around the edges
  \ifhbox\treebox \topsubfalse          % If picked up box is a \vbox,
    \else \topsubtrue \fi               % ..this is the top, otherwise not.
  \addsubtreebox                        % Stack subtree with the rest.
  \iftopsub \global\leaffalse           % If top, remember not a leaf
    \let\next\relax \else               % ..(after recursion), set return.
    \botsubfalse \let\next\makesubtree  % Otherwise, we have more subtrees.
  \fi \next}                            % Do tail recursion or return.
\endvb

Each subtree in the list is processed and stacked in *\box0*; this is
done by *\addsubtreebox*, which calls *\subtreebox* to add connecting
rules to the subtree in *\box1*, and appends to that the old contents
of *\box0*.  The vertical connecting rules in the tree are made with
tall narrow *\hrule*s rather than a more simple calls to *\vrule*,
because they are made inside a *\vbox*.

\vb
\def\addsubtreebox{\setbox0=\vbox{\subtreebox\unvbox0}}
\def\subtreebox{\hbox\bgroup            % Start \hbox of tree and lines
  \vbox to \treeheight\bgroup           % Start \vbox for vertical rules.
    \ifbotsub \iftopsub \vfil           % If both bottom and top subtree
        \hrule width 0.4pt              % ..vertical rule is just a dot.
      \else \treehalfrule \fi \vfil     % Bottom gets half-height rule.
    \else \iftopsub \vfil \treehalfrule % Top gets half-height the other way.
      \else \hrule width 0.4pt height \treeheight \fi\fi % Middle, full height.
    \egroup                             % Finish vertical rule \vbox.
  \treectrbox{\hrule width 1em}\hskip 0.2em\treectrbox{\box1}\egroup}
\endvb

The last line of the definition of *\subtreebox* calls *\treectrbox*
twice: once for the horizontal connecting rule, and once for the
subtree box itself.  This macro centers its argument in a *\vbox* the
height of this subtree and surrounding space.  We also define here
*\treehalfrule*, the macro called to make an *\hrule* half the height
of the subtree (with half the height of the horizontal connection
added to make the corners come out square).

\vb
\def\treectrbox#1{\vbox to \treeheight{\vfil #1\vfil}}
\def\treehalfrule{\dimen0=\treeheight   % Get total height.
  \divide\dimen0 2\advance\dimen0 0.2pt % Divide by two, add half horiz height.
  \hrule width 0.4pt height \dimen0}    % Make a vertical rule that high.
\endvb

That completes *\makesubtree*.  If this subtree has no sub-subtrees
under it, *\maketree* will now run *\makeleaf*; this merely adds the
tree text to the *\hbox* opened in *\maketree*.  Otherwise we call
*\makeparent* to attach the sub-subtrees and connecting rules to the
text at the root of the subtree.

\vb
\def\makeleaf{\box\treebox}             % Add leaf box to horiz list.
\def\makeparent{\ifdim\ht\treebox>\ht0  % If text is higher than subtrees
    \treeheight=\ht\treebox             % ..use that height.
  \else \treeheight=\ht0 \fi            % Otherwise use height of subtrees.
  \advance\treewidth-\wd\treebox        % Take remainder of level width
  \advance\treewidth 1em                % ..after accounting for text and glue.
  \treectrbox{\box\treebox}\hskip 0.2em % Add text, space before connection.
  \treectrbox{\hrule width \treewidth}\treectrbox{\box0}} % Add \hrule, subs.
\endvb

%	restore * to type ordinary
\catcode`\*=12

%\endinput
\bye
