%%@texfile{%
%% filename="chap2.tex",
%% version="1.1",
%% date="21-JUN-1991",
%% filetype="AMS-LaTeX: documentation",
%% copyright="Copyright (C) American Mathematical Society, all rights
%%   reserved.  Copying of this file is authorized only if either:
%%   (1) you make absolutely no changes to your copy, including name;
%%   OR (2) if you do make changes, you first rename it to some other
%%   name.",
%% author="American Mathematical Society",
%% address="American Mathematical Society,
%%   Technical Support Department,
%%   P. O. Box 6248,
%%   Providence, RI 02940,
%%   USA",
%% telephone="401-455-4080 or (in the USA) 800-321-4AMS",
%% email="Internet: Tech-Support@Math.AMS.org",
%% checksumtype="line count",
%% checksum="89",
%% codetable="ISO/ASCII",
%% keywords="latex, amslatex, ams-latex",
%% abstract="This file is part of the AMS-\LaTeX{} package, version 1.1."
%%   It is part of the monograph sample, testbook.tex (q.v.)."
%%}
%%% end of file header
%
\chapter{Keldysh Pencils}
\section{Holomorphic operator-valued functions}

The main object of study in this book is polynomial operator pencils
(operator polynomials). However, it is more convenient for us to give
certain definitions and results for the more general case of
holomorphic operator-valued functions.

Let $U$ be a domain in $\bold C$, $\cal{B}$ be a complex Banach space,
and $f(\lambda)$ a $\cal{B}$-valued function defined in $U$. Such a
function $f(\lambda)$ is said to be {\it strongly \rom(weakly\rom)
holomorphic} in $U$ if for any $\lambda_0\in U$ the strong (weak)
limit
\begin{equation}
\lim_{\lambda\to\lambda_0}\frac{f(\lambda)-f(\lambda_0)}{\lambda-\lambda_0}\
(=f'(\lambda_0))
\end{equation}
exists.

Obviously,  $f(\lambda)$ is weakly holomorphic if and only if any function
$\psi(f(\lambda))$ is holomorphic, where $\psi\in\cal{B}^*$.

An operator-valued function $A(\lambda)$ $(\lambda\in U)$ with values in
$L(\cal H)$ is said to be {\it uniformly \rom(strongly, weakly\rom) holomorphic}
 if for any $\lambda_0\in U$ the uniform (strong, weak) limit
\begin{equation}
\lim_{\lambda\to\lambda_0}\frac{A(\lambda)-A(\lambda_0)}{\lambda-\lambda_0}
\ (=A'(\lambda_0))
\end{equation}
exists.

Test of citations: 
\cite{dihe:newdir},
\cite{fre:cichon},
\cite{gouja:lagrmeth},
\cite{hapa:graphenum},
\cite{imlelu:oneway},
\cite{komiyo:unipfunc},
\cite{komiyo:lincomp},
\cite{liuchow:formalsum},
\cite{mami:matrixth},
\cite{miyoki:lincomp},
\cite{moad:quadpro},
\cite{ste:sint},
\cite{ye:intalg}.

%% \CharacterTable
%%  {Upper-case    \A\B\C\D\E\F\G\H\I\J\K\L\M\N\O\P\Q\R\S\T\U\V\W\X\Y\Z
%%   Lower-case    \a\b\c\d\e\f\g\h\i\j\k\l\m\n\o\p\q\r\s\t\u\v\w\x\y\z
%%   Digits        \0\1\2\3\4\5\6\7\8\9
%%   Exclamation   \!     Double quote  \"     Hash (number) \#
%%   Dollar        \$     Percent       \%     Ampersand     \&
%%   Acute accent  \'     Left paren    \(     Right paren   \)
%%   Asterisk      \*     Plus          \+     Comma         \,
%%   Minus         \-     Point         \.     Solidus       \/
%%   Colon         \:     Semicolon     \;     Less than     \<
%%   Equals        \=     Greater than  \>     Question mark \?
%%   Commercial at \@     Left bracket  \[     Backslash     \\
%%   Right bracket \]     Circumflex    \^     Underscore    \_
%%   Grave accent  \`     Left brace    \{     Vertical bar  \|
%%   Right brace   \}     Tilde         \~}
\endinput
