% xeplain.tex: macros for nonformatting.  Written 1989--94 by (mostly)
% Karl Berry.  These macros are in the public domain.
% 
% This is the ``extended plain'' TeX format that's described in
% `eplain.texinfo', which you should have received with this file.  We
% assume plain has been loaded.
%
% N.B.: A version number is defined at the beginning and end of this file;
% please change those numbers whenever the file is modified!
% And don't modify the file under any circumstances; rename it first.
%
% Some macros were written and/or suggested by Paul Abrahams.
% Other sources (e.g., The TeXbook) are cited at the appropriate places.
% 
%% @texfile{
%%   author = "Karl Berry",
%%   version = "REPLACE-WITH-VERSION",
%%   date = "REPLACE-WITH-DATE",
%%   filename = "xeplain.tex",
%%   email = "karl@cs.umb.edu",
%%   address = "135 Center Hill Rd. // Plymouth, MA 02360"
%%   checksum = "REPLACE-WITH-CHECKSUM",
%%   codetable = "ISO/ASCII",
%%   supported = "yes",
%%   docstring = "This file defines macros that extend and expand on
%%                plain TeX. eplain.tex is xeplain.tex and the other
%%                source files with comments stripped; see the original
%%                files for author credits, etc.  And please base diffs
%%                or other contributions on xeplain.tex, not the
%%                stripped-down eplain.tex.",
%% }
% 
% 
% Category codes, etc.
%
\def\makeactive#1{\catcode`#1 = \active \ignorespaces}%
\chardef\letter = 11
\chardef\other = 12
% 
% 
% So we can have user-inaccessible control sequences.
% 
\edef\leftdisplays{\the\catcode`@}%
\catcode`@ = \letter
\let\@eplainoldatcode = \leftdisplays
%
% Save miniscule amounts of memory and time by writing \toks@ii instead
% of \toks2.
\toksdef\toks@ii = 2
% 
% 
% This macro is defined in The TeXbook, but it never made it
% into plain TeX.  \dospecials is defined there, though.
% 
\def\uncatcodespecials{%
   \def\do##1{\catcode`##1 = \other}%
   \dospecials
}%
%
% 
% Here is a way to do \let^^M = \cs, where the \let need not be global.
{%
   \makeactive\^^M %
   \long\gdef\letreturn#1{\let^^M = #1}%
}%
% 
% 
% Swallow parameters, etc.
% 
\let\@eattoken = \relax  % Define this, so \eattoken can be used in \edef.
\def\eattoken{\let\@eattoken = }%
\def\gobble#1{}%
\def\gobbletwo#1#2{}%
\def\gobblethree#1#2#3{}%
%
% We can't just use \empty as the identity function, since then outer
% braces which would supposedly delimit the argument would define a group.
\def\identity#1{#1}%
% 
% True if #1 is the empty string, i.e., called like `\ifempty{}'.
\def\ifempty#1{\@@ifempty #1\@emptymarkA\@emptymarkB}%
\def\@@ifempty#1#2\@emptymarkB{\ifx #1\@emptymarkA}%
%
% Turn a definition into the characters that compose it.  See
% ``Sanitizing control sequences under \write'', by Ron Whitney, TUGboat
% 11(4), p.620.
\def\@gobblemeaning#1:->{}%
\def\sanitize{\expandafter\@gobblemeaning\meaning}%
%
% 
% From p.308 of the TeXbook.  This cannot be used in places where TeX
% might be skipping tokens, e.g., in conditionals.
% 
\def\ifundefined#1{\expandafter\ifx\csname#1\endcsname\relax}%
% 
% 
% \csname constructions come up an awful lot, so we save typing with the
% following.  (But the extra macro expansion does take time, so we don't
% use these in frequently-executed code.)
% 
\def\csn#1{\csname#1\endcsname}%
\def\ece#1#2{\expandafter#1\csname#2\endcsname}%
%
% 
% \expandonce{TOKEN} abbreviates \expandafter\noexpand TOKEN.
% 
\def\expandonce{\expandafter\noexpand}%
%
%
% Don't show our register allocations in the log.
% 
\let\@plainwlog = \wlog
\let\wlog = \gobble
%
% 
% Make it convenient to put newlines in error messages.
% 
\newlinechar = `^^J
% 
%
% Sometimes it is convenient to have everything in the transcript file
% and nothing on the terminal.  We don't just call \tracingall here,
% since that produces some useless output on the terminal.
%
\def\gloggingall{\begingroup \globaldefs = 1 \loggingall \endgroup}%
\def\loggingall{\tracingcommands\tw@\tracingstats\tw@
   \tracingpages\@ne\tracingoutput\@ne\tracinglostchars\@ne
   \tracingmacros\tw@\tracingparagraphs\@ne\tracingrestores\@ne
   \showboxbreadth\maxdimen\showboxdepth\maxdimen
}%
% Show the complete contents of boxes.
%
\def\tracingboxes{\showboxbreadth = \maxdimen \showboxdepth = \maxdimen}%
%
% Don't trace anything, except restore \showbox... to plain's values.
% 
\def\gtracingoff{\begingroup \globaldefs = 1 \tracingoff \endgroup}%
\def\tracingoff{\tracingonline\z@\tracingcommands\z@\tracingstats\z@
  \tracingpages\z@\tracingoutput\z@\tracinglostchars\z@
  \tracingmacros\z@\tracingparagraphs\z@\tracingrestores\z@
  \showboxbreadth5 \showboxdepth3
}%
%
% 
% Definitions to produce actual `{' (et al.) characters in an output
% file via \write.  We omit the line break after the first }, since we
% have no comment character at that point.
% 
\begingroup
  \catcode`\{ = 12 \catcode`\} = 12
  \catcode`\[ = 1 \catcode`\] = 2
  \gdef\lbracechar[{]%
  \gdef\rbracechar[}]%
  \catcode`\% = \other
  \gdef\percentchar[%]\endgroup
%
% 
% In order to do anything with ^^L inside a macro, it must
% be made non-\outer.
% 
\def^^L{\par}%
%
% 
% Leave horizontal mode (if we're in it), then insert a penalty.
% And conversely.
% 
\def\vpenalty{\ifhmode\par\fi \penalty}%
\def\hpenalty{\ifvmode\leavevmode\fi \penalty}%
%
% 
% Make \else usable in \loop.  From Victor Eijkhout's TeX by Topic (page
% 104).  See also Alois Kabelschacht, TUGboat 8(2), page 184.
%
\def\iterate{%
  \let\next\relax
  \body
  \let\next\iterate
  \fi
  \next
}%
% 
% 
% Add #2 (which is expanded in an \edef) to the end of the definition of
% #1 (which must be a previously-defined control sequence).  This is a
% way to construct simple lists.
% 
\def\edefappend#1#2{%
  \toks@ = \expandafter{#1}%
  \edef#1{\the\toks@ #2}%
}%
%
% 
% 
%  Hooks.
%
% \hookaction{HOOK}{TOKENS} adds TOKENS to the list of actions for
% HOOK.  We avoid defining a \toks register for each hook, although
% maybe that isn't so important.
% 
% \hookappend and \hookprepend add TOKENS specificially to the end or
% the beginning.  When the argument is used, \toks@ will be the previous
% value of the hook, and \toks@ii the new tokens.
%
\long\def\hookprepend{\@hookassign{\the\toks@ii \the\toks@}}%
\long\def\hookappend{\@hookassign{\the\toks@ \the\toks@ii}}%
\let\hookaction = \hookappend % either one should be ok
%
% 
% \@hookassign{LAST-DEF}{HOOK}{TOKENS} makes \toks@ the previous value
% of HOOK, and \toks@ii TOKENS, and then assigns the new value using
% LASTDEF.  We store the hook in a control sequence \@HOOKhook.
% 
\long\def\@hookassign#1#2#3{%
  % Make \toks@ be the expansion (to one level) of \@HOOKhook, or empty.
  \expandafter\ifx\csname @#2hook\endcsname \relax
    % If \@HOOKhook was undefined, let it be empty.
    \toks@ = {}%
  \else
    % Otherwise, expand it to one level.  We can't just assign from
    % \expandafter{\csname ...} since then the \toks register would
    % contain the control sequence, not its definition.
    \expandafter\let\expandafter\temp \csname @#2hook\endcsname
    \toks@ = \expandafter{\temp}%
  \fi
  \toks2 = {#3}% Don't expand the argument all the way.
  \ece\edef{@#2hook}{#1}%
}%
%
% 
% \hookactiononce{HOOK}\CS adds `\global\let\CS=\relax' to the
% definition of \CS, then adds to HOOK.  Thus, \CS is expanded the next
% time HOOK is called, but then it goes away.  This only works if \CS is
% expandable, though.
% 
\long\def\hookactiononce#1#2{%
  \edefappend#2{\global\let\noexpand#2\relax}
  \hookaction{#1}#2%
}%
%
% 
% \hookrun{HOOKNAME} runs whatever actions have been defined for HOOK.
% 
\def\hookrun#1{%
  \expandafter\ifx\csname @#1hook\endcsname \relax \else
    % Isn't this fun?  We want to get rid of the \fi before expanding
    % the actions, so that they can read what's coming up next.
    \def\temp{\csname @#1hook\endcsname}%
    \expandafter\temp
  \fi
}%
%
% 
% 
%  Properties a la Lisp.
%
% \setproperty{ATOM}{PROPNAME}{VALUE} defines the property PROPNAME on the
% ``atom'' ATOM to have VALUE.
% 
\def\setproperty#1#2#3{\ece\edef{#1@p#2}{#3}}%
%
% 
% \getproperty{ATOM}{PROPNAME} expands to the value of the property
% PROPNAME on ATOM, or to nothing (i.e., \empty), if the property isn't
% present.
% 
\def\getproperty#1#2{%
  \expandafter\ifx\csname#1@p#2\endcsname\relax
  % then \empty
  \else \csname#1@p#2\endcsname
  \fi
}%
%
% 
% 
%  Macros to support BibTeX are in a separate file, btxmac.tex.
% 
% (They are maintained separately, too, by Oren Patashnik,
% opbibtex@cs.stanford.edu.)  btxmac.tex also defines other macros we
% want to use and make available.
% 
% But not all people want to read the BibTeX macros, because of either
% space or time considerations.  Therefore, we look for \nobibtex,
% which, if defined, causes btxmac.tex not to be read.  But we still
% have to get \tokstostring et al. defined---so eplain.tex contains
% those definitions, automatically edited in from btxmac.tex.  All the
% documentation has been removed, so you must read btxmac.tex if you
% want the comments.
% 
% 
% We want to give a slightly different message than btxmac if no .aux
% file exists (unless the person using us has already define some
% message, possibly empty.)
% 
\ifx\@undefinedmessage\@undefined
  \def\@undefinedmessage
    {No .aux file; I won't warn you about undefined labels.}%
\fi
%
% 
% We use a token register to define all the BibTeX definitions, to avoid
% problems with the \if... constructions when they are conditionally
% read.
% 
%% [[[here is the first set of common definitions from btxmac]]]
\toks0 = {%
  %% [[[here are the BibTeX-specific definitions from btxmac]]]
}%
\ifx\nobibtex\@undefined \the\toks0 \fi
%% [[[here is the second set of common definitions from btxmac]]]
%
% Here are the control sequences that btxmac.tex defines using an @,
% because btxmac.tex wants to absolutely minimize the chance of
% conflicts.  But these control sequence implement documented features
% of eplain, so we want to allow people to use them without the @.
% 
\let\auxfile = \@auxfile
\let\for = \@for
\let\futurenonspacelet = \@futurenonspacelet
\def\iffileexists{\if@fileexists}%
\let\innerdef = \@innerdef
\let\innernewcount = \@innernewcount
\let\innernewdimen = \@innernewdimen
\let\innernewif = \@innernewif
\let\innernewwrite = \@innernewwrite
\let\linenumber = \@linenumber
\let\readauxfile = \@readauxfile
\let\spacesub = \@spacesub
\let\testfileexistence = \@testfileexistence
\let\writeaux = \@writeaux
% 
%
% btxmac.tex defines \innerdef.  Let's use it to make an abbreviation
% for \innerdef\inner<name>{<name>}.
% 
\def\innerinnerdef#1{\expandafter\innerdef\csname inner#1\endcsname{#1}}%
%
% Use that in turn to make non-outer versions of the rest of plain TeX's
% allocation macros.  (btxmac.tex already did a few of them.)
% 
\innerinnerdef{newbox}%
\innerinnerdef{newfam}%
\innerinnerdef{newhelp}%
\innerinnerdef{newinsert}%
\innerinnerdef{newlanguage}%
\innerinnerdef{newmuskip}%
\innerinnerdef{newread}%
\innerinnerdef{newskip}%
\innerinnerdef{newtoks}%
% 
% 
% Besides doing a \write to the aux file, we also need to do an
% \immediate\write.
% 
\def\immediatewriteaux#1{%
  \ifx\noauxfile\@undefined
    \immediate\write\@auxfile{#1}%
  \fi
}%
% 
% 
% 
%  Macros that produce output.
% 
% \obeywhitespace makes both end-of-lines and spaces in the input be
% respected in the output.  Even spaces at the beginning of lines turn
% into blank space the size of the natural space of the current font.
% The reason why plain TeX's \obeyspaces does not do this last is that
% it produces actual space characters, i.e., glue, and glue is discarded
% at a(n output) line break, and so if line breaks in the input are
% line breaks in the output...
% 
% Tabs are not affected; they will still produce glue (a single space).
%
\begingroup
   \makeactive\^^M \makeactive\ % No spaces or ^^M's from here on.
\gdef\obeywhitespace{%
\makeactive\^^M\def^^M{\par\futurelet\next\@finishobeyedreturn}%
\makeactive\ \let =\ %
%
% The group we use here is the one \obeywhitespace must be enclosed in.
% If we don't do this, then if the obeyed stuff ends in a newline, the
% indent produced by the definition of ^^M will make that line indented,
% even if it isn't the end of the paragraph.
\aftergroup\@removebox%
\futurelet\next\@finishobeywhitespace%
}%
%
% \@finishobeywhitespace eats any spaces and/or the end-of-line after
% the \obeywhitespace command itself.  The group here is the one that it
% itself creates.
%
\gdef\@finishobeywhitespace{{%
\ifx\next %
\aftergroup\@obeywhitespaceloop%
\else\ifx\next^^M%
\aftergroup\gobble%
\fi\fi}}%
%
% \@finishobeyedreturn is invoked at the end of every input line.  We
% check if the next thing is also a return, and, if so, insert extra
% space.  Then we start the next line.
% 
\gdef\@finishobeyedreturn{%
\ifx\next^^M\vskip\blanklineskipamount\fi%
\indent%
}%
\endgroup
%
% The argument here is the space that we are supposed to eat after the
% \obeywhitespace command.
% 
\def\@obeywhitespaceloop#1{\futurelet\next\@finishobeywhitespace}%
% 
% 
% This removes the last box, if it was a empty box of width \parindent.
% We might have been called inside a \vbox, so we have to test if we are
% in horizontal mode before using \lastbox.
% 
\def\@removebox{%
  \ifhmode
    \setbox0 = \lastbox
    \ifdim\wd0=\parindent
      \setbox2 = \hbox{\unhbox0}%
      \ifdim\wd2=0pt
        % Don't put it back: it was an indentation box.
        % This \ignorespaces ignores spaces after the group.
        \ignorespaces
      \else
        \box2 % Put it back: it wasn't empty.
      \fi
    \else
       \box0 % Put it back: it wasn't the right width.
    \fi
  \fi
}%
%
% We allow for extra (possibly negative) space when we hit blank lines.
% 
\newskip\blanklineskipamount
\blanklineskipamount = 0pt
%
% 
% A good way to print fractions in text when you don't want
% to use \over (which should be most of the time), and yet
% just `1/2' doesn't look right.  (From the TeXbook,
% the answer to exercise 11.6, p.311.)
% 
\def\frac#1/#2{\leavevmode
   \kern.1em \raise .5ex \hbox{\the\scriptfont0 #1}%
   \kern-.1em $/$%
   \kern-.15em \lower .25ex \hbox{\the\scriptfont0 #2}%
}%
%
% 
% The `e' just means `Eplain', as in `Eplain's hrule'.  The advantage
% to using these is that you can change the default thickness.
% 
\newdimen\hruledefaultheight  \hruledefaultheight = 0.4pt
\newdimen\hruledefaultdepth   \hruledefaultdepth = 0.0pt
\newdimen\vruledefaultwidth   \vruledefaultwidth = 0.4pt
%
\def\ehrule{\hrule height\hruledefaultheight depth\hruledefaultdepth}%
\def\evrule{\vrule width\vruledefaultwidth}%
%
%
% The texnames.sty and path.sty files included below were originally
% written by Nelson Beebe and Philip Taylor, respectively. See the
% complete source files (e.g., in this distribution) for comments.
%% [[[include texnames.sty]]]
%% [[[include path.sty]]]
%
% 
% A square box, suitable for being a marker in lists.
% 
\def\blackbox{\vrule height .8ex width .6ex depth -.2ex \relax}% square bullet
%
% 
% From p.311 of the TeXbook.
% 
% Make an unfilled rectangle with the dimensions of \box0.  #1 is the
% height of the rules, #2 the depth (i.e., the thicknesses).
% 
\def\makeblankbox#1#2{%
  \ifvoid0
    \errhelp = \@makeblankboxhelp
    \errmessage{Box 0 is void}%
  \fi
  \hbox{\lower\dp0
    \vbox{\hidehrule{#1}{#2}%
      \kern -#1% overlap rules
      \hbox to \wd0{\hidevrule{#1}{#2}%
        \raise\ht0\vbox to #1{}% vrule height
        \lower\dp0\vtop to #1{}% vrule depth
        \hfil\hidevrule{#2}{#1}%
      }%
      \kern-#1\hidehrule{#2}{#1}%
    }%
  }%
}%
%
\newhelp\@makeblankboxhelp{Assigning to the dimensions of a void^^J%
  box has no effect.  Do `\string\setbox0=\string\null' before you^^J%
  define its dimensions.}%
%
% 
% Produce an hrule with height #1 and depth #2, and insert kerning so it
% doesn't change the current position.
% 
\def\hidehrule#1#2{\kern-#1\hrule height#1 depth#2 \kern-#2}%
%
% Produce a vrule with width #1+#2, kerning so as not to change the
% current position.
% 
\def\hidevrule#1#2{%
  \kern-#1%
  \dimen0=#1\advance\dimen0 by #2%
  \vrule width\dimen0
  \kern-#2%
}%
% 
% 
% The \boxit macro from the TeXbook, trivially generalized to allow
% something other than 3pt around the TeX box being boxed.
% 
\newdimen\boxitspace \boxitspace = 3pt
%
\long\def\boxit#1{%
  \vbox{%
    \ehrule
    \hbox{%
      \evrule
      \kern\boxitspace
      \vbox{\kern\boxitspace \parindent = 0pt #1\kern\boxitspace}%
      \kern\boxitspace
      \evrule
    }%
    \ehrule
  }%
}%
%
% 
% Produce the written-out form of a number.
% 
\def\numbername#1{\ifcase#1%
   zero%
   \or one%
   \or two%
   \or three%
   \or four%
   \or five%
   \or six%
   \or seven%
   \or eight%
   \or nine%
   \or ten%
   \or #1%
   \fi
}%
%
% The following arrow macros were written by Steven Smith. See arrow.tex.
\let\@plainnewif = \newif
\let\@plainnewdimen = \newdimen
\let\newif = \innernewif
\let\newdimen = \innernewdimen
\edef\@eplainoldandcode{\the\catcode`& }%
\catcode`& = 11
\toks0 = {%
  %% [[[include arrow]]]
}%
\let\newif = \@plainnewif
\let\newdimen = \@plainnewdimen
\ifx\noarrow\@undefined \the\toks0 \fi
\catcode`& = \@eplainoldandcode
% 
% 
% 
%  Environments.
% 
% Define an ``environment'': arbitrary text, enclosed by \begingroup and
% \endgroup.  But you get to label the group, so that if you forget an
% \environment or an \endenvironment, you will probably get an error
% message about it.
% 
% Since the environment names appear in \errmessage arguments, it's best
% to keep them to `letter' and `other' characters.  I suppose we could
% call \tokstostring to allow more general labels.
% 
% These macros improve slightly on the answer to exercise 5.7 in
% The TeXbook, by making some checks on \begingroup and \endgroup, as
% well as just making sure \environment and \endenvironment's match.
% 
%
\def\environment#1{%
   \ifx\@groupname\@undefined\else
      % This gets invoked if we have two \environments (and no matching
      % \endenvironment to the first) with an \endgroup in between.
      \errhelp = \@unnamedendgrouphelp
      \errmessage{`\@groupname' was not closed by \string\endenvironment}%
   \fi
   % Use \edef in case we are passed a macro that contains the name,
   % instead of the name.
   \edef\@groupname{#1}%
   \begingroup
      \let\@groupname = \@undefined
}%
%
\def\endenvironment#1{%
   \endgroup
   \edef\@thearg{#1}%
   \ifx\@groupname\@thearg
   \else
      \ifx\@groupname\@undefined
         % Unfortunately, one gets an `extra \endgroup' message before
         % seeing this.  But we have to restore \@groupname, so I see no
         % alternative. 
         \errhelp = \@isolatedendenvironmenthelp
         \errmessage{Isolated \string\endenvironment\space for `#1'}%
      \else
         \errhelp = \@mismatchedenvironmenthelp
         \errmessage{Environment `#1' ended, but `\@groupname' started}%
         \endgroup % Probably a typo in the names.
      \fi
   \fi
   \let\@groupname = \@undefined
}%
%
\newhelp\@unnamedendgrouphelp{Most likely, you just forgot an^^J%
   \string\endenvironment.  Maybe you should try inserting another^^J%
   \string\endgroup to recover.}%
%
\newhelp\@isolatedendenvironmenthelp{You ended an environment X, but^^J%
   no \string\environment{X} to start it is anywhere in sight.^^J%
   You might also be at an \string\endenvironment\space that would match^^J%
   a \string\begingroup, i.e., you forgot an \string\endgroup.}%
%
\newhelp\@mismatchedenvironmenthelp{You started an environment named X, but^^J%
   you ended one named Y.  Maybe you made a typo in one^^J%
   or the other of the names?}%
% 
% 
% The above sort of environment allows nesting.  But environments
% shouldn't always be allowed to nest (like the \flushright,
% \flushleft, and \center ones defined below).  Here are some macros to
% help deal with that.
% 
% \checkenvironmentnesting goes at the beginning of a macro that is
% going to define the environment.
% 
\newif\ifenvironment
\def\checkenv{\ifenvironment \errhelp = \@interwovenenvhelp
   \errmessage{Interwoven environments}%
   \egroup \fi
}%
%
\newhelp\@interwovenenvhelp{Perhaps you forgot to end the previous^^J%
   environment? I'm finishing off the current group,^^J%
   hoping that will fix it.}%
%
% 
% 
%  Mathematics displays.
%
% By default, TeX centers displayed material.  To get left-justified
% displays, say \leftdisplays.  To go back to centered displays, say
% \centereddisplays.
% 
% This is based on an approach developed by Donald Arseneau,
% asnd@triumfrg.bitnet.
% 
\newtoks\previouseverydisplay
\newdimen\leftdisplayindent
\newif\if@leftdisplays
%
\def\leftdisplays{%
  % IF we've already been called, do nothing.
  \if@leftdisplays\else
    \previouseverydisplay = \everydisplay
    \everydisplay = {\the\previouseverydisplay \leftdisplaysetup}%
    %
    % Since we don't want to start a group (the \leftdisplays might
    % well continue to the end of the document), we have to explicitly
    % save and restore everything we change.
    \let\@save@maybedisableeqno = \@maybedisableeqno
    \let\@saveeqno = \eqno
    \let\@saveleqno = \leqno
    \let\@saveeqalignno = \eqalignno
    \let\@saveleqalignno = \leqalignno
    \let\@maybedisableeqno = \relax
    \def\eqno{\hfill\textstyle\enspace}%
    \def\leqno{%
      \hfill
      \hbox to0pt\bgroup
        \kern-\displaywidth
        \kern-\displayindent
        $\aftergroup\@leftleqnoend
    }%
    \@redefinealignmentdisplays
    \@leftdisplaystrue
  \fi
}%
%
\def\centereddisplays{%
  % If \leftdisplays hasn't been called, don't try to restore all the
  % stuff it changes.
  % 
  \if@leftdisplays
    \everydisplay = \previouseverydisplay
    \let\@maybedisableeqno = \@save@maybedisableeqno
    \let\eqno = \@saveeqno
    \let\leqno = \@saveleqno
    \let\eqalignno = \@saveeqalignno
    \let\leqalignno = \@saveleqalignno
    \@leftdisplaysfalse
  \fi
}%
%
\def\leftdisplaysetup{%
  \hbox to\displaywidth\bgroup
    \strut
    \dimen@ = \parindent
      \advance\dimen@ by \leftdisplayindent 
      \advance\dimen@ by \leftskip
    \hskip\dimen@
    % We can't use \displayindent all along, because then we would
    % \hskip by the above plus \hangindent or \parshape indentation.
    % See p.188.
    \advance\displayindent by \dimen@
    \advance\displaywidth by -\parindent
      \advance\displaywidth by -\leftdisplayindent
      \advance\displaywidth by -\leftskip
    $%
    %
    % If we do a \noalign inside an \eqalignno, we will be in horizontal
    % mode, and the line will be as wide as \hsize.  This will cause an
    % overfull alignment, since we also have to fit the indentation in,
    % before the alignment.
    \advance\hsize by -\displayindent
    %
    \aftergroup\@lefteqend
    \displaystyle
}%
%
\def\@lefteqend{\hfil\egroup$}% gets inserted between the ending $$
\def\@leftleqnoend{\hss \egroup$}%
%
% Since the equation is typeset in math mode and not display math mode,
% \halign is illegal.  So we use \vcenter, a la \eqalign, to make the
% \halign legal.  Also, we remove the \hfil from the left of the
% template, to make the display come out flush left.
%
\def\@redefinealignmentdisplays{%
  \def\displaylines##1{\displ@y
    \vcenter{%
      % We add another \hfil outside math mode in this template to cure
      % underfull hboxes that sometimes result (from big fractions as in
      % exercise 19.19).  I'm not sure why DEK does \halign{\hbox to...}
      % instead of just \halign to... here.
      \halign{\hbox to\displaywidth{$\@lign\displaystyle####\hfil$\hfil}\crcr
              ##1\crcr}}}%
  %
  \def\eqalignno##1{\displ@y
    \let\noalign = \@lefteqalignonoalign
    \vcenter{%
      \halign to\displaywidth{%
         $\@lign\displaystyle{####}$\tabskip\z@skip
        &$\@lign\displaystyle{{}####}$\hfil\tabskip\centering
        &\llap{$\@lign####$}\tabskip\z@skip\crcr
        ##1\crcr}}}%
  %
  \def\leqalignno##1{\displ@y
    % Don't use the \eqno in \eqdef.
    \let\eqno = \relax
    \vcenter{%
      \halign to\displaywidth{%
         \hfil$\@lign\displaystyle{####}$\tabskip\z@skip
        &$\@lign\displaystyle{{}####}$\hfil\tabskip\centering
        &\kern-\displaywidth
         \rlap{\kern-\displayindent $\@lign####$}%
         \tabskip\displaywidth\crcr
        ##1\crcr}}}%
}%
% 
% \noalign is typically used to insert a few words (`and', for example)
% between two aligned equations.  So I don't think the \noaligned
% material should be indented.  Since \noalign takes <vertical mode
% material>, we would end up with double indentation, anyway: one
% because we're indenting the whole display, and one at the start of the
% <v.m.m.>.  (If you want to change any of this, you can put something in
% \@everynoalign.)  So, we use this definition for \noalign in a
% left-justified \eqalignno:
% 
\let\@primitivenoalign = \noalign
\newtoks\@everynoalign
\def\@lefteqalignonoalign#1{%
  \@primitivenoalign{%
    % Is it right to set \leftskip=0pt first, thus perhaps making this
    % work in lists and so forth?  We just compensate for the other ways
    % the display is indented here.
    \advance\leftskip by -\parindent
    \advance\leftskip by -\leftdisplayindent
    \parskip = 0pt
    %
    % We use \parindent=0pt instead of \noindent because the latter
    % starts unrestricted horizontal mode, which means the alignment
    % we're inside will wind up being as wide as the page.  When the arg
    % is just vertical material, this is wrong.  For example, using
    % \matrix inside \eqalignno fails if \noindent is used.
    \parindent = 0pt
    \the\@everynoalign
    #1%
  }%
}%
% 
% 
% 
%  Time macros.
% 
% TeX sets \time, \day, \month, and \year when it begins.  (And does not
% update them as it runs!)
% 
% 
% \monthname produces the name of the month, abbreviated to three
% letters.  The primitive \month should never be zero.
% 
\def\monthname{%
   \ifcase\month
      \or Jan\or Feb\or Mar\or Apr\or May\or Jun%
      \or Jul\or Aug\or Sep\or Oct\or Nov\or Dec%
   \fi
}%
%
% \fullmonthname is like \monthname, except it doesn't abbreviate.
% 
\def\fullmonthname{%
   \ifcase\month
      \or January\or February\or March\or April\or May\or June%
      \or July\or August\or September\or October\or November\or December%
   \fi
}%
%
% \timestring produces the current time, in a format like `1:14 p.m.'.
% 
\def\timestring{\begingroup
   \count0 = \time
   \divide\count0 by 60
   \count2 = \count0   % The hour, from zero to 23.
   %
   \count4 = \time
   \multiply\count0 by 60
   \advance\count4 by -\count0   % The minute, from zero to 59.
   % But we need the minutes with a leading zero, if necessary.
   \ifnum\count4<10
      \toks1 = {0}%
   \else
      \toks1 = {}%
   \fi
   % 
   % Convert the hour into `a.m.' or `p.m.', and make it mod 12.
   \ifnum\count2<12
      \toks0 = {a.m.}%
   \else
      \toks0 = {p.m.}%
      \advance\count2 by -12
   \fi
   %
   % If it's midnight, call it `12', not `0'.
   \ifnum\count2=0
      \count2 = 12
   \fi
   %
   % Produce the output.
   \number\count2:\the\toks1 \number\count4 \thinspace \the\toks0
\endgroup}%
%
%
% \timestamp produces a text string for the whole thing like
% `23 Apr 1964  1:14 p.m.'.
%
\def\timestamp{\number\day\space\monthname\space\number\year\quad\timestring}%
%
% 
% \today produces the current date, as in `23 April 1964'.
% 
\def\today{\the\day\ \fullmonthname\ \the\year}%
%
% 
% 
%  (Typographical) lists.
% 
% These macros can produce numbered or unnumbered lists.
% 
% You can change the spacing by assigning new values to these registers.
% They are used by both kinds of lists.  \listleftindent is relative to
% the current paragraph indentation, while \listrightindent is an
% absolute value.  I do this for two reasons: (1) it is more useful, if not
% more ``logical'', to make list indentation depend on the paragraph
% indentation; (2) footnotes do not work if \parindent is zero, and
% having a footnote in a list item is perfectly reasonable.
% 
% If you change \baselineskip and want \abovelistskip and \belowlistskip
% to retain their meanings here, you will have to reassign to them.  The
% \baselineskip here is the value at the time eplain.tex is read, i.e.,
% 12pt (most likely).
% 
% If the items in your lists are very long, you might want to
% make \interitemskipamount nonzero.  
% 
\newskip\abovelistskipamount      \abovelistskipamount = .5\baselineskip
  \newcount\abovelistpenalty      \abovelistpenalty    = 10000
  \def\abovelistskip{\vpenalty\abovelistpenalty \vskip\abovelistskipamount}%
\newskip\interitemskipamount      \interitemskipamount = 0pt
  \newcount\belowlistpenalty      \belowlistpenalty    = -50
  \def\belowlistskip{\vpenalty\belowlistpenalty \vskip\belowlistskipamount}%
\newskip\belowlistskipamount      \belowlistskipamount = .5\baselineskip
  \newcount\interitempenalty      \interitempenalty    = 0
  \def\interitemskip{\vpenalty\interitempenalty \vskip\interitemskipamount}%
\newdimen\listleftindent    \listleftindent = 0pt
\newdimen\listrightindent   \listrightindent = 0pt        
\let\listmarkerspace = \enspace
%
% To do arbitrary things at the start of each list:
\newtoks\everylist
%
% If you want no space between items for a particular list
% (perhaps because the items in it are short), you can say,
% e.g., \numberedlist\listcompact.
% 
\def\listcompact{\interitemskipamount = 0pt \relax}%
%
% This is called to set up the parameters by both sorts of lists.
% Because we set \rightskip, we finish off the current paragraph.
% 
\newdimen\@listindent
%
\def\beginlist{%
  % Insert the space above this list, before we change \leftskip
  % (because the \vskip in here might be what ends the paragraph).
  \abovelistskip
  %
  \@listindent = \parindent
  \advance\@listindent by \listleftindent
  %
  % \leftskip shifts nested lists to the right on the page.
  \advance\leftskip by \@listindent
  \advance\rightskip by \listrightindent
  %
  % We always need \itemnumber, so we can know whether an item is the
  % first one or not.
  \itemnumber = 1
  %
  \the\everylist
}%
%
% A list item, for both kinds of lists.
% 
\def\li{\@getoptionalarg\@finli}%
\def\@finli{%
  \ifx\@optionalarg\empty \else
    \expandafter\writeitemxref\expandafter{\@optionalarg}%
  \fi
  \ifnum\itemnumber=1 \else \interitemskip \fi
  \printitem
  \advance\itemnumber by 1
  \advance\itemletter by 1
  \advance\itemromannumeral by 1
  %
  % Just in case somebody creeps in with an argument or something.
  \ignorespaces
}%
%
% \writeitemxref{LABEL} writes out a definition for LABEL to be \marker
% for the aux file.
% 
\def\writeitemxref#1{\definexref{#1}\marker{item}}%
%
% \printitem is used to print items by both sorts of lists.  A \par gets
% produced before every item -- even the first one.  We also want to
% make paragraphs after the first appear to be indented -- i.e., they
% will have double indentation.  It is usually bad exposition to have
% lists with multiparagraph items, but sometimes it is unavoidable.
% 
\def\printitem{%
  \par
  \vskip-\parskip
  \noindent
  \printmarker\marker
}%
%
% Output the list marker.
% 
\def\printmarker#1{\llap{\marker \enspace}}%
%
% Common ending.
% 
\def\endlist{\belowlistskip}%
%
% 
% \numberedlist produces items which are numbered sequentially, starting
% from one.  You start items with \li (`list item').  End the list with
% \endnumberedlist.
% 
% A nested \numberedlist produces items labelled `(a)', `(b)', etc.  A
% doubly (and deeper) nested \numberedlist labels items with `*'.
% 
% These registers keep track of where we are.
% 
\newcount\numberedlistdepth
\newcount\itemnumber
\newcount\itemletter
\newcount\itemromannumeral
%
\def\numberedmarker{%
  \ifcase\numberedlistdepth
      (impossible)%
  \or \printitemnumber
  \or \printitemletter
  \or \printitemromannumeral
  \else *%
  \fi
}%
%
% These produce the text of the labels.  We use \the\itemletter so that
% the value will expand.
% 
\def\printitemnumber{\number\itemnumber}%
\def\printitemletter{\char\the\itemletter}%
\def\printitemromannumeral{\romannumeral\itemromannumeral}%
\def\numberedprintmarker#1{\llap{#1) \listmarkerspace}}%
%
\def\numberedlist{\environment{@numbered-list}%
  % This is set back to zero by getting to the end of the group.
  \advance\numberedlistdepth by 1
  \itemletter = `a
  \itemromannumeral = 1
  \beginlist
  \let\marker = \numberedmarker
  \let\printmarker = \numberedprintmarker
}%
%
\def\endnumberedlist{%
  \par
  \endenvironment{@numbered-list}%
  \endlist
}%
%
% Allow synonyms for \numberedlist.
\let\orderedlist = \numberedlist
\let\endorderedlist = \endnumberedlist
% 
% 
% 
% \unorderedlist produces items which are labelled with bullets.  You
% start an item with \li, just as with numbered lists.  You end the list
% with \endunorderedlist.
% 
% A nested \unorderedlist produces items labelled with em-dashes.  A
% doubly (and deeper) nested \unorderedlist uses `*'.
% 
\newcount\unorderedlistdepth
%
\def\unorderedmarker{%
  \ifcase\unorderedlistdepth
      (impossible)%
  \or \blackbox
  \or ---%
  \else *%
  \fi
}%
\def\unorderedprintmarker#1{\llap{#1\listmarkerspace}}%
%
\def\unorderedlist{\environment{@unordered-list}%
  \advance\unorderedlistdepth by 1
  \beginlist
  \let\marker = \unorderedmarker
  \let\printmarker = \unorderedprintmarker
}%
%
\def\endunorderedlist{%
  \par
  \endenvironment{@unordered-list}%
  \endlist
}%
%
% 
% 
%  Verbatim listing.
% 
% ... well, almost verbatim.  We assume the font \tt has all the
% characters that will appear.  Control characters, except for tabs and
% form feeds (and returns) won't produce anything useful.  Tabs produce
% a fixed amount of space, and form feeds produce a page break.
% 
% This is based on Knuth's ideas in Appendix D of the TeXbook, p. 380.
% The argument should be a filename.
% 
% if you need to do something more for your particular fonts and/or
% environment before the file is input, give a definition to
% \setuplistinghook.  If you want line numbers on the output, you can
% say \let\setuplistinghook = \linenumberedlisting.
% 
\def\listing#1{%
   \par \begingroup
   \@setuplisting
   \setuplistinghook
   \input #1
   \endgroup
}%
%
\let\setuplistinghook = \relax
%
\def\linenumberedlisting{%
  \ifx\lineno\undefined \innernewcount\lineno \fi
  \lineno = 0
  \everypar = {\advance\lineno by 1 \printlistinglineno}%
}%
\def\printlistinglineno{\llap{[\the\lineno]\quad}}%
%
% 
% \uncatcodespecials must come before \obeywhitespace, lest a space
% character in the input produce character 32 from the \tt font.
% 
\def\listingfont{\tt}%
\def\@setuplisting{%
   \uncatcodespecials
   \obeywhitespace
   \makeactive\`
   \makeactive\^^I
   \def^^L{\vfill\eject}%
   \parskip = 0pt
   \listingfont
}%
%
% Give definitions to the characters we want to be special.
% 
% Do ` separately, so can use ` in the \catcode commands elsewhere.
% 
{%
   \makeactive\`
   \gdef`{\relax\lq}% Defeat ligatures.
}%
{%
   \makeactive\^^I
   \gdef^^I{\hskip8\fontdimen2\font \relax}%
}%
%
% 
% \verbatim ... |endverbatim typesets the ... in typewriter.  To produce a |
% in the ..., use ||.  This macro was contributed by beebe@math.utah.edu.
% Generalized to characters other than | by dorai@cs.rice.edu.
% 
\def\verbatimescapechar#1{%
  \gdef\@makeverbatimescapechar{%
    \@makeverbatimdoubleescape #1%
    \catcode`#1 = 0
  }%
}%
\def\@makeverbatimdoubleescape#1{%
  \catcode`#1 = \other
  \begingroup
    \lccode`\* = `#1%
    \lowercase{\endgroup \ece\def*{*}}%
}%
\verbatimescapechar\|  % initially escapechar is |
%
\def\verbatim{\begingroup
  \uncatcodespecials
  \obeywhitespace
  \makeactive\` % make space character a single space, not stretchable
  \@makeverbatimescapechar
  \tt}
\let\endverbatim = \endgroup
% 
% 
% 
%  Contents.
% 
% Entries for the table of contents are recorded in \jobname.toc, which
% we open for writing at the first \writetocentry or when \readtocfile
% is invoked (after we read it, in the latter case).  We use
% \tocfilebasename as the root of the filename to read; this defaults to
% \jobname.
% 
\newif\if@tocfileopened
\newwrite\tocfile
\let\tocfilebasename = \jobname
%
\def\opentocfile{%
  \if@tocfileopened\else
     \global\@tocfileopenedtrue
     \immediate\openout\tocfile = \tocfilebasename.toc
  \fi
}%
%
% \writetocentry#1#2 produces a line in the .toc file that
% looks like:
% \toc#1entry{#2}{page number}
% e.g., 
% \tocchapterentry{Introduction}{2}
% would be written by
% \writetocentry{chapter}{Introduction}
% if the chapter started on page two.
% 
% Thus, #1 is intended to be something like `chapter' or `section', #2
% to be the text of the title.
% 
% Of course, if you want, you can \write\tocfile yourself with whatever
% you like.  In that case, you must also call \opentocfile.
% 
% By the way, it would be wrong to put a \percentchar at the end of the
% output line.  Then, when the .toc file is read, if each line is turned
% into a \leftline, say, there would be no legal breakpoint between the
% boxes, and one extremely long line would result.
% 
\def\writetocentry#1#2{\writenumberedtocentry{#1}{#2}{}}%
%
% Sometimes you want the control sequence to take another number (e.g.,
% a chapter number) as a parameter.  (Although you can pass anything you
% want as the third parameter, naturally.)  The third parameter is
% expanded at the point of the \writenumberedtocentry, not when the
% \write actually happens.  This makes the usual case---the third
% parameter being \the\someregister---work.
% 
% For example:
% \writenumberedtocentry{chapter}{The second chapter}{2}
% would produce:
% \tocchapterentry{The second chapter}{2}{14}
% 
% if the second chapter started on page 14.
% 
\def\writenumberedtocentry#1#2#3{%
  \ifrewritetocfile
    \opentocfile
    \toks0 = {\expandafter\noexpand \csname toc#1entry\endcsname}%
    \def\temp{#2}%
    %
    % Usually #3 is just `\the\register', which we want to expand.  But
    % if it's not a number at all -- e.g., if it's an author's name, we
    % don't want to expand control sequences for accents and the like.
    % So we play some games here.
    \toks2 = \expandafter{#3}%
    \edef\cs{\the\toks2}%
    \edef\@wr{%
      \write\tocfile{%
        % The \toc...entry control sequence:
        \the\toks0
        %
        % The text:
        {\sanitize\temp}%
        %
        % A secondary number, or nothing:
        \ifx\empty\cs\else {\sanitize\cs}\fi
        %
        % The page number:
        {\noexpand\folio}%
      }%
    }%
    \@wr
  \fi
  \ignorespaces
}%
%
% 
% The entries are read in when the user invokes \readtocfile (which
% should be before the first \writetocentry).  The .toc file is also
% opened for writing (i.e., emptied) here, unless \rewritetocfilefalse.
% You might want to set that to, e.g., read it in twice.
% 
\newif\ifrewritetocfile   \rewritetocfiletrue
%
\def\readtocfile{%
   \testfileexistence[\tocfilebasename]{toc}%
   \if@fileexists
      \input \tocfilebasename.toc
      \ifrewritetocfile \opentocfile \fi
   \fi
}%
%
% 
% Here are some sample definitions of the \toc...entry macros.  Perhaps
% you or your book designer can come up with a better way of handling
% contents than leaders.  These definitions are just as examples, not as
% something you might want to actually use to print a document.
% 
\def\tocchapterentry#1#2{\line{\bf #1 \dotfill\ #2}}%
\def\tocsectionentry#1#2{\line{\quad\sl #1 \dotfill\ \rm #2}}%
\def\tocsubsectionentry#1#2{\line{\qquad\rm #1 \dotfill\ #2}}%
%
% 
% 
%  Cross-references.
% 
% Definitions of references are recorded in \jobname.aux, called
% \auxfile in the macros, which btxmac.tex has opened.
% 
% When a label isn't defined, we only want to complain if
% \xrefwarningtrue; btxmac uses \if@citewarning for this, so we have to
% reuse that name.  We can't just say \let\ifxrefwarning =
% \if@citewarning, since then changes to the latter won't be reflected
% in the former.  On the other hand, we have to have a true \if...
% command, so \if's and \fi's match properly.  What a mess.
%
\let\ifxrefwarning = \iftrue
\def\xrefwarningtrue{\@citewarningtrue \let\ifxrefwarning = \iftrue}%
\def\xrefwarningfalse{\@citewarningfalse \let\ifxrefwarning = \iffalse}%
%
% 
% \xref{foo} produces ``p.\thinspace <page#>''.  \xrefn{foo} produces
% ``<page#>''.  \xrdef{foo} produces nothing, but defines the label
% `foo' to be on the current page.
%
% As usual, it takes two passes to get the cross-references right.  I
% would like to check for labels being defined twice, but I don't know
% how to do that.  If the cross-reference file has been read in, many
% cross-references will be defined that I don't want to complain about.
% It is only if two \xrdef commands are given to the same string that I
% want to complain.  I could define a new control sequence for each
% cross-reference, and check that, but that seems like too high of a
% price to pay.
%
% 
% \xrlabel{LABEL} expands to a cross-reference internal name.  We append
% an _ character to NAME, to help avoid conflicts.  And we append an `x'
% so that we don't redefine \_ on an empty label.
% 
\begingroup
  % Mike Spivak's MathTime macros for Times Roman fonts changes the
  % catcode of _ to be active.  (From adam@symcom.math.uiuc.edu.)
  \catcode`\_ = 8
  \gdef\xrlabel#1{#1_x}%
\endgroup
%
%
% \xrdef{LABEL} defines LABEL to be the current page number.  But we
% don't define the label here, because the page number might be off: if
% this is not the first time through, the label would already be
% defined, and we would redefine it with the wrong information.
% 
\def\xrdef#1{\definexref{#1}{\noexpand\folio}{page}}%
%
% \definexref{LABEL}{DEFINITION}{CLASS} defines a cross-reference named
% LABEL of label class CLASS to be DEFINITION.  (Or LABEL can be a
% control sequence; it's expanded to get the label text.)  To get a
% possible page number right, we have to write the definition out to the
% auxiliary file, instead of only defining it directly.
% 
\def\definexref#1#2#3{%
  % Remember what we're given; it might be `\@optionalarg', which
  % \readauxfile trashes.  (No loss of generality here, since \csname
  % will fully expand the label anyway.)
  \edef\temp{#1}%
  %
  % Be sure we've read the aux file before we zap it:
  \readauxfile
  %
  % When we read in the aux file next time, define the label:
  \edef\@wr{\noexpand\writeaux{\string\@definelabel{\temp}{#2}{#3}}}%
  \@wr
  \ignorespaces
}%
%
% \@definelabel{LABEL}{DEFINITION}{CLASS} actually defines LABEL of
% label class CLASS to be DEFINITION.
% 
\def\@definelabel#1#2#3{%
  % Define the control sequence.
  \expandafter\gdef\csname\xrlabel{#1}\endcsname{#2}%
  %
  % Remember what kind of label this is, so \ref will know what to do.
  \global\setproperty{\xrlabel{#1}}{class}{#3}%
}%
%
% 
% Typeset a reference to the label #1.
% 
\def\xrefn#1{%
  \readauxfile
  %
  \expandafter \ifx\csname\xrlabel{#1}\endcsname\relax
    \if@citewarning
       \message{\linenumber Undefined label `#1'.}%
    \fi
    %
    % Give it a dummy definition, though, to stop multiple error messages.
    \expandafter\def\csname\xrlabel{#1}\endcsname{%
      `{\tt
        \escapechar = -1
        \expandafter\string\csname#1\endcsname
      }'%
    }%
  \fi
  \csname\xrlabel{#1}\endcsname % Always produce something.
}%
%
% \refn is just a synonym.
% 
\let\refn = \xrefn
%
% One common case: print `p. ' before the page number.
% 
\def\xref{p.\thinspace\xrefn}%
%
% \ref{LABEL} typesets \CLASSword for LABEL's class (if it's defined)
% and then does \refn on LABEL. But amstex also has a \ref, so tell the
% user if they try to use \ref and have loaded amsppt.sty.
% 
% \refs is similar, but puts the letter `s' after the \...word, thus
% producing (for example) `Figures 1.2' (presumably to be followed by
% `and~\refn{fig-1.3}').
% 
\def\@maybewarnref{%
  \ifundefined{amsppt.sty}%
    % No amsppt.sty, so just use ours.
  \else
    \message{Warning: amsppt.sty and Eplain both define \string\ref. See
             the Eplain manual.}%
    % Remember their definition.
    \let\amsref = \ref
  \fi
  \let\ref = \eplainref
  \ref
}
\let\ref = \@maybewarnref
%
\def\eplainref{\@generalref{}}%
\def\refs{\@generalref s}%
%
% #1 is the text to follow the \...word, and is supplied by the macros
% above.  #2 comes from the document, and is the LABEL.
% 
\def\@generalref#1#2{%
  \readauxfile
  %
  \edef\temp{\getproperty{\xrlabel{#2}}{class}}%
  %
  % If the word for this class is not defined, don't complain.
  \expandafter\ifx\csname \temp word\endcsname\relax \else
    % Produce the word.
    \csname \temp word\endcsname
    % Add the suffix and then put in a tie before the \refn.  Do not
    % rely on `~' being defined as a tie.
    #1\penalty\@M \ 
  \fi
  %
  \refn{#2}%
}%
%
%
% References to equations are similar.
%
% \eqref{foo} produces ``(<text for equation label foo>)''.
% \eqdefn{foo} advances \eqnumber, resets \eqsubnumber, and defines
%   `foo' to be the new number. 
% \eqsubdefn{foo} advances \eqsubnumber and defines `foo'.  \eqref works
%   for both equations and subequations,
% \eqdef{foo} does \eqdefn, then inserts an \eqno and \eqref.
% \eqsubdef{foo} does \eqsubdefn, then what \eqdef does.
% 
% The non-``sub'' macros also take an optional argument; if it's
% present, we use it as the text for the equation label, instead of the
% various counters.
%
% Because there are no page break issues with equations, we can
% immediately define the control sequence.  But we also need to write
% the definition out, in case the user wants to forward reference an
% equation (bad style as that may be).
%
% The current equation number is in \eqnumber; we just advance it by one
% for each \eqdef.  You can handle fancier equation numbers (e.g., ones
% that include a chapter number) by redefining \eqprint, below, and
% using your own counters.  We do provide for one level of substructure,
% since that's more painful to implement than superstructures.
%
\newcount\eqnumber
\newcount\subeqnumber
%
%
% \eqdefn[TEXT]{LABEL} defines LABEL to be TEXT (if it's present),
% otherwise it advances \eqnumber and defines LABEL to be that.  It
% doesn't produce anything.
% 
\def\eqdefn{\@getoptionalarg\@fineqdefn}%
\def\@fineqdefn#1{%
  \ifx\@optionalarg\empty
    \global\advance\eqnumber by 1
    % We call \eqconstruct here instead of in \@eqdefn because we don't
    % want to expand it for \eqsubdefn -- \eqsubdefn already includes an
    % \eqrefn which includes the text of the label which was \eqconstructed.
    \def\temp{\eqconstruct{\number\eqnumber}}%
  \else
    \def\temp{\noexpand\@optionalarg}%
  \fi
  %
  % Always reset the current subequation number:
  \global\subeqnumber = 0
  %
  % Remember this label, so that we can define subequations:
  \gdef\@currenteqlabel{#1}%
  \toks0 = \expandafter{\@currenteqlabel}%
  %
  % Actually do the definition, taking precautions not to expand \eqrefn
  % in what we output to the aux file.  \eqrefn expands to many things,
  % including \count@'s and \edef's and the expansion of \xrlabel, and
  % it's just a real mess.
  \begingroup
    \def\eqrefn{\noexpand\eqrefn}%
    \edef\temp{\noexpand\@eqdefn{\the\toks0}{\temp}}%
    \temp
  \endgroup
}%
% 
% 
% \eqsubdefn defines its argument as a ``subequation'' of the last \eqdef.
%
\def\eqsubdefn#1{%
  \global\advance\subeqnumber by 1
  \toks0 = {#1}%
  %
  % Get the text of the label;
  \toks2 = \expandafter{\@currenteqlabel}%
  %
  % We must expand \@currenteqlabel.  We have to not expand
  % \eqsubreftext here, as well \eqrefn, since the first arg to
  % \eqsubreftext could also include lots of complicated things.
  \begingroup
    \def\eqrefn{\noexpand\eqrefn}%
    \def\eqsubreftext{\noexpand\eqsubreftext}%
    \edef\temp{%
      \noexpand\@eqdefn
        {\the\toks0}%
        {\eqsubreftext{\eqrefn{\the\toks2}}{\the\subeqnumber}}%
    }%
    \temp           
  \endgroup
}%
% 
% \@eqdefn{LABEL}{REF-TEXT} actually handles the equation number
% definitions and writing to the aux file.
% 
% In contrast to \xrdef, we define LABEL right away (as REF-TEXT).  We
% can do this since we know right now what the right equation number is.
% This eliminates some unnecessary warning.  It also lets the user put
% \eqdef{} on all equations and have it work, since then \eqref
% refers to the just-defined new value.
% 
\def\@eqdefn#1#2{%
  \definexref{#1}{#2}{eq}%
  \@definelabel{#1}{#2}{eq}%
}%
% 
% \eqdef{LABEL} defines LABEL, with \eqdefn, then prints it.  We allow
% an optional argument to explicitly specify the text which we define
% the label as.
%
\def\eqdef{\@getoptionalarg\@fineqdef}%
\def\@fineqdef{%
  \toks0 = \expandafter{\@optionalarg}%
  \edef\temp{\noexpand\@eqdef{\noexpand\eqdefn[\the\toks0]}}%
  \temp
}%
%
% \eqsubdef is to \eqdef as \eqsubdefn is to \eqdefn.  No optional
% argument allowed here.
% 
\def\eqsubdef{\@eqdef\eqsubdefn}%
%
% \@eqdef{DEFN-CMD}{LABEL} defines LABEL, using DEFN-CMD.  Then it
% inserts an \eqno (unless it's called when an \eqno would be invalid).
% Then it prints the newly-defined value using \eqprint.
% 
\def\@eqdef#1#2{%
  #1{#2}% Define the label.
  %
  \@maybedisableeqno
  \eqno \eqref{#2}% Print the text.
  \@mayberestoreeqno
  \ignorespaces
}%
% 
% 
% If we are in an alignment or some other inner place, \eqno won't work.
% 
\let\@mayberestoreeqno = \relax
%
\def\@maybedisableeqno{%
  \ifinner
    \global\let\eqno = \relax
    \global\let\@mayberestoreeqno = \@restoreeqno
  \fi
}%
%
% This makes `\eqno' mean \eqno again.
% 
\let\@primitiveeqno = \eqno
\def\@restoreeqno{%
  \global\let\eqno = \@primitiveeqno
  \global\let\@mayberestoreeqno = \empty
}%
%
% 
% \eqrefn{LABEL} produces the text for the equation label LABEL, or
% something suitable if LABEL is undefined.  (It possibly issues a
% warning in the latter case as well.)
%
\let\eqrefn = \xrefn
%
% \eqref{LABEL} is the usual way to refer to equation labels; it calls
% \eqprint on the text of LABEL.
% 
\def\eqref#1{\eqprint{\eqrefn{#1}}}%
%
% 
% \eqconstruct{EQ-TEXT} constructs an equation number, i.e., the text to
% be defined as the value of a label.
% 
\let\eqconstruct = \identity
%
% \eqprint{EQ-TEXT} produces the typeset equation number EQ-TEXT.
%
\def\eqprint#1{(#1)}%
%
% \eqsubreftext{EQ-TEXT}{SUBEQ-TEXT} produces the text of a subequation
% reference.  (\eqprint is later called on the result of this to produce
% output for subequations; I didn't define any \subeqprint.)
% 
\def\eqsubreftext#1#2{#1.#2}%
%
%
% 
%  Justification of multiple input lines.
%
% You use these by saying 
% {\flushright 
% <flush right text>
% }
% 
% and similarly for \flushleft and \center.  The command must be
% embedded in a group.  The lines are set in paragraphs as usual, i.e.,
% blank lines start a new paragraph (by virtue of the
% \blanklineskipamount vertical glue being inserted).
% 
% \environment ... \endenvironment isn't appropriate in this case, since
% these ``environments'' can't be nested.
% 
\begingroup
  \catcode `\^^M = \active %
  \gdef\flushleft{%
    \def\@endjustifycmd{\@endflushleft}%
    \def\@eoljustifyaction{\null\hfil\break}%
    \let\@firstlinejustifyaction = \relax
    \@startjustify %
  }%
  \gdef\flushright{%
    \def\@endjustifycmd{\@endflushright}%
    \def\@eoljustifyaction{\break\null\hfil}%
    \def\@firstlinejustifyaction{\hfil\null}%
    \@startjustify %
  }%
  \gdef\center{%
    \def\@endjustifycmd{\@endcenter}%
    \def\@eoljustifyaction{\hfil\break\null\hfil}%
    \def\@firstlinejustifyaction{\hfil\null}%
    \@startjustify %
  }%
  %
  % We do this before starting any of the justification commands.
  \gdef\@startjustify{%
    \parskip = 0pt
    \catcode`\^^M = \active %
    \def^^M{\futurelet\next\@finjustifyreturn}%
    %
    % \@eateol is called at the beginning of each justified paragraph.
    \def\@eateol##1^^M{%
      \def\temp{##1}%
      \@firstlinejustifyaction %
      \ifx\temp\empty\else \temp^^M\fi %
    }%
    \expandafter\aftergroup\@endjustifycmd %
    \checkenv \environmenttrue %
    \par\noindent %
    \@eateol %
  }%
  %
  % If the next thing is a ^^M, insert \blanklineskipamount glue.  Then
  % do \@eoljustifyaction (which each justification command defines).
  \gdef\@finjustifyreturn{%
    \@eoljustifyaction %
    \ifx\next^^M%
      % Insert extra glue when the \@end... command does the \par.
      \def\par{\endgraf\vskip\blanklineskipamount \global\let\par = \endgraf}%
      \@endjustifycmd %
      % Get back into horizontal mode for the next line.
      \noindent %
      \@firstlinejustifyaction %
    \fi %
  }%
\endgroup
% 
\def\@endflushleft{\unpenalty{\parfillskip = 0pt plus1fil\par}\ignorespaces}%
\def\@endflushright{% Remove the \hfil\null\break we just put on.
   \unskip \setbox0=\lastbox \unpenalty
   % We have fil glue at the left of the line; \parfillskip shouldn't
   % affect that.
   {\parfillskip = 0pt \par}\ignorespaces
}%
\def\@endcenter{% Remove the \hfil\null\break we just put on.
   \unskip \setbox0=\lastbox \unpenalty
   % We have fil glue at the left of the line; \parfillskip must balance it.
   {\parfillskip = 0pt plus1fil \par}\ignorespaces
}%
% 
%
% 
%  Automatically-columnated tables.
% 
% \makecolumns N/K: organizes the entries on the following N lines into
% K columns.  If N is too small, some text beyond the end of the table
% will be incorporated into the table, probably producing an error
% message.  If N is too large, some of the entries will appear after the
% table, probably looking very out of place.
% 
% You can adjust the position of the table on the page by changing
% \parindent (space to the left of the block) and \hsize (distance from
% the left margin to the right of the block).  (No doubt inside a
% group.)  And you can allow a page break above the valign by changing
% \abovecolumnspenalty.
% 
\newcount\abovecolumnspenalty   \abovecolumnspenalty = 10000
\newcount\@linestogo         % Lines remaining to process.
\newcount\@linestogoincolumn % Lines remaining in column.
\newcount\@columndepth       % Number of lines in a column.
\newdimen\@columnwidth       % Width of each column.
\newtoks\crtok  \crtok = {\cr}%
\newcount\currentcolumn
%
% The space matches an end-of-line that will probably be there.
% 
\def\makecolumns#1/#2: {\par \begingroup
   % Set \@columndepth to the number of items we will put in a column:
   % (N - 1) / K.
   \@columndepth = #1
   \advance\@columndepth by #2
   \advance\@columndepth by -1
   \divide \@columndepth by #2
   \@linestogoincolumn = \@columndepth
   \@linestogo = #1
   %
   % We start in the first column.
   \currentcolumn = 1
   %
   \def\@endcolumnactions{%
      \ifnum \@linestogo<2 
         \the\crtok \egroup \endgroup \par % End \valign and \makecolumns.
      \else
         % We've done one more line out of the total.
         \global\advance\@linestogo by -1
         % 
         % How many left in the column?
         % 
         \ifnum\@linestogoincolumn<2
            % End this column, that was the last line.
            \global\advance\currentcolumn by 1
            \global\@linestogoincolumn = \@columndepth
            \the\crtok
         \else
            % Still got more lines to go.
            &\global\advance\@linestogoincolumn by -1
         \fi
      \fi
   }%
   %
   % Set up to read the table.
   % 
   \makeactive\^^M
   \letreturn \@endcolumnactions
   % 
   % Figure out how wide our columns are going to be; each column has
   % exactly the same template, so we can use the feature described on
   % p.241 of the TeXbook for repeating preambles.
   % 
   \@columnwidth = \hsize
     \advance\@columnwidth by -\parindent
     \divide\@columnwidth by #2
   \penalty\abovecolumnspenalty
   \noindent % It's not a paragraph (usually).
   \valign\bgroup
     &\hbox to \@columnwidth{\strut \hsize = \@columnwidth ##\hfil}\cr
     %
     % The next end-of-line starts everything going.
}%
%
% 
% 
% \numberedfootnote is like plain TeX's \footnote, but automatically
% numbered.  When you want to reset the footnote number, say
% \footnotenumber = 0.
% 
% We also provide for more general formatting than \footnote:
%   \footnotemarkseparation is the space between the reference mark and
%     the footnote text;
%  \interfootnoteskip is the space between footnotes;
%  \everyfootnote is expanded just before we typeset the footnote.
% 
% The dimensions of the footnote rule are controlled by
% \footnoterulewidth and \footnoteruleheight (the depth is always zero);
% the space after the rule is \belowfootnoterulespace.
% 
\newcount\footnotenumber
\newdimen\footnotemarkseparation \footnotemarkseparation = .5em
\newskip\interfootnoteskip \interfootnoteskip = 0pt
\newtoks\everyfootnote
\newdimen\footnoterulewidth \footnoterulewidth = 2in
\newdimen\footnoteruleheight \footnoteruleheight = 0.4pt
\newdimen\belowfootnoterulespace \belowfootnoterulespace = 2.6pt
%
\let\@plainfootnote = \footnote
\let\@plainvfootnote = \vfootnote
%
\def\vfootnote#1{\insert\footins\bgroup
  \interlinepenalty\interfootnotelinepenalty
  \splittopskip\ht\strutbox % top baseline for broken footnotes
  \advance\splittopskip by \interfootnoteskip
  \splitmaxdepth\dp\strutbox
  \floatingpenalty\@MM
  \leftskip\z@skip \rightskip\z@skip \spaceskip\z@skip \xspaceskip\z@skip
  \everypar = {}%
  \parskip = 0pt % because of the vskip
  % Even if typesetting in multicolumns, do footnotes in normal page width.
  % (We don't have any provision in the output routine for having
  % footnotes per column, anyway.)
  \ifnum\@numcolumns > 1 \hsize = \@normalhsize \fi
  \the\everyfootnote
  \vskip\interfootnoteskip
  \indent\llap{#1\kern\footnotemarkseparation}\footstrut\futurelet\next\fo@t
}%
%
\def\footnoterule{\dimen0 = \footnoteruleheight
  \advance\dimen0 by \belowfootnoterulespace
  \kern-\dimen0
  \hrule width\footnoterulewidth height\footnoteruleheight depth0pt
  \kern\belowfootnoterulespace
  \vskip-\interfootnoteskip
}%
%
\def\numberedfootnote{%
  \global\advance\footnotenumber by 1
  \@plainfootnote{$^{\number\footnotenumber}$}%
}%
%
%
% 
%  Margins.
% 
% TeX's primitives determine the type area.  But some users prefer to
% think in terms of margins.  These definitions allow one to say, for
% example, `\topmargin = 2in', instead of `\voffset=1in\advance\vsize by
% -1in'.  Constructions like `\advance\topmargin by 1in' give an error
% message, though, since \topmargin is not a parameter.  Instead, the
% macro \advancetopmargin has to be used.
% 
% 
\newdimen\paperheight \paperheight = 11in
%
\def\topmargin{\afterassignment\@finishtopmargin \dimen0}%
\def\@finishtopmargin{%
  \dimen2 = \voffset		% Remember the old \voffset.
  \voffset = \dimen0 \advance\voffset by -1in
  \advance\dimen2 by -\voffset	% Compute the change in \voffset.
  \advance\vsize by \dimen2	% Change type area accordingly.
}%
\def\advancetopmargin{%
  \dimen0 = 0pt \afterassignment\@finishadvancetopmargin \advance\dimen0
}%
\def\@finishadvancetopmargin{%
  \advance\voffset by \dimen0
  \advance\vsize by -\dimen0
}%
%
%
\def\bottommargin{\afterassignment\@finishbottommargin \dimen0}%
\def\@finishbottommargin{%
  \@computebottommargin		% Result in \dimen2.
  \advance\dimen2 by -\dimen0	% Compute the change in the bottom margin.
  \advance\vsize by \dimen2	% Change the type area.
}%
\def\advancebottommargin{%
  \dimen0 = 0pt \afterassignment\@finishadvancebottommargin \advance\dimen0
}%
\def\@finishadvancebottommargin{%
  \advance\vsize by -\dimen0
}%
%
% Find the current bottom margin, putting the result in \dimen2.
% 
\def\@computebottommargin{%
  \dimen2 = \paperheight	% The total paper size.
  \advance\dimen2 by -\vsize	% Less the text size.
  \advance\dimen2 by -\voffset	% Less the offset at the top.
  \advance\dimen2 by -1in	% Less the default offset.
}%
% 
% 
\newdimen\paperwidth \paperwidth = 8.5in
%
\def\leftmargin{\afterassignment\@finishleftmargin \dimen0}%
\def\@finishleftmargin{%
  \dimen2 = \hoffset		% Remember the old \hoffset.
  \hoffset = \dimen0 \advance\hoffset by -1in
  \advance\dimen2 by -\hoffset	% Compute the change in \hoffset.
  \advance\hsize by \dimen2	% Change type area accordingly.
}%
\def\advanceleftmargin{%
  \dimen0 = 0pt \afterassignment\@finishadvanceleftmargin \advance\dimen0
}%
\def\@finishadvanceleftmargin{%
  \advance\hoffset by \dimen0
  \advance\hsize by -\dimen0
}%
%
%
\def\rightmargin{\afterassignment\@finishrightmargin \dimen0}%
\def\@finishrightmargin{%
  \@computerightmargin		% Result in \dimen2.
  \advance\dimen2 by -\dimen0	% Compute the change in the right margin.
  \advance\hsize by \dimen2	% Change the type area.
}%
\def\advancerightmargin{%
  \dimen0 = 0pt \afterassignment\@finishadvancerightmargin \advance\dimen0
}%
\def\@finishadvancerightmargin{%
  \advance\hsize by -\dimen0
}%
%
% Find the current right margin, putting the result in \dimen2.
% 
\def\@computerightmargin{%
  \dimen2 = \paperwidth		% The total paper size.
  \advance\dimen2 by -\hsize	% Less the text size.
  \advance\dimen2 by -\hoffset	% Less the offset at the left.
  \advance\dimen2 by -1in	% Less the default offset.
}%
% 
% 
% 
%  Double column output.
%
% \doublecolumns begins double column output.  It can be called
% in the midst of a page.  \singlecolumn restores single column 
% output.  (It would be wrong to require \enddoublecolumns, because 
% often one wants double column mode to continue to the end of 
% the document.)
%
% The basic approach is that of Appendix E of the TeXbook, p.417.
% David Guichard made significant improvements to my original implementation.
%
% The glue here (the default is intended to be one linespace) is inserted
% before double columns start, and after they end.
%
\newskip\abovecolumnskip \abovecolumnskip = \bigskipamount
\newskip\belowcolumnskip \belowcolumnskip = \bigskipamount
%
% Space between the columns. It can be changed as desired.
\newdimen\gutter \gutter = 2pc
%
% These registers are needed for dealing with switching back and forth.
\newbox\@partialpage
\newdimen\@columnhsize
\newdimen\@normalhsize
\newdimen\@normalvsize
\newtoks\previousoutput
%
% Some synonymous ways to refer to multiple column modes.
\def\quadcolumns{\@columns4}%
\def\triplecolumns{\@columns3}%
\def\doublecolumns{\@columns2}%
\def\begincolumns#1{\ifcase#1\relax \or \singlecolumn \or \@columns2 \or
                            \@columns3 \or \@columns4 \else \relax \fi}%
\let\endcolumns = \singlecolumn
\let\@ndcolumns = \relax
%
% Set this by default so \vfootnote can unconditionally inspect it.
\chardef\@numcolumns = 1
%
% Start typesetting with #1 columns.
\def\@columns#1{%
  \@ndcolumns
  %
  \let\@ndcolumns = \@endcolumns
  \chardef\@numcolumns = #1
  %
  \par                     % Shouldn't start in horizontal mode.
  \previousoutput = \expandafter{\the\output}%
  %
  % Figure out how wide the columns should be -- for n columns,
  % decrement by n - 1 gutters.
  \@columnhsize = \hsize
  \count@ = \@numcolumns
  \advance\count@ by -1
  \advance\@columnhsize by -\count@\gutter
  \divide\@columnhsize by \@numcolumns
  %
  % Set up to grab the page so far and save it in \@partialpage.
  \output = {\global\setbox\@partialpage =
    \vbox{\unvbox255\vskip\abovecolumnskip}%
  }%
  %
  % \pagegoal is the size that TeX will make \box255.  We want a box
  % exactly the size of the current height of the page, i.e., \pagetotal.
  \pagegoal = \pagetotal
  %
  % Expand the \output we just defined.
  \eject
  %
  % Reset \output to prepare for the first real page break.
  \output = {\@columnoutput}%
  \@normalhsize = \hsize
  \@normalvsize = \vsize
  \hsize = \@columnhsize
  %
  % Compute \vsize based on what's already on the page
  % and the number of columns. Also change the mag factor for insertions.
  \advance\vsize by -\ht\@partialpage
  %
  \advance\vsize by -\ht\footins
  \ifvoid\footins\else \advance\vsize by -\skip\footins \fi
  \multiply\count\footins by \@numcolumns
  %
  \advance\vsize by -\ht\topins
  \ifvoid\topins\else \advance\vsize by -\skip\topins \fi
  \multiply\count\topins by \@numcolumns
  %
  \global\vsize = \@numcolumns\vsize
}%
%
% When this is invoked box 255 contains just the right amount of
% material, whether triggered by an output routine or a change in the
% number of columns. Because columns have to contain an integral number
% of lines of type, we take a bit of care with balancing the heights of
% the columns to prevent either losing material or having a very short
% last column.
%
% when a page ends due to \bye or \eject, box 255 will contain lots of
% white space, so the columns will not look balanced. To fix this use
% \singlecolumn before ending the page.
%
\def\@columnsplit{%
  \splittopskip = \topskip
  \splitmaxdepth = \baselineskip
  %
  % \dimen0 will be the height that the double-column material on this
  % page should have, i.e., the height of the page (\singlecolumvsize)
  % minus single-column material, which includes insertions.  (If you
  % want your insertions to respect the columns, you will have to
  % change the output routine.)  If you add more insertions, they
  % should be taken into account both here and in \singlecolumn.
  %
  % Unfortunately, we lose on flexible glue because we must
  % \vsplit to a <dimen>.
  \dimen@ = \ht255
    \divide\dimen@ by \@numcolumns
 %
 % Split the long scroll into columns.
 \begingroup
    % We do not want to see underfull \vbox messages unless the final
    % page is underfull.
    \vbadness = 10000
    %
    % The first (leftmost) column.
    \global\setbox1 = \vsplit255 to \dimen@  \global\wd1 = \hsize
    %
    % The second column.
    \global\setbox3 = \vsplit255 to \dimen@  \global\wd3 = \hsize
    %
    \ifnum\@numcolumns > 2
      % The third column, if requested.
      \global\setbox5 = \vsplit255 to \dimen@ \global\wd5 = \hsize
    \fi
    \ifnum\@numcolumns > 3
      % The fourth column, likewise if requested.
      \global\setbox7 = \vsplit255 to \dimen@ \global\wd7 = \hsize
    \fi
  \endgroup
  % 
  % Preserve what's left over.
  \setbox0 = \box255
  %
  % Set up \box255 with the real output page, as the previous output
  % routine expects.
  \global\setbox255 = \vbox{%
    \unvbox\@partialpage
    \ifcase\@numcolumns \relax\or\relax
      \or \hbox to \@normalhsize{\box1\hfil\box3}%
      \or \hbox to \@normalhsize{\box1\hfil\box3\hfil\box5}%
      \or \hbox to \@normalhsize{\box1\hfil\box3\hfil\box5\hfil\box7}%
    \fi
  }%
  %
  % Save what's left over in a private register before calling their
  % output routine.
  \setbox\@partialpage = \box0
}%
%
% Our output routine splits the columns and then calls the previous one.
% 
\def\@columnoutput{%
  \@columnsplit
  \hsize = \@normalhsize % Local to \output's group.
  \vsize = \@normalvsize
  \the\previousoutput
  %
  % Put back what didn't fit.
  \unvbox\@partialpage
  \penalty\outputpenalty
  %
  % The correct vsize is the original vsize times the
  % number of columns.
  \global\vsize = \@numcolumns\@normalvsize
}%
%
% Go back to single-column typesetting.  Assume \doublecolumns has
% been called.
% 
\def\singlecolumn{%
  \@ndcolumns
  \chardef\@numcolumns = 1
  \vskip\belowcolumnskip
  \nointerlineskip
}%
%
\def\@endcolumns{%
  \global\let\@ndcolumns = \relax
  \par % Shouldn't start in horizontal mode.
  %
  \global\output = {\global\setbox1 = \box255}%
  \pagegoal = \pagetotal
  \eject                    % Exercise the page builder, i.e., \output.
  \global\setbox255 = \box1 % Retrieve what the fake \output set.
  %
  % \box255 now has the double-column material.  On the page where we
  % switch back to one column, the double-column material might not
  % fill up the page.  We want to split whatever is there.
  \@columnsplit
  %
  \global\vsize = \@normalvsize
  \global\hsize = \@normalhsize
  \global\output = \expandafter{\the\previousoutput}%
  %
  \ifvoid\topins\else\topinsert\unvbox\topins\endinsert\fi
  \unvbox255
}%
%
% We don't have any way to force a column eject, since the \output
% routine is only prepared to split up a full page of material. Instead,
% we provide the following as a guess at enough space to fill up the
% current column.
\def\columnfill{%
  \dimen@ = \@normalvsize
  \advance\dimen@ by -\pagetotal
  \kern\dimen@
}
%
%
\let\wlog = \@plainwlog
\catcode`@ = \@eplainoldatcode
%
\def\fmtname{eplain}%
\def\eplain{t}%
{\edef\plainversion{\fmtversion}%
 \xdef\fmtversion{REPLACE-WITH-VERSION: REPLACE-WITH-DAY-MONTH-YEAR (and plain \plainversion)}%
}%
%
% 
% 
% 
% Local variables:
% page-delimiter: "^% \f"
% End:
