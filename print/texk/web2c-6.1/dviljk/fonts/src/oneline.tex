% oneline.tex for HP fonts.
\advance\hsize by 1in
\hoffset = -.5in
\def\dofont{\fontline}

\nopagenumbers
\def\startfamily#1{\def\family{#1}\bigbreak} % glue disappears at page breaks

% Print a one-line sample of the font #1.
\def\fontline#1{%
  \def\variant{#1}%
  \font\testfont = \family\variant
  \testfont
  \vskip3pt
  \line{%
    \hbox to 42pt{\tt \fontname\testfont\hfil}%
    \fullname
    \hfil
    ABCDEFGHIJKLMNOPQRSTUVWXYZ
    abcdefghijklmnopqrstuvwxyz
    0123456789%
  }
  \vfil
}

\def\fullname{%
  \csname \family family\endcsname
  \space
  \csname \variant variant\endcsname
  \space
}

\def\calfamily{Albertus}
\def\caofamily{Antique Olive}
\def\ccdfamily{Clarendon}
\def\ccrfamily{Courier}
\def\cgmfamily{Garamond}
\def\clgfamily{Letter Gothic}
\def\cmgfamily{Marigold}
\def\copfamily{Omega}
\def\cotfamily{Coronet}
\def\ctmfamily{Times}
\def\cunfamily{Univers}
\def\hwifamily{Wingdings}
\def\mhvfamily{Arial}
\def\mntfamily{Times New}
\def\msyfamily{Symbol}

\def\bicvariant{bold italic condensed}
\def\bivariant{bold italic}
\def\brcvariant{bold condensed}
\def\bvariant{bold}
\def\micvariant{medium italic condensed}
\def\mrcvariant{medium condensed}
\def\mvariant{medium}
\def\rivariant{italic}
\def\rrcvariant{condensed}
\def\rvariant{}
\def\xvariant{extra bold}

\startfamily{cal}
\dofont{m}
\dofont{x}

\startfamily{cao}
\dofont{r}
\dofont{ri}
\dofont{b}

\startfamily{ccd}
\dofont{rrc}

\startfamily{ccr}
\dofont{r}
\dofont{ri}
\dofont{b}
\dofont{bi}

\startfamily{cgm}
\dofont{r}
\dofont{ri}
\dofont{b}
\dofont{bi}

\startfamily{clg}
\dofont{r}
\dofont{ri}
\dofont{b}

\startfamily{cmg}
\dofont{r}

\startfamily{cop}
\dofont{r}
\dofont{ri}
\dofont{b}
\dofont{bi}

\startfamily{cot}
\dofont{r}

\startfamily{ctm}
\dofont{r}
\dofont{ri}
\dofont{b}
\dofont{bi}

\startfamily{cun}
\dofont{m}
\dofont{mi}
\dofont{b}
\dofont{bi}

\startfamily{cun}
\dofont{mrc}
\dofont{mic}
\dofont{brc}
\dofont{bic}

\startfamily{hwi}
\dofont{r}

\startfamily{mhv}
\dofont{r}
\dofont{ri}
\dofont{b}
\dofont{bi}

\startfamily{mnt}
\dofont{r}
\dofont{ri}
\dofont{b}
\dofont{bi}

\startfamily{msy}
\dofont{r}

\bye
